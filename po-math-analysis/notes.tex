\documentclass{article}

\usepackage{amsmath,amssymb,amsthm}
\usepackage{csquotes}
\usepackage[shortlabels]{enumitem}

\newtheorem{theorem}{Theorem}
\numberwithin{theorem}{section}

\title{Notes to ``Principles of Mathematical Analysis'', 3rd edition (Rudin)}
\author{Ng Wei En}

\begin{document}

\maketitle
\tableofcontents
\newpage

\section{The Real and Complex Number Systems}

\subsection*{Ordered Sets}
\addcontentsline{toc}{subsection}{Ordered Sets}

\paragraph{Supremum \& infimum.} Suppose $S$ is an ordered set, $E \subset S$,
and $E$ is bounded above. Suppose there exists an $\alpha \in S$ with the
following properties:
\begin{enumerate}[(i)]
  \item $\alpha$ is an upper bound of $E$.
  \item If $\gamma < \alpha$ then $\gamma$ is not an upper bound of $E$.
\end{enumerate}
Then $\alpha$ is called the \textbf{least upper bound} of $E$ or the
\textbf{supremum} of $E$, and we write \[
  \alpha = \sup E.
\]
The \textbf{greatest lower bound}, or \textbf{infimum}, of a set $E$ which is
bounded below is defined in the same manner: The statement \[
  \alpha = \inf E
\] means that $\alpha$ is a lower bound of $E$ and that no $\beta$ with $\beta >
\alpha$ is a lower bound of $E$.

\paragraph{Least-upper-bound property.} An ordered set $S$ is said to have the
\textbf{least-upper-bound property} if the following is true: \[
  \text{If } E \subset S, E \text{ is not empty}, \text{ and } E \text{ is
  bounded above}, \text{ then } \sup E \text{ exists in } S.
\]

\setcounter{theorem}{10}
\begin{theorem}
  Suppose $S$ is an ordered set with the least-upper-bound property, $B \subset
  S$, $B$ is not empty, and $B$ is bounded below. Let $L$ be the set of all
  lower bounds of $B$. Then \[
    \alpha = \sup L
  \] exists in $S$, and $\alpha = \inf B$. In particular, $\inf B$ exists in
  $S$.
\end{theorem}

\subsection*{The Real Field}
\addcontentsline{toc}{subsection}{The Real Field}

\setcounter{theorem}{18}
\begin{theorem}
  There exists an ordered field $R$ which has the least-upper-bound property.
  Moreover, $R$ contains $Q$ as a subfield.
\end{theorem}

\begin{theorem}
  \begin{enumerate}[(a)]
    \item If $x \in R$, $y \in R$, and $x > 0$, then there is a positive integer
      $n$ such that $nx > y$.
    \item If $x \in R, y \in R$, and $x < y$, then there exists a $p \in Q$ such
      that $x < p < y$.
  \end{enumerate}
\end{theorem}

\begin{theorem}
  For every real $x > 0$ and every integer $n > 0$ there is one and only one
  positive real $y$ such that $y^n = x$.
\end{theorem}

\subsection*{The Complex Field}
\addcontentsline{toc}{subsection}{The Complex Field}

\setcounter{theorem}{34}
\begin{theorem}[Schwarz's inequality]
  If $a_1, \ldots, a_n$ and $b_1, \ldots, b_n$ are complex numbers, then \[
    \left|\sum_{j=1}^n a_j\overline{b}_j\right|^2 \leq
    \sum_{j=1}^n |a_j|^2 \sum_{j=1}^n |b_j|^2
  \].
\end{theorem}

\end{document}
