\documentclass{article}

\usepackage{amsmath,amssymb,amsthm}
\usepackage[shortlabels]{enumitem}

\newtheorem{theorem}{Theorem}
\numberwithin{theorem}{section}
\newtheorem*{corollary*}{Corollary}

\title{Notes to ``Principles of Mathematical Analysis'', 3rd edition (Rudin)}
\author{Ng Wei En}

\begin{document}

\maketitle
\tableofcontents
\newpage

\section{The Real and Complex Number Systems}

\subsection*{Ordered Sets}
\addcontentsline{toc}{subsection}{Ordered Sets}

\paragraph{Supremum \& infimum.} Suppose $S$ is an ordered set, $E \subset S$,
and $E$ is bounded above. Suppose there exists an $\alpha \in S$ with the
following properties:
\begin{enumerate}[(i)]
  \item $\alpha$ is an upper bound of $E$.
  \item If $\gamma < \alpha$ then $\gamma$ is not an upper bound of $E$.
\end{enumerate}
Then $\alpha$ is called the \textbf{least upper bound} of $E$ or the
\textbf{supremum} of $E$, and we write \[
  \alpha = \sup E.
\]
The \textbf{greatest lower bound}, or \textbf{infimum}, of a set $E$ which is
bounded below is defined in the same manner: The statement \[
  \alpha = \inf E
\] means that $\alpha$ is a lower bound of $E$ and that no $\beta$ with $\beta >
\alpha$ is a lower bound of $E$.

\paragraph{Least-upper-bound property.} An ordered set $S$ is said to have the
\textbf{least-upper-bound property} if the following is true: \[
  \text{If } E \subset S, E \text{ is not empty}, \text{ and } E \text{ is
  bounded above}, \text{ then } \sup E \text{ exists in } S.
\]

\setcounter{theorem}{10}
\begin{theorem}
  Suppose $S$ is an ordered set with the least-upper-bound property, $B \subset
  S$, $B$ is not empty, and $B$ is bounded below. Let $L$ be the set of all
  lower bounds of $B$. Then \[
    \alpha = \sup L
  \] exists in $S$, and $\alpha = \inf B$. In particular, $\inf B$ exists in
  $S$.
\end{theorem}

\subsection*{The Real Field}
\addcontentsline{toc}{subsection}{The Real Field}

\setcounter{theorem}{18}
\begin{theorem}
  There exists an ordered field $R$ which has the least-upper-bound property.
  Moreover, $R$ contains $Q$ as a subfield.
\end{theorem}

\begin{theorem}
  \begin{enumerate}[(a)]
    \item If $x \in R$, $y \in R$, and $x > 0$, then there is a positive integer
      $n$ such that $nx > y$.
    \item If $x \in R, y \in R$, and $x < y$, then there exists a $p \in Q$ such
      that $x < p < y$.
  \end{enumerate}
\end{theorem}

\begin{theorem}
  For every real $x > 0$ and every integer $n > 0$ there is one and only one
  positive real $y$ such that $y^n = x$.
\end{theorem}

\subsection*{The Complex Field}
\addcontentsline{toc}{subsection}{The Complex Field}

\setcounter{theorem}{34}
\begin{theorem}[Schwarz's inequality]
  If $a_1, \ldots, a_n$ and $b_1, \ldots, b_n$ are complex numbers, then \[
    \left|\sum_{j=1}^n a_j\overline{b}_j\right|^2 \leq
    \sum_{j=1}^n |a_j|^2 \sum_{j=1}^n |b_j|^2
  \].
\end{theorem}

\section{Basic Topology}

\subsection*{Finite, Countable, and Uncountable Sets}
\addcontentsline{toc}{subsection}{Finite, Countable, and Uncountable Sets}

For any positive integer $n$, let $J_n$ be the set whose elements are the
integers $1, 2, \ldots, n$; let $J$ be the set consisting of all positive
integers. For any set $A$, we say:
\begin{itemize}[(a)]
  \item $A$ is \emph{finite} if $A \sim J_n$ for some $n$ (the empty set is also
    considered to be finite).
  \item $A$ is \emph{infinite} if $A$ is not finite.
  \item $A$ is \emph{countable} if $A \sim J$.
  \item $A$ is \emph{uncountable} if $A$ is neither finite nor countable.
  \item $A$ is \emph{at most countable} if $A$ is finite or countable.
\end{itemize}

\paragraph{Sequences.} A \emph{sequence} is a function $f$ defined on the set
$J$ of all positive integers. If $f(n) = x_n$, for $n \in J$, the sequence $f$
can be denoted by the symbol $\{x_n\}$, or by $x_1, x_2, x_3, \ldots$. The
values of $f$, that is, the elements $x_n$, are called the \emph{terms} of the
sequence. If $A$ is a set and if $x_n \in A$ for all $n \in J$, then $\{x_n\}$
is said to be a \emph{sequence} in $A$, or a \emph{sequence of elements of $A$}.

\setcounter{theorem}{7}
\begin{theorem}
  Every infinite subset of a countable set $A$ is countable.
\end{theorem}

\setcounter{theorem}{11}
\begin{theorem}
  Let $\{E_n\}$, $n = 1, 2, 3, \ldots$, be a sequence of countable sets, and put
  \[
    S = \bigcup_{n=1}^{\infty} E_n.
  \] Then $S$ is countable.
\end{theorem}

\begin{theorem}
  Let $A$ be a countable set, and let $B_n$ be the set of all $n$-tuples $(a_1,
  \ldots, a_n)$, where $a_k \in A$ ($k = 1, \ldots, n$), and the elements $a_1,
  \ldots, a_n$ need not be distinct. Then $B_n$ is countable.
\end{theorem}

\begin{corollary*}
  The set of all rational numbers is countable.
\end{corollary*}

\setcounter{theorem}{13}
\begin{theorem}
  Let $A$ be the set of all sequences whose elements are the digits 0 and 1.
  This set is uncountable.
\end{theorem}

\subsection*{Metric Spaces}
\addcontentsline{toc}{subsection}{Metric Spaces}

\paragraph{Metric spaces.} A set $X$, whose elements we shall call
\emph{points}, is said to be a \emph{metric space} if with any two points $p$
and $q$ of $X$ there is associated a real number $d(p, q)$, called the
\emph{distance} from $p$ to $q$, such that
\begin{enumerate}[(a)]
  \item $d(p, q) > 0$ if $p \neq 0$; $d(p, p) = 0$;
  \item $d(p, q) = d(q, p)$;
  \item $d(p, q) \leq d(p, r) + d(r, q)$, for any $r \in X$.
\end{enumerate}
Any function with these three properties is called a \emph{distance function},
or a \emph{metric}.

\paragraph{$k$-cells.} If $a_i < b_i$ for $i = 1, \ldots, k$, the set of all
points $\mathbf{x} = (x_1, \ldots, x_k)$ in $R^k$ whose coordinates satisfy the
inequalities $a_i \leq x_i \leq b_i$ ($1 \leq i \leq k$) is called a
\emph{$k$-cell}.

\paragraph{Open \& closed balls.} If $\mathbf{x} \in R^k$ and $r > 0$, the
\emph{open} (or \emph{closed}) \emph{ball} $B$ with center at $\mathbf{x}$ and
radius $r$ is defined to be the set of all $\mathbf{y} \in R^k$ such that
$|\mathbf{y} - \mathbf{x}| < r$ (or $|\mathbf{y} - \mathbf{x}| \leq r$).

\paragraph{Convex.} We call a set $E \subset R^k$ \emph{convex} if \[
  \lambda\mathbf{x} + (1 - \lambda)\mathbf{y} \in E
\] whenever $\mathbf{x} \in E$, $\mathbf{y} \in E$, and $0 < \lambda < 1$.

\paragraph{Neighbourhoods.} A \emph{neighborhood} of $p$ is a set $N_r(p)$
consisting of all $q$ such that $d(p, q) < r$, for some $r > 0$. The number $r$
is called the \emph{radius} of $N_r(p)$.

\paragraph{Limit \& isolated points.} A point $p$ is a \emph{limit point} of the
set $E$ if \emph{every} neighborhood of $p$ contains a point $q \neq p$ such
that $q \in E$. Otherwise, it is called an \emph{isolated point} of $E$.

\paragraph{Interior points.} A point $p$ is an \emph{interior point} of $E$ if
there is a neighborhood $N$ of $p$ such that $N \subset E$.

\paragraph{Closed \& open sets.} The set $E$ is \emph{closed} if every limit
point of $E$ is a point of $E$, and \emph{open} if every point of $E$ is an
interior point of $E$.

\paragraph{Complement set.} The \emph{complement} of $E$ (denoted by $E^c$)
is the set of all points $p \in X$ such that $p \notin E$.

\paragraph{Perfect sets.} The set $E$ is \emph{perfect} if $E$ is closed and if
every point of $E$ is a limit point of $E$.

\paragraph{Bounded sets.} The set $E$ is \emph{bounded} if there is a real
number $M$ and a point $q \in X$ such that $d(p, q) < M$ for all $p \in E$.

\paragraph{Dense sets.} The set $E$ is \emph{dense in the set $X$} if every
point of $X$ is a limit point of $E$, or a point of $E$ (or both).

\setcounter{theorem}{18}
\begin{theorem}
  Every neighborhood is an open set.
\end{theorem}

\begin{theorem}
  If $p$ is a limit point of a set $E$, then every neighborhood of $p$ contains
  infinitely many points of $E$.
\end{theorem}

\begin{corollary*}
  A finite point set has no limit points.
\end{corollary*}

\setcounter{theorem}{21}
\begin{theorem}
  Let $\{E_{\alpha}\}$ be a (finite or infinite) collection of sets
  $E_{\alpha}$. Then \[
    \left(\bigcup_{\alpha} E_{\alpha}\right)^c =
    \bigcap_{\alpha} E_{\alpha}^c.
  \]
\end{theorem}

\begin{theorem}
  A set $E$ is open if and only if its complement is closed.
\end{theorem}

\begin{corollary*}
  A set $F$ is closed if and only if its complement is open.
\end{corollary*}

\setcounter{theorem}{23}
\begin{theorem}
  \begin{enumerate}[(a)]
    \item For any collection $\{G_{\alpha}\}$ of open sets, $\cup_{\alpha}
      G_{\alpha}$ is open.
    \item For any collection $\{F_{\alpha}\}$ of closed sets, $\cap_{\alpha}
      F_{\alpha}$ is closed.
    \item For any finite collection $G_1, \cdots, G_n$ of open sets,
      $\cap_{i=1}^n G_i$ is open.
    \item For any finite collection $F_1, \cdots, F_n$ of closed sets,
      $\cup_{i=1}^n F_i$ is closed.
  \end{enumerate}
\end{theorem}

\end{document}
