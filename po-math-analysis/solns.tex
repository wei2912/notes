\documentclass{article}

\usepackage{amsmath,amssymb,amsthm}
\usepackage[shortlabels]{enumitem}

\title{Solutions to ``Principles of Mathematical Analysis'', 3rd edition (Rudin)}
\author{Ng Wei En}

\begin{document}

\maketitle
\tableofcontents
\newpage

\section{The Real and Complex Number Systems}

\paragraph{6.} Fix $b > 1$.
\begin{enumerate}[(a)]
  \item If $m, n, p, q$ are integers, $n > 0$, $q > 0$, and $r = m/n = p/q$,
    prove that \[
      (b^m)^{1/n} = (b^p)^{1/q}.
    \] Hence it makes sense to define $b^r = (b^m)^{1/n}$.
    \begin{proof}
      Since $mq = np$, \[
        ((b^m)^{1/n})^{mq} = ((b^m)^{1/n})^{np} = b^{mp}.
      \] Applying \textbf{Theorem 1.21} twice, we have \[
        (b^m)^{1/n} = (b^p)^{1/q}.
      \]
    \end{proof}

  \item Prove that $b^{r+s} = b^rb^s$ if $r$ and $s$ are rational.
    \begin{proof}
      Let $r = m/n$, $s = p/q$. Then
      \begin{align*}
        (b^{r+s})^{nq}
        &= ((b^{mq+np})^{1/nq})^{nq} \\
        &= b^{mq+np} \\
        &= b^{mq}b^{np} \\
        &= ((b^m)^{1/n}(b^p)^{1/q})^{nq} = (b^rb^s)^{nq}.
      \end{align*}
      Hence, by \textbf{Theorem 1.21}, $b^{r+s} = b^rb^s$.
    \end{proof}

  \item If $x$ is real, define $B(x)$ to be the set of all numbers $b^t$, where
    $t$ is rational and $t \leq x$. Prove that \[
      b^r = \sup B(r)
    \] when $r$ is rational. Hence it makes sense to define \[
      b^x = \sup B(x)
    \] for every real $x$.
    \begin{proof}
      It is clear that $b^r$ is an upper bound for $B(r)$. For $q \in Q^+$, $b
      > 1 \implies b^q > 1$. Hence if $t < r$, $b^t < b^tb^{r-t} < b^{t+(r-t)}
      = b^r$, and so $b^r$ is the least upper bound.
    \end{proof}

  \item Prove that $b^{x+y} = b^xb^y$ for all real $x$ and $y$.
    \begin{proof}
      By definition $b^{x+y} = \sup B(x + y)$, where $B(x + y)$ is the set of
      all numbers $b^t$ with $t$ rational and $t \leq x + y$. Let $r$ be any
      rational number satisfying $t - y < r < x$, and let $s = t - r$. Then $t$
      can be rewritten as $r + s$, with $r \leq x$ and $s \leq y$. Conversely,
      any pair of rational numbers $r, s$ with $r \leq x$ and $s \leq y$ yields
      a rational sum $t = r + s \leq x + y$. Hence $B(x + y)$ can be described
      as the set of all products $uv$, where $u \in B(x)$ and $v \in B(y)$.

      Let $M = \sup B(x)\sup B(y)$. Since any such product is at most $M$, we
      can conclude that $M$ is an upper bound for $B(x + y)$. Suppose that for
      some real $c$ we have $0 < c < M = \sup B(x)\sup B(y)$. Then \[
        \frac{c}{\sup B(x)} < \sup B(y).
      \] Let $m = \frac{1}{2}\left(\frac{c}{\sup B(x)} + \sup B(y)\right)$; then
      \[
        \frac{c}{\sup B(x)} < m < \sup B(y)
      \] and there exists $u \in B(x), v \in B(y)$ such that $c/m \leq u$ and $m
      \leq v$. But $c < M$, so $c$ cannot be an upper bound for $B(x + y)$ and
      it follows that $\sup B(x + y) = \sup B(x)\sup B(y)$, or \[
        b^{x+y} = b^xb^y,
      \] as required.
    \end{proof}
\end{enumerate}

\paragraph{7.} Fix $b > 1, y > 0$, and prove that there is a unique real $x$
such that $b^x = y$, by completing the following outline. (This $x$ is called
the \emph{logarithm of $y$ to the base $b$}.)
\begin{enumerate}[(a)]
  \item For any positive integer $n$, $b^n - 1 \geq n(b - 1)$.
    \begin{proof}
      Using the binomial theorem, \[
        b^n = (1 + (b - 1))^n \geq 1^n + n1^{n-1}(b - 1),
      \] so $b^n - 1 \geq n(b - 1)$.
    \end{proof}

  \item Hence $b - 1 \geq n(b^{1/n} - 1)$.
    \begin{proof}
      Replacing $b$ with $b^{1/n}$, $b - 1 \geq n(b^{1/n} - 1)$.
    \end{proof}

  \item If $t > 1$ and $n > (b - 1)/(t - 1)$, then $b^{1/n} < t$.
    \begin{proof}
      From \[
        \frac{b - 1}{t - 1}(b^{1/n} - 1) < n(b^{1/n} - 1) \leq b - 1,
      \] it is clear that $b^{1/n} - 1 < t - 1$ and so $b^{1/n} < t$.
    \end{proof}

  \item If $w$ is such that $b^w < y$, then $b^{w+(1/n)} < y$ for sufficiently
    large $n$; to see this, apply part (c) with $t = y \cdot b^{-w}$.
    \begin{proof}
      If $n > (b - 1)/(y \cdot b^{-w} - 1)$, then from (c) $b^{1/n} < y \cdot
      b^{-w}$ and so $b^{w+(1/n)} < y$.
    \end{proof}

  \item If $b^w > y$, then $b^{w-(1/n)} > y$ for sufficiently large $n$.
    \begin{proof}
      Applying part (c) with $t = y^{-1} \cdot b^{w}$, if $n > (b - 1)/(y^{-1}
      \cdot b^{w} - 1)$ then $b^{1/n} < y^{-1} \cdot b^{w}$ and so $b^{w-(1/n)}
      > y$.
    \end{proof}

  \item Let $A$ be the set of all $w$ such that $b^w < y$, and show that $x =
    \sup A$ satisfies $b^x = y$.
    \begin{proof}
      Suppose $b^x < y$. Then by (d), $b^{x+(1/n)} < y$ for sufficiently large
      $n$ and so $x + (1/n) \in A$, contradicting that $x$ is an upper bound on
      $A$.

      On the other hand, suppose $b^x > y$. Then by (e), $b^{x-(1/n)} > y$ for
      sufficiently large $n$ and so so $x - (1/n)$ is an upper bound on $A$,
      contradicting that $x$ is the least upper bound on $A$.
    \end{proof}

  \item Prove that this $x$ is unique.
    \begin{proof}
      Suppose $x_1, x_2 \in R$, $x_1 < x_2$ and $b^{x_1} = b^{x_2}$. Then \[
        b^{x_1}b^{x_2-x_1} = b^{x_2} = b^{x_1},
      \] so $b^{x_2-x_1} = 1$ for $x_2 - x_1 > 0$ which is impossible.
    \end{proof}
\end{enumerate}

\end{document}
