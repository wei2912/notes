\documentclass{article}

\usepackage{amsmath,amssymb,amsthm}
\usepackage[shortlabels]{enumitem}

\newtheorem{theorem}{Theorem}
\newtheorem{lemma}{Lemma}
\newtheorem*{lemma*}{Lemma}

\begin{document}

\title{Chapter 1: The Wave Function}
\maketitle

\section{Schr\"{o}dinger's Equation}

A particle's position is defined by its \emph{wave function} $\Psi(x, t)$,
which can be obtained by solving \textbf{Schr\"{o}dinger's equation}:
\begin{equation} \label{eq:sch-eqn}
  \boxed{
    i\hbar \frac{\partial \Psi}{\partial t}
    = -\frac{\hbar^2}{2m} \frac{\partial^2 \Psi}{\partial x^2} + V\Psi.
  }
\end{equation}

Born's \emph{statistical interpretation} of the wave function states the
probability of finding the particle between $a$ and $b$ at time $t$ is \[
  \int_a^b |\Psi(x, t)|^2 \,dx.
\] It follows that the integral of $|\Psi|^2$
must be 1, i.e. the particle must be somewhere:
\begin{equation} \label{eq:sch-prob}
  \boxed{
    \int_{-\infty}^{+\infty} |\Psi(x, t)|^2 \,dx = 1.
  }
\end{equation}

\subsection{Normalisation}

If $\Psi(x, t)$ is a solution, then $A \Psi(x, t)$ for any complex constant $A$
must also be a solution, so any appropriate multiplicative factor can be chosen
to normalise the wave function.

\begin{lemma*}
  The Schr\"{o}dinger equation preserves the normalization of the wave function
  as time goes on and $\Psi$ evolves.
\end{lemma*}

\begin{proof}
  To begin with, \[
    \frac{d}{dt} \int_{-\infty}^{+\infty} |\Psi(x, t)|^2 \,dx
    = \int_{-\infty}^{+\infty} \frac{\partial}{\partial t} |\Psi(x, t)|^2 \,dx.
  \]
  By the product rule,
  \begin{equation} \label{eq:sch-norm-1}
    \frac{\partial}{\partial t}|\Psi|^2
    = \frac{\partial}{\partial t} (\Psi^* \Psi)
    = \Psi^* \frac{\partial \Psi}{\partial t} +
    \frac{\partial \Psi^*}{\partial t} \Psi.
  \end{equation}
  Now the Schr\"{o}dinger equation says that
  \begin{equation} \label{eq:sch-norm-2}
    \frac{\partial \Psi}{\partial t}
    = \frac{i\hbar}{2m} \frac{\partial^2 \Psi}{\partial x^2} -
    \frac{i}{\hbar} V \Psi,
  \end{equation}
  and hence also
  \begin{equation} \label{eq:sch-norm-3}
    \frac{\partial \Psi^*}{\partial t}
    = -\frac{i\hbar}{2m} \frac{\partial^2 \Psi^*}{\partial x^2} +
    \frac{i}{\hbar} V \Psi^*,
  \end{equation}
  so
  \begin{equation} \label{eq:sch-norm-4}
    \frac{\partial}{\partial t} |\Psi|^2
    = \frac{i\hbar}{2m} \left(
      \Psi^* \frac{\partial^2 \Psi}{\partial x^2} -
      \frac{\partial^2 \Psi^*}{\partial x^2} \Psi
    \right)
    = \frac{\partial}{\partial t} \left[
      \frac{i\hbar}{2m} \left(
        \Psi^* \frac{\partial \Psi}{\partial x} -
        \frac{\partial \Psi^*}{\partial x} \Psi
      \right)
    \right]
  \end{equation}
  The integral in \eqref{eq:sch-norm-1} can now be evaluated explicitly: \[
    \frac{\partial}{\partial t} \int_{-\infty}^{+\infty} |\Psi(x, t)|^2 \,dx
    = \frac{i\hbar}{2m} \left.\left(
      \Psi^* \frac{\partial \Psi}{\partial x} -
      \frac{\partial \Psi^*}{\partial x} \Psi
    \right)\right|_{-\infty}^{+\infty}.
  \] But $\Psi(x, t)$ must go to zero as $x$ goes to $\pm\infty$ --- otherwise
  the wave function would not be normalizable.\footnote{While mathematical
  counterexamples to this exist, they do not arise in physics.} It follows that
  \begin{equation} \label{eq:sch-norm-5}
    \frac{\partial}{\partial t} \int_{-\infty}^{+\infty} |\Psi(x, t)|^2 \,dx
    = 0,
  \end{equation}
  and hence that the integral is \emph{constant} (independent of time); if
  $\Psi$ is normalized at $t = 0$, it \emph{stays} normalized for all future
  time.
\end{proof}

\paragraph{Problem 1.5}
\begin{enumerate}[(a)]
  \item Since \[
      |\Psi(x, t)|^2 = |A|^2 e^{-2 \lambda |x|},
    \] the integral evaluates to
    \begin{align*}
      \int_{-\infty}^{+\infty} |\Psi(x, t)|^2 \,dx
      &= |A|^2 \left[
        \int_{-\infty}^0 e^{-2 \lambda x} \,dx +
        \int_0^{+\infty} e^{-2 \lambda x} \,dx
      \right] \\
      &= 2|A|^2 \int_0^{+\infty} e^{-2 \lambda x} \,dx \\
      &= 2|A|^2 \cdot \left.
        \frac{1}{-2\lambda} e^{-2 \lambda x}
      \right|_0^{+\infty} = \frac{|A|^2}{\lambda},
    \end{align*}
    so $A = \sqrt{\lambda}$.
  \item By symmetry, $\langle x \rangle = 0$, and
    \begin{equation*}
      \langle x^2 \rangle
      = \lambda \int_{-\infty}^{+\infty} x^2 e^{-2 \lambda x} \,dx
      = \frac{1}{2\lambda^2}.
    \end{equation*}
  \item The standard deviation $\sigma = \sqrt{\langle x^2 \rangle -
    \langle x \rangle^2} = \frac{1}{\sqrt{2} \lambda}$.
\end{enumerate}

\subsection{Momentum}

From equation \eqref{eq:sch-norm-4}, the expectation value of \emph{velocity}
can be calculated through \[
  \frac{d\langle x \rangle}{dt}
  = \int_{-\infty}^{+\infty} x \frac{\partial}{\partial t} |\Psi|^2 \,dx
  = \frac{i\hbar}{2m} \int_{-\infty}^{+\infty} x\frac{\partial}{\partial x}
  \left(
    \Psi^*\frac{\partial \Psi}{\partial x} -
    \frac{\partial \Psi^*}{\partial x}\Psi
  \right) \,dx.
\]
This can be simplified using integration by parts, \[
  \frac{d\langle x \rangle}{dt}
  = -\frac{i\hbar}{m} \int_{-\infty}^{+\infty} \Psi^*
  \frac{\partial \Psi}{\partial x} \,dx.
\]
Working with \emph{momentum} instead, rather than velocity, \[
  \langle p \rangle
  = m \frac{d \langle x \rangle}{dt}
  = -i\hbar \int_{-\infty}^{+\infty}
    \Psi^*\frac{\partial \Psi}{\partial x}
  \,dx.
\]
Finally, writing the equations for $\langle x \rangle$ and $\langle p \rangle$
in a more suggestive way:
\begin{gather}
  \label{eq:sch-vel}
  \langle x \rangle = \int_{-\infty}^{+\infty} \Psi^* (x) \Psi \,dx, \\
  \label{eq:sch-mom}
  \langle p \rangle
  = \int_{-\infty}^{+\infty} \Psi^*\left(
    \frac{\hbar}{i} \frac{\partial}{\partial x}
  \right)\Psi \,dx.
\end{gather}
To calculate the expectation value of any quantity $Q(x, p)$, we integrate
\begin{equation} \label{eq:sch-qty}
  \boxed{
    \langle Q(x, p) \rangle = \int_{-\infty}^{+\infty} \Psi^* Q\left(
      x, \frac{\hbar}{i}\frac{\partial}{\partial x}
    \right)\Psi \,dx.
  }
\end{equation}

\paragraph{Problem 1.7}
\begin{align*}
  \frac{d\langle p \rangle}{dt}
  &= \frac{d}{dt} \int_{-\infty}^{+\infty} \Psi^*\left(
    \frac{\hbar}{i}\frac{\partial}{\partial x}
  \right)\Psi \,dx \\
  &= \frac{\hbar}{i} \int_{-\infty}^{+\infty} \frac{\partial}{\partial t}
  \left(
    \Psi^*\frac{\partial \Psi}{\partial x}
  \right) \,dx \\
  &= \frac{\hbar}{i} \int_{-\infty}^{+\infty} \left(
    \Psi^*_t\Psi_x + \Psi^*\Psi_{xt}
  \right) \,dx.
\end{align*}
Using equations \eqref{eq:sch-norm-2} and \eqref{eq:sch-norm-3} to work out the
time derivatives of $\Psi$ and $\Psi^*$,
\begin{align*}
  &\frac{d\langle p \rangle}{dt} \\
  &= \frac{\hbar}{i} \int_{-\infty}^{+\infty} \left[
    \left(-\frac{i\hbar}{2m}\Psi^*_{xx} + \frac{i}{\hbar}V\Psi^*\right)\Psi_x
    + \Psi^*\left(\frac{i\hbar}{2m}\Psi_{xx} - \frac{i}{\hbar}V\Psi\right)_x
  \right] \,dx \\
  &= \int_{-\infty}^{+\infty} \left[
    \left(-\frac{\hbar^2}{2m}\Psi^*_{xx} + V\Psi^*\right)\Psi_x
    + \Psi^* \left(\frac{\hbar^2}{2m}\Psi_{xx} - V\Psi\right)_x
  \right] \,dx \\
  &= \int_{-\infty}^{+\infty} \left[
    \left(-\frac{\hbar^2}{2m}\Psi^*_{xx}\Psi_x + V\Psi^*\Psi_x\right)
    + \left(
      \frac{\hbar^2}{2m}\Psi^*\Psi_{xxx} - V_x\Psi^*\Psi - V\Psi^*\Psi_x
    \right)
  \right] \,dx \\
  &= \int_{-\infty}^{+\infty} \left[
    \left(-\frac{\hbar^2}{2m}\Psi^*_{xx}\right)\Psi_x
    + \Psi^*\left( \frac{\hbar^2}{2m}\Psi_{xxx} - V_x\Psi\right)
  \right] \,dx.
\end{align*}
Performing integration by parts twice on the first term, we find that \[
  \int_{-\infty}^{+\infty} -\frac{h^2}{2m}\Psi^*_{xx}\Psi_x \,dx
  = \int_{-\infty}^{+\infty} -\frac{h^2}{2m}\Psi^*\Psi_{xxx} \,dx
\]
since the boundary terms can be ignored, as the wave function and all its
derivatives are zero at infinity. This simplifies $d\langle p \rangle/dt$ to \[
  \frac{d\langle p \rangle}{dt}
  = -\int_{-\infty}^{+\infty} \Psi^*V_x\Psi \,dx
  = \left\langle-\frac{\partial V}{\partial x}\right\rangle.
\]

\subsection{The Uncertainty Principle}

The wavelength of $\Psi$ is related to the \emph{momentum} of the particle by
the \textbf{de Broglie formula}:
\begin{equation} \label{eq:de-brog}
  p = \frac{h}{\lambda} = \frac{2\pi\hbar}{\lambda},
\end{equation}
which leads to
\begin{equation} \label{eq:unc-prin}
  \boxed{
    \sigma_x \sigma_p \geq \frac{\hbar}{2},
  }
\end{equation}
where $\sigma_x$ and $\sigma_p$ are the standard deviation in $x$ and $p$
respectively.

\paragraph{Problem 1.9}
\begin{enumerate}[(a)]
  \item Since \[
      \Psi(x, t) = Ae^{-a(mx^2/\hbar)}e^{-iat},
    \] the integral evaluates to
    \begin{align*}
      \int_{-\infty}^{+\infty} |\Psi(x, t)|^2 \,dx
      &= |A|^2\int_{-\infty}^{+\infty} e^{-2a(mx^2/\hbar)} \,dx \\
      &= |A|^2\sqrt{\frac{\pi\hbar}{2ma}},
    \end{align*}
    so \[
      A = \left(\frac{2ma}{\pi\hbar}\right)^{1/4}.
    \]
  \item Working with these equations,
    \begin{align*}
      \Psi
      &= Ae^{-a[(mx^2/\hbar) + it]}, \\
      \frac{\partial \Psi}{\partial t}
      &= -iaA e^{-a[(mx^2/\hbar) + it]} = -ia\Psi, \\
      \frac{\partial \Psi}{\partial x}
      &= -\frac{2ma}{\hbar} Ax e^{-a[(mx^2/\hbar) + it]}
      = -\frac{2ma}{\hbar} x \Psi, \\
      \frac{\partial^2 \Psi}{\partial x^2}
      &= -\frac{2ma}{\hbar}\left(
        \Psi + x \frac{\partial \Psi}{\partial x}
      \right)
      = -\frac{2ma}{\hbar}\Psi\left(1 - \frac{2ma}{\hbar} x^2\right),
    \end{align*}
    this leads us to \[
      i\hbar(-ia\Psi) = \frac{-\hbar^2}{2m} \cdot \left[
        -\frac{2ma}{\hbar}\Psi\left(1 - \frac{2ma}{\hbar} x^2\right)
      \right] + V\Psi
    \] and finally resolves to \[
      V = 2ma^2x^2.
    \]
  \item
    \begin{gather*}
      \langle x \rangle = \int_{-\infty}^{+\infty} |\Psi|^2x \,dx = 0. \\
      \langle x^2 \rangle = \int_{-\infty}^{+\infty} |\Psi|^2x^2 \,dx =
        \frac{\hbar}{4ma}. \\
      \langle p \rangle
      = -i\hbar \int_{-\infty}^{+\infty}
        \Psi^*\frac{\partial \Psi}{\partial x}
      \,dx = 0. \\
      \langle p^2 \rangle
      = -\hbar^2 \int_{-\infty}^{+\infty}
        \Psi^*\frac{\partial^2 \Psi}{\partial x^2}
      \,dx
      = ma\hbar.
    \end{gather*} The standard deviations are \[
      \sigma_x = \frac{1}{2}\sqrt{\frac{\hbar}{ma}}, \quad
      \sigma_p = \sqrt{ma\hbar},
    \] so \[
      \sigma_x \sigma_p = \frac{\hbar}{2}
    \] which is consistent with the uncertainty principle, and in fact takes on
    the minimum possible value.
\end{enumerate}

\end{document}

