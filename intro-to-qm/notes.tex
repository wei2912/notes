\documentclass{article}

\usepackage{amsmath,amssymb,amsthm}
\usepackage[shortlabels]{enumitem}

\newtheorem{theorem}{Theorem}
\newtheorem{lemma}{Lemma}

\title{Notes to ``Introduction to Quantum Mechanics'', 2nd edition (Griffiths)}
\author{Ng Wei En}

\begin{document}

\maketitle

\section{The Wave Function}

\subsection{Schr\"{o}dinger's Equation}

A particle's position is defined by its \emph{wave function} $\Psi(x, t)$,
which can be obtained by solving \textbf{Schr\"{o}dinger's equation}:
\begin{equation} \label{eq:sch-eqn}
  \boxed{
    i\hbar \frac{\partial \Psi}{\partial t}
    = -\frac{\hbar^2}{2m} \frac{\partial^2 \Psi}{\partial x^2} + V\Psi.
  }
\end{equation}

Born's \emph{statistical interpretation} of the wave function states the
probability of finding the particle between $a$ and $b$ at time $t$ is \[
  \int_a^b |\Psi(x, t)|^2 \,dx.
\] It follows that the integral of $|\Psi|^2$
must be 1, i.e. the particle must be somewhere:
\begin{equation} \label{eq:sch-prob}
  \boxed{
    \int_{-\infty}^{+\infty} |\Psi(x, t)|^2 \,dx = 1.
  }
\end{equation}

\subsubsection{Normalisation}

If $\Psi(x, t)$ is a solution, then $A \Psi(x, t)$ for any complex constant $A$
must also be a solution, so any appropriate multiplicative factor can be chosen
to normalise the wave function.

\begin{lemma}
  The Schr\"{o}dinger equation preserves the normalization of the wave function
  as time goes on and $\Psi$ evolves.
\end{lemma}

\begin{proof}
  To begin, \[
    \frac{d}{dt} \int_{-\infty}^{+\infty} |\Psi(x, t)|^2 \,dx
    = \int_{-\infty}^{+\infty} \frac{\partial}{\partial t} |\Psi(x, t)|^2 \,dx.
  \] By the product rule, \[
    \frac{\partial}{\partial t}|\Psi|^2
    = \frac{\partial}{\partial t} (\Psi^* \Psi)
    = \Psi^* \frac{\partial \Psi}{\partial t} +
    \frac{\partial \Psi^*}{\partial t} \Psi.
  \] With the Schr\"{o}dinger equation we derive $\partial\Psi/\partial t$ and
  $\partial\Psi^*/\partial t$ to obtain
  \begin{equation} \label{eq:sch-norm-1}
    \frac{\partial}{\partial t} |\Psi|^2
    = \frac{i\hbar}{2m} \left(
      \Psi^* \frac{\partial^2 \Psi}{\partial x^2} -
      \frac{\partial^2 \Psi^*}{\partial x^2} \Psi
    \right)
    = \frac{\partial}{\partial x} \left[
      \frac{i\hbar}{2m} \left(
        \Psi^* \frac{\partial \Psi}{\partial x} -
        \frac{\partial \Psi^*}{\partial x} \Psi
      \right)
    \right].
  \end{equation}
  The integral can now be evaluated explicitly: \[
    \frac{d}{dt} \int_{-\infty}^{+\infty} |\Psi(x, t)|^2 \,dx
    = \frac{i\hbar}{2m} \left.\left(
      \Psi^* \frac{\partial \Psi}{\partial x} -
      \frac{\partial \Psi^*}{\partial x} \Psi
    \right)\right|_{-\infty}^{+\infty}.
  \] But $\Psi(x, t)$ (and by consequence its derivatives) must go to zero as
  $x$ goes to $\pm\infty$ --- otherwise the wave function would not be
  normalizable. It follows that \[
    \frac{d}{dt} \int_{-\infty}^{+\infty} |\Psi(x, t)|^2 \,dx = 0,
  \] and hence that the integral is constant (independent of time); if $\Psi$
  is normalized at $t = 0$, it stays normalized for all future time.
\end{proof}

\subsection{Velocity and Momentum}

From equation \eqref{eq:sch-norm-1}, the time derivative of the expectation
value of \emph{velocity} can be calculated through integration by parts to
obtain \[
  \langle p \rangle
  = m\frac{d\langle x \rangle}{dt}
  = m\int_{-\infty}^{+\infty} x \frac{\partial}{\partial t} |\Psi|^2 \,dx
  = -i\hbar\int_{-\infty}^{+\infty} \Psi^*
  \frac{\partial \Psi}{\partial x} \,dx.
\] Writing the equations for $\langle x \rangle$ and $\langle p \rangle$ in a
more suggestive way:
\begin{gather}
  \label{eq:sch-vel}
  \langle x \rangle = \int_{-\infty}^{+\infty} \Psi^* (x) \Psi \,dx, \\
  \label{eq:sch-mom}
  \langle p \rangle
  = \int_{-\infty}^{+\infty} \Psi^*\left(
    \frac{\hbar}{i} \frac{\partial}{\partial x}
  \right)\Psi \,dx.
\end{gather}
To calculate the expectation value of any quantity $Q(x, p)$, we insert the
resulting operator between $\Psi^*$ and $\Psi$, and integrate:
\begin{equation} \label{eq:sch-qty}
  \boxed{
    \langle Q(x, p) \rangle = \int_{-\infty}^{+\infty} \Psi^* Q\left(
      x, \frac{\hbar}{i}\frac{\partial}{\partial x}
    \right)\Psi \,dx.
  }
\end{equation}

\subsection{The Uncertainty Principle}

The wavelength of $\Psi$ is related to the \emph{momentum} of the particle by
the \textbf{de Broglie formula}:
\begin{equation} \label{eq:de-brog}
  p = \frac{h}{\lambda} = \frac{2\pi\hbar}{\lambda},
\end{equation}
which leads to
\begin{equation} \label{eq:unc-prin}
  \boxed{
    \sigma_x \sigma_p \geq \frac{\hbar}{2},
  }
\end{equation}
where $\sigma_x$ and $\sigma_p$ are the standard deviation in $x$ and $p$
respectively.

\section{Time-independent Schr\"{o}dinger Equation}

\subsection{Stationary States}

Consider separable solutions to the wave function $\Psi(x, t)$,
\begin{equation} \label{eq:sch-time-ind-1}
  \Psi(x, t) = \psi(x)\phi(t).
\end{equation}
With some manipulation the Schr\"{o}dinger equation resolves to \[
  i\hbar \frac{1}{\phi} \frac{d\phi}{dt}
  = -\frac{h^2}{2m} \frac{1}{\psi} \frac{d^2\psi}{dx^2} + V.
\] Since the left side is a function of $t$ alone, while the right side is a
function of $x$ alone, both sides must be \emph{constant}. With $E$ as the
constant value of both sides, we get
\begin{equation} \label{eq:sch-time-ind-psi}
  \boxed{
    \phi = e^{-iEt/\hbar}
  }
\end{equation}
and
\begin{equation} \label{eq:sch-time-ind}
  \boxed{
    -\frac{\hbar^2}{2m} \frac{d^2\psi}{dx^2} + V\psi = E\psi.
  }
\end{equation}
which is called the \textbf{time-independent Schr\"{o}dinger equation}.

\begin{lemma}
  Separable solutions to the Schr\"{o}dinger equation are \emph{stationary
  states}.
\end{lemma}
\begin{proof}
  The probability density of the wave function \[
    |\Psi(x, t)|^2 = \Psi^*\Psi = \psi^* e^{+iEt/\hbar} \psi e^{-iEt/\hbar} =
    |\psi(x)|^2
  \] is independent of time. Similarly in the calculation of the expectation
  value of any dynamical variable, \eqref{eq:sch-time-ind} reduces to \[
    \langle Q(x, p) \rangle
    = \int_{-\infty}^{+\infty} \psi^*
      Q\left(x, \frac{\hbar}{i}\frac{\partial}{\partial x}\right)
    \psi dx,
  \] and so every expectation value is \emph{constant in time}.
\end{proof}

\begin{lemma}
  Separable solutions to the Schr\"{o}dinger equation are states of
  \emph{definite total energy}.
\end{lemma}
\begin{proof}
  In classical mechanics, the total energy (kinetic plus potential) is called
  the \textbf{Hamiltonian}: \[
    H(x, p) = \frac{p^2}{2m} + V(x).
  \] The corresponding Hamiltonian operator $\hat{H}$, obtained by the
  canonical substitution $p \rightarrow (\hbar/i)(\partial/\partial x)$, is
  therefore
  \begin{equation} \label{eq:ham-op}
    \hat{H} = -\frac{\hbar^2}{2m} \frac{\partial^2}{\partial x^2} + V(x).
  \end{equation}
  Thus, the time-independent Schr\"{o}dinger equation \eqref{eq:sch-time-ind}
  can be written
  \begin{equation} \label{eq:sch-time-ind-ham}
    \hat{H}\psi = E\psi
  \end{equation}
  and the expectation value of the total energy is \[
    \langle H \rangle
    = \int_{-\infty}^{+\infty} \psi^* \hat{H} \psi \,dx
    = E\int_{-\infty}^{+\infty} |\psi|^2 \,dx
    = E\int_{-\infty}^{+\infty} |\Psi|^2 \,dx
    = E.
    \] Moreover, $\langle H^2 \rangle = \int_{-\infty}^{+\infty} \psi^* 
    \hat{H}^2 \psi \,dx = E^2$, and hence $\sigma_H^2 = \langle H^2 \rangle -
    \langle H \rangle^2 = E^2 - E^2 = 0$. So, a separable solution has the
    property that every measurement of the total energy is certain to return
    the value $E$.
\end{proof}

\begin{lemma}
  The general solution to the Schr\"{o}dinger equation is a \emph{linear
  combination} of separable solutions.
\end{lemma}

There are a few other useful lemmas for solutions to the time-independent
Schr\"{o}dinger equation:

\begin{lemma}[cf. Problem 2.1(a)]
  For normalizable solutions, the separation constant $E$ must be real.
\end{lemma}

\begin{lemma}[cf. Problem 2.1(b)]
  The time-independent wave function $\psi(x)$ can always be taken to be real
  (unlike $\Psi(x, t)$, which is necessarily complex).
\end{lemma}

\begin{lemma}[cf. Problem 2.1(c)]
  If $V(x)$ is an even function then $\psi(x)$ can always be taken to be either
  even or odd.
\end{lemma}

\begin{lemma}[cf. Problem 2.2]
  $E$ must exceed the minimum value of $V(x)$, for every normalizable solution
  to the time-independent Schr\"{o}dinger equation.
\end{lemma}

\subsection{Infinite Square Well}

\begin{equation} \label{eq:inf-sq-well}
  V(x) =
  \begin{cases}
    0, &\text{ if } 0 \leq x \leq a, \\
    \infty, &\text{ otherwise}.
  \end{cases}
\end{equation}

Outside the well, $\psi(x) = 0$; inside the well, where $V = 0$, the
time-independent Schr\"{o}dinger equation can be rewritten into
\begin{equation} \label{eq:inf-sq-well-1}
  \frac{d^2\psi}{dx^2} = -k^2\psi,
  \text{ where } k \equiv \frac{\sqrt{2mE}}{\hbar}
\end{equation}
(with the assumption that $E \geq 0$, since by Problem 2.2 $E < 0$ would lead
to a non-normalizable solution). \eqref{eq:inf-sq-well-1} is the classical
\textbf{simple harmonic oscillator equation}; the general solution is \[
  \psi(x) = A \sin kx + B \cos kx,
\] where $A$ and $B$ are arbitrary constants, fixed by the \emph{boundary
conditions} for $\psi$:
\begin{enumerate}
  \item $\psi$ is always continuous;
  \item $d\psi/dx$ is continuous except at points where the potential is
    infinite.
\end{enumerate}
Ordinarily, we assume both $\psi$ and $d\psi/dx$ are continuous, so we set
$\psi(0) = \psi(a) = 0$ to join onto the solution outside the wall, leading to
the distinct solutions
\begin{equation} \label{eq:inf-sq-well-2}
  k_n = \frac{n\pi}{a}, \text{ with } n = 1, 2, 3, \ldots
\end{equation}
and the following stationary states:
\begin{equation} \label{eq:inf-sq-well-psi}
  \boxed{
    \psi_n(x) = \sqrt{\frac{2}{a}} \sin\left(\frac{n\pi}{a}x\right),
    \quad E_n = \frac{\hbar^2k_n^2}{2m} = \frac{n^2\pi^2\hbar^2}{2ma^2}.
  }
\end{equation}
$\psi_1$, which contains the lowest energy, is called the \textbf{ground
state}; the other states, whose energies increase in proportion to $n^2$, are
called \textbf{excited states}.

\begin{lemma}
  The states are alternately \emph{even} and \emph{odd} with respect to the
  center of the well: $\psi_1$ is even, $\psi_2$ is odd, $\psi_3$ is even, and
  so on.
\end{lemma}

\begin{lemma}
  As you go up in energy, each successive state has one more \textbf{node}
  (zero-crossing): $\psi_1$ has none (the end-points don't count), $\psi_2$ has
  one, $\psi_3$ has two, and so on.
\end{lemma}

\begin{lemma}
  The states are mutually \textbf{orthogonal}, in the sense that \[
    \int_{-\infty}^{+\infty} \psi_m(x)^* \psi_n(x) \,dx = \delta_{mn},
  \] where $\delta_{mn}$ is the \textbf{Kronecker delta} defined usually by \[
    \delta_{mn} =
    \begin{cases}
      0, &\text{ if } m \neq n; \\
      1, &\text{ if } m = n.
    \end{cases}
  \]
\end{lemma}

\begin{lemma}
  The states are \textbf{complete}, in the sense that any \emph{other} function
  $f(x)$ can be expressed as a linear combination of them: \[
    f(x)
    = \sum_{n = 1}^{\infty} c_n\psi_n(x)
    = \sqrt{\frac{2}{a}} \sum_{n = 1}^{\infty}
      c_n\sin\left(\frac{n\pi}{a}x\right).
  \]
\end{lemma}
This expression is the \textbf{Fourier series} of $f(x)$, and
\textbf{Dirichlet's theorem} can be used to prove completeness. The
coefficients $c_n$ can be evaluated with \textbf{Fourier's trick}, which
exploits the orthogonality of $\{\psi_n\}$. Multiplying by $\psi_m(x)^*$ and
integrating, \[
  \int_{-\infty}^{+\infty} \psi_m(x)^* f(x) \,dx
  = \sum_{n = 1}^{\infty} c_n \int_{-\infty}^{\infty}
    \psi_m(x)^* \psi_n(x)
  \,dx
  = \sum_{n = 1}^{\infty} c_n\delta_{mn}
  = c_m.
\] Thus the $n$th coefficient in the expansion of $f(x)$ is
\begin{equation} \label{eq:sim-har-osc-Psi-coeff}
  \boxed{
    c_n = \int_{-\infty}^{+\infty} \psi_n(x)^* f(x) \,dx.
  }
\end{equation}

Hence, we can derive the general solution to the (time-dependent)
Schr\"{o}dinger equation as a linear combination of stationary states:
\begin{equation} \label{eq:sim-har-osc-Psi}
  \boxed{
    \begin{gathered}
      \Psi(x, t) = \sum_{n = 1}^{\infty}
        c_n\sqrt{\frac{2}{a}} \sin\left(\frac{n\pi}{a}x\right)
        e^{-i(n^2\pi^2\hbar/2ma^2)t}, \\
      c_n = \sqrt{\frac{2}{a}} \int_0^a
        \sin\left(\frac{n\pi}{a}x\right) \Psi(x, 0)
      \,dx.
    \end{gathered}
  }
\end{equation}

\subsection{Harmonic Oscillator}

Practically any potential is \emph{approximately} parabolic with the use of the
\textbf{Taylor series}, in the neighbourhood of a local minimum. We consider
the following potential in the rest of this section,
\begin{equation} \label{eq:har-osc}
  V(x) = \frac{1}{2} m \omega^2 x^2,
\end{equation}
which can be solved with the time-independent Schr\"{o}dinger equation.

\subsubsection{Algebraic Method}

First, we rewrite the Schr\"{o}dinger equation as \[
  \frac{1}{2m}[p^2 + (m \omega x)^2]\psi = E\psi
\] where $p \equiv (\hbar/i)d/dx$ is the momentum operator. Let
\begin{equation} \label{eq:har-osc-alg-a}
  \boxed{
    a_{\pm} \equiv \frac{1}{\sqrt{2 \hbar m \omega}}(\mp ip + m \omega x)
  }
\end{equation}
Consider the product
\begin{align*}
  a_{\pm}a_{\mp}
  &= \frac{1}{2 \hbar m \omega}(\mp ip + m \omega x)(\pm ip + m \omega x) \\
  &= \frac{1}{2 \hbar m \omega}[p^2 + (m \omega x)^2 \pm im\omega(xp - px)].
\end{align*}
The \textbf{commutator} of operators $A$ and $B$ is \[
  [A, B] \equiv AB - BA.
\] In this notation, \[
  a_{\pm}a_{\mp}
  = \frac{1}{2 \hbar m \omega}[p^2 + (m \omega x)^2]
  \pm \frac{i}{2\hbar}[x, p].
\] Rewriting this equation with the \textbf{canonical commutation relation},
$[x, p] = i\hbar$ (which can be shown easily by evaluating $[x, p]f(x)$ for
some function $f$), and \eqref{eq:ham-op} we obtain the following equation:
\begin{equation} \label{eq:har-osc-alg}
  \hbar\omega\left(a_{\pm}a_{\mp} \pm \frac{1}{2}\right)\psi = E\psi.
\end{equation}
We can now prove that $H(a_{\pm}\psi) = (E \pm \hbar\omega)(a_{\pm}\psi)$.
Hence, $a_{\pm}$ are called \textbf{ladder operators}, with $a_+$ the
\textbf{raising operator} and $a_-$ the \textbf{lowering operator}.

The ground state $\psi_0$ is such that applying the lowering operator to it
causes the solution to be zero and non-normalizable, i.e. $a_-\psi_0 = 0$.
Applying the operator and solving for the differential equation, upon
normalization we obtain the solution \[
  \psi_0(x) = \left(\frac{m\omega}{\pi\hbar}\right)^{1/4}
  e^{-\frac{m\omega}{2\hbar}x^2},\quad
  \text{with } E_0 = \frac{1}{2}\hbar\omega.
\] Also, by applying the raising operator repeatedly we obtain \[
  \psi_n(x) = A_n(a_+)^n\psi_0(x), \quad
  \text{with } E_n = \left(n + \frac{1}{2}\right)\hbar\omega.
\] We know that $a_{\pm}\psi_n$ is proportional to $\psi_{n \pm 1}$, so let \[
  a_+\psi_n = c_n\psi_{n + 1}, \quad a_-\psi_n = d_n\psi_{n - 1}.
\] First note that one can prove \[
  \int_{-\infty}^{\infty} f^*(a_{\pm}g) \,dx
  = \int_{-\infty}^{\infty} (a_{\mp}f)^*g \,dx
\] for any functions $f$ and $g$. In particular, \[
  \int_{-\infty}^{\infty} (a_{\pm}\psi_n)^*(a_{\pm}\psi_n) \,dx
  = \int_{-\infty}^{\infty} (a_{\mp}a_{\pm}\psi_n)^*\psi_n \,dx.
\] But $a_+a_-\psi_n = n\psi_n$ and $a_-a_+\psi_n = (n + 1)\psi_n$, so $|c_n|^2
= n + 1$ and $|d_n|^2 = n$. Hence,
\begin{equation} \label{eq:har-osc-alg-op}
  \boxed{
    a_+\psi_n = \sqrt{n + 1}\psi_{n + 1}, \quad
    a_-\psi_n = \sqrt{n}\psi_{n - 1}
  }
\end{equation}
and so
\begin{equation} \label{eq:har-osc-alg-sta-st}
  \boxed{
    \psi_n = \frac{1}{\sqrt{n!}}(a_+)^n\psi_0, \quad
    \text{with } E_n = \left(n + \frac{1}{2}\right)\hbar\omega
  }
\end{equation}
As in the case of the stationary square well, the stationary states of the
harmonic oscillator are \emph{orthogonal}.

\subsubsection{Analytic Method}

Introducing $\xi \equiv \sqrt{m\omega/\hbar}x$, we can rewrite the
Schr\"{o}dinger equation as
\begin{equation} \label{eq:har-osc-ana-1}
  \frac{d^2\psi}{d\xi^2} = (\xi^2 - K)\psi,\quad
  \text{with } K \equiv \frac{2E}{\hbar\omega}.
\end{equation}
At very large $\xi$, $\xi^2$ dominates over the constant $K$, so in this regime 
$\frac{d^2\psi}{d\xi^2} \approx \xi^2\psi$, which has the approximate solution
$\psi(\xi) \approx Ae^{-\xi^2/2} + Be^{\xi^2/2}$. The $B$ term is clearly not
normalizable; the physically acceptable solutions, then, have the asymptotic
form
\begin{equation} \label{eq:har-osc-ana-psi}
  \psi(\xi) = h(\xi)e^{-\xi^2/2}.
\end{equation}
Differentiating \eqref{eq:har-osc-ana-psi}, equation \eqref{eq:har-osc-ana-1}
becomes
\begin{equation} \label{eq:har-osc-ana-2}
  \frac{d^2h}{d\xi^2} - 2\xi\frac{dh}{d\xi} + (K - 1)h = 0.
\end{equation}
We propose to look for solutions to $h$ in the form of power series in $\xi$:
\begin{align*}
  h(\xi) &= \sum_{j=0}^{\infty} a_j\xi^j, \\
  \frac{dh}{d\xi} &= \sum_{j=0}^{\infty} ja_j\xi^{j-1}, \\
  \frac{d^2h}{d\xi^2} &= \sum_{j=0}^{\infty} (j+1)(j+2)a_{j+2}\xi^j.
\end{align*}
Putting these into equation \eqref{eq:har-osc-ana-1}, the summations reduce to
\begin{equation} \label{eq:har-osc-ana-a}
  a_{j+2} = \frac{2j + 1 - K}{(j+1)(j+2)}a_j.
\end{equation}
Starting with $a_0$, this formula generates all the even-numbered coefficients:
\[
  a_2 = \frac{1 - K}{2}a_0, \quad
  a_4 = \frac{5 - K}{12}a_2 = \frac{(5 - K)(1 - K)}{24}a_0, \quad
  \ldots,
\] and starting with $a_1$, it generates the odd coefficients: \[
  a_3 = \frac{3 - K}{6}a_1, \quad
  a_5 = \frac{7 - K}{20}a_3 = \frac{(7 - K)(3 - K)}{120}a_1, \quad
  \ldots.
\] We view the complete solution as
\begin{equation} \label{eq:har-osc-ana-h-2}
  h(\xi) = h_{\mathrm{even}}(\xi) + h_{\mathrm{odd}}(\xi),
\end{equation} where \[
  h_{\mathrm{even}}(\xi) \equiv a_0 + a_2\xi^2 + a_4\xi^4 + \ldots
\] is an even function of $\xi$, built on $a_0$, and \[
h_{\mathrm{odd}}(\xi) \equiv a_1\xi + a_3\xi^3 + a_5\xi^5 + \ldots
\] is an odd function, built on $a_1$. Thus \eqref{eq:har-osc-ana-a} determines
$h(\xi)$ in terms of two arbitrary constants ($a_0$ and $a_1$). However, not
all the solutions are normalizable. At very large $j$, the recursion formula
becomes approximately $a_{j+2} \approx \frac{2}{j}a_j$ with the approximate
solution $a_j \approx \frac{C}{(j/2)!}$ for some constant $C$, and this yields
(at large $\xi$) \[
  h(\xi)
  \approx C\sum_{j=0}^{\infty} \frac{1}{(j/2)!}\xi^j
  \approx C\sum_{j=0}^{\infty} \frac{1}{j!}\xi^{2j}
  \approx Ce^{\xi^2},
\] leading to $\psi(\xi) \approx Ce^{\xi^2/2}$ which cannot be normalized. For
normalizable solutions, it is therefore crucial that the power series
terminate, with some highest $j$ (called $n$) such that the formula yields
$a_{n+2} = 0$ (truncating either $h_{\mathrm{odd}}$ or $h_{\mathrm{even}}$; the
other one must be 0 from the start). Hence, \eqref{eq:har-osc-ana-a} requires
that $K = 2n + 1$ for some non-negative integer $n$, which is to say that the
energy must be
\begin{equation} \label{eq:har-osc-ana-ene}
  E_n = \left(n + \frac{1}{2}\right)\hbar\omega, \quad
  \text{for } n = 0, 1, 2, \ldots.
\end{equation}
We have returned to the fundamental quantization condition we found
algebraically previously. For the allowed values of $K$, the recursion formula
reads
\begin{equation} \label{eq:har-osc-ana-a-2}
  a_{j+2} = \frac{-2(n - j)}{(j+1)(j+2)}a_j.
\end{equation}
Define $h_n(\xi)$ as the polynomial corresponding to a specific value of $n$;
apart from the overall factor ($a_0$ or $a_1$) they are called \textbf{Hermite
polynomials}, $H_n(\xi)$. By tradition, the arbitrary multiplication factor is
chosen so that the coefficient of the highest power of $\xi$ is $2^n$. With
this convention, the normalized stationary states for the harmonic oscillator
are
\begin{equation} \label{eq:har-osc-ana-sta-st}
  \boxed{
    \psi_n(x) = \left(\frac{m\omega}{\pi\hbar}\right)^{1/4}
    \frac{1}{\sqrt{2^nn!}}H_n(\xi)e^{-\xi^2/2},
  }
\end{equation}
identical to the ones we obtained in \eqref{eq:har-osc-alg-sta-st}.

\subsection{The Free Particle}

The free particle has $V(x) = 0$ everywhere, which leads to the
time-independent Schr\"{o}dinger equation rewritten as
\begin{equation} \label{eq:free-par}
  \frac{d^2\psi}{dx^2} = -k^2\psi,
  \text{ where } k \equiv \frac{\sqrt{2mE}}{\hbar}.
\end{equation}

This is similar to the infinite square well in \eqref{eq:inf-sq-well-1};
however, we will write the general solution as $\psi(x) = Ae^{ikx} +
Be^{-ikx}$. Unlike the infinite square well, there are no boundary conditions
to restrict the possible values of $k$ (and hence of $E$); the free particle
can carry any (positive) energy. Tacking on the standard time dependence,
$\exp(-iEt/\hbar)$,
\begin{equation} \label{eq:free-par-1}
  \Psi(x, t)
  = Ae^{ik(x - \frac{\hbar k}{2m}t)} + Be^{-ik(x - \frac{\hbar k}{2m}t)}
\end{equation}

Any function of $(x \pm vt)$ (for some constant $v$) represents a wave of fixed
profile, travelling in the $\mp x$-direction, at speed $v$. To simplify
equation \eqref{eq:free-par-1}, we write
\begin{equation} \label{eq:free-par-2}
  \Psi_k(x, t) = Ae^{i(kx - \frac{\hbar k^2}{2m}t)}
\end{equation}
and let $k$ run negative to cover the case of waves travelling to the left:
\begin{equation} \label{eq:free-par-k}
  k \equiv \pm\frac{\sqrt{2mE}}{\hbar},
  \text{ with}
  \begin{cases}
    k > 0 \implies \text{travelling to the right}, \\
    k < 0 \implies \text{travelling to the left}.
  \end{cases}
\end{equation}
Evidently the "stationary states" of the free particle are propagating waves;
their wavelength is $\lambda = 2\pi/|k|$, and, according to the de Broghlie
formula \eqref{eq:de-brog}, they carry momentum $p = \hbar k$.

While the wave function \emph{is not normalizable}, and the separable solutions
do not represent physically realizable states, they play a mathematical role
that is entirely independent of their physical interpretation. The general
solution to the time-independent Schr\"{o}dinger equation is still a linear
combination of separable solutions (but in the form of an integral over the
continuous variable $k$, instead of a sum over the discrete index $n$):
\begin{equation} \label{eq:free-par-psi}
  \boxed{
    \Psi(x, t)
    = \frac{1}{\sqrt{2\pi}}\int_{-\infty}^{+\infty}
      \phi(k)e^{i(kx - \frac{\hbar k^2}{2m}t)}
    \,dk.
  }
\end{equation}
This wave function can be normalized (for appropriate $\phi(k)$), but it
necessarily carries a range of $k$'s, and hence a range of energies and speeds.
We call it a \textbf{wave packet}.\footnote{Sinusoidal waves extend out to
infinity, and they are not normalizable. But superpositions of such waves lead
to interference, which allows for localization and normalizability.}

In the generic quantum problem, we are given $\Psi(x, 0)$, and we are asked to
find $\Psi(x, t)$. For a free particle the solution takes the form of
\eqref{eq:free-par-psi}; the only question is how to determine $\phi(k)$ so as
to match the initial wave function \[
  \Psi(x, 0)
  = \frac{1}{\sqrt{2\pi}}\int_{-\infty}^{+\infty} \phi(k)e^{ikx} \,dk.
\] The answer is provided by \textbf{Plancherel's theorem}:
\begin{equation} \label{eq:plan-thm}
  f(x) = \frac{1}{\sqrt{2\pi}}\int_{-\infty}^{+\infty} F(k)e^{ikx} \,dk.
  \iff
  F(k) = \frac{1}{\sqrt{2\pi}}\int_{-\infty}^{+\infty} f(x)e^{-ikx} \,dx.
\end{equation}
$F(k)$ is called the \textbf{Fourier transform} of $f(x)$; $f(x)$ is the
\textbf{inverse Fourier transform} of $F(k)$.\footnote{The necessary and
sufficient condition on $f(x)$ is that $\int_{-\infty}^{\infty} |f(x)|^2 \,dx$
be \emph{finite}. (In that case $\int_{-\infty}^{\infty} |F(k)|^2 \,dk$ is also
finite, and in fact the two integrals are equal.) For our purposes, the
integrals are guaranteed to exist by the physical requirement that $\Psi(x, 0)$
itself is normalized.} So the solution to the generic quantum problem, for the free
particle, is \eqref{eq:free-par-psi} with
\begin{equation} \label{eq:free-par-phi}
  \boxed{
    \phi(k)
    = \frac{1}{\sqrt{2\pi}}\int_{-\infty}^{+\infty}
      \Psi(x, 0)e^{-ikx}
    \,dx.
  }
\end{equation}

\subsubsection{Wave Packet}

A wave packet is a superposition of sinusoidal functions whose amplitude is
modulated by $\phi$. What corresponds to the particle velocity is not the
speed of the individual ripples (the \textbf{phase velocity}), but rather the
speed of the envelope (the \textbf{group velocity}).\footnote{Depending on the
nature of the waves, the group velocity can be greater than, less than, or
equal to the phase velocity.}

Take a wave packet with the general form \[
  \Psi(x, t) = \frac{1}{\sqrt{2\pi}}\int_{-\infty}^{+\infty}
    \phi(k)e^{i(kx - \omega t)}
  \,dk
\] with the \textbf{dispersion relation} determining the formula for $\omega$
as a function of $k$. Let us assume that $\phi(k)$ is normally peaked about
some particular value $k_0$.\footnote{Wave packets with a broad spread in $k$
change speed rapidly as different components travel at different speeds, so the
whole notion of a "group", with a well-defined velocity, loses its meaning.}
Taylor-expanding the function $\omega(k)$ about $k_0$, and keeping only the
leading terms, we have $\omega(k) \approx \omega_0 + \omega'_0(k - k_0)$ where
$\omega'_0$ is the derivative of $\omega$ with respect to $k$, at the point
$k_0$. Changing variables from $k$ to $s \equiv k - k_0$ (to center the
integral at $k_0$), we have \[
  \Psi(x, t) \approx \frac{1}{\sqrt{2\pi}}\int_{-\infty}^{+\infty}
    \phi(k_0 + s)e^{i[(k_0 + s)x - (\omega_0 + \omega'_0s)t]}
  \,ds.
\] At $t = 0$, we have \[
  \Psi(x, 0) = \frac{1}{\sqrt{2\pi}}\int_{-\infty}^{+\infty}
    \phi(k_0 + s)e^{i(k_0 + s)x}
  \,ds,
\] and at later times, we rewrite the previous equation for $\Psi(x, t)$ to get
\[
  \Psi(x, t)
  \approx \frac{1}{\sqrt{2\pi}}e^{i(-\omega_0 + k_0\omega'_0)t}
  \int_{-\infty}^{+\infty}
    \phi(k_0 + s)e^{i(k_0 + s)(x - \omega'_0t)}
  \,ds.
\] Except for the shift from $x$ to $(x - \omega'_0t)$, the integral is the
same as the one in $\Psi(x, 0)$. Thus
\begin{equation} \label{eq:wave-pac-psi}
  \boxed{
    \Psi(x, t) \approx e^{-i(\omega_0 - k_0\omega'_0)t}\Psi(x - \omega'_0, 0).
  }
\end{equation}
Apart from the phase factor in front (which won't affect $|\Psi|^2$ in any
event) the wave packet evidently moves along at a speed $\omega'_0$:
\begin{equation} \label{eq:wave-pac-vel-grp}
  v_{\text{group}} = \frac{d\omega}{dk}
  \text{ (evaluated at $k = k_0$)}.
\end{equation}
This is to be contrasted with the ordinary phase velocity
\begin{equation} \label{eq:wave-pac-vel-pha}
  v_{\text{phase}} = \frac{\omega}{k}
\end{equation}

In our case, $\omega = \hbar k^2/2m$, so $\omega/k = \hbar k/2m$, whereas
$d\omega/dk = \hbar k/m$ which is twice as great. This confirms that it is the
group velocity of the wave packet, not the phase velocity of the stationary
states, that matches the classical particle velocity:
\begin{equation} \label{eq:wave-pac-vel}
  v_{\text{classical}} = v_{\text{group}} = 2v_{\text{phase}}.
\end{equation}

\subsection{The Delta-Function Potential}

\subsubsection{Bound States and Scattering States}

In \emph{classical} mechanics, if $V(x)$ rises higher than a particle's total
energy ($E$) on either side, it cannot escape the potential well unless it is
provided with a source of extra energy and is in a \emph{bound state}. If, on
the other hand, $E$ exceeds $V(x)$ on one side (or both), then the particle
cannot be trapped in the potential unless there is some mechanism to dissipate
energy and is a \emph{scattering state}.

In the quantum domain, \emph{tunneling} allows the particle to "leak" through
any finite potential barrier, so the only thing that matters is the potential
at infinity:
\begin{equation} \label{eq:qtm-stat}
  \begin{cases}
    E < [V(-\infty) \text{ and } V(+\infty)] &\implies \text{bound state}, \\
    E > [V(-\infty) \text{ or } V(+\infty)] &\implies \text{scattering state}.
  \end{cases}
\end{equation}

\subsubsection{The Delta-Function Well}

The \textbf{Dirac delta function} is an infinitely high, infinitesimally
narrow spike at the origin, \emph{whose area is 1}:
\begin{equation}
  \delta(x) = \left.
  \begin{cases}
    0, &\text{ if } x \neq 0 \\
    \infty, &\text{ if } x = 0
  \end{cases}
\right\}, \text{ with } \int_{-\infty}^{+\infty} \delta(x) \,dx = 1.
\end{equation}

Multiplying $\delta(x - a)$ by an ordinary function $f(x)$ is the same as
multiplying by $f(a)$, \[
  f(x)\delta(x - a) = f(a)\delta(x - a),
\] because the product is \emph{zero} anyway except at the point $a$. In
particular, \[
  \int_{-\infty}^{+\infty} f(x)\delta(x - a) \,dx
  = f(a) \int_{-\infty}^{+\infty} \delta(x - a) \,dx
  = f(a).
\]

Let us consider a potential of the form
\begin{equation} \label{eq:dir-del}
  V(x) = -\alpha\delta(x),
\end{equation}
where $\alpha$ is some positive constant. The Schr\"{o}dinger equation for the
delta-function well reads
\begin{equation} \label{eq:dir-del-1}
  -\frac{\hbar^2}{2m}\frac{d^2\psi}{dx^2} - \alpha\delta(x)\psi = E\psi;
\end{equation}
and yields both bound states ($E < 0$) and scattering states ($E > 0$).

We first examine the bound states for $E < 0$. In the region $x < 0$, $V(x) =
0$, so
\begin{equation} \label{eq:dir-del-bou-1}
  \frac{d^2\psi}{dx^2} = -\frac{2mE}{\hbar^2}\psi = \kappa^2\psi,
\end{equation}
where $\kappa \equiv \sqrt{-2mE}/\hbar$ is real and positive. The general
solution to \eqref{eq:dir-del-1} is $\psi(x) = Ae^{-\kappa x} + Be^{\kappa x}$,
but the first term blows up as $x \to -\infty$, so we must choose $A = 0$.
Similarly, the general solution in the region $x > 0$ is of the form
$F\exp(-\kappa x) + G\exp(\kappa x)$, but the second term blows up as $x \to
\infty$. Stitching the two functions together using the first boundary
condition at $x = 0$ ($\psi$ is always continuous), $B = F$, so
\begin{equation} \label{eq:dir-del-bou-2}
  \psi(x) =
  \begin{cases}
    Be^{\kappa x}, &\text{ if } x \leq 0, \\
    Be^{-\kappa x}, &\text{ if } x \geq 0.
  \end{cases}
\end{equation}
The delta function must determine the continuity in $d\psi/dx$ at $x = 0$ for
the second boundary condition to be satisfied ($d\psi/dx$ is always continuous
except at points where the potential is infinite). The idea is to integrate the
Schr\"{o}dinger equation, from $-\epsilon$ to $\epsilon$, and then take the
limit as $\epsilon \to 0$. In the equation
\begin{equation} \label{eq:dir-del-bou-3}
  -\frac{\hbar^2}{2m}\int_{-\epsilon}^{+\epsilon} \frac{d^2\psi}{dx^2} \,dx
  + \int_{-\epsilon}^{+\epsilon} V(x)\psi(x) \,dx
  = E\int_{-\epsilon}^{+\epsilon} \psi(x) \,dx,
\end{equation}
the first integral is $d\psi/dx$ evaluated at the two end points and the last
integral is \emph{zero} in the limit $\epsilon \to 0$. Thus
\begin{equation} \label{eq:dir-del-bou-4}
  \Delta\left(\frac{d\psi}{dx}\right)
  \equiv \lim_{\epsilon \to 0}\left(
    \left.\frac{d\psi}{dx}\right|_{+\epsilon}
    - \left.\frac{d\psi}{dx}\right|_{-\epsilon}
  \right)
  = \frac{2m}{\hbar^2}\lim_{\epsilon \to 0}
    \int_{-\epsilon}^{+\epsilon} V(x)\psi(x) \,dx.
\end{equation}
For $V(x) = -\alpha\delta(x)$, this simplifies to
\begin{equation} \label{eq:dir-del-bou-5}
  \Delta\left(\frac{d\psi}{dx}\right)
  = -\frac{2ma}{\hbar^2}\psi(0).
\end{equation}
With some algebra we prove that $\kappa = m\alpha/\hbar^2$ and the allowed
energy is $E = -\hbar\kappa^2/2m = -m\alpha^2/2h^2$. Normalizing $\psi$, we
have $B = \sqrt{\kappa} = \sqrt{ma}/\hbar$ (choosing the positive square root
for convenience). Evidently the delta-function well, regardless of its
"strength" $\alpha$, has \emph{exactly one bound state}:
\begin{equation} \label{eq:dir-del-bou}
  \boxed{
    \psi(x) = \frac{\sqrt{ma}}{\hbar}e^{-m\alpha|x|/\hbar^2};
    \quad E = -\frac{m\alpha^2}{2\hbar^2}.
  }
\end{equation}

For scattering states with $E > 0$, the Schr\"{o}dinger equation reads
\begin{equation} \label{eq:dir-del-sca-1}
  \frac{d^2\psi}{dx^2} = -\frac{2mE}{\hbar^2}\psi = -k^2\psi,
\end{equation}
where $k \equiv \sqrt{2mE}{\hbar}$ is real and positive. The general solution
is
\begin{equation} \label{eq:dir-del-sca-2}
  \psi(x) =
  \begin{cases}
    Ae^{ikx} + Be^{-ikx}, &\text{if } x < 0, \\
    Fe^{ikx} + Ge^{-ikx}, &\text{if } x > 0,
  \end{cases}
\end{equation}
but this time we cannot rule out any of the terms since none of them blow up.
The continuity of $\psi(x)$ at $x = 0$ requires that
\begin{equation} \label{eq:dir-del-sca-3a}
  F + G = A + B.
\end{equation}
From the derivatives, $\Delta(d\psi/dx) = ik(F - G - A + B)$. Meanwhile,
$\psi(0) = A + B$, so the second boundary condition says \[
  ik(F - G - A + B) = -\frac{2m\alpha}{\hbar^2}(A + B)
\] or more compactly
\begin{equation} \label{eq:dir-del-sca-3b}
  F - G = A(1 + 2i\beta) - B(1 - 2i\beta),
  \quad \text{ where } \beta \equiv \frac{m\alpha}{\hbar^2k}.
\end{equation}

With equations \eqref{eq:dir-del-sca-3a} and \eqref{eq:dir-del-sca-3b}, we now
consider a typical scattering experiment where particles are fired in from the
left. In that case the amplitude of the wave coming in from the \emph{right}
will be \emph{zero} ($G = 0$); $A$ is the amplitude of the \emph{incident
wave}, $B$ is the amplitude of the \emph{reflected wave}, and $F$ is the
amplitude of the \emph{transmitted wave}. Solving the two equations for $B$ and
$F$, we find
\begin{equation} \label{eq:dir-del-sca-4}
  B = \frac{i\beta}{1 - i\beta}A, \quad
  F = \frac{1}{1 - i\beta}A.
\end{equation}
Calculating the \emph{reflection coefficient} $R \equiv |B|^2/|A|^2$ and
\emph{transmission coefficient} $T \equiv |F|^2/|A|^2$ (which sum up to 1),
we express them in terms of $E$:
\begin{equation} \label{eq:dir-del-sca-coeff}
  \boxed{
    R = \frac{1}{1 + (2\hbar^2E/m\alpha^2)}, \quad
    T = \frac{1}{1 + (m\alpha^2/2\hbar^2E)}.
  }
\end{equation}
This suggests that the higher the energy, the greater the probability of
transmission. However, these scattering wave functions are not normalizable, so
we must form normalizable linear combinations of the stationary states, just as
we did for the free particle. Though straightforward in principle, this turns
out to be messy and best done with a computer. Meanwhile, since it is
impossible to create a normalizable free-particle wave function without
involving a range of energies, $R$ and $T$ should be interpreted as approximate
reflection and transmission probabilities for particles in the vicinity of $E$.

Interestingly, by flipping the sign of $\alpha$ we obtain a delta-function
barrier instead of a delta-function well, which kills the bound state but does
not affect the reflection and transmission coefficients (which only rely on
$\alpha^2$). The particle is just as likely to pass through the barrier as to
cross over the well. We call this phenomenon \textbf{tunneling}.

\subsubsection{The Finite Square Well}

Consider the potential
\begin{equation} \label{eq:fin-sq-well}
  V(x) =
  \begin{cases}
    -V_0, &\text{for } -a \leq x \leq a, \\
    0, &\text{for } |x| > a,
  \end{cases}
\end{equation}
where $V_0$ is a positive constant.

We first examine the bound states ($E < 0$). Note that the potential is an
even function, so we can assume with no loss of generality that the solutions
are either odd or even (Problem 2.1(c)). In the region $x < -a$ the potential
is zero, so the Sch\"{o}rdinger equation reads
\begin{equation} \label{eq:fin-sq-well-bou-1}
  \frac{d^2\psi}{dx^2} = \kappa^2\psi,
\end{equation}
where $\kappa \equiv \sqrt{-2mE}/\hbar$ is real and positive. The physically
admissible solution is
\begin{equation} \label{eq:fin-sq-well-bou-2}
  \psi(x) = Be^{\kappa x}, \text{ for } x < -a.
\end{equation}
In the region $-a < x < a$, $V(x) = -V_0$, and the Schr\"{o}dinger equation
reads
\begin{equation} \label{eq:fin-sq-well-bou-3}
  \frac{d^2\psi}{dx^2} = -l^2\psi,
\end{equation}
where $l \equiv \sqrt{2m(E + V_0)}/\hbar$. Although $E$ is negative, for bound
states, it must be greater than $-V_0$, by the old theorem $E > V_{\min}$
(Problem 2.2); so $l$ is also real and positive. The general solution is
\begin{equation} \label{eq:fin-sq-well-bou-4}
  \psi(x) = C\sin(lx) + D\cos(lx), \text{ for } -a < x < a,
\end{equation}
where $C$ and $D$ are arbitrary constants. Finally for the region $x > a$, we
are left with
\begin{equation} \label{eq:fin-sq-well-bou-5}
  \psi(x) = Fe^{-\kappa x}, \text{ for } x > a.
\end{equation}
We know that as $\psi(-x) = \pm\psi(x)$, we need only impose the boundary
conditions on one side. We work out the even solutions here; as the cosine is
even (and the sine is odd), we are looking for solutions of the form
\begin{equation} \label{eq:fin-sq-well-bou-6}
  \psi(x) =
  \begin{cases}
    Fe^{-\kappa x}, &\text{for } x > a, \\
    D\cos(lx), &\text{for } 0 < x < a, \\
    \psi(-x), &\text{for } x < 0.
  \end{cases}
\end{equation}
The continuity of $\psi(x)$ and $d\psi/dx$ at $x = a$ says that $\kappa =
l\tan(la)$. To solve for $E$, we adopt some nicer notation; let \[
  z \equiv la, \quad \text{ and } \quad z_0 \equiv \frac{a}{\hbar}\sqrt{2mV_0}.
\] $\kappa^2 + l^2 = 2mV_0/\hbar^2$, so $\kappa a = \sqrt{z_0^2 - z^2}$, and
therefore \[
  \tan z = \sqrt{(z_0/z)^2 - 1}.
\] Two limiting cases are of special interest:
\begin{enumerate}
  \item \textbf{Wide, deep well.} If $z_0$ is very large, the intersections
    between $\tan z$ and $\sqrt{(z_0/z)^2 - 1}$ occur just slightly below $z_n
    = n\pi/2$, with $n$ odd; it follows that \[
      E_n + V_0 \approx \frac{n^2\pi^2\hbar^2}{2m(2a)^2}.
    \] On the right side we have half of the infinite square well energies,
    for a well of width $2a$, as $n$ is odd. So the finite square well goes
    over to the infinite square well, as $V_0 \to \infty$; however, for any
    finite $V_0$ there are only a finite number of bound states.
  \item \textbf{Shallow, narrow well.} As $z_0$ decreases, there are fewer and
    fewer bound states, until finally one state is left. Note, however, that
    there is always one bound state no matter how "weak" the well becomes.
\end{enumerate}

For the scattering states ($E > 0$), to the left, where $V(x) = 0$, we have \[
  \psi(x) = Ae^{ikx} + Be^{-ikx}, \text{ for } x < -a,
\] where $k \equiv \sqrt{2mE}/\hbar$. Inside the well, where $V(x) = -V_0$, \[
  \psi(x) = C\sin(lx) + D\cos(lx), \text{ for } -a < x < a,
\] where $l \equiv \sqrt{2m(E + V_0)}/\hbar$. To the right, assuming there is
no incoming wave in this region, we have \[
  \psi(x) = Fe^{ikx}.
\] Here $A$ is the incident amplitude, $B$ is the reflected amplitude, and $F$
is the transmitted amplitude. Applying the boundary conditions, we obtain the
following:
\begin{equation} \label{eq:fin-sq-well-sca-1}
  \begin{gathered}
    B = i\frac{\sin(2la)}{2kl}(l^2 - k^2)F, \\
    F = \frac{e^{-2ika}A}{\cos(2la) - i\frac{(k^2 + l^2)}{2kl}\sin(2la)}.
  \end{gathered}
\end{equation}
The transmission coefficient is given by
\begin{equation} \label{eq:fin-sq-well-sca-tran}
  T^{-1} = 1 + \frac{V_0^2}{4E(E + V_0)}\sin^2\left(
    \frac{2a}{\hbar}\sqrt{2m(E + V_0)}
  \right).
\end{equation}
Notice that $T = 1$ whenever the sine is zero, which is when \[
  \frac{2a}{\hbar}\sqrt{2m(E_n + V_0)} = n\pi,
\] where $n$ is any integer. The energies for perfect transmission, then, are
given by \[
  E_n + V_0 = \frac{n^2\pi^2\hbar^2}{2m(2a)^2},
\] which happen to be precisely the allowed energies for the infinite square
well.

\end{document}

