\documentclass{article}

\usepackage{amsmath,amssymb,amsthm}
\usepackage[shortlabels]{enumitem}

\newtheorem{theorem}{Theorem}
\newtheorem{lemma}{Lemma}
\newtheorem*{lemma*}{Lemma}

\title{Notes to ``Introduction to Quantum Mechanics'', 2nd edition (Griffiths)}
\author{Ng Wei En}

\begin{document}

\maketitle

\section{The Wave Function}

\subsection{Schr\"{o}dinger's Equation}

A particle's position is defined by its \emph{wave function} $\Psi(x, t)$,
which can be obtained by solving \textbf{Schr\"{o}dinger's equation}:
\begin{equation} \label{eq:sch-eqn}
  \boxed{
    i\hbar \frac{\partial \Psi}{\partial t}
    = -\frac{\hbar^2}{2m} \frac{\partial^2 \Psi}{\partial x^2} + V\Psi.
  }
\end{equation}

Born's \emph{statistical interpretation} of the wave function states the
probability of finding the particle between $a$ and $b$ at time $t$ is \[
  \int_a^b |\Psi(x, t)|^2 \,dx.
\] It follows that the integral of $|\Psi|^2$
must be 1, i.e. the particle must be somewhere:
\begin{equation} \label{eq:sch-prob}
  \boxed{
    \int_{-\infty}^{+\infty} |\Psi(x, t)|^2 \,dx = 1.
  }
\end{equation}

\subsubsection{Normalisation}

If $\Psi(x, t)$ is a solution, then $A \Psi(x, t)$ for any complex constant $A$
must also be a solution, so any appropriate multiplicative factor can be chosen
to normalise the wave function.

\begin{lemma*}
  The Schr\"{o}dinger equation preserves the normalization of the wave function
  as time goes on and $\Psi$ evolves.
\end{lemma*}

\begin{proof}
  To begin with, \[
    \frac{d}{dt} \int_{-\infty}^{+\infty} |\Psi(x, t)|^2 \,dx
    = \int_{-\infty}^{+\infty} \frac{\partial}{\partial t} |\Psi(x, t)|^2 \,dx.
  \]
  By the product rule,
  \begin{equation} \label{eq:sch-norm-1}
    \frac{\partial}{\partial t}|\Psi|^2
    = \frac{\partial}{\partial t} (\Psi^* \Psi)
    = \Psi^* \frac{\partial \Psi}{\partial t} +
    \frac{\partial \Psi^*}{\partial t} \Psi.
  \end{equation}
  Now the Schr\"{o}dinger equation says that
  \begin{equation} \label{eq:sch-norm-2}
    \frac{\partial \Psi}{\partial t}
    = \frac{i\hbar}{2m} \frac{\partial^2 \Psi}{\partial x^2} -
    \frac{i}{\hbar} V \Psi,
  \end{equation}
  and hence also
  \begin{equation} \label{eq:sch-norm-3}
    \frac{\partial \Psi^*}{\partial t}
    = -\frac{i\hbar}{2m} \frac{\partial^2 \Psi^*}{\partial x^2} +
    \frac{i}{\hbar} V \Psi^*,
  \end{equation}
  so
  \begin{equation} \label{eq:sch-norm-4}
    \frac{\partial}{\partial t} |\Psi|^2
    = \frac{i\hbar}{2m} \left(
      \Psi^* \frac{\partial^2 \Psi}{\partial x^2} -
      \frac{\partial^2 \Psi^*}{\partial x^2} \Psi
    \right)
    = \frac{\partial}{\partial t} \left[
      \frac{i\hbar}{2m} \left(
        \Psi^* \frac{\partial \Psi}{\partial x} -
        \frac{\partial \Psi^*}{\partial x} \Psi
      \right)
    \right]
  \end{equation}
  The integral in \eqref{eq:sch-norm-1} can now be evaluated explicitly: \[
    \frac{\partial}{\partial t} \int_{-\infty}^{+\infty} |\Psi(x, t)|^2 \,dx
    = \frac{i\hbar}{2m} \left.\left(
      \Psi^* \frac{\partial \Psi}{\partial x} -
      \frac{\partial \Psi^*}{\partial x} \Psi
    \right)\right|_{-\infty}^{+\infty}.
  \] But $\Psi(x, t)$ must go to zero as $x$ goes to $\pm\infty$ --- otherwise
  the wave function would not be normalizable. It follows that
  \begin{equation} \label{eq:sch-norm-5}
    \frac{\partial}{\partial t} \int_{-\infty}^{+\infty} |\Psi(x, t)|^2 \,dx
    = 0,
  \end{equation}
  and hence that the integral is \emph{constant} (independent of time); if
  $\Psi$ is normalized at $t = 0$, it \emph{stays} normalized for all future
  time.
\end{proof}

\subsection{Velocity and Momentum}

From equation \eqref{eq:sch-norm-4}, the expectation value of \emph{velocity}
can be calculated through \[
  \frac{d\langle x \rangle}{dt}
  = \int_{-\infty}^{+\infty} x \frac{\partial}{\partial t} |\Psi|^2 \,dx
  = \frac{i\hbar}{2m} \int_{-\infty}^{+\infty} x\frac{\partial}{\partial x}
  \left(
    \Psi^*\frac{\partial \Psi}{\partial x} -
    \frac{\partial \Psi^*}{\partial x}\Psi
  \right) \,dx.
\]
This can be simplified using integration by parts, \[
  \frac{d\langle x \rangle}{dt}
  = -\frac{i\hbar}{m} \int_{-\infty}^{+\infty} \Psi^*
  \frac{\partial \Psi}{\partial x} \,dx.
\]
Working with \emph{momentum} instead, rather than velocity, \[
  \langle p \rangle
  = m \frac{d \langle x \rangle}{dt}
  = -i\hbar \int_{-\infty}^{+\infty}
    \Psi^*\frac{\partial \Psi}{\partial x}
  \,dx.
\]
Finally, writing the equations for $\langle x \rangle$ and $\langle p \rangle$
in a more suggestive way:
\begin{gather}
  \label{eq:sch-vel}
  \langle x \rangle = \int_{-\infty}^{+\infty} \Psi^* (x) \Psi \,dx, \\
  \label{eq:sch-mom}
  \langle p \rangle
  = \int_{-\infty}^{+\infty} \Psi^*\left(
    \frac{\hbar}{i} \frac{\partial}{\partial x}
  \right)\Psi \,dx.
\end{gather}
To calculate the expectation value of any quantity $Q(x, p)$, we integrate
\begin{equation} \label{eq:sch-qty}
  \boxed{
    \langle Q(x, p) \rangle = \int_{-\infty}^{+\infty} \Psi^* Q\left(
      x, \frac{\hbar}{i}\frac{\partial}{\partial x}
    \right)\Psi \,dx.
  }
\end{equation}

\subsection{The Uncertainty Principle}

The wavelength of $\Psi$ is related to the \emph{momentum} of the particle by
the \textbf{de Broglie formula}:
\begin{equation} \label{eq:de-brog}
  p = \frac{h}{\lambda} = \frac{2\pi\hbar}{\lambda},
\end{equation}
which leads to
\begin{equation} \label{eq:unc-prin}
  \boxed{
    \sigma_x \sigma_p \geq \frac{\hbar}{2},
  }
\end{equation}
where $\sigma_x$ and $\sigma_p$ are the standard deviation in $x$ and $p$
respectively.

\end{document}

