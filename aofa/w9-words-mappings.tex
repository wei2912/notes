\documentclass{article}

\usepackage{amsmath,amssymb,amsthm}

\newtheorem{corollary}{Corollary}
\newtheorem{lemma}{Lemma}
\newtheorem{theorem}{Theorem}

\begin{document}

\title{Week 9: Words and Mappings}
\maketitle

\section{Words}

A \textbf{word} is a sequence of $M$ urns holding $N$ objects in total. Let
$W_M$ be the class of $M$-sequences of words. Then,
\begin{gather*}
  W_M = \mathrm{SEQ}_M(\mathrm{SET}(Z)) \\
  W_M(z) = (e^z)^M = e^{Mz} \\
  N![z^N]W_M(z) = M^N
\end{gather*}

There is a natural correspondence between a \textbf{string}, a sequence of $N$
characters (from an $M$-char alphabet), and a \textbf{word}, a sequence of $M$
labelled sets (having $N$ objects in total). The former studies the sequence
of characters, while the latter studies the sets of indices.

\subsection{Birthday Problem}

A birthday sequence is a word where no set has more than one element. Let $B_M$
be the class of birthday sequences of length $M$.
\begin{gather*}
  B_M = \mathrm{SEQ}_M(E + Z) \\
  B_M(z) = (1 + z)^M \\
  \begin{aligned}
    N![z^N]B_M(z)
    &= N!\binom{M}{N} = \frac{M!}{(M - N)!} \\
    &= M(M - 1)(M - 2)\ldots(M - N + 1)
  \end{aligned}
\end{gather*}

The probability that no character is repeated in a random $M$-word of length
$N$ is \[
  \frac{M!}{M^N(M - N)!},
\] and the expected position of the first repeat is \[
  \sum_{0 \leq N \leq M} \frac{M!}{M^N(M - N)!} = 1 + Q(M) \sim \sqrt{\pi M/2}
\] where $Q(M)$ is the Ramanujan Q-function.

\subsection{Coupon Collector Problem}

A coupon collector sequence is an $M$-word with no empty set. Let $R_M$ be the
class of coupon collector sequences of length $M$.
\begin{gather*}
  R_M = \mathrm{SEQ}_M(\mathrm{SET}_{> 0}(Z)) \\
  R_M(z) = (e^z - 1)^M \\
  \begin{aligned}
    N![z^N]R_M(z)
    &= N![z^N] \sum_j \binom{M}{j} (-1)^j e^{(M - j)z} \\
    &= \sum_j \binom{M}{j} (-1)^j (M - j)^N \sim M^N.
  \end{aligned}
\end{gather*}

The probability that a random $M$-word of length $N$ is a coupon collector
sequence is \[
  \frac{1}{M^N} \sum_j \binom{M}{j} (-1)^j (M - j)^N
  = \sum_j \binom{M}{j} (-1)^j \left(1 - \frac{j}{M}\right)^N.
\] Hence, the probability that a collection in a random $M$-word completes in
$> N$ chars is \[
  1 - \sum_j \binom{M}{j} (-1)^j \left(1 - \frac{j}{M}\right)^N
\] and the average number of chars to complete a collection in a random
$M$-word is \[
  \sum_{N \geq 0} \left(
    1 - \sum_j \binom{M}{j} (-1)^j \left(1 - \frac{j}{M}\right)^N
  \right)
  = -M\sum_{j \geq 1} \binom{M}{j} \frac{(-1)^j}{j}
  = MH_M
\]

An alternative analysis considers $W_{Mk}$, the class of $M$-words with $k$
different letters and the last letter appearing only once. From the OGF \[
  W_{Mk}(z) = \sum_{N \geq 0} W_{MNk}z^N
\] a PGF can be constructed: \[
  W_{Mk}(z/M) = \sum_{N \geq 0} W_{MNk}\frac{z^N}{M^N}.
\] The mean wait time for $k$ coupons is \[
  w_{Mk} \equiv \left.W_{Mk}'(z/M)\right|_{z = 1}
  = \sum_{N \geq 0} N\frac{W_{MNk}}{M^N}z^N.
\] With the construction \[
  W_{Mk} = (k - 1)Z \times W_{Mk} + (M - k + 1)Z \times W_{M(k - 1)}
\] and some tedious algebra, it can be shown that $w_{Mk} = M(H_M - H_{M - k})$
with $W_{MM} = MH_M$.

\subsection{Surjection}

An $M$-surjection is an $M$-word with no empty set. Let $R_M$ be the class of
$M$-surjections.

\begin{gather*}
  R_M = \mathrm{SEQ}_M(\mathrm{SET}_{> 0}(M)) \\
  R_M(z) = (e^z - 1)^M \\
  R_{MN} \sim M^N
\end{gather*}

A surjection is a word that is an $M$-surjection for some $M$.

\begin{gather*}
  R = \mathrm{SEQ}(\mathrm{SET}_{> 0}(M)) \\
  R(z) = \frac{1}{1 - (e^z - 1)} = \frac{1}{2 - e^z} \\
  N![z^N]R(z) \sim \frac{N!}{2(\ln 2)^{N + 1}}
\end{gather*}

\end{document}

