\documentclass{article}

\usepackage{amsmath,amssymb,amsthm}

\newtheorem{corollary}{Corollary}
\newtheorem{lemma}{Lemma}
\newtheorem{theorem}{Theorem}

\begin{document}

\title{Week 8: Strings and Tries}
\maketitle

\section{Bitstrings}

\subsection{Bitstrings without long runs of 0s}

We define the combinatorial class $B_p$, the class of bitstrings with no $0^p$,
as: \[
  B_p = Z_{< p}(E + Z_1 B_p),
\] which has a corresponding OGF \[
  B_p(z) = (1 + z + \cdots + z^{p - 1})(1 + zB_p(z))
  = \frac{1 - z^p}{1 - 2z + z^{p + 1}}.
\] leading to the coefficients $[z^N]B_k(z) \sim c_k \beta_k^N$ where $\beta_k$
is the dominant root of $1 - 2z + z^k$. It can be shown that the expected
position of the end of the first $0^p$ sequence is $B_p(1/2) = 2^{p + 1} - 2$.

\subsection{Autocorrelation}

Given a binary pattern $p$ and two combinatorial classes $S_p$ and $T_p$, the
former the class of bitstrings that do not contain $p$, and the latter the
class of bitstrings that end in $p$ and have no other occurence of $p$, \[
  S_p + T_p = E + S_p \times \{Z_0 + Z_1\}
\] Every pattern has an autocorrelation polynomial, constructed by sliding the
pattern to the left over itself, and including a term $z^{|p| - i}$ for each
match of $i$ trailing bits (e.g. the pattern $101001010$ has an autocorrelation
polynomial $C_{101001010}(z) = 1 + z^5 + z^7$).

For every 1-bit in the autocorrelation of any string in $T_p$, adding a "tail"
forms a string in $S_p$ followed by the pattern $p$. Hence, \[
  S_p \times \{p\} = T_p \times \sum_{c_i \neq 0} \{t_i\}.
\]

From the two constructions of $S_p$ and $T_p$, we obtain the two OGFs
\begin{gather*}
  S_p(z) + T_p(z) = 1 + 2zS_p(z), \\
  S_p(z)z^{|p|} = T_p(z)c_p(z),
\end{gather*} which leads to \[
  S_p(z) = \frac{c_p(z)}{z^p + (1 - 2z)c_p(z)}.
\] leading to the coefficients $[z^N]S_p(z) \sim c_p \beta_p^N$ where $\beta_p$
is the dominant root of $z^p + (1 - 2z)c_p(z)$.

\section{Regular Expressions}

Let $A$ and $B$ be unambiguous REs with OGFs $A(z)$ and $B(z)$. If $A + B$,
$AB$ and $A^*$ are also unambiguous, then
\begin{gather*}
  A(z) + B(z) \text{ enumerates } A + B \\
  A(z)B(z) \text{ enumerates } AB \\
  \frac{1}{1 - A(z)} \text{ enumerates } A^*.
\end{gather*}

\begin{corollary}
  OGFs that enumerate regular languages are rational.
\end{corollary}
\begin{proof}
  \begin{enumerate}
    \item There exists a FSA for the language.
    \item \emph{Kleene's theorem} gives an unambiguous RE for the language
      defined by any FSA.
  \end{enumerate}
\end{proof}

\section{Context-free Languages}

Let $\langle A \rangle$ and $\langle B \rangle$ be nonterminals in an
unambiguous CFG with OGFs $A(z)$ and $B(z)$. If $\langle A \rangle | \langle B
\rangle$ and $\langle A \rangle \langle B \rangle$ are also unambiguous, then
\begin{gather*}
  A(z) + B(z) \text{ enumerates } \langle A \rangle | \langle B \rangle \\
  A(z)B(z) \text{ enumerates } \langle A \rangle \langle B \rangle \\
\end{gather*}

\begin{corollary}
  OGFs that enumerate unambiguous CF languages are algebraic.
\end{corollary}
\begin{proof}
  "Gr\"{o}bner basis" elimination -- see text.
\end{proof}

\section{Tries}

The probability that, with $N$ random strings, $k$ of them would start with a
zero bit is \[
  \frac{1}{2^N}\binom{N}{k};
\] this leads to the recurrence relation for the total path length of the $N$
external nodes of a trie, \[
  C_N = N + \frac{1}{2^N}\sum_k \binom{N}{k}(C_k + C_{N - k})
\] for $N > 1$ with $C_0 = C_1 = 0$. The corresponding EGF is
\begin{align*}
  C(z)
  &= ze^z - z + 2e^{z/2}C(z/2)
  &&\left(C(z) = \sum_{N \geq 0} C_N\frac{z^N}{N!}\right)
\end{align*} which iterates to give \[
  C(z) = z \sum_{j \geq 0} \left(e^z - e^{(1 - 2^{-j})z}\right)
\] Hence,
\begin{align*}
  C_N
  &= N![z^N]C(z) \\
  &= N\sum_{j \geq 0} \left(
    1 - \left(1 - \frac{1}{2^j}\right)^{N - 1}
  \right) \\
  &\sim N\sum_{j \geq 0} \left(1 - e^{-N/2}\right) \sim N \lg N
\end{align*}
(The proof of the last line is omitted. It relies on the isolation of the
periodic term in the sum.)

\end{document}

