\documentclass{article}

\usepackage{amsmath,amssymb,amsthm}

\newtheorem{corollary}{Corollary}
\newtheorem{lemma}{Lemma}
\newtheorem{theorem}{Theorem}

\begin{document}

\title{Week 7: Permutations}
\maketitle

\section{Involution}

An \textbf{involution} is a mapping of the numbers $1$ through $N$ to itself
with all 1- or 2-cycles.

We define the combinatorial class $I$ of involutions
\begin{equation*}
  I = \mathrm{SET}(\mathrm{CYC}_1(z)) \star \mathrm{SET}(\mathrm{CYC}_2(z)),
\end{equation*} which has a corresponding EGF \begin{equation*}
  I(z) = e^{z + z^2/2}.
\end{equation*} Extracting the coefficient to obtain the number of involutions
of length $N$, \begin{equation*}
I_N \equiv N![z^N]e^{z + z^2/2} = \sum_{0 \leq 2k \leq N}
\frac{N!}{k!2^k(N - 2k)!} \sim \frac{1}{\sqrt{2\sqrt{e}}}
\left(\frac{N}{e}\right)^{N/2}e^{\sqrt{N}}.
\end{equation*}

\section{Left-Right Minima}

A \textbf{left-right minimum} (lrm) is a permutation that is smaller than any
item to its left. Define $\mathrm{lrm}(p)$ as the number of left-right minima
of permutation $p$.

By taking the star product of a permutation $p$ with an atom, $|p| + 1$
permutations are created, with one permutation having an extra left-right
minima. Hence, the set of constructed permutations has $(|p| + 1)
\mathrm{lrm}(p) + 1$ left-right-minima.

To find the average number of left-right minimum, we construct a cumulative
generating function \begin{align*}
  B(z) &= \sum_{p \in P} \mathrm{lrm}(p)\frac{z^{|p|}}{|p|!} \\
       &= \sum_{p \in P} ((|p| + 1)\mathrm{lrm}(p) + 1)
       \frac{z^{|p| + 1}}{(|p| + 1)!} \\
       &= \sum_{p \in P} \mathrm{lrm}(p)\frac{z^{|p| + 1}}{|p|!} +
       \sum_{p \in P} \frac{z^{|p| + 1}}{(|p| + 1)!} \\
       &= zB(z) + \sum_{k \geq 0} \frac{z^{k + 1}}{k + 1} = zB(z) +
       \ln\frac{1}{1 - z}
\end{align*} which solves to \begin{equation*}
  B(z) = \frac{1}{1 - z}\ln\frac{1}{1 - z}
\end{equation*} and yields coefficients $[z^N]B(z) = \frac{B_N}{N!} = H_N$.

\section{Cycles}

To find the average number of cycles, the exact same derivation for left-right
minimum is used.

\subsection{1-cycles}

Creating $|p| + 1$ permutations from a permutation $p$ by inserting $|p| + 1$
into every position in every cycle (including the null cycle), the set of
constructed permutations has $(|p| + 1)\mathrm{cyc}_1(p) + 1 -
\mathrm{cyc}_1(p) = |p|\mathrm{cyc}_1(p) + 1$ 1-cycles.

To find the average number of 1-cycles, we construct a cumulative generating
function \begin{align*}
  B(z) &= \sum_{p \in P} \mathrm{cyc}_1(p)\frac{z^{|p|}}{|p|!} \\
       &= \sum_{p \in P} (|p|\mathrm{cyc}_1(p) + 1)
       \frac{z^{|p| + 1}}{(|p| + 1)!},
\end{align*} which when differentiated yields \begin{equation*}
  B'(z) = \sum_{p \in P} |p|\mathrm{cyc}_1(p)\frac{z^{|p|}}{|p|!} +
  \sum_{p \in P} \frac{z^{|p|}}{|p|!} = zB'(z) + \frac{1}{1 - z}
\end{equation*} and solves to \begin{equation*}
  B(z) = \frac{1}{1 - z}.
\end{equation*} with coefficient $[z^N]B(z) = \frac{B_N}{N!} = 1$.

\section{Inversions}

The number of \textbf{inversions} in a permutation is the number of pairs $(i,
j)$ with $i > j$. We define $\mathrm{inv}(p)$ to be the number of inversions in
the permutation $p$.

Creating $|p| + 1$ permutations from a permutation $p$ leads to the set of
constructed permutations having $(|p| + 1)\mathrm{inv}(p) + (|p| + 1)|p|/2$
inversions.

To find the average number of inversions, we construct a cumulative generating
function \begin{align*}
  B(z) &= \sum_{p \in P} \mathrm{inv}(p)\frac{z^{|p|}}{|p|!} \\
       &= \sum_{p \in P} ((|p| + 1)\mathrm{inv}(p) + (|p| + 1)|p|/2)
       \frac{z^{|p| + 1}}{(|p| + 1)!} \\
       &= \sum_{p \in P} \mathrm{inv}(p)\frac{z^{|p| + 1}}{|p|!} +
       \frac{1}{2}\sum_{p \in P} |p|\frac{z^{|p| + 1}}{|p|!} = zB(z) +
       \frac{z}{2}\sum_{k \geq 0} kz^k \\
       &= zB(z) + \frac{1}{2}\frac{z^2}{(1 - z)^2}
\end{align*} with coefficient $[z^N]B(z) = \frac{B_N}{N!} =
\frac{N(N - 1)}{4}$.

\section{Bivariate Generating Functions (BGFs)}

A \textbf{bivariate generating function} associated with an unlabelled class
is the formal power series \begin{equation*}
  A(z, u) = \sum_{a \in A} z^{|a|}u^{\mathrm{cost}(a)}
\end{equation*} where $|a|$ is the size and $\mathrm{cost}(a)$ is the value of
the parameter. Similarly, for a labelled class, one can define the power series
\begin{equation*}
  A(z, u) = \sum_{a \in A} \frac{z^{|a|}}{|a|!}u^{\mathrm{cost}(a)} =
  \sum_{N \geq 0} \sum_{k \geq 0} A_{Nk}\frac{z^N}{N!}u^k \end{equation*}

The average value of a parameter of permutation can be calculated by taking
the partial derivative \begin{equation*}
  \frac{\partial}{\partial u} A(z, u) = \sum_{N \geq 0} \sum_{k \geq 0} kA_{Nk}
  \frac{z^N}{N!}u^{k - 1}
\end{equation*} and defining the function \begin{equation*}
  A_u(z, 1) \equiv \left.\frac{\partial}{\partial u}A(z, u)\right|_{u = 1} =
  \sum_{N \geq 0} \sum_{k \geq 0} kA_{Nk}\frac{z^N}{N!}.
\end{equation*} The average value of the parameter is therefore
\begin{equation*}
  [z^N]A_u(z, 1) = \left.\frac{\partial}{\partial u}A(z, u)\right|_{u = 1} =
  \sum_{k \geq 0} k\frac{A_{Nk}}{N!}.
\end{equation*}

\section*{Exercises}

\paragraph{7.29} An \emph{arrangement} of $N$ elements is a sequence formed
from a subset of the elements. Prove that the EGF for arrangements is $e^z/(1 -
z)$. Express the coefficients as a simple sum and interpret that sum
combinatorially.

\paragraph{Solution:} An arrangement of $N$ elements can be viewed as a pair of
permutation of selected elements and a set of remaining elements. Hence,
\begin{equation*}
  \mathrm{ARR}(z) = \mathrm{SET}(z) \star \mathrm{SEQ}(z)
\end{equation*} and its corresponding EGF is \begin{equation*}
  A(z) = e^z \cdot \frac{1}{1 - z}
\end{equation*}. The number of arrangements of size $N$ is therefore
$N![z^N]A(z) = \sum_{k = 0}^N \frac{N!}{k!} = \sum_{k = 0}^N \binom{N}{k}k!$.

\paragraph{7.45} Find the CGF for the total number of inversions in all
involutions of length $N$. Use this to find the average number of inversions in
an involution.

\paragraph{Solution (using BGFs):} Defining an involution as \begin{equation*}
  I = \mathrm{SET}(\mathrm{CYC}_1(z)) \star \mathrm{SET}(u\mathrm{CYC}_2(z)),
\end{equation*} its corresponding BGF is \begin{equation*}
  J(z, u) = e^{z + uz^2/2}
\end{equation*} Taking its partial derivative with respect to $u$,
\begin{equation*}
  J_u(z, 1) = \left.\frac{\partial}{\partial u}I(z, u)\right|_{u = 1} =
  \frac{z^2}{2} e^{z + z^2/2}
\end{equation*} and so the average number of inversions in an involution of
length $N$ is \begin{equation*}
  [z^N]J_u(z, 1) = \frac{I_{N - 2}}{2}
\end{equation*} where $I_N$ is the number of involutions of length $N$.

\paragraph{7.61} Use asymptotics from generating functions or a direct argument
to show that the probability for a random permutation to have $j$ cycles of
length $k$ is asymptotic to the Poisson distribution $e^{-\lambda}\lambda^j/j!$
with $\lambda = 1/k$.

\paragraph{Solution:} The class of such permutations can be constructed as
\begin{equation*}
  \mathrm{SET}(\mathrm{CYC}_{\neq k}(Z)) \star
  \mathrm{SET}_j(\mathrm{CYC}_k(Z))
\end{equation*} which translates to the EGF \begin{align*}
  F(z) &= \exp\left(\log\frac{1}{1 - z} - \frac{z^k}{k}\right) \cdot
  \frac{(z^k/k)^j}{j!} \\
       &= \frac{1}{1 - z}e^{-z^k/k}z^{jk}\frac{\lambda^j}{j!}.
\end{align*} Considering the coefficients of $F(z)$, it suffices to prove that
\begin{equation*}
  [z^N]F(z) = [z^{N - jk}]\frac{1}{1 - z}e^{-z^k/k}\frac{\lambda^j}{j!} \sim
  \frac{e^{-\lambda}\lambda^j}{j!},
\end{equation*} which is done by considering the asymptotic growth of the
coefficients, \begin{equation*}
  \lim_{N \rightarrow \infty} [z^N]\frac{1}{1 - z}e^{-z^k/k} =
  \sum_{i \geq 0} \frac{\lambda^i}{i!} \sim e^{-\lambda}.
\end{equation*}

\end{document}

