\documentclass{article}

\usepackage{amsmath,amssymb,amsthm}

\newtheorem{corollary}{Corollary}
\newtheorem{lemma}{Lemma}
\newtheorem{theorem}{Theorem}

\begin{document}

\title{Week 5: Analytic Combinatorics}
\maketitle

\section{Transfer Theorem (unlabelled classes)}

Let $A$ and $B$ be combinatorial classes of unlabelled objects with OGFs $A(z)$
and $B(z)$.

\paragraph{$A + B$} \begin{equation*}
  \sum_{\gamma \in A + B} z^{|\gamma|} = \sum_{\alpha \in A} z^{|\alpha|} +
    \sum_{\beta in B} z^{|\beta|} = A(z) + B(z)
\end{equation*}

\paragraph{$A \times B$} \begin{equation*}
  \sum_{\gamma \in A \times B} z^{|\gamma|}
    = \sum_{\alpha \in A} \sum_{\beta \in B} z^{|\alpha| + |\beta|}
    = \left(\sum_{\alpha \in A} z^{|\alpha|}\right)
      \left(\sum_{\beta \in B} z^{|\beta|}\right)
    = A(z)B(z)
\end{equation*}

\section{Transfer Theorem (labelled classes)}

Let $A$ and $B$ be combinatorial classes of unlabelled objects with EGFs $A(z)$
and $B(z)$.

\paragraph{$A + B$} \begin{equation*}
  \sum_{\gamma \in A + B} \frac{z^{|\gamma|}}{|\gamma|!}
  = \sum_{\alpha \in A} \frac{z^{|\alpha|}}{|\alpha|!}
  + \sum_{\beta \in B} \frac{z^{|\beta|}}{|\beta|!}
  = A(z) + B(z)
\end{equation*}

\paragraph{$A \star B$} \begin{equation*}
  \sum_{\gamma \in A \star B} \frac{z^{|\gamma|}}{|\gamma|!}
  = \sum_{\alpha \in A} \sum_{\beta \in B} \binom{|\alpha| + |\beta|}{|\alpha|}
    \frac{z^{|\alpha| + |\beta|}}{(|\alpha| + |\beta|)!}
  = \left(\sum_{\alpha \in A} \frac{z^{|\alpha|}}{|\alpha|!}\right)
\left(\sum_{\beta \in B} \frac{z^{|\beta|}}{|\beta|!}\right)
  = A(z)B(z)
\end{equation*}

\section{Asymptotic Transfer Theorem}

\begin{theorem}[Radius-of-convergence Transfer Theorem]
  If $f(z)$ has radius of convergence $> 1$ with $f(1) \neq 0$, then
  \begin{equation*}
    [z^n]\frac{f(z)}{(1 - z)^{\alpha}}
      \sim f(1)\binom{n + \alpha - 1}{n}
      \sim \frac{f(1)}{\Gamma(\alpha)}n^{\alpha - 1}
  \end{equation*} for any real $\alpha \notin \mathbb{Z}_{\leq 0}$.
\end{theorem}

\begin{corollary}
  If $f(z)$ has radius of convergence $> p$ with $f(p) \neq 0$, then
  \begin{equation*}
    [z^n]\frac{f(z)}{(1 - z/p)^{\alpha}}
      \sim \frac{f(p)}{\Gamma(\alpha)}p^{-n}n^{\alpha - 1}
  \end{equation*} for any real $\alpha \notin \mathbb{Z}_{\leq 0}$.
\end{corollary}

\section*{Exercises}

\paragraph{5.3} Let $U$ be the class of binary trees with the size of a tree
defined to be the total number of nodes (internal plus external), so that the
generating function for its counting sequence is $U(z) = z + z^3 + 2z^5 + 5z^7
+ 14z^9 + \ldots$. Derive an explicit expression for $U(z)$.

\paragraph{Solution:} Defining $U = Z + U \times Z \times U$, the corresponding
OGF is $U(z) = z + U(z) \times z \times U(z)$ which reduces to
\begin{equation*}
  U(z) = \frac{1 \pm \sqrt{1 - 4z^2}}{2z}.
\end{equation*} Expanding $\sqrt{1 - 4z^2}$, \begin{equation*}
\sqrt{1 - 4z^2} = \sum_{k \geq 0} \binom{1/2}{k}(-4z^2)^k
  = 1 - 2z\sum_{k \geq 0} \frac{1}{k + 1}\binom{2k}{k}z^{2k + 1}
\end{equation*} after some tedious working. This leaves $U(z)$ with only one
solution, \begin{equation*}
  U(z) = \frac{1 - \sqrt{1 - 4z^2}}{2z}
  = \sum_{k \geq 0} \frac{1}{k + 1}\binom{2k}{k}z^{2k + 1}
  = \sum_{k \geq 0} C_kz^{2k + 1}.
\end{equation*}

\paragraph{5.7} Derive an EGF for the number of permutations whose cycles are
all of odd length.

\paragraph{Solution:} Since $\mathrm{CYC}_{\mathrm{odd}}(A) = \mathrm{CYC}(A) -
\mathrm{CYC}(A^2)$, its corresponding EGF is \begin{equation*}
  A(z) = \sum_{k \geq 0} \frac{z^k}{k} - \sum_{k \geq 0} \frac{z^{2k}}{2k}
    = \ln\frac{1}{1 - z} - \frac{1}{2}\ln\frac{1}{1 - z^2}
    = \ln\sqrt{\frac{1 + z}{1 - z}}.
\end{equation*} Therefore, the EGF of
$\mathrm{SET}(\mathrm{CYC}_{\mathrm{odd}(Z)})$ is $e^{A(z)} =
\sqrt{\frac{1 + z}{1 - z}}$.

\paragraph{5.15} Find the average number of internal nodes in a binary tree of
size $N$ with both children internal.

\paragraph{Solution:} Defining $T(z, u) := \sum_{t \in T} z^{|t|}
u^{\mathrm{int}(t)}$ where $\mathrm{int}(t)$ is the number of internal nodes in
$t$ with both children internal, it is clear that \begin{equation*}
  T(z, 1) = \frac{1 - \sqrt{1 - 4z}}{2z}.
\end{equation*} One can define the combinatorial class of binary trees as
\begin{align*}
  T &= E + Z \\
    &+ E \times Z \times (T - E) + (T - E) \times Z \times E \\
    &+ (T - E) \times Z \times (T - E)
\end{align*} to obtain \begin{align*}
  T(z, u) &= 1 + z \\
    &+ 2z(T(z, u) - 1) \\
    &+ zu(T(z, u) - 1)^2.
\end{align*} Finding the partial derivative of $T(z, u)$ with respect to $u$
for $u = 1$, \begin{equation*}
  \left.\frac{\partial T(z, u)}{\partial u}\right|_{u = 1}
    = \frac{z(T(z, 1) - 1)^2}{1 - 2zT(z, 1)}
    = 1 + \frac{z}{\sqrt{1 - 4z}} + \frac{\sqrt{1 - 4z} - 1}{2z},
\end{equation*} and $[z^N]\left.\frac{\partial T(z, u)}{\partial u}\right
  |_{u = 1} \sim \frac{4^{N - 1}}{\sqrt{\pi N}} \sim C_N/4$, so the average
number of internal nodes with both children internal is $N/4$.

\paragraph{5.16} Find the average number of internal nodes in a binary tree of
size $N$ with one child internal and one child external.

\paragraph{Solution:} The average number of internal nodes in a binary tree of
size $N$ with both children external is $N/4$ by a similar analysis, so the
average number of internal nodes in a binary tree of size $N$ with one child
internal and one child external must be $N/2$.

\end{document}

