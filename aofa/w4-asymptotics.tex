\documentclass{article}

\usepackage{amsmath,amssymb,amsthm}

\newtheorem{lemma}{Lemma}
\newtheorem{theorem}{Theorem}

\begin{document}

\title{Week 4: Asymptotics}
\maketitle

\paragraph{Asymptotic Scale.} A decreasing series $g_k(N)$ with $g_{k+1}(N) =
o(g_k(N))$ is called an \emph{asymptotic scale}. The series \begin{equation*}
  f(N) \sim c_0g_0(N) + c_1g_1(N) + c_2g_2(N) + \ldots
\end{equation*} is called an asymptotic expansion of $f$. The expansion
represents the collection of formulae \begin{align*}
  f(N) &= O(g_0(N)) \\
  f(N) &= c_0g_0(N) + O(g_1(N)) \\
  f(N) &= c_0g_0(N) + c_1g_1(N) + O(g_2(N)) \\
  f(N) &= c_0g_0(N) + c_1g_1(N) + c_2g_2(N) + O(g_3(N)) \\
  \vdots
\end{align*} The standard scale is products of powers of $N$, $\log N$,
iterated logs and exponentials.

\begin{theorem}
  Assume that a rational GF $f(z)/g(z)$ with $f(z)$ and $g(z)$ relatively
  prime and $g(0) = 0$ has a unique pole $1/\beta$ of smallest modulus and that
  the multiplicity of $\beta$ is $\nu$. Then \begin{equation*}
    [z^n]\frac{f(z)}{g(z)} \sim C\beta^nn^{\nu - 1} \text{ where } C = \nu
      \frac{(-\beta)^{\nu}f(1/\beta)}{g^{(\nu)}(1/\beta)}.
  \end{equation*}
\end{theorem}

\begin{proof}
  \begin{equation*}
    \sum_{0 \leq j < m_1} c_{1j}n^j\beta_1^n + \sum_{0 \leq j < m_2}c_{2j}n^j
      \beta_2^n + \ldots + \sum_{0 \leq j < m_r}c_{rj}n^j\beta_r^n
  \end{equation*} Largest term dominates.
\end{proof}

\subsection{Asymptotics of finite sums}

\paragraph{Bounding the tail.} Make a \emph{rapidly decreasing sum} infinite.
\begin{align*}
  N!\sum_{0 \leq k \leq N}\frac{(-1)^k}{k!} &= N!e^{-1} - R_n \text{ where }
    R_n = N!\sum_{k > N}\frac{(-1)^k}{k!} \\
    &= \frac{N!}{e} + O\left(\frac{1}{N}\right).
\end{align*}

\paragraph{Using the tail.} The last term of a \emph{rapidly increasing sum}
may dominate. \begin{align*}
  \sum_{0 \leq k \leq N}k! = N!\left(1 + \frac{1}{N} + \sum_{0 \leq k \leq N -
    2}\frac{k!}{N!}\right) = N!\left(1 + O\left(\frac{1}{N}\right)\right)
\end{align*}

\begin{theorem}[Euler-Maclaurin Summation]
  Let $f$ be a function defined on $[1, \infty)$ whose derivatives exist and
  are absolutely integrable. Then \begin{equation*}
    \sum_{1 \leq k \leq N}f(k) = \int_1^N f(x)\,\mathrm{d}x + \frac{1}{2}f(N) +
      C_f + \frac{1}{12}f'(N) - \frac{1}{720}f'''(N) + \ldots
  \end{equation*}
\end{theorem}

\end{document}

