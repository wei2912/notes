\documentclass{article}

\usepackage{amsmath,amssymb,amsthm}

\begin{document}

\title{Week 1: Analysis of Algorithms}
\maketitle

\paragraph{Quicksort.} Assume array of size $N$ with entries distinct and
randomly ordered. Let $C_N$ be the expected number of compares used by
quicksort. Given that $N + 1$ compares are used for partitioning, and the
probability that $k$ is the partitioning element, a recurrence relation can
be formed for $C_N$ with $C_0 = 0$: \begin{equation*}
  C_N = N + 1 + \sum_{1 \leq k \leq N}\frac{1}{N}(C_{k - 1} + C_{N - k}).
\end{equation*} By symmetry, \begin{equation*}
  C_N = N + 1 + \frac{2}{N}\sum_{1 \leq k \leq N}C_{k - 1},
\end{equation*} and multiplying both sides by $N$, \begin{equation*}
  NC_N = N(N + 1) + 2\sum_{1 \leq k \leq N}C_{k - 1}.
\end{equation*} Subtracting the same formula by $(N - 1)C_{N - 1}$,
\begin{equation*}
  NC_N - (N - 1)C_{N - 1} = 2N + 2C_{N - 1}
\end{equation*} and thus \begin{equation*}
  NC_N = (N + 1)C_{N - 1} + 2N.
\end{equation*}

The tricky step is to divide the whole equation by $N(N + 1)$:
\begin{equation*}
  \frac{C_N}{N + 1} = \frac{C_{N - 1}}{N} + \frac{2}{N + 1}.
\end{equation*} Applying telescoping, \begin{align*}
  \frac{C_N}{N + 1} &= \frac{C_{N - 1}}{N} + \frac{2}{N + 1} =
    \frac{C_{N - 2}}{N - 1} + \frac{2}{N} + \frac{2}{N + 1} \\
    &= \frac{C_1}{2} + \frac{2}{3} + \cdots + \frac{2}{N} + \frac{2}{N + 1}.
\end{align*} This leads us to \begin{equation*}
  C_N \sim 2N\sum_{1 \leq k \leq N}\frac{1}{k} - 2N,
\end{equation*} which can be approximated by an integral: \begin{align*}
  C_N &\sim 2N\left(\int^{\infty}_1 \frac{1}{x} \mathrm{d}x + \gamma\right)
    - 2N \\
    &= 2N\ln{N} - 2(1 - \gamma)N.
\end{align*}

\section*{Exercises}

\paragraph{1.14} Follow through the steps above to solve the recurrence
\begin{equation*}
  A_N = 1 + \frac{2}{N}\sum_{1 \leq j \leq N}A_{j - 1} \text{ for } N > 0
\end{equation*} with $A_0 = 0$. This is the number of times quicksort is called
with \verb|hi >= lo|.

\paragraph{Solution:} Multiplying by $N$ and subtracting away the same equation
for $N - 1$, \begin{align*}
  NA_N &= N + 2\sum_{1 \leq j \leq N}A_{j - 1}, \\
  (N - 1)A_{N - 1} &= N - 1 + 2\sum_{1 \leq j \leq N - 1}A_{j - 1},
\end{align*} which leads to the much simpler recurrence relation
\begin{equation*}
  NA_N = (N + 1)A_{N - 1} + 1,
\end{equation*} that can be rewritten into \begin{equation*}
  \frac{A_N}{N + 1} = \frac{A_{N - 1}}{N} + \frac{1}{N(N + 1)}.
\end{equation*} Telescoping leads to the neat result \begin{equation*}
  \frac{A_N}{N + 1} = \frac{A_0}{1} + \sum_{1 \leq j \leq N}\frac{1}{j(j + 1)}
    = \sum_{1 \leq j \leq N}\frac{1}{j(j + 1)}
\end{equation*} since $A_0 = 0$. It should be noted that $\sum_{1 \leq j \leq
N}\frac{1}{j(j + 1)} = \sum_{1 \leq j \leq N}\left(\frac{1}{j} - \frac{1}{j +
1}\right)$ has an exact solution $\frac{1}{1} - \frac{1}{N + 1} = \frac{N}{N +
1}$. Therefore, $A_N = N$.

\paragraph{1.15} Show that the average number of exchanges used during the
first partitioning stage (before the pointers cross) is $(N - 2)/6$. (Thus, by
linearity of the recurrences, the average number of exchanges used by quicksort
is $\frac{1}{6}C_N - \frac{1}{2}A_N$.)

\paragraph{Solution:} The average number of exchanges used during the first
partitioning stage (before the pointers cross), given that $i$ is the rank of
the pivot, is equal to the expected number of elements of rank more than $i -
1$ in the first $i - 1$ elements.

Noting that the probability of any element (other than the pivot) being of rank
more than $i - 1$ is $\frac{(N - 1) - (i - 1)}{N - 1} = \frac{N - i}{N - 1}$
and summing up the probabilities for the first $i - 1$ elements, the expected
number of exchanges given that $i$ is the rank of the pivot is $(i - 1)\frac{N
- i}{N - 1}$.

Since the probability of $i$ being the rank of the pivot is $\frac{1}{N}$, the
expected number of exchanges in the first partitioning stage is
\begin{align*}
  \sum_{i = 1}^N{\frac{1}{N} \cdot (i - 1)\frac{N - i}{N - 1}} &= -\frac{1}{N(N
    - 1)}\sum_{i = 1}^N{i^2} + \frac{N + 1}{N(N - 1)}\sum_{i = 1}^N{i} -
    \frac{1}{N - 1}\sum_{i = 1}^N{1} \\
    &= -\frac{1}{N(N - 1)} \cdot \frac{N(N + 1)(2N + 1)}{6} \\
      &+ \frac{N + 1}{N(N - 1)} \cdot \frac{N(N + 1)}{2} - \frac{1}{N - 1}
      \cdot N \\
    &= \frac{N - 2}{6}
\end{align*} after a fair bit of algebra.

\paragraph{1.16} How many subarrays of size $k$ or less (not counting subarrays
of size 0 or less) are encountered, on the average, when sorting a random file
of size $N$ with quicksort?

\paragraph{Solution:} Let $B_N$ be the average number of subarrays of size $k$
or less encountered when sorting a random file of size $N$ with quicksort.
The probability that the left subarray takes on $m - 1$ number of elements is
equal to the probability that the rank of the pivot is $m$, or $\frac{1}{N -
1}$. A recurrence relation can be formulated for $B_N$,
\begin{equation*}
  B_N = \begin{cases}
    \frac{1}{N - 1}\sum_{1 \leq j \leq N}(B_{j - 1} + B_{N - j}), &N > k; \\
    1 + \frac{1}{N - 1}\sum_{1 \leq j \leq N}(B_{j - 1} + B_{N - j}), &N \leq
      k,
  \end{cases}
\end{equation*} with $B_1 = 1$. For $N - 1 > k$, the recurrence relation can be
simplified to $(N - 1)B_N - (N - 2)B_{N - 1} = 2B_{N - 1}$, or
\begin{equation*}
  \frac{B_N}{N} = \frac{B_{N - 1}}{N - 1} \text{ for } N - 1 > k.
\end{equation*} With telescoping, $B_N = \frac{N}{k}B_k$. Now, we proceed to
solve for $B_k$.

For $N \leq k$, the recurrence relation can be simplified to $(N - 1)B_N - (N -
2)B_{N - 1} = 1 + 2B_{N - 1}$, or \begin{equation*}
  \frac{B_N}{N} = \frac{B_{N - 1}}{N - 1} + \frac{1}{N(N - 1)} \text{ for } N
    \leq k.
\end{equation*} Applying to this $B_k$, \begin{align*}
  \frac{B_k}{k} &= \frac{B_{k - 1}}{k - 1} + \frac{1}{k(k - 1)} \\
    &= \frac{B_1}{1} + \sum_{2 \leq j \leq k}\frac{1}{j(j - 1)} \\
    &= 1 + \sum_{2 \leq j \leq k}\frac{1}{j(j - 1)}.
\end{align*} Noting that $\sum_{2 \leq j \leq k}\frac{1}{j(j - 1)} = \sum_{2
\leq j \leq k}\left(\frac{1}{j - 1} - \frac{1}{j}\right) = \frac{1}{1} -
\frac{1}{k}$, \begin{equation*}
  B_k = 2k - 1
\end{equation*} and therefore \begin{equation*}
  B_N = N\left(2 - \frac{1}{k}\right), \text{ for } N - 1 > k.
\end{equation*}

\paragraph{1.17} If we change the first line of the quicksort implementation
above to call insertion sort when \verb|hi - lo <= M| then the total number of
comparisons to sort $N$ elements is described by the recurrence
\begin{equation*}
  C_N = \begin{cases}
    N + 1 + \frac{1}{N}\sum_{1 \leq j \leq N}(C_{j - 1} + C_{N - j}), &N > M;
      \\
    \frac{1}{4}N(N - 1), &N \leq M.
  \end{cases}
\end{equation*}

\paragraph{Solution:} For $N - 1 > M$, following the steps in the lecture to
the point before telescoping, \begin{align*}
  \frac{C_N}{N + 1} &= \frac{C_{N - 1}}{N} + \frac{2}{N + 1} \\
    &= \frac{C_M}{M + 1} + \sum_{M + 1 \leq j \leq N}\frac{2}{j + 1} \\
    &= \frac{M(M - 1)}{4(M + 1)} + 2(H_{N + 1} - H_{M + 1})
\end{align*} and so \begin{equation*}
  C_N = (N + 1)\left(\frac{M(M - 1)}{4(M + 1)} + 2(H_{N + 1} - H_{M + 1})
    \right).
\end{equation*}

\paragraph{1.18} Ignoring small terms (those significantly less than $N$) in
the answer to the previous exercise, find a function $f(M)$ so that the number
of comparisons is approximately \begin{equation*}
  2N\ln N + f(M)N.
\end{equation*}
Plot the function $f(M)$, and find the value of $M$ that minimizes the
function.

\paragraph{Solution:} Using the approximation $H_{N + 1} \sim \ln N$,
\begin{align*}
  C_N &\sim N\left(\frac{M(M - 1)}{4(M + 1)} + 2(\ln N - H_{M + 1})\right)
    \\
    &= 2N\ln N + \left(\frac{M(M - 1)}{4(M + 1)} - 2H_{M + 1}\right)N.
\end{align*}
A quick plot reveals that $M = 7$ yields the minimum value of $f(M)$.

\end{document}
