\documentclass{article}

\usepackage{amsmath,amssymb,amsthm}
\usepackage[shortlabels]{enumitem}

\newtheorem{corollary}{Corollary}
\newtheorem*{definition*}{Definition}
\newtheorem{lemma}{Lemma}
\newtheorem{proposition}{Proposition}
\numberwithin{proposition}{subsection}
\newtheorem{theorem}{Theorem}

\DeclareMathOperator{\ord}{ord}

\begin{document}

\title{Notes to Ireland \& Rosen, "A Classical Introduction to Modern Number
Theory"}
\maketitle

\section{Unique Factorisation}

\subsection{Unique Factorization in $\mathbb{Z}$}

$\mathbb{Z}$ denotes the ring of integers, with 1 and -1 the \emph{units} of
$\mathbb{Z}$.

\begin{definition*}[Order]
  Let $n \in \mathbb{Z}$ and let $p$ be a prime. Then if $n$ is not zero, there
  is a nonnegative integer $a$ such that $p^a \mid n$ but $p^{a + 1} \nmid n$.
  The number $a$ is called the \emph{order} of $n$ at $p$ and is denoted by
  $\ord_p n$.
\end{definition*}

\begin{lemma}
  Every nonzero integer can be written as a product of primes.
\end{lemma}

\begin{theorem}[Unique Factorisation Theorem]
  For every nonzero integer $n$ there is a prime factorisation \[ n =
  (-1)^{\epsilon(n)} \prod_p p^{a(p)}, \] with the exponents uniquely
  determined by $n$. In fact, we have $a(p) = \ord_p n$.
\end{theorem}

We establish a few preliminary results to produce the proof of Theorem 1.

\begin{lemma}[Euclidean division]
  If $a, b \in \mathbb{Z}$, there exist $q, r \in \mathbb{Z}$ such that $a = qb
  + r$ with $0 \leq r < b$.
\end{lemma}

\begin{definition*}[Ideal]
  If $a_1, a_2, \ldots, a_n \in \mathbb{Z}$, we define $(a_1, a_2, \ldots,
  a_n)$ to be the set of all integers of the form $a_1x_1 + a_2x_2 + \ldots +
  a_nx_n$ with $x_1, x_2, \ldots, x_n \in \mathbb{Z}$.

  Let $A = (a_1, a_2, \ldots, a_n)$. Notice that the sum and difference of two
  elements in $A$ are again in $A$. Also, if $a \in A$ and $r \in \mathbb{Z}$,
  then $ra \in A$. In ring-theoretic language, $A$ is an \emph{ideal} in the
  ring $\mathbb{Z}$.
\end{definition*}

\begin{lemma}
  If $a, b \in \mathbb{Z}$, then there is a $d \in \mathbb{Z}$ such that
  $(a, b) = (d)$.
\end{lemma}

\begin{definition*}[Greatest Common Divisor]
  Let $a, b \in \mathbb{Z}$. An integer $d$ is called a \emph{greatest common
  divisor} of $a$ and $b$ if $d$ is a divisor of both $a$ and $b$ and if every
  other common divisor of $a$ and $b$ divides $d$.
\end{definition*}

\begin{lemma}
  Let $a, b \in \mathbb{Z}$. If $(a, b) = (d)$ then $d$ is a greatest common
  divisor of $a$ and $b$.
\end{lemma}

\begin{definition*}[Relatively prime]
  We say that two integers $a$ and $b$ are \emph{relatively prime} if the only
  common divisors are $\pm 1$, the units.
\end{definition*}
Since $(a, b) = (d)$ with $d$ the greatest common divisor of $a$ and $b$, it
will not be too confusing to use the symbol $(a, b)$ for both the set and the
greatest common divisor.

\begin{proposition}
  Suppose that $a \mid bc$ and that $(a, b) = 1$. Then $a \mid c$.
\end{proposition}

\begin{corollary}
  If $p$ is a prime and $p \mid bc$, then either $p \mid b$ or $p \mid c$.
\end{corollary}

\begin{corollary}
  Suppose that $p$ is a prime and that $a, b \in \mathbb{Z}$. Then $\ord_p ab =
  \ord_p a + \ord_p b$.
\end{corollary}

Now, we are in a position to prove the main theorem.

\begin{proof}
  Apply the function $\ord_q$ to both sides of the equation \[ n =
  (-1)^{\epsilon(n)} \prod_p p^{a(p)}, \] and use the property of $\ord_q$
  given by Corollary 2. The result is \[ \ord_q n = \epsilon(n)\ord_q(-1) +
  \sum_p a(p)\ord_q p. \]

  Now, from the definition of $\ord_q$ we have $\ord_q(-1) = 0$ and $\ord_q(p)
  = 0$ if $p \neq q$ and 1 if $p = q$. Thus the right-hand side collapses to
  the single term $a(q)$, i.e., $\ord_q n = a(q)$.
\end{proof}

\subsection{Unique Factorization in $k[x]$}

We consider the ring $k[x]$ of polynomials with coefficients in a field $k$. If
$f, g \in k[x]$, we say that $f$ divides $g$ if there is an $h \in k[x]$ such
that $g = hf$.

If $\deg f$ denotes the degree of $f$, we have $\deg fg = \deg f + \deg g$.
Also, remember that $\deg f = 0$ iff $f$ is a nonzero constant. It follows
that $f \mid g$ and $g | f$ iff $f = cg$, where c is a nonzero constant. It
also follows that the only polynomials that divide all the others are the
nonzero constants. These are the \emph{units} of $k[x]$.

\begin{definition*}[Irreduciblity]
  A nonconstant polynomial $p$ is said to be \emph{irreducible} if $q \mid p$
  implies that $q$ is either a constant or a constant times $p$.
\end{definition*}

\setcounter{lemma}{0}
\begin{lemma}
  Every nonconstant polynomial is the product of irreducible polynomials.
\end{lemma}

\begin{definition*}[Monic]
  A polynomial $f$ is called \emph{monic} if its leading coefficient is 1.
  Every polynomial (except zero) is a constant times a monic polynomial.
\end{definition*}

\begin{definition*}[Order]
  Let $p$ be a monic irreducible polynomial. We define $\ord_p f$ to be the
  integer $a$ defined by the property that $p^a \mid f$ but that $p^{a + 1}
  \mid f$. Such an integer must exist since the degree of the powers of $p$ get
  larger and larger. Notice that $\ord_p f = 0$ iff $p \nmid f$.
\end{definition*}

\begin{theorem}
  Let $f \in k[x]$. Then we can write \[ f = c\prod_p p^{a(p)}, \] where the
  product is over all monic irreducible polynomials and $c$ is a constant. The
  constant $c$ and the exponents $a(p)$ are uniquely determined by $f$; in
  fact, $a(p) = \ord_p f$.
\end{theorem}

The existence of such a product follows immediately from Lemma 1. As before,
the uniqueness is more difficult to prove.

\begin{lemma}[Euclidean division]
  Let $f, g \in k[x]$. If $g \neq 0$, there exist polynomials $h, r \in k[x]$
  such that $f = hg + r$, where either $r = 0$ or $r \neq 0$ and $\deg r < \deg
  g$.
\end{lemma}

\begin{definition*}
  If $f_1, f_2, \ldots, f_n \in k[x]$, then $(f_1, f_2, \ldots, f_n)$ is the
  set of all polynomials of the form $f_1h_1 + f_2h_2$
\end{definition*}

\begin{definition*}[Ideal]
  If $f_1, f_2, \ldots, f_n \in k[x]$, then $(f_1, f_2, \ldots, f_n)$ is the
  set of all polynomials of the form $f_1h_1 + f_2h_2 + \ldots + f_nh_n$, where
  $h_1, h_2, \ldots, h_n \in k[x]$.

  In ring-theoretic language $(f_1, f_2, \ldots, f_n)$ is the ideal generated
  by $f_1, f_2, \ldots, f_n$.
\end{definition*}

\begin{lemma}
  Given $f, g \in k[x]$ there is a $d \in k[x]$ such that $(f, g) = (d)$.
\end{lemma}

\begin{definition*}[Greatest Common Divisor]
  Let $f, g \in k[x]$. Then $d \in k[x]$ is said to be a \emph{greatest common
  divisor} of $f$ and $g$ if $d$ divides $f$ and $g$ and every common divisor
  of $f$ and $g$ divides $d$.
\end{definition*}

Notice that the greatest common divisor of two polynomials is determined up to
multiplication by a constant. If we require it to be monic, it is uniquely
determined.

\begin{lemma}
  Let $f, g \in k[x]$. By Lemma 3 there is a $d \in k[x]$ such that $(f, g) =
  (d)$. $d$ is a greatest common divisor of $f$ and $g$.
\end{lemma}

\begin{definition*}[Relatively prime]
  Two polynomials $f$ and $g$ are said to be \emph{relatively prime} if the
  only common divisors of $f$ and $g$ are constants. In other words, $(f, g) =
  (1)$.
\end{definition*}

\begin{proposition}
  If $f$ and $g$ are relatively prime and $f \mid gh$, then $f \mid h$.
\end{proposition}

\setcounter{corollary}{0}
\begin{corollary}
  If $p$ is an irreducible polynomial and $p \mid fg$, then $p \mid f$ or $p
  \mid g$.
\end{corollary}

\begin{corollary}
  If $p$ is a monic irreducible polynomial and $f, g \in k[x]$, we have
  $\ord_p fg = \ord_p f + \ord_p g$.
\end{corollary}

The proof of Theorem 2 is now easy.

\begin{proof}
  Apply the function $\ord_q$ to both sides of the relation \[ f =
  c\prod_p p^{a(p)}. \] We find that \[ \ord_q f = \ord_q c +
  \sum_p a(p)\ord_q p. \] Now, since $c$ is a constant $q \nmid c$ and
  $\ord_q c = 0$. Moreover, $\ord_q p = 0$ if $q \neq p$ and 1 if $q = p$. Thus
  the above relation yields $\ord_q f = a(q)$. It is clear that if the
  exponents are uniquely determined by $f$, then so is $c$. This completes the
  proof.
\end{proof}

\end{document}

