\chapter{Postulates of Quantum Mechanics}

\section{State space}

\begin{postulate}
  Associated to any isolated physical system is a complex vector space with
  inner product (that is, a Hilbert space) known as the \emph{state space} of
  the system. The system is completely described by its \emph{state vector},
  which is a unit vector in the system's state space.
\end{postulate}

The simplest quantum mechanical system, and the system which we will be most
concerned with, is the \emph{qubit}. A qubit has a two-dimensional state space.
Suppose $\ket{0}$ and $\ket{1}$ form an orthonormal basis for that state space.
Then an arbitrary state vector in the state space can be written
\begin{equation*}
  \ket{\psi} = a\ket{0} + b\ket{1},
\end{equation*} where $a$ and $b$ are complex numbers. The condition that
$\psi$ be a unit vector, $\ketbra{\psi}{\psi} = 1$, is therefore equivalent to
$|a|^2 + |b|^2 = 1$. The condition $\ketbra{\psi}{\psi} = 1$ is often known as
the \emph{normalization condition} for state vectors.

\paragraph{Superposition.} We say that any linear combination $\sum_i \alpha_i
\ket{\psi_i}$ is a \emph{superposition} of the states $\ket{\psi_i}$ with
\emph{amplitude} $\alpha_i$ for the state $\ket{\psi_i}$.

\section{Evolution}

\begin{postulate}
  The evolution of a \emph{closed} quantum system is described by a
  \emph{unitary transformation}. That is, the state $\ket{\psi}$ of the system
  at time $t_1$ is related to the state $\ket{\psi'}$ of the system at time
  $t_2$ by a unitary operator $U$ which depends only on the times $t_1$ and
  $t_2$, \begin{equation*}
    \ket{\psi'} = U\ket{\psi}.
  \end{equation*}

  The evolution of the state of a \emph{closed} quantum system is also
  described by the \emph{Schr{\"o}dinger equation}, \begin{equation}
    i\hbar\frac{\mathrm{d}\ket{\psi}}{\mathrm{d}t} = H\ket{\psi}.
      \label{eq:sch-eq}
  \end{equation}
\end{postulate}

In Schr{\"o}dinger's equation, $\hbar$ is a physical constant known
as \emph{Planck's constant} whose value must be experimentally determined. The
exact value is not important to us. In practice, it is common to absorb the
factor $\hbar$ into $H$, effectively setting $\hbar = 1$. $H$ is a fixed
Hermitian operator known as the \emph{Hamiltonian} of the closed system.

Because the Hamiltonian is a Hermitian operator it has a spectral decomposition
\begin{equation}
  H = \sum_E E\ketbra{E}{E}, \label{eq:ham-spec-dec}
\end{equation} with eigenvalues $E$ and corresponding normalized eigenvectors
$\ket{E}$. The states $\ket{E}$ are conventionally referred to as \emph{energy
eigenstates}, or sometimes as \emph{stationary states}, and $E$ is the
\emph{energy} of the state $\ket{E}$. The lowest energy is known as the
\emph{ground state energy} for the system, and the corresponding energy
eigenstate (or eigenspace) is known as the \emph{ground state}. The reason the
states $\ket{E}$ are sometimes known as stationary states is because their only
change in time is to acquire an overall numerical factor, \begin{equation*}
  \ket{E} \rightarrow \exp(-iEt/\hbar)\ket{E}.
\end{equation*}
The solution to Sch{\"o}dinger's equation will be verified later to be:
\begin{equation*} \ket{\psi(t_2)} = \exp\left[\frac{-iH(t_2 - t_1)}{\hbar}\right]
    \ket{\psi(t_1)} = U(t_1, t_2)\ket{\psi(t_1)},
\end{equation*} where we define \begin{equation}
  U(t_1, t_2) \equiv \exp\left[\frac{-iH(t_2 - t_1)}{\hbar}\right].
    \label{eq:unit-sch-eq}
\end{equation}

\paragraph{\cite{mikeandike} Exercise 2.54:} Suppose $A$ and $B$ are commuting
Hermitian operators. Prove that $\exp(A)\exp(B) = \exp(A + B)$.

\paragraph{Solution:} Since $A$ and $B$ are commuting Hermitian operators, they
are simultaneously diagonalizable. We write $A = \sum_i a_i\ketbra{i}{i}$,
$B = \sum_j b_j\ketbra{j}{j}$, noting that $A + B = \sum_i (a_i + b_i)
\ketbra{i}{i}$. Then, $\exp(A) = \sum_i e^{a_i}\ketbra{i}{i}$, $\exp(B) =
\sum_j e^{b_j}\ketbra{j}{j}$, $\exp(A + B) = \sum_i e^{a_i + b_i}
\ketbra{i}{i}$. It is clear that \begin{align*}
  \exp(A)\exp(B)
    &= \left(\sum_i e^{a_i}\ketbra{i}{i}\right)\left(\sum_j e^{b_j}
      \ketbra{j}{j}\right) \\
    &= \sum_{i, j} e^{a_i}e^{b_j}\ket{i}\delta_{ij}\bra{j} \\
    &= \sum_i e^{a_i}e^{b_i}\ketbra{i}{i} \\
    &= \sum_i e^{a_i + b_i}\ketbra{i}{i} = \exp(A + B),
\end{align*} which proves the result.

\paragraph{\cite{mikeandike} (modified) Exercise 2.55:} Prove that $U(t_1,
t_2)$ as defined in \eqref{eq:unit-sch-eq} is unitary.

\paragraph{Solution:} Consider $U(t_1, t_2) = \exp\left[\frac{-iH(t_2 - t_1)}
{\hbar}\right]$ and $U(t_1, t_2)^\dagger = \exp\left[\frac{iH^\dagger(t_2 -
t_1)}{\hbar}\right]$. Rewriting $U(t_1, t_2) = \exp(A)$ and $U(t_1,
t_2)^\dagger = \exp(B)$ where $A$ and $B$ are Hermitian operators, it is clear
that $A = -B$ and commute. Hence, $\exp(A)\exp(B) = \exp(A + B) = I$.

\paragraph{\cite{mikeandike} Exercise 2.56:} Use the spectral decomposition to
show that $K \equiv -i\log(U)$ is Hermitian for any unitary $U$, and thus $U =
\exp(iK)$ for some Hermitian $K$.

\paragraph{Solution:} Noting that all eigenvalues of $U$ are of modulus 1, one
could express $U = \sum_i e^{i\theta_i}\ketbra{i}{i}$ with the spectral
decomposition theorem, leading to $K = -i\log(U) = \sum_i \theta_i\ketbra{i}
{i}$. Clearly, $K = K^\dagger$, so $K$ is Hermitian.

\paragraph{}

These exercises show that $U(t_1, t_2)$ is unitary. There is therefore a
one-to-one correspondence between the discrete-time description of dynamics
using unitary operators, and the continuous time description using
Hamiltonians.

