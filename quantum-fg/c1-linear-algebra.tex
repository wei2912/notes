\chapter{Linear Algebra in Quantum Mechanics}

\section{Bases and linear independence}

\paragraph{Spanning set.} A \emph{spanning set} for a vector space is a set of
vectors $\ket{v_1}, \ldots, \ket{v_n}$ such that any vector $\ket{v}$ in the
vector space can be written as a linear combination $\ket{v} = \sum_i a_i
\ket{v_i}$ of vectors in that set. We say that the vectors $\ket{v_1}, \ldots,
\ket{v_n}$ \emph{span} the vector space.

\paragraph{Linear independence.} A set of vectors is \emph{linearly
independent} if there exists a set of complex numbers $a_1, \ldots, a_n$ with
$a_i \neq 0$ for at least one value of $i$, such that \begin{equation}
  a_1\ket{v_1} + a_2\ket{v_2} + \cdots + a_n\ket{v_n} = 0. \label{eq:lin-ind}
\end{equation} A set of vectors is \emph{linearly independent} if it is not
linearly dependent.

\paragraph{Basis.} It can be shown that any two sets of linearly independent
vectors which span a vector space $V$ contain the same number of elements. We
call such a set a \emph{basis} for $V$. Furthermore, such a basis set always
exists. The number of elements in the basis is defined to be the
\emph{dimension} of $V$.

\section{Linear operators and matrices}

\paragraph{Linear operator.} A \emph{linear operator} between vector spaces $V$
and $W$ is defined to be any function $A : V \rightarrow W$ which is linear in
its inputs, \begin{equation}
  A\left(\sum_i a_i\ket{v_i} \right) = \sum_i a_i A(\ket{v_i}).
    \label{eq:lin-op}
\end{equation} Usually we just write $A\ket{v}$ to denote $A(\ket{v})$. When we
say that a linear operator $A$ is defined \emph{on} a vector space $V$, we mean
that $A$ is a linear operator from $V$ to $V$.

An important linear operator on any vector space $V$ is the \emph{identity
operator}, $I_V$, defined by the equation $I_V\ket{v} \equiv \ket{v}$ for all
vectors $\ket{v}$. When no chance of confusion arises we drop the subscript $V$
and just write $I$ to denote the identity operator.

Another important linear operator is the \emph{zero operator}, which we denote
0. The zero operator maps all vectors to the zero vector, $0\ket{v} \equiv 0$.

\paragraph{Matrix representation of a linear operator.} Suppose $A : V
\rightarrow W$ is a linear operator between vector spaces $V$ and $W$. Suppose
$\ket{v_1}, \ldots, \ket{v_m}$ is a basis for $V$ and $\ket{w_1}, \ldots,
\ket{w_n}$ is a basis for $W$. Then for each $j$ in the range $1, \ldots, m$,
there exist complex numbers $A_{1j}$ through $A_{nj}$ such that
\begin{equation}
  A\ket{v_j} = \sum_i A_{ij}\ket{w_i}.
\end{equation} The matrix whose entries are the values $A_{ij}$ is said to form
a \emph{matrix representation} of the operator $A$.

To make sense of this formula, suppose we have an orthonormal basis where each
vector $\ket{v_i}$ is represented as a column vector with all entries equal to
zero except for the $i$th element, which is 1. In that case, the result of
operating on one particular basis vector $\ket{v_k}$ with $A$ is
\begin{align*}
  A\ket{v_k}
    &= \begin{bmatrix}
        \cdots & A_{1k} & \cdots \\
        \vdots & A_{2k} & \vdots \\
        \vdots & \vdots & \vdots \\
        \cdots & A_{nk} & \cdots
      \end{bmatrix} \begin{bmatrix}
        0 \\
        \vdots \\
        1 \\
        \vdots \\
        0
      \end{bmatrix} \\
    &= \begin{bmatrix}
        A_{1k} \\
        A_{2k} \\
        \vdots \\
        A_{nk}
      \end{bmatrix} \\
    &= A_{1k}\begin{bmatrix}1 \\ 0 \\ \vdots \\ 0\end{bmatrix}
      + A_{2k}\begin{bmatrix}0 \\ 1 \\ \vdots \\ 0\end{bmatrix}
      + \cdots
      + A_{nk}\begin{bmatrix}0 \\ 0 \\ \vdots \\ 1\end{bmatrix} \\
    &= \sum_{i=1}^n A_{ik}\ket{v_i}.
\end{align*}
Naturally, the matrix entries $A_{ij}$ in general depend on the basis.

\paragraph{\cite{mikeandike} Exercise 2.2: (Matrix representations: example)}
Suppose $V$ is a vector space with basis vector $\ket{0}$ and $\ket{1}$, and
$A$ is a linear operator from $V$ to $V$ such that $A\ket{0} = \ket{1}$ and
$A\ket{1} = \ket{0}$. Give a matrix representation for $A$, with respect to the
input basis $\ket{0}, \ket{1}$, and the output basis $\ket{0}, \ket{1}$. Find
input and output bases which give rise to a different matrix representation of
$A$.

\paragraph{Solution:} A matrix representation is $A = \begin{bmatrix}
  0 & 1 \\
  1 & 0
\end{bmatrix},$ which is also one of the Pauli matrices, $X$. With input basis
$\ket{0}, \ket{1}$, and output basis $\ket{1}, \ket{0}$ (such that $A\ket{0} =
\ket{0}, A\ket{1} = \ket{1}$), $A = \begin{bmatrix}
  1 & 0 \\
  0 & 1
\end{bmatrix},$ which is the identity matrix $I$.

\paragraph{\cite{mikeandike} Exercise 2.3: (Matrix representation for operator
products)} Suppose $A$ is a linear operator from vector space $V$ to vector
space $W$, and $B$ is a linear operator from vector space $W$ to vector space
$X$. Let $\ket{v_i}$, $\ket{w_j}$ and $\ket{x_k}$ be bases for the vector
spaces $V$, $W$, and $X$, respectively. Show that the matrix representation for
the linear transformation $BA$ is the matrix product of the matrix
representations for $B$ and $A$, with respect to the appropriate bases.

\paragraph{Solution:} Writing the linear operators $A$ and $B$ using the
definition \eqref{eq:lin-op}, \begin{equation*}
  \begin{split}
    A\ket{v_i} = \sum_j A_{ji}\ket{w_j}, \\
    B\ket{w_j} = \sum_k B_{kj}\ket{x_k}.
  \end{split}
\end{equation*} Thus, \begin{align*}
  BA\ket{v_i} &= B \sum_j A_{ji}\ket{w_j} \\
    &= \sum_j A_{ji} (B\ket{w_j}) \\
    &= \sum_j A_{ji} \left(\sum_k B_{kj} \ket{x_k}\right) \\
    &= \sum_k \left(\sum_j B_{kj}A_{ji}\right) \ket{x_k} 
\end{align*} which proves the statement.

\paragraph{\cite{mikeandike} Exercise 2.4: (Matrix representation for
identity)} Show that the identity operator on a vector space $V$ has a matrix
representation which is one along the diagonal and zero everywhere else, if the
matrix representation is taken with respect to the same input and output bases.
This matrix is known as the \emph{identity matrix}.

\paragraph{Solution:} From \eqref{eq:lin-op}, the identity matrix $I$ must
satisfy the property \begin{equation*}
  I\ket{v_j} = \sum_i I_{ij}\ket{v_j} = \ket{v_j}
\end{equation*} where $v_1, \ldots, v_n$ is a basis for $V$. Naturally, the
solution $I_{ij} = \delta_{ij}$ leads to the aforedescribed matrix. (It
suffices to show that a solution satisfies the definition of the linear
operator, as it is clear that linear transformations from a basis $V$ to $W$
are unique.)

\section{Inner products}

\paragraph{Inner product.} An \emph{inner product} is a function which takes as
input two vectors $\ket{v}$ and $\ket{w}$ from a vector space and produces a
complex number as output.

The standard quantum mechanical notation for the inner product $(\ket{v},
\ket{w})$ is $\braket{v}{w}$, where $\ket{v}$ and $\ket{w}$ are vectors in the
inner product space, and the notation $\bra{v}$ is used for the \emph{dual
vector} to the vector $\ket{v}$; the dual is a linear operator from the inner
product space $V$ to the complex numbers $\mathbb{C}$, defined by
$\bra{v}(\ket{w}) \equiv \braket{v}{w} \equiv (\ket{v}, \ket{w})$.

A function $(\cdot, \cdot)$ from $V \cross V$ to $\mathbb{C}$ is an inner
product if it satisfies the requirements that: \begin{enumerate}
  \item $(\cdot, \cdot)$ is linear in the second argument, \begin{equation}
      \left(\ket{v}, \sum_i \lambda_i\ket{w_i}\right) = \sum_i
      \lambda_i(\ket{v}, \ket{w_i}).
    \end{equation}
  \item $(\ket{v}, \ket{w}) = (\ket{w}, \ket{v})^*$
  \item $(\ket{v}, \ket{v}) \geq 0$ with equality if and only if $\ket{v} = 0$.
\end{enumerate}

For example, $\mathbb{C}^n$ has an inner product defined by \begin{equation}
  ((y_1, \ldots, y_n), (z_1, \ldots, z_n)) \equiv \sum_i y_i^*z_i =
    \begin{bmatrix}y_1^* & \dots & y_n^*\end{bmatrix}
    \begin{bmatrix}z_1 \\ \vdots \\ z_n\end{bmatrix}.
\end{equation}

We call a vector space equipped with an inner product an \emph{inner product
space}.\footnote{Discussions of quantum mechanics often refer to \emph{Hilbert
space}. In the finite dimensional complex vector spaces that come up in quantum
computation and quantum information, a Hilbert space is \emph{exactly the same
thing} as an inner product space. In infinite dimensions Hilbert spaces satisfy
additional technical restrictions above and beyond inner product spaces, which
we will not need to worry about.}

\paragraph{\cite{mikeandike} Exercise 2.6:} Show that any inner product
$(\cdot, \cdot)$ is \emph{conjugate-linear} in the first argument,
\begin{equation}
  \left(\sum_i \lambda_i\ket{w_i}, \ket{v}\right)
    = \sum_i \lambda_i^*(\ket{w_i}, \ket{v}). \label{eq:conj-lin}
\end{equation}

\paragraph{Solution:} From $(\ket{v}, \ket{w}) = (\ket{w}, \ket{v})^*$,
\begin{align*}
  \left(\sum_i \lambda_i\ket{w_i}, \ket{v}\right)
    &= \left(\ket{v}, \sum_i \lambda_i\ket{w_i}\right)^* \\
    &= \sum_i (\ket{v}, \lambda_i\ket{w_i})^* \\
    &= \sum_i \lambda_i^* (\ket{v}, \ket{w_i})^* \\
    &= \sum_i \lambda_i^*(\ket{w_i}, \ket{v}).
\end{align*}

\paragraph{Orthogonal.} Vectors $\ket{w}$ and $\ket{v}$ are \emph{orthogonal}
if their inner product is zero.

\paragraph{Norm.} We define the \emph{norm} of a vector $\ket{v}$ such that
$\|\ket{v}\| = 1$. We also say that $\ket{v}$ is \emph{normalized} if
$\|\ket{v}\| = 1$. It is convenient to talk of \emph{normalizing} a vector by
dividing by its norm; thus $\ket{v}/\|\ket{v}\|$ is the \emph{normalized} form
of $\ket{v}$, for any non-zero vector $\ket{v}$.

\paragraph{Orthonormal.} A set $\ket{i}$ of vectors with index $i$ is
\emph{orthonormal} if each vector is a unit vector, and distinct vectors in the
set are orthogonal, that is, $\braket{i}{j} = \delta_{ij}$, where $i$ and $j$
are both chosen from the index set.

\paragraph{Gram-Schmidt procedure.} Suppose $\ket{w_1}, \ldots, \ket{w_d}$ is a
basis set for some vector space $V$ with an inner product. There is a useful
method, the \emph{Gram-Schmidt} procedure, which can be used to produce an
orthonormal basis set $\ket{v_1}, \ldots, \ket{v_d}$ for the vector space $V$.
Define $\ket{v_1} \equiv \ket{w_1}/\|\ket{w_1}\|$, and for $1 \leq k \leq d -
1$ define $\ket{v_{k+1}}$ inductively by \begin{equation}
  \ket{v_{k+1}} \equiv \frac{\ket{w_{k+1}} - \sum_{i=1}^k \braket{v_i}{w_{k+1}}
    \ket{v_i}}{\|\ket{w_{k+1}} - \sum_{i=1}^k \braket{v_i}{w_{k+1}}
    \ket{v_i}\|}. \label{eq:gs-proc}
\end{equation}

\paragraph{\cite{mikeandike} Exercise 2.8:} Prove that the Gram-Schmidt
procedure produces an orthonormal basis for V.

\paragraph{Solution:} The base case for a 1-dimensional vector space $V$ is
trivial, as the orthonormal basis set must contain only $\ket{v_1} \equiv
\ket{w_1}/\|\ket{w_1}\|$ as defined in the Gram-Schmidt procedure.

Consider a (k+1)-dimensional vector space $V$ with basis set $\ket{w_1},
\ldots, \ket{w_{k+1}}$. Suppose an orthonormal set $\ket{v_1}, \ldots,
\ket{v_k}$ has been derived from the set $\ket{w_1}, \ldots, \ket{w_k}$ using
the Gram-Schmidt procedure. Then, it is clear that $\braket{v_j}{v_k} =
\delta_{jk}$. The vector $\ket{v_{k+1}}$ as defined by \eqref{eq:gs-proc} can
be shown to be orthogonal to $\ket{v_j}$ for $1 \leq j \leq k$ by working from
the inductive definition: \begin{align*}
  \braket{v_j}{v_{k+1}}
    &= \frac{\bra{v_j} \left(\ket{w_{k+1}} - \sum_{i=1}^k
      \braket{v_i}{w_{k+1}}\ket{v_i}\right)}{\|\ket{w_{k+1}} - \sum_{i=1}^k
      \braket{v_i}{w_{k+1}}\ket{v_i}\|} \\
    &= \frac{\braket{v_j}{w_{k+1}} - \sum_{i=1}^k \braket{v_i}{w_{k+1}}
      \delta_{ji}}{\|\ket{w_{k+1}} - \sum_{i=1}^k \braket{v_i}{w_{k+1}}
      \ket{v_i}\|} \\
    &= \frac{\braket{v_j}{w_{k+1}} - \braket{v_j}{w_{k+1}}}{\|\ket{w_{k+1}} -
      \sum_{i=1}^k \braket{v_i}{w_{k+1}}\ket{v_i}\|} \\
    &= 0.
\end{align*} It is also clear that $\ket{v_{k+1}}$ as defined is already
normalized. By induction, the correctness of the Gram-Schmidt procedure is
proven.

\paragraph{Outer product.} Suppose $\ket{v}$ is a vector in an inner product
space $V$, and $\ket{w}$ is a vector in an inner product space $W$. Define
$\ketbra{w}{v}$ to be the \emph{outer product}, or the linear operator from $V$
to $W$ whose action is defined by \begin{equation}
  (\ketbra{w}{v})(\ket{v'}) \equiv \ket{w}\braket{v}{v'} = \braket{v}{v'}
    \ket{w}. \label{eq:out-prod}
\end{equation} We can take linear combinations of outer product operators
$\ketbra{w}{v}$ in the obvious way. By definition $\sum_i a_i
\ketbra{w_i}{v_i}$ is the linear operator, which, when acting on $\ket{v'}$,
produces $\sum_i a_i\ket{w_i} \braket{v_i}{v'}$ as output.

\paragraph{Completness relation.} Let $\ket{i}$ be any orthonormal basis for
the vector space $V$, so an arbitrary vector $\ket{v}$ can be written $\ket{v}
= \sum_i v_i\ket{i}$ for some set of complex numbers $v_i$. Note that
$\braket{i}{v} = v_i$ and therefore \begin{equation*}
  \left(\sum_i \ketbra{i}{i}\right)\ket{v} = \sum_i \ket{i}\braket{i}{v} =
    \sum_i v_i\ket{i} = \ket{v}.
\end{equation*} Since the last equation is true for all $\ket{v}$ it follows
that \begin{equation} \label{eq:comp-rel}
  \sum_i \ketbra{i}{i} = I.
\end{equation} This equation is known as the \emph{completeness relation}.

One application of the completeness relation is to give a means for
representing any operator in the outer product notation. Suppose $A : V
\rightarrow W$ is a linear operator, $\ket{v_i}$ is an orthonormal basis for
$V$, and $\ket{w_j}$ an orthonormal basis for $W$. Using the completeness
relation twice we obtain \begin{align}
  A &= I_WAI_V \nonumber \\
    &= \sum_{ij} \ket{w_j}\mel{w_j}{A}{v_i}\bra{v_i} \nonumber \\
    &= \sum_{ij} \mel{w_j}{A}{v_i}\ketbra{w_j}{v_i}, \label{eq:out-prod-rep}
\end{align} which is the outer product representation for $A$. We also see from
this equation that $A$ has matrix element $\mel{w_j}{A}{v_i}$ in the $i$th
column and $j$th row, with respect to the input basis $\ket{v_i}$ and output
basis $\ket{w_j}$.

\begin{theorem}[Cauchy-Schwarz inequality]
  For any two vectors $\ket{v}$ and $\ket{w}$, \begin{equation}
    |\braket{v}{w}|^2 \leq \braket{v}{v}\braket{w}{w}. \label{eq:cau-sch-ine}
  \end{equation}
\end{theorem}
\begin{proof}
  To see this, use the Gram-Schmidt procedure to construct an orthonormal basis
  $\ket{i}$ for this vector space such that the first member of the basis
  $\ket{i}$ is $\ket{w}/\sqrt{\braket{w}{w}}$. Using the completeness relation
  $\sum_i \ketbra{i}{i} = I$, and dropping some non-negative terms gives
  \begin{align*}
    \braket{v}{v}\braket{w}{w}
      &= \sum_i \braket{v}{i}\braket{i}{v} \braket{w}{w} \\
      &\geq \frac{\braket{v}{w}\braket{w}{v}}{\braket{w}{w}} \braket{w}{w} \\
      &= \braket{v}{w}\braket{w}{v} = \|\braket{v}{w}\|^2,
  \end{align*} as required. A little thought shows that equality occurs if and
  only if $\ket{v}$ and $\ket{w}$ are linearly related, $\ket{v} = z\ket{w}$ or
  $\ket{w} = z\ket{v}$, for some scalar $z$.
\end{proof}

\section{Eigenvectors and eigenvalues}

\paragraph{Eigenvector and eigenvalue.} An \emph{eigenvector} of a linear
operator $A$ on a vector space is a non-zero vector $\ket{v}$ such that
$A\ket{v} = v\ket{v}$, where $v$ is a complex number known as the
\emph{eigenvalue} of $A$ corresponding to $\ket{v}$.

\paragraph{Characteristic function.} The \emph{characteristic function} is
defined to be $c(\lambda) \equiv \text{det} |A - \lambda I|$, where
$\text{det}$ is the \emph{determinant} function for matrices; it can be shown
that the characteristic function depends only upon the operator $A$, and not on
the specific matrix representation used for $A$.

The solutions of the \emph{characteristic equation} $c(\lambda) = 0$ are the
eigenvalues of the operator $A$. By the fundamental theorem of algebra, every
polynomial has at least one complex root, so every operator $A$ has at least
one eigenvalue, and a corresponding eigenvector.

\paragraph{Eigenspace.} The \emph{eigenspace} corresponding to an eigenvalue
$v$ is the set of vectors which have eigenvalue $v$. It is a vector subspace of
the vector space on which $A$ acts.

\paragraph{Diagonal representation.} A \emph{diagonal representation} for an
operator $A$ on a vector space $V$ is a representation $A = \sum_i \lambda_i
\ketbra{i}{i}$, where the vectors $\ket{i}$ form an orthonormal set of
eigenvectors for $A$, with corresponding eigenvalues $\lambda_i$. An operator
is said to be \emph{diagonalizable} if it has a diagonal representation. In the
next section we will find a simple set of necessary and sufficient conditions
for an operator on a Hilbert space to be diagonalizable.

\paragraph{Degenerate.} When an eigenspace is more than one dimensional we say
that it is \emph{degenerate}. This occurs when there are multiple linearly
independent eigenvectors with the same eigenvalue.

\section{Adjoints and Hermitian operators}

\paragraph{Adjoint/Hermitian conjugate.} Suppose $A$ is any linear operator on
a Hilbert space, $V$. It turns out that there exists a unique linear operator
$A^\dagger$ on $V$ such that for all vectors $\ket{v}, \ket{w} \in V$,
\begin{equation}
  (\ket{v}, A\ket{w}) = (A^\dagger \ket{v}, \ket{w}). \label{eq:adj}
\end{equation}
This linear operator is known as the \emph{adjoint} or \emph{Hermitian
conjugate} of the operator $A$. From the definition it is easy to see that
$(AB)^\dagger = B^\dagger A^\dagger$. By convention, if $\ket{v}$ is a vector,
then we define $\ket{v}^\dagger \equiv \bra{v}$. With this definition it is not
difficult to see that $(A\ket{v})^\dagger = \bra{v}A^\dagger$.

In a matrix representation of the operator $A$, the action of the Hermitian
conjugation operation is to take the matrix of $A$ to the conjugate-transpose
matrix, $A^\dagger \equiv (A^*)^T$, where the $*$ indicates complex
conjugation, and $T$ indicates the transpose operation.

\paragraph{\cite{mikeandike} Exercise 2.14: (Anti-linearity of the adjoint)}
Show that the adjoint operation is anti-linear, \begin{equation}
  \left(\sum_i a_iA_i\right)^\dagger = \sum_i a_i^* A_i^\dagger.
    \label{eq:adj-anti-lin}
\end{equation}

\paragraph{Solution:} Let $A = \sum_i a_iA_i$. For all vectors $\ket{v},
\ket{w} \in V$, \begin{align*}
  (A^\dagger \ket{v}, \ket{w}) &= (\ket{v}, A\ket{w}) \\
    &= \sum_i a_i(\ket{v}, A_i\ket{w}) \\
    &= \sum_i a_i(A_i\ket{w}, \ket{v})^* \\
    &= \sum_i (A_i\ket{w}, a_i^*\ket{v})^* \\
    &= \sum_i (a_i^*\ket{v}, A_i\ket{w}) \\
    &= \sum_i (a_i^* A_i^\dagger \ket{v}, \ket{w}) \\
    &= \left(\left(\sum_i a_i^* A_i^\dagger\right)\ket{v}, \ket{w}\right).
\end{align*}

\paragraph{Hermitian/self-adjoint operators.} An operator $A$ whose adjoint is
$A$ is known as a \emph{Hermitian} or \emph{self-adjoint} operator.

\paragraph{Projectors.} An important class of Hermitian operators is the
\emph{projectors}. Suppose $W$ is a $k$-dimensional vector subspace of the
$d$-dimensional vector space $V$. Using the Gram-Schmidt procedure it is
possible to construct an orthonormal basis $\ket{1}, \ldots, \ket{d}$ for $V$
such that $\ket{1}, \ldots, \ket{k}$ is an orthonormal basis for $W$. By
definition, \begin{equation} \label{eq:proj}
  P \equiv \sum_{i=1}^k \ketbra{i}{i}
\end{equation} is the \emph{projector} onto the subspace $W$. It is easy to
check that this definition is independent of the orthonormal basis $\ket{1},
\ldots, \ket{k}$ used for $W$. From the definition it can be shown that
$\ketbra{v}{v}$ is Hermitian for any vector $\ket{v}$, so $P$ is Hermitian,
$P^\dagger = P$. We will often refer to the 'vector space' P, as shorthand for
the vector space onto which $P$ is a projector.

\paragraph{Orthogonal complement.} The \emph{orthogonal complement} of the
projector $P$ defined above is the operator $Q \equiv I - P$. It is easy to see
that $Q$ is a projector onto the vector space spanned by $\ket{v+1}, \ldots,
\ket{d}$, which we also refer to as the \emph{orthogonal complement} of $P$,
and may denote by $Q$.

\paragraph{\cite{mikeandike} Exercise 2.16:} Show that any projector $P$
satisfies the equation $P^2 = P$.

\paragraph{Solution:} From the definition in \eqref{eq:proj}, \begin{align*}
  P^2 &= \left(\sum_{i=1}^k \ketbra{i}{i}\right) \left(\sum_{i=1}^k
      \ketbra{i}{i}\right) \\
    &= \left(\sum_{i=1}^k \ket{i}\braket{i}{i}\bra{i}\right) + \left(\sum_{i, j
      \leq k, i \neq j} \ket{i}\braket{i}{j}\bra{j}\right) \\
    &= \left(\sum_{i=1}^k \ket{i} \|i\|^2 \bra{i}\right) + 0 \\
    &= \sum_{i=1}^k \ketbra{i}{i} \\
    &= P.
\end{align*}

\paragraph{Normal.} An operator $A$ is said to be \emph{normal} if $A A^\dagger
= A^\dagger A$.

Clearly, an operator which is Hermitian is also normal.

\begin{theorem}[Spectral decomposition]
  Any normal operator $M$ on a vector space $V$ is diagonal with respect to
  some orthonormal basis for $V$. Conversely, any diagonalizable operator is
  normal.
\end{theorem}
\begin{proof}
  The converse is a simple exercise, so we prove merely the forward
  implication, by induction on the dimension $d$ of $V$. The case $d = 1$ is
  trivial. Let $\lambda$ be an eigenvalue of $M$, $P$ the projector onto the
  $\lambda$ eigenspace, and $Q$ the projector onto the orthogonal complement.
  Then $M = (P + Q)M(P + Q) = PMP + QMP + PMQ + QMQ$. As $MP\ket{v} = \lambda P
  \ket{v}$ for any vector $\ket{v}$, $PMP = P(\lambda P) = \lambda P^2 =
  \lambda P$. Furthermore, $QMP = (I - P)MP = MP - PMP = \lambda P - \lambda P
  = 0$.

  We claim that $PMQ = 0$ also. To see this, let $\ket{v}$ be an element of the
  subspace $P$. Then $M M^\dagger \ket{v} = M^\dagger M \ket{v} = \lambda
  M^\dagger \ket{v}$. Thus, $M^\dagger \ket{v}$ has eigenvalue $\lambda$ and
  therefore is an element of the subspace $P$. It follows that $Q M^\dagger P =
  0$. Taking the adjoint of this equation gives $PMQ = 0$, as the projectors
  are Hermitian i.e. $P = P^\dagger$ and $Q = Q^\dagger$. Thus $M = PMP + QMQ$.

  Next, we prove that $QMQ$ is normal. To see this, note that $QM = QM(P + Q) =
  QMQ$, and $Q M^\dagger = Q M^\dagger (P+Q) = Q M^\dagger Q$. Therefore, by
  the normality of $M$, and the observation that $Q^2 = Q$, \begin{align*}
    Q M Q Q M^\dagger Q &= Q M Q M^\dagger Q \\
      &= Q M M^\dagger Q \\
      &= Q M^\dagger M Q \\
      &= Q M^\dagger Q M Q \\
      &= Q M^\dagger Q Q M Q,
  \end{align*} so $QMQ$ is normal.

  By induction, $QMQ$ is diagonal with respect to some orthonormal basis for
  the subspace $Q$, and $PMP$ is already diagonal with respect to some
  orthonormal basis for the subspace $P$. It follows that $M = PMP + QMQ$ is
  diagonal with respect to some orthonormal basis for the total vector space.
\end{proof}

In terms of the outer product representation, this means that $M$ can be
written as $M = \sum_i \lambda_i \ketbra{i}{i}$, where $\lambda_i$ are the
eigenvalues of $M$, $\ket{i}$ is an orthonormal basis for $V$, and each
$\ket{i}$ an eigenvector of $M$ with eigenvalue $\lambda_i$.

In terms of projectors, $M = \sum_i \lambda_i P_i$ where $\lambda_i$ are again
the eigenvalues of $M$, and $P_i$ is the projector onto the $\lambda_i$
eigenspace of $M$. These projectors satisfy the completeness relation $\sum_i
P_i = I$, and the orthonormality relation $P_iP_j = \delta_{ij}P_i$.

\paragraph{Exercise 2.17:} Show that a normal matrix is Hermitian if and only
if it has real eigenvalues.

\paragraph{Solution:} Consider a normal matrix $M = \sum_i \lambda_i
\ketbra{i}{i}$ where the vectors $\ket{i}$ form an orthonormal set of
eigenvectors for $M$, with corresponding eigenvalues $\lambda_i$. From the
anti-linearity of the adjoint shown in \eqref{eq:adj-anti-lin},
\begin{equation*}
  M^\dagger = \left(\sum_i \lambda_i \ketbra{i}{i}\right)^\dagger = \sum_i
    \lambda_i^* (\ketbra{i}{i})^\dagger = \sum_i \lambda_i^* \ketbra{i}{i},
\end{equation*} which is equal to $M$ if and only if $\lambda_i = \lambda_i^*$
for all $i$.

\paragraph{Unitary.} A matrix $U$ is said to be \emph{unitary} if $U^\dagger U
= I$. Similarly an operator $U$ is unitary if $U^\dagger U = I$. It is easily
checked that an operator is unitary if and only if each of its matrix
representations is unitary. A unitary operator also satisfies $U U^\dagger =
I$, and therefore $U$ is normal and has a spectral decomposition.

Geometrically, unitary operators are important because they preserve inner
products between vectors. To see this, let $\ket{v}$ and $\ket{w}$ be any two
vectors. Then the inner product of $U\ket{v}$ and $U\ket{w}$ is the same as the
inner product of $\ket{v}$ and $\ket{w}$, \begin{equation*}
  (U\ket{v}, U\ket{w}) = \mel{v}{U^\dagger U}{w} = \mel{v}{I}{w} =
    \braket{v}{w}.
\end{equation*} This result suggests the following elegant outer product
representation of any unitary $U$. Let $\ket{v_i}$ be any orthonormal basis
set. Define $\ket{w_i} \equiv U\ket{v_i}$, so $\ket{w_i}$ is also an
orthonormal basis set, since unitary operators preserve inner products. Note
that $U = UI = U\sum_i \ketbra{v_i}{v_i} = \sum_i (U\ket{v_i})\bra{v_i} =
\sum_i \ketbra{w_i}{v_i}$. Conversely, if $\ket{v_i}$ and $\ket{w_i}$ are any
two orthonormal bases, then it is easily checked that the operator $U$ defined
by $U \equiv \sum_i \ketbra{w_i}{v_i}$ is a unitary operator.

\paragraph{\cite{mikeandike} Exercise 2.18:} Show that all eigenvalues of a
unitary matrix have modulus 1, that is, can be written in the form
$e^{i\theta}$ for some real $\theta$.

\paragraph{Solution:} By spectral decomposition, $U$ is diagonalizable and can
be expressed as $U = \sum_i \lambda_i \ketbra{i}{i}$. Observing that
anti-linearity of the adjoint holds by \eqref{eq:adj-anti-lin}, \begin{align*}
  U^\dagger U
    &= \left(\sum_i \lambda_i \ketbra{i}{i}\right)^\dagger \left(\sum_j
      \lambda_j \ketbra{j}{j}\right) \\
    &= \left(\sum_i \lambda_i^* (\ketbra{i}{i})^\dagger\right) \left(\sum_j
      \lambda_j \ketbra{j}{j}\right) \\
    &= \left(\sum_i \lambda_i^*\ketbra{i}{i}\right) \left(\sum_j \lambda_j
      \ketbra{j}{j}\right) \\
    &= \sum_{i,j} \lambda_i^*\lambda_j\ket{i}\braket{i}{j}\bra{j} \\
    &= \sum_{i,j} \lambda_i^*\lambda_j\ket{i}\delta_{ij}\bra{j} \\
    &= \sum_i \lambda_i^*\lambda_i\ketbra{i}{i}
\end{align*}
The eigenvalues of $U^\dagger U = I$ are $\lambda_i^*\lambda_i = 1$ which leads
to the desired result.

\paragraph{\cite{mikeandike} Exercise 2.20: (Basis changes)} Suppose $A'$ and
$A''$ are matrix representations of an operator $A$ on a vector space $V$ with
respect to two different orthonormal bases, $\ket{v_i}$ and $\ket{w_i}$. Then
the elements of $A'$ and $A''$ are $A'_{ij} = \mel{v_i}{A}{v_j}$ and $A''_{ij}
= \mel{w_i}{A}{w_j}$. Characterize the relationship between $A'$ and $A''$.

\paragraph{Solution:} Noting that one could define a unitary matrix $U = \sum_i
\ketbra{w_i}{v_i}$ with $U^\dagger = \sum_i \ketbra{v_i}{w_i}$, \begin{align*}
  A'_{ij} &= \mel{v_i}{A}{v_j} \\
    &= \mel{v_i}{U U^\dagger A U U^\dagger}{v_j} \\
    &= \sum_{p,q,r,s} \braket{v_i}{w_p} \braket{v_p}{v_q} \mel{w_q}{A}{w_r}
      \braket{v_r}{v_s} \braket{w_s}{v_j} \\
    &= \sum_{p,q,r,s} \braket{v_i}{w_p} \delta_{pq} \mel{w_q}{A}{w_r}
      \delta_{rs} \braket{w_s}{v_j} \\
    &= \sum_{p,r} \braket{v_i}{w_p} A''_{pr} \braket{w_r}{v_j} \\
\end{align*}

\paragraph{Positive operators.} A \emph{positive operator} $A$ is defined to be
an operator such that for any vector $\ket{v}$, $(\ket{v}, A\ket{v})$ is a
real, non-negative number. If $(\ket{v}, A\ket{v})$ is \emph{strictly} greater
than zero for all $\ket{v} \neq 0$ then we say that $A$ is \emph{positive
definite}.

In Exercise 2.24 you will show that any positive operator is automatically
Hermitian, and therefore by the spectral decomposition has diagonal
representation $\sum_i \lambda_i \ketbra{i}{i}$, with non-negative eigenvalues
$\lambda_i$.

\paragraph{\cite{mikeandike} Exercise 2.22:} Prove that two eigenvectors of a
Hermitian operator with different eigenvalues are necessarily orthogonal.

\paragraph{Solution:} Suppose $A$ is a Hermitian operator with eigenvectors
$\ket{v_i}, \ket{v_j}$ corresponding to the eigenvalues $\lambda_i, \lambda_j$.
Note that $\mel{v_i}{A}{v_j} = (\ket{v_i}, A\ket{v_j}) = (A^\dagger \ket{v_i},
\ket{v_j})$. Then, \begin{equation*}
  (\ket{v_i}, A\ket{v_j}) = (\ket{v_i}, \lambda_j\ket{v_j}) = \lambda_j
    (\ket{v_i}, \ket{v_j}).
\end{equation*}
On the other hand, noting that $A = A^\dagger$ and that from Exercise 2.17,
Hermitian matrices only have real eigenvalues, \begin{align*}
  (A^\dagger \ket{v_i}, \ket{v_j}) &= (A\ket{v_i}, \ket{v_j}) \\
    &= (\ket{v_j}, A\ket{v_i})^* \\
    &= (\ket{v_j}, \lambda_i\ket{v_i})^* \\
    &= \lambda_i^*(\ket{v_j}, \ket{v_i})^* \\
    &= \lambda_i(\ket{v_i}, \ket{v_j})
\end{align*}
Since $\lambda_i \neq \lambda_j$, $(\ket{v_i}, \ket{v_j}) = 0$ must hold for
the expressions of $\mel{v_i}{A}{v_j}$ to be consistent.

\paragraph{\cite{mikeandike} Exercise 2.24: (Hermiticity of positive
operators)} Show that a positive operator is necessarily Hermitian.
(\emph{Hint}: Show that an arbitrary operator $A$ can be written $A = B + iC$
where $B$ and $C$ are Hermitian.)

\paragraph{Solution:} Suppose $A$ is a positive operator. Then, $A$ can be
decomposed into $A = \frac{A + A^\dagger}{2} + i\frac{A - A^\dagger}{2i} = B +
iC$ where $B$ and $C$ are Hermitian.

For any vector $\ket{v}$, $\mel{v}{A}{v} = \mel{v}{B + iC}{v} = \mel{v}{B}{v} +
i\mel{v}{C}{v}$. Since $B$ and $C$ are Hermitian, by Exercise 2.17,
$\mel{v}{B}{v}, \mel{v}{C}{v} \in \mathbb{R}$. However, as $\mel{v}{A}{v} \in
\mathbb{R}$ by definition, $\mel{v}{C}{v} = 0$ for all $\ket{v}$. Hence,
$\mel{v}{A}{v} = \mel{v}{B}{v}$ which implies that $A = A^\dagger$.

\section{Tensor products}

\paragraph{Tensor product.} Suppose $V$ and $W$ are vector spaces of dimension
$m$ and $n$ respectively. For convenience we also suppose that $V$ and $W$ are
Hilbert spaces. Then $V \otimes W$ is an $mn$ dimensional vector space. The
elements of $V \otimes W$ are linear combinations of 'tensor products' $\ket{v}
\otimes \ket{w}$ of elements $\ket{v}$ of $V$ and $\ket{w}$ of $W$. In
particular, if $\ket{i}$ and $\ket{j}$ are orthonormal bases for the spaces $V$
and $W$ then $\ket{i} \otimes \ket{j}$ is a basis for $V \otimes W$. We often
use the abbreviated notations $\ket{v}\ket{w}$, $\ket{v, w}$ or even $\ket{vw}$
for the tensor product $\ket{v} \otimes \ket{w}$.

By definition the tensor product satisfies the following basic properties:
\begin{enumerate}
  \item For an arbitrary scalar $z$ and elements $\ket{v}$ of $V$ and $\ket{w}$
    of $W$, \begin{equation*}
      z(\ket{v} \otimes \ket{w}) = (z\ket{v}) \otimes \ket{w} = \ket{v} \otimes
        (z\ket{w}).
  \end{equation*}
  \item For arbitrary $\ket{v_1}$ and $\ket{v_2}$ in $V$ and $\ket{w}$ in $W$,
    \begin{equation*}
      (\ket{v_1} + \ket{v_2}) \otimes \ket{w} = \ket{v_1} \otimes \ket{w} +
        \ket{v_2} \otimes \ket{w}.
    \end{equation*}
  \item For arbitrary $\ket{v}$ in $V$ and $\ket{w_1}$ and $\ket{w_2}$ in $W$,
    \begin{equation*}
      \ket{v} \otimes (\ket{w_1} + \ket{w_2}) = \ket{v} \otimes \ket{w_1} +
        \ket{v} \otimes \ket{w_2}.
    \end{equation*}
\end{enumerate}

Suppose $\ket{v}$ and $\ket{w}$ are vectors in $V$ and $W$, and $A$ and $B$ are
linear operators on $V$ and $W$, respectively. Then we can define a linear
operator $A \otimes B$ on $V \otimes W$ by the equation \begin{equation*}
  (A \otimes B)(\ket{v} \otimes \ket{w}) \equiv A\ket{v} \otimes B\ket{w}.
\end{equation*}
This can be extended to all elements of $V \otimes W$ inthe natural way to
ensure linearity of $A \otimes B$, that is, \begin{equation*}
  (A \otimes B)\left(\sum_i a_i\ket{v_i} \otimes \ket{w_i}\right)
    \equiv \sum_i a_iA\ket{v_i} \otimes B\ket{w_i}.
\end{equation*}

It can be shown that $A \otimes B$ defined in this way is a well-defined linear
operator on $V \otimes W$. This notion of the tensor product of two operators
extends in the obvious way to the case where $A: V \rightarrow V'$ and $B: W
\rightarrow W'$ map between different vector spaces. Indeed, an arbitrary
linear operator $C$ mapping $V \otimes W$ to $V' \otimes W'$ can be represented
as a linear combination of tensor products of operators mapping $V$ to $V'$ and
$W$ to $W'$, \begin{equation*}
  C = \sum_i c_iA_i \otimes B_i,
\end{equation*}
where by definition \begin{equation*}
  \left(\sum_i c_iA_i \otimes B_i\right) \ket{v} \otimes \ket{w}
    \equiv \sum_i c_iA_i\ket{v} \otimes B_i\ket{w}.
\end{equation*}
The inner products on the spaces $V$ and $W$ can be used to define a natural
inner product on $V \otimes W$. Define \begin{equation*}
  \left(\sum_i a_i\ket{v_i} \otimes \ket{w_i}, \sum_j b_j\ket{v'_j} \otimes
    \ket{w'_j}\right) \equiv \sum_{i,j} a_i^*b_j \braket{v_i}{v'_j}
    \braket{w_i}{w'_j}.
\end{equation*}
It can be shown that the function so defined is a well-defined inner product.
From this inner product, the inner product space $V \otimes W$ inherits the
other structure we are familiar with, such as notions of an adjoint, unitarity,
normality, and Hermiticity.

\paragraph{Kronecker product.} Suppose $A$ is an $m$ by $n$ matrix, and $B$ is
a $p$ by $q$ matrix. Then we have the matrix representation: \begin{equation}
  A \otimes B \equiv \begin{bmatrix}
    A_{11}B & A_{12}B & \cdots & A_{1n}B \\
    A_{21}B & A_{22}B & \cdots & A_{2n}B \\
    \vdots & \vdots & \vdots & \vdots \\
    A_{m1}B & A_{m2}B & \cdots & A_{mn}B
  \end{bmatrix} \label{eq:kro-prod}
\end{equation}
In this representation terms like $A_{11}B$ denote $p$ by $q$ submatrices whose
entries are proportional to $B$, with overall proportionality constant $A_{11}$
.

We mention the useful notation $\ket{\phi}^{\otimes k}$, which means
$\ket{\phi}$ tensored with itself $k$ times. Ananalogous notation is also used
for operations on tensor product spaces.

\paragraph{\cite{mikeandike} Exercise 2.31:} Show that the tensor product of
two positive operators is positive.

\paragraph{Solution:} Consider the positive operators $A$ and $B$, and observe
that \begin{equation*}
  (\bra{\psi} \otimes \bra{\phi})(A \otimes B)(\ket{\psi} \otimes \ket{\phi})
    = \mel{\psi}{A}{\psi} \otimes \mel{\phi}{B}{\phi}
\end{equation*} by linearity.

Since $A$ and $B$ are positive operators, it is clear that $\mel{\psi}{A}{\psi}
\geq 0$ and $\mel{\phi}{B}{\phi} \geq 0$, and therefore $\mel{\psi}{A}{\psi}
\mel{\phi}{B}{\phi} \geq 0$. Hence, $A \otimes B$ is a positive operator.

\section{Operator functions}

Generally speaking, given a function $f$ from the complex numbers to the
complex numbers, it is possible to define a corresponding matrix function on
normal matrices (or some subclass, such as the Hermitian matrices) by the
following construction. Let $A = \sum_a a\ketbra{a}{a}$ be a spectral
decomposition for a normal operator $A$. Define $f(A) = \sum_a f(a)
\ketbra{a}{a}$. A little thought shows that $f(A)$ is uniquely defined. This
procedure can be used, for example, to define the square root of a positive
operator, the logarithm of a positive-definite operator, or the exponential
of a normal operator.

\paragraph{\cite{mikeandike} Exercise 2.34:} Find the square root and the
logarithm of the matrix \begin{equation*}
  \begin{bmatrix}
    4 & 3 \\
    3 & 4
  \end{bmatrix}.
\end{equation*}

\paragraph{Solution:} Let the above mentioned matrix be $M$.

As $M$ has eigenvalues 7 and 1 corresponding to the eigenvectors
\begin{equation*}
  \ket{+} = \frac{1}{\sqrt{2}}\begin{bmatrix}1 \\ 1\end{bmatrix},
  \ket{-} = \frac{1}{\sqrt{2}}\begin{bmatrix}1 \\ -1\end{bmatrix}
\end{equation*} which form an orthonormal basis, by the spectral decomposition
theorem, $M = 7\ketbra{+}{+} + \ketbra{-}{-}$. Therefore, \begin{align*}
  \sqrt{M} &= \sqrt{7}\ketbra{+}{+} + \sqrt{1}\ketbra{-}{-} \\
    &= \frac{1}{2}\begin{bmatrix}
      \sqrt{7} + 1 & \sqrt{7} - 1 \\
      \sqrt{7} - 1 & \sqrt{7} + 1
    \end{bmatrix}, \\
  \log{M} &= (\log{7})\ketbra{+}{+} + (\log{1})\ketbra{-}{-} \\
    &= \frac{\log{7}}{2}\begin{bmatrix}
      1 & 1 \\
      1 & 1
    \end{bmatrix}.
\end{align*}

\paragraph{\cite{mikeandike} Exercise 2.35: (Exponential of the Pauli matrices)
} Let $\vec{v}$ be any real, three-dimensional unit vector and $\theta$ a real
number. Prove that \begin{equation*}
  \mathrm{exp}(i \theta \vec{v} \cdot \vec{\sigma}) = \cos(\theta)I + i
    \sin(\theta)\vec{v} \cdot \vec{\sigma},
\end{equation*} where $\vec{v} \cdot \vec{\sigma} \equiv \sum_{i=1}^3 v_i
\sigma_i$.

\paragraph{Solution:} First consider $\vec{v} \cdot \vec{\sigma}$:
\begin{align*}
  \vec{v} \cdot \vec{\sigma} &= v_1 \begin{bmatrix}
      1 & 0 \\
      0 & 1
    \end{bmatrix} + v_2 \begin{bmatrix}
      0 & -i \\
      i & 0
    \end{bmatrix} + v_3 \begin{bmatrix}
      1 & 0 \\
      0 & -1
    \end{bmatrix} \\
    &= \begin{bmatrix}
      v_3 & v_1 - iv_2 \\
      v_1 + iv_2 & - v_3
    \end{bmatrix}.
\end{align*}
Solving for the characteristic equation, \begin{align*}
  \det(\vec{v} \cdot \vec{\sigma} - \lambda I)
    &= (v_3 - \lambda)(-v_3 - \lambda) - (v_1 - iv_2)(v_1 + iv_2) \\
    &= \lambda^2 - (v_1^2 + v_2^2 + v_3^2) \\
    &= \lambda^2 - |\vec{v}|^2 \\
    &= \lambda^2 - 1,
\end{align*} and $\vec{v} \cdot \vec{\sigma}$ has eigenvalues $\pm 1$. Let
the eigenvectors $\ket{\lambda_{\pm 1}}$ correspond to the eigenvalues $\pm 1$,
forming an orthonormal basis.

Since $\vec{v} \cdot \vec{\sigma}$ is Hermitian, $\vec{v} \cdot \vec{\sigma}$
is diagonalizable. Then \begin{equation*}
  \vec{v} \cdot \vec{\sigma} = \ketbra{\lambda_1}{\lambda_1} -
    \ketbra{\lambda_{-1}}{\lambda_{-1}}
\end{equation*}
and \begin{align*}
  \mathrm{exp}(i\theta\vec{v} \cdot \vec{\sigma})
    &= e^{i\theta \cdot 1} \ketbra{\lambda_1}{\lambda_1} +
      e^{i\theta \cdot (-1)} \ketbra{\lambda_{-1}}{\lambda_{-1}} \\
    &= (\cos\theta + i\sin\theta) \ketbra{\lambda_1}{\lambda_1} +
      (\cos\theta - i\sin\theta) \ketbra{\lambda_{-1}}{\lambda_{-1}} \\
    &= (\cos\theta)(\ketbra{\lambda_1}{\lambda_1} + \ketbra{\lambda_{-1}}
      {\lambda_{-1}}) + (i\sin\theta)(\ketbra{\lambda_1}{\lambda_1} -
      \ketbra{\lambda_{-1}}{\lambda_{-1}}) \\
    &= \cos(\theta) I + i\sin(\theta) \vec{v} \cdot \vec{\sigma}.
\end{align*}

\paragraph{Trace.} The \emph{trace} of $A$ is defined to be the sum of its
diagonal elements, \begin{equation}
  \mathrm{tr}(A) \equiv \sum_i A_{ii}. \label{eq:tr}
\end{equation}

As an example of the trace, suppose $\ket{\psi}$ is a unit vector and $A$ is an
arbitrary operator. To evaluate $\mathrm{tr}(A\ketbra{\psi}{\psi})$ use the
Gram-Schmidt procedure as defined in \eqref{eq:gs-proc} to extend $\ket{\psi}$
to an orthonormal basis $\ket{i}$ which includes $\ket{\psi}$ as the first
element. Then we have \begin{align*}
  \mathrm{tr}(A\ketbra{\psi}{\psi})
    &= \sum_i \mel{i}{A}{\psi} \braket{\psi}{i} \\
    &= \mel{\psi}{A}{\psi}.
\end{align*}

The trace is easily seen to be \emph{cyclic}, $\mathrm{tr}(AB) =
\mathrm{tr}(BA)$, and \emph{linear}, $\mathrm{tr}(A + B) = \mathrm{tr}(A) +
\mathrm{tr}(B)$ and $\mathrm{tr}(zA) = z\mathrm{tr}(A)$, where $A$ and $B$ are
arbitrary matrices, and $z$ is a complex number.

From the cyclic property it follows that the trace of a matrix is invariant
under the unitary \emph{similarity transformation} $A \rightarrow U A
U^\dagger$, as $\mathrm{tr}(U A U^\dagger) = \mathrm{tr}(U^\dagger U A) =
\mathrm{tr}(A)$. In light of this result, it makes sense to define the trace
of an operator $A$ to be the trace of any matrix representation of $A$. The
invariance of the trace under unitary similarity transformations ensures that
the trace of an operator is well defined.

\paragraph{\cite{mikeandike} Exercise 2.39: (The Hilbert-Schmidt inner product
on operators)} The set $L_V$ of linear operators on a Hilbert space $V$ is
obviously a vector space - the sum of two linear operatotrs is a linear
operator, $zA$ is a linear operator if $A$ is a linear operator and $z$ is a
complex number, and there is a zero element 0. An important additional result
is that the vector space $L_V$ can be given a natural inner product structure,
turning it into a Hilbert space. \begin{enumerate}
  \item Show that the function $(\cdot, \cdot)$ on $L_V \cross L_V$ defined by
    \begin{equation}
      (A, B) \equiv \mathrm{tr}(A^\dagger B) \label{eq:hs-inn-prod}
    \end{equation}
    is an inner product function. This inner product is known as the
    \emph{Hilbert-Schmidt} or \emph{trace} inner product.
  \item If $V$ has $d$ dimensions show that $L_V$ has dimension $d^2$.
  \item Find an orthonormal basis of Hermitian matrices for the Hilbert space
    $L_V$.
\end{enumerate}

\paragraph{Solution:} \begin{enumerate}
  \item To show that the function defined is an inner product function, we need
    to prove that a few requirements are satisfied: \begin{enumerate}
      \item \emph{Linearity in the second argument.} This is clear from the
        linearity of the operators $A$ and $B$, and the trace function.
      \item \emph{$(A, B) = (B, A)^*$}, which can be rewritten as $\mathrm{tr}
        (A^\dagger B) = \mathrm{tr}(B^\dagger A)^*$. Working from the
        right-hand side, \begin{equation*}
          \mathrm{tr}(B^\dagger A)^* = \mathrm{tr}((B^\dagger)^* A^*) =
            \mathrm{tr}(B^T A^*).
        \end{equation*}
        As the trace function is preserved under transposition of the matrices,
        one can consider the transpose \begin{equation*}
          \mathrm{tr}((B^T A^*)^T) = \mathrm{tr}((A^*)^T (B^T)^T) = \mathrm{tr}
            (A^\dagger B).
        \end{equation*}
      \item \emph{$(A, A) \geq 0$ with equality if and only if $A = 0$.} The
        case of equality is trivial. As for any non-zero $A$, since $A^\dagger
        A > 0$ for any linear operator $A$, $\mathrm{tr}(A^\dagger A) > 0$ too.
    \end{enumerate}
  \item Given an orthonormal basis $\ket{v_1}, \ldots, \ket{v_d}$ for $V$, we
    construct linear operators $A_{ij}$ on $V$ such that $A_{ij}\ket{v_j} =
    \ket{v_i}$. (Such a construction is based off \eqref{eq:lin-op}.) It is
    clear that such a construction of $d^2$ linear operators would be linearly
    independent and span the whole of $L_V$.
  \item The construction mentioned in the previous part will produce an
    orthonormal basis, albeit not solely of Hermitian matrices. However, noting
    that $A_{ij} = A_{ji}^\dagger$, one can construct an orthonormal basis for
    any $i < j$ which are provably Hermitian:
    \begin{align*}
      B_{ij} &= \frac{1}{\sqrt{2}}(A_{ij} + A_{ji}), \\
      B_{ii} &= A_{ii}, \\
      B_{ji} &= \frac{i}{\sqrt{2}}(A_{ij} - A_{ji}). \\
    \end{align*}
\end{enumerate}

\section{Commutator and anti-commutator}

\paragraph{Commutator.} The \emph{commutator} between two operators $A$ and $B$
is defined to be \begin{equation}
  [A, B] \equiv AB - BA. \label{eq:comm}
\end{equation}
If $[A, B] = 0$, that is, $AB = BA$, then we say $A$ \emph{commutes} with $B$.

\paragraph{Anti-Commutator.} The \emph{anti-commutator} of two operators $A$
and $B$ is defined by \begin{equation}
  {A, B} \equiv AB + BA; \label{eq:anti-comm}
\end{equation}
we say $A$ \emph{anti-commutes} with $B$ if ${A, B} = 0$.

It turns out that many important properties of pairs of operators can be
deduced from their commutator and anti-commutator. Perhaps the most useful
relation is the following connection between the commutator and the property
of being able to \emph{simultaneously diagonalize} Hermitian operators $A$ and
$B$, that is, write $A = \sum_i a_i\ketbra{i}{i}$, $B = \sum_i b_i\ketbra{i}{i}
$, where $\ket{i}$ is some common orthonormal set of eigenvectors for $A$ and
$B$.

\begin{theorem}[Simultaneous diagonalization theorem]
  Suppose $A$ and $B$ are Hermitian operators. Then $[A, B] = 0$ if and only if
  there exists an orthonormal basis such that both $A$ and $B$ are diagonal
  with respect to that basis. We say that $A$ and $B$ are \emph{simultaneously
  diagonalizable} in this case.
\end{theorem}

This result connects the commutator of two operators, which is often easy to
compute, to the property of being simultaneously diagonalizable, which is
\emph{a priori} rather difficult to determine.

\begin{proof}
  You can (and should!) easily verify that if $A$ and $B$ are diagonal in the
  same orthonormal basis then $[A, B] = 0$. To show the converse, let $\ket{a,
  j}$ be an orthonormal basis for the eigenspace $V_a$ of $A$ with eigenvalue
  $a$; the index $j$ is used to label possible degeneracies. Note that
  \begin{equation*}
    AB\ket{a, j} = BA\ket{a, j} = aB\ket{a, j}
  \end{equation*}
  and therefore $B\ket{a, j}$ is an element of the eigenspace $V_a$. Let $P_a$
  denote the projector onto the space $V_a$ and define $B_a \equiv P_aBP_a$.
  It is easy to see that the restriction of $B_a$ to the space $V_a$ is
  Hermitian on $V_a$, and therefore has a spectral decomposition in terms of
  an orthonormal set of eigenvectors which span the space $V_a$. Let's call
  these eigenvectors $\ket{a, b, k}$, where the indices $a$ and $b$ label the
  eigenvalues of $A$ and $B_a$, and $k$ is an extra index to allow for the
  possibility of a degenerate $B_a$. Note that $B\ket{a, b, k}$ is an element
  of $V_a$, so \begin{equation*}
    B\ket{a, b, k} = P_aBP_a\ket{a, b, k} = b\ket{a, b, k}.
  \end{equation*}
  It follows that $\ket{a, b, k}$ is an eigenvector of $B$ with eigenvalue $b$,
  and therefore $\ket{a, b, k}$ is an orthonormal set of eigenvectors of both
  $A$ and $B$, spanning the entire vector space on which $A$ and $B$ are
  defined. That is, $A$ and $B$ are simultaneously diagonalizable.
\end{proof}

\section{The polar and singular value decomposition}

The \emph{polar} and \emph{singular value} decompositions are useful ways of
breaking linear operators up into simpler parts. In particular, these
decompositions allow us to break general linear operators up into products of
unitary operators and positive operators.

\begin{theorem}[Polar decomposition]
  Let $A$ be a linear operator on a vector space $V$. Then there exists unitary
  $U$ and positive operators $J$ and $K$ such that \begin{equation*}
    A = UJ = KU,
  \end{equation*}
  where the unique positive operators $J$ and $K$ satisfying these equations
  are defined by $J \equiv \sqrt{A^\dagger A}$ and $K \equiv \sqrt{A
  A^\dagger}$. Moreover, if $A$ is invertible then $U$ is unique.
\end{theorem}

We call the expression $A = UJ$ the \emph{left polar decomposition} of $A$, and
$A = KU$ the \emph{right polar decomposition} of $A$.

\begin{proof}
  $J \equiv \sqrt{A^\dagger A}$ is a positive operator, so it can be given a
  spectral decomposition, $J = \sum_i \lambda_i \ketbra{i}{i}$ ($\lambda_i \geq
  0$). Define $\ket{\psi_i} \equiv A\ket{i}$. From the definition, we see that
  $\braket{\psi_i}{\psi_i} = \lambda_i^2$. Consider for now only those $i$ for
  which $\lambda_i \neq 0$. For those $i$ define $\ket{e_i} \equiv \ket{\psi_i}
  /\lambda_i$, so that the $\ket{e_i}$ are normalized. Moreover, they are
  orthogonal, since if $i \neq j$ then $\braket{e_i}{e_j} = \mel{i}{A^\dagger A
  }{j}/\lambda_i\lambda_j = \mel{i}{J^2}{j}/\lambda_i\lambda_j = 0$.

  We have been considering $i$ such that $\lambda_i \neq 0$. Now use the
  Gram-Schmidt procedure to extend the orthonormal set $\ket{e_i}$ so it forms
  an orthonormal basis, which we also label $\ket{e_i}$. Define a unitary
  operator $U \equiv \sum_i \ketbra{e_i}{i}$. When $\lambda_i \neq 0$ we have
  $UJ\ket{i} = \lambda_i\ket{e_i} = \ket{\psi_i} = A\ket{i}$. When $\lambda_i =
  0$ we have $UJ\ket{i} = 0 = \ket{\psi_i}$. We have proved that the action of
  $A$ and $UJ$ agree on the basis $\ket{i}$, and thus that $A = UJ$.

  $J$ is unique, since multiplying $A = UJ$ on the left by the adjoint equation
  $A^\dagger = JU^\dagger$ gives $J^2 = A^\dagger A$, from which we see that $J
  = \sqrt{A^\dagger A}$, uniquely. A little thought shows that if $A$ is
  invertible, then so is $J$, so $U$ is uniquely determined by the equation $U
  = AJ^{-1}$. The proof of the right polar decomposition follows, since $A = UJ
  = UJU^\dagger U = KU$, where $K \equiv UJU^\dagger$ is a positive operator.
  Since $AA^\dagger = KUU^\dagger K = K^2$ we must have $K = \sqrt{A^\dagger
  A}$, as claimed.
\end{proof}

The singular value decomposition combines the polar decomposition and the
spectral theorem.

\begin{corollary}[Singular value decomposition]
  Let $A$ be a square matrix. Then there exist unitary matrices $U$ and $V$,
  and a diagonal matrix $D$ with non-negative entries such that
  \begin{equation*}
    A = UDV.
  \end{equation*}
  The diagonal elements of $D$ are called the \emph{singular values} of $A$.
\end{corollary}

\begin{proof}
  By the polar decomposition, $A = SJ$, for unitary $S$, and positive $J$. By
  the spectral theorem, $J = TDT^\dagger$, for unitary $T$ and diagonal $D$
  with non-negative entries. Setting $U \equiv ST$ and $V \equiv T^\dagger$
  completes the proof.
\end{proof}

