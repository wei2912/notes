\chapter{Conventions}

\section{Vectors}

For two-level quantum systems used as qubits, we shall usually identify the
state $\ket{0}$ with the vector $(1, 0)$, and similarly $\ket{1}$ with $(0,
1)$.

\section{The Pauli matrices}

\paragraph{Pauli matrices.} The Pauli matrices are defined accordingly:

\begin{center}
  \begin{tabular}{r r}
    \addlinespace[1em]
    $\sigma_0 \equiv I \equiv \begin{bmatrix}
      1 & 0 \\
      0 & 1
    \end{bmatrix}$ & $\sigma_1 \equiv \sigma_x \equiv X \equiv \begin{bmatrix}
      0 & 1 \\
      1 & 0
    \end{bmatrix}$ \\
    \addlinespace[1em]
    $\sigma_2 \equiv \sigma_y \equiv Y \equiv \begin{bmatrix}
      0 & -i \\
      i & 0
    \end{bmatrix}$ & $\sigma_3 \equiv \sigma_z \equiv Z \equiv \begin{bmatrix}
      1 & 0 \\
      0 & -1
    \end{bmatrix}$
  \end{tabular}
\end{center}

