\documentclass{article}

\usepackage{amsmath,amssymb,amsthm}
\usepackage{csquotes}
\usepackage[shortlabels]{enumitem}

\newtheorem{theorem}{Theorem}
\newtheorem*{theorem*}{Theorem}
\newtheorem{definition}{Definition}
\newtheorem*{definition*}{Definition}
\newtheorem{corollary}{Corollary}
\newtheorem*{corollary*}{Corollary}
\newtheorem{lemma}{Lemma}
\newtheorem*{lemma*}{Lemma}

\title{Notes to ``Linear Algebra'', 2nd edition (Hoffman, Kunze)}
\author{Ng Wei En}

\begin{document}

\maketitle
\tableofcontents
\newpage

\section{Linear Equations}

\subsection{Fields}

\paragraph{Field.} Suppose one has a set $F$ of objects $x, y, z, \ldots$ and
two operations on the elements of $F$ as follows:

\begin{enumerate}
  \item Addition, which associates with each pair of elements $x, y \in F$ an
    element $(x + y) \in F$.
  \item Multiplication, which associates with each pair $x, y$ an element $xy
    \in F$.
\end{enumerate}

If these two operations satisfy conditions (1)-(9) below, then the set $F$,
together with these two operations, is called a \textbf{field}.

\begin{enumerate}
  \item Addition is commutative (i.e. $x + y = y + x$ for all $x, y \in F$).
  \item Addtion is associative (i.e. $x + (y + z) = (x + y) + z$ for all $x, y,
    z \in F$).
  \item There is a unique element $0 \in F$ such that $x + 0 = x$, for every $x
    \in F$.
  \item To each $x \in F$ there corresponds a unique element $(-x) \in F$ such
    that $x + (-x) = 0$.
  \item Multiplication is commutative (i.e. $xy = yx$ for all $x, y \in F$).
  \item Multiplication is associative (i.e. $x(yz) = (xy)z$ for all $x, y, z
    \in F$).
  \item There is a unique non-zero element $1 \in F$ such that $x1 = x$, for
    every $x \in F$.
  \item To each non-zero $x \in F$ there corresponds a unique element $x^{-1}
    \in F$ such that $xx^{-1} = 1$.
  \item Multiplication distributes over addition (i.e. $x(y + z) = xy + xz$ for
    all $x, y, z \in F$).
\end{enumerate}

\paragraph{Subfield.} A \textbf{subfield} of a field $F$ is a subset $F_1
\subseteq F$ which is itself a field under the same two operations of addition
and multiplication.

\paragraph{Characteristic.} For a field $F$, if it is possible to add the unit
1 to itself a finite number of times and obtain 0: \[
  1 + 1 + \cdots + 1 = 0,
\] then the least $n$ such that the sum of $n$ 1's is 0 is called the
\textbf{characteristic} of $F$. Otherwise, $F$ is called a field of
\textbf{characteristic zero}.

\subsection{Systems of Linear Equations}

\paragraph{System of linear equations.} Suppose $F$ is a field. We consider the
problem of finding $n$ scalars $x_1, \ldots, x_n \in F$ which satisfy the
conditions
\begin{equation}\label{eq:1-1}\tag{1-1}
  \begin{aligned}
    A_{11}x_1 + A_{12}x_2 + \cdots + A_{1n}x_n &= y_1 \\
    A_{21}x_1 + A_{22}x_2 + \cdots + A_{2n}x_n &= y_2 \\
    &\vdots \\
    A_{m1}x_1 + A_{m2}x_2 + \cdots + A_{mn}x_n &= y_m
  \end{aligned}
\end{equation}
while $y_1, \ldots, y_m$ and $A_{ij}$, $1 \leq i \leq m, 1 \leq j \leq n$, are
given elements of $F$. We call \eqref{eq:1-1} a \textbf{system of $m$ linear
equations in $n$ unknowns}. Any $n$-tuple $(x_1, \ldots, x_n)$ of elements of
$F$ which satisfies each of the equations in \eqref{eq:1-1} is called a
\textbf{solution} of the system. If $y_1 = y_2 = \cdots = y_m = 0$, we say that
the system is \textbf{homogeneous}.

\paragraph{Equivalence.} Two systems of linear equations are \textbf{equivalent}
if each equation in each system is a linear combination of the equations in the
other system.

\begin{theorem}
  Equivalent systems of linear equations have exactly the same solutions.
\end{theorem}

\subsection{Matrices and Elementary Row Operations}

\paragraph{Matrix.} An $m \times n$ \textbf{matrix over the field $F$} is a
function $A$ from the set of pairs of integers $(i, j)$, $1 \leq i \leq m, 1
\leq j \leq n$, into the field $F$. The \textbf{entries} of the matrix $A$ are
the scalars $A(i, j) = A_{ij}$.

\paragraph{Matrix of coefficients.} We can abbreviate the system \eqref{eq:1-1}
by $AX = Y$ where
\begin{gather*}
  A = \begin{bmatrix}
    A_{11} & \cdots & A_{1n} \\
    \vdots & & \vdots \\
    A_{m1} & \cdots & A_{mn}
  \end{bmatrix}, \\
  X = \begin{bmatrix}
    x_1 \\
    \vdots \\
    x_n
  \end{bmatrix}
  \quad\text{ and }\quad
  Y = \begin{bmatrix}
    y_1 \\
    \vdots \\
    y_n
  \end{bmatrix}.
\end{gather*}
We call $A$ the \textbf{matrix of coefficients} of the system.

\paragraph{Elementary row operations (EROs).} Three \textbf{elementary row
operations} can be carried out on a $m \times n$ matrix $A$ over the field $F$:
\begin{enumerate}
  \item multiplication of one row of $A$ by a non-zero scalar $c$;
  \item replacement of the $r$th row of $A$ by row $r + cs$, $c$ any scalar and
    $r \neq s$;
  \item interchange of two rows of $A$.
\end{enumerate}

\begin{theorem}
  To each ERO $e$ there corresponds an ERO $e_1$, of the same type as $e$, such
  that $e_1(e(A)) = e(e_1(A)) = A$ for each $A$.
\end{theorem}

\begin{definition*}
  If $A$ and $B$ are $m \times n$ matrices over the field $F$, we say that $B$
  \textbf{is row-equivalent to} $A$ if $B$ can be obtained from $A$ by a finite
  sequence of EROs.
\end{definition*}

\begin{theorem}
  If $A$ and $B$ are row-equivalent $m \times n$ matrices, the homogeneous
  systems of linear equations $AX = 0$ and $BX = 0$ have exactly the same
  solutions.
\end{theorem}

\begin{definition*}
  An $m \times n$ matrix $R$ is called \textbf{row-reduced} if:
  \begin{enumerate}[(a)]
    \item the first non-zero entry in each non-zero row of $R$ is equal to 1;
    \item each column of $R$ which contains the leading non-zero entry of some
      row has all its other entries 0.
  \end{enumerate}
\end{definition*}

\begin{theorem} \label{thm:4}
  Every $m \times n$ matrix over the field $F$ is row-equivalent to a
  row-reduced matrix.
\end{theorem}

\subsection{Row-Reduced Echelon Matrices}

\begin{definition*}
  An $m \times n$ matrix $R$ is called a \textbf{row-reduced echelon matrix} if:
  \begin{enumerate}[(a)]
    \item $R$ is row-reduced;
    \item every row of $R$ which has all its entries 0 occurs below every row
      which has a non-zero entry;
    \item if rows $1, \ldots, r$ are the non-zero rows of $R$, and if the
      leading non-zero entry of row $i$ occurs in column $k_i$, $i = 1, \ldots,
      r$, then $k_1 < k_2 < \cdots < k_r$.
  \end{enumerate}
\end{definition*}

\begin{theorem}
  Every $m \times n$ matrix $A$ is row-equivalent to a row-reduced echelon
  matrix (cf. \textbf{Theorem~\ref{thm:4}}).
\end{theorem}

\begin{theorem}
  If $A$ is an $m \times n$ matrix and $m < n$, then the homogeneous system of
  linear equations $AX = 0$ has a non-trivial solution.
\end{theorem}

\begin{theorem}
  If $A$ is an $n \times n$ (square) matrix, then $A$ is row-equivalent to the
  $n \times n$ identity matrix if and only if the system of equations $AX = 0$
  has only the trivial solution.
\end{theorem}

\paragraph{Augmented matrix.} The \textbf{augmented matrix} $A'$ of the system
$AX = Y$ is the $m \times (n + 1)$ matrix whose first $n$ columns are the
columns of $A$ and whose last column is $Y$.

\subsection{Matrix Multiplication}

\begin{definition*}
  Let $A$ be an $m \times n$ matrix over the field $F$ and let $B$ be an $n
  \times p$ matrix over $F$. The \textbf{product} $AB$ is the $m \times p$
  matrix $C$ whose $i, j$ entry is \[
    C_{ij} = \sum_{r=1}^n A_{ir}B_{rj}.
  \]
\end{definition*}

\begin{theorem}
  If $A, B, C$ are matrices over the field $F$ such that the products $BC$ and
  $A(BC)$ are defined, then so are the products $AB$, $(AB)C$ and \[
    A(BC) = (AB)C.
  \]
\end{theorem}

\begin{definition*}
  An $m \times n$ matrix is said to be an \textbf{elementary matrix} if it can
  be obtained from the $m \times n$ identity matrix by means of a single ERO.
\end{definition*}

\begin{theorem}
  Let $e$ be an ERO and let $E$ be the $n \times m$ elementary matrix $E =
  e(I)$. Then, for every $m \times n$ matrix $A$, \[
    e(A) = EA.
  \]
\end{theorem}

\begin{corollary*}
  Let $A$ and $B$ be $m \times n$ matrices over the field $F$. Then $B$ is
  row-equivalent to $A$ if and only if $B = PA$, where $P$ is a product of $m
  \times n$ elementary matrices.
\end{corollary*}

\subsection{Invertible Matrices}

\begin{definition*}
  Let $A$ be an $n \times n$ matrix over the field $F$. An $n \times n$ matrix
  $B$ such that $BA = I$ is called a \textbf{left inverse} of $A$; an $n \times
  n$ matrix $B$ such that $AB = I$ is called a \textbf{right inverse} of $A$. If
  $AB = BA = I$, then $B$ is called a \textbf{two-sided inverse} (or simply
  \textbf{the inverse}) of $A$, denoted as $A^{-1}$, and $A$ is said to be
  \textbf{invertible}.
\end{definition*}

\begin{lemma*}
  If $A$ has a left inverse $B$ and a right inverse $C$, then $B = C$.
\end{lemma*}

\begin{theorem}
  Let $A$ and $B$ be $n \times n$ matrices over $F$.
  \begin{enumerate}[(i)]
    \item If $A$ is invertible, so is $A^{-1}$ and $(A^{-1})^{-1} = A$.
    \item If both $A$ and $B$ are invertible, so is $AB$, and $(AB)^{-1} =
      B^{-1}A^{-1}$.
  \end{enumerate}
\end{theorem}

\begin{corollary*}
  A product of invertible matrices is invertible.
\end{corollary*}

\begin{theorem}
  An elementary matrix is invertible.
\end{theorem}

\begin{theorem}
  If $A$ is an $n \times n$ matrix, the following are equivalent.
  \begin{enumerate}[(i)]
    \item $A$ is invertible.
    \item $A$ is row-equivalent to the $n \times n$ identity matrix.
    \item $A$ is a product of elementary matrices.
  \end{enumerate}
\end{theorem}

\begin{corollary*}
  If $A$ is an invertible $n \times n$ matrix and if a sequence of EROs reduces
  $A$ to the identity, then that same sequence of operations when applied to $I$
  yields $A^{-1}$.
\end{corollary*}

\begin{corollary*}
  Let $A$ and $B$ be $m \times n$ matrices. Then $B$ is row-equivalent to $A$ if
  and only if $B = PA$ where $P$ is an invertible $m \times m$ matrix.
\end{corollary*}

\begin{theorem}
  For an $n \times n$ matrix $A$, the following are equivalent.
  \begin{enumerate}[(i)]
    \item $A$ is invertible.
    \item The homogeneous system $AX = 0$ has only the trivial solution $X = 0$.
    \item The system of equations $AX = Y$ has a solution $X$ for each $n \times
      1$ matrix $Y$.
  \end{enumerate}
\end{theorem}

\begin{corollary*}
  A square matrix with either a left or right inverse is invertible.
\end{corollary*}

\begin{corollary*}
  Let $A = A_1A_2 \cdots A_k$, where $A_1, \ldots, A_k$ are $n \times n$
  matrices. Then $A$ is invertible if and only if each $A_j$ is invertible.
\end{corollary*}

\end{document}
