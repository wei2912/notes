\documentclass{article}

\usepackage{amsmath,amssymb,amsthm}
\usepackage{csquotes}
\usepackage[shortlabels]{enumitem}

\newtheorem{theorem}{Theorem}
\newtheorem*{theorem*}{Theorem}

\title{Notes to ``Linear Algebra'', 2nd edition (Hoffman, Kunze)}
\author{Ng Wei En}

\begin{document}

\maketitle
\tableofcontents
\newpage

\section{Linear Equations}

\subsection{Fields}

\paragraph{Characteristic.} If $F$ is a field, it may be possible to add the
unit 1 to itself a finite number of times and obtain 0: \[
  1 + 1 + \cdots + 1 = 0.
\] If it does happen in $F$, then the least $n$ such that the sum of $n$ 1's is
0 is called the \textbf{characteristic} of the field $F$. If it does not happen
in $F$, then $F$ is called a field of \textbf{characteristic zero}.

\subsection{Systems of Linear Equations}

\begin{theorem}
  Equivalent systems of linear equations have exactly the same solutions.
\end{theorem}

\end{document}
