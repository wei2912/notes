\documentclass{article}

\usepackage{amsmath,amssymb,amsthm}
\usepackage{csquotes}

\title{Solutions to ``Linear Algebra'', 2nd edition (Hoffman, Kunze)}
\author{Ng Wei En}

\begin{document}

\maketitle
\tableofcontents
\newpage

\section{Linear Equations}

\setcounter{subsection}{1}
\subsection{Systems of Linear Equations}

\paragraph{1.} Verify that the set of complex numbers described in Example 4 is
a subfield of $\mathbb{C}$.

\begin{displayquote}
  \textbf{Example 4.} The set of all complex numbers of the form $x +
  y\sqrt{2}$, where $x$ and $y$ are rational, is a subfield of $\mathbb{C}$. We
  leave it to the reader to verify this.
\end{displayquote}

\begin{proof}
  Let $F = \{x + y\sqrt{2} \mid x, y \in \mathbb{Q}\}$. Then,
  \begin{enumerate}
    \item $0 + 0\sqrt{2} = 0 \in F$.
    \item $1 + 0\sqrt{2} = 1 \in F$.
    \item For $x, y \in F$, take $x = a + b\sqrt{2}$ and $y = c + d\sqrt{2}$.
      Then $x + y = (a + c) + (b + d)\sqrt{2} \in F$.
    \item For $x \in F$, take $x = a + b\sqrt{2}$. Then $-x = (-a) +
      (-b)\sqrt{2} \in F$.
    \item For $x, y \in F$, take $x = a + b\sqrt{2}$ and $y = c + d\sqrt{2}$.
      Then $xy = (ac + 2bd) + (ad + bc)\sqrt{2} \in F$.
    \item For $x \in F$, $x \neq 0$, take $x = a + b\sqrt{2}$ where at least one
      of $a$ and $b$ are not zero. Then for some $y = a + (-b)\sqrt{2}$, $xy = a^2
      - 2b^2 \neq 0$ and thus $x\left(\frac{y}{a^2 - 2b^2}\right) = 1$. Since
      $\frac{a}{a^2 - 2b^2}, \frac{-b}{a^2 - 2b^2} \in \mathbb{Q}$, $x^{-1} =
      \frac{y}{a^2 - 2b^2} \in F$.
  \end{enumerate}
  So $F$ is also a field, and since $F \subseteq \mathbb{C}$, $F$ is therefore a
  subfield of $\mathbb{C}$.
\end{proof}

\paragraph{6.} Prove that if two homogeneous systems of linear equations in two
unknowns have the same solutions, then they are equivalent.

\begin{proof}
  Consider these two systems:
  \begin{align*}
    a_1x + b_1y &= 0 & a'_1x + b'_1y &= 0 \\
    a_2x + b_2y &= 0 & a'_2x + b'_2y &= 0 \\
    &\vdots & &\vdots \\
    a_mx + b_my &= 0 & a'_mx + b'_my &= 0
  \end{align*}
  Each system consists of a set of lines through $(0, 0)$ in the $x$-$y$ plane.
  Thus the two systems have the same solutions if and only if they both have
  either $(0, 0)$ as their only solution, or a single line $ux + vy = 0$ as
  their common solution.

  In the latter case, all equations are simply multiples of the same line, so
  clearly the two systems are equivalent.

  In the former case, assume without loss of generality that the first two
  equations in the first system represent different lines. Then \[
    \frac{a_1}{a_2} \neq \frac{b_1}{b_2}.
  \] We need to show that there exists some solution $(u, v)$ to the system
  \begin{align*}
    a_1u + a_2v &= a'_i \\
    b_1u + b_2v &= b'_i
  \end{align*}
  for some $i$. Solving for both $u$ and $v$, \[
    u = \frac{b_2a'_i - b_1b'_i}{a_1b_2 - a_2b_1}, \quad
    v = \frac{a_1b'_i - a_2a'_i}{a_1b_2 - a_2b_1}.
  \] Since $a_1b_2 - b_1a_2 \neq 0$, both $u$ and $v$ are
  well-defined and therefore the two systems are equivalent.
\end{proof}

\paragraph{7.} Prove that each subfield of the field of complex numbers contains
every rational number.

\begin{proof}
  For a subfield $F$ of $\mathbb{C}$, $n \cdot 1 = 0 \in F \implies n \cdot 1 =
  0 \in \mathbb{C}$. But $\mathbb{C}$ has characteristic zero, so $n \cdot 1 = 0
  \in \mathbb{C} \implies n = 0$ and thus $1, 2, 3, \cdots$ are all distinct
  elements of $F$. Each of their additive inverses $-1, -2, -3, \cdots$ are also in
  $F$, so $\mathbb{Z} \in F$. Similarly, each of the multiplicative inverses
  $\frac{1}{n}$ for all $n \in \mathbb{Z}, n \neq 0$ are also in $F$.

  For all numbers $\frac{m}{n} \in \mathbb{Q}$, $m, n \in \mathbb{Z}$ which
  implies $m, \frac{1}{n} \in F$ and so $m \cdot \frac{1}{n} = \frac{m}{n} \in
  F$. Hence $\mathbb{Q} \subseteq F$.
\end{proof}

\paragraph{8.} Prove that each field of characteristic zero contains a copy of
the rational number field.

\begin{proof}
  For a field $F$ of characteristic zero, we define the additive and
  multiplicative identities as $0_F$ and $1_F$ respectively. Let $n_F =
  \underbrace{1_F + 1_F + \cdots + 1_F}_{n\text{ terms}}$, and its additive and
  multiplicative inverses be $-n_F$ and $n_F^{-1}$ respectively. As $F$ has
  characteristic zero, if $n \neq m$ then $n_F \neq m_F$.

  For $m, n \in \mathbb{Z}$, let $\left(\frac{m}{n}\right)_F = m_F \cdot
  n_F^{-1}$. If $\frac{m}{n} \neq \frac{m'}{n'}$ then
  $\left(\frac{m}{n}\right)_F \neq \left(\frac{m'}{n'}\right)_F$. So, the map
  $h: \frac{m}{n} \mapsto \left(\frac{m}{n}\right)_F$ is a bijection from a
  subset of $F$ to $\mathbb{Q}$.
\end{proof}

\end{document}
