\documentclass{article}

\usepackage{amsmath,amssymb,amsthm}
\usepackage[shortlabels]{enumitem}
\usepackage{systeme}

\title{Solutions to ``Linear Algebra'', 2nd edition (Hoffman, Kunze)}
\author{Ng Wei En}

\begin{document}

\maketitle
\tableofcontents
\newpage

\section{Linear Equations}

\setcounter{subsection}{1}
\subsection{Systems of Linear Equations}

\paragraph{6.} Prove that if two homogeneous systems of linear equations in two
unknowns have the same solutions, then they are equivalent.

\begin{proof}
  We express these two systems of linear equations as:
  \begin{align*}
    a_{11}x + a_{12}y &= 0 & b_{11}x + b_{12}y &= 0 \\
    a_{21}x + a_{22}y &= 0 & b_{21}x + b_{22}y &= 0 \\
    &\vdots & &\vdots \\
    a_{m1}x + a_{m2}y &= 0 & b_{m1}x + b_{m2}y &= 0 \\
  \end{align*}
  Each system consists of a set of lines through $(0, 0)$ in the $x$-$y$ plane.
  Thus the two systems have the same solution iff they both have either $(0, 0)$
  as their only solution (Case 1) or single line $ux + vy = 0$ as their common
  solution (Case 2).

  In Case 1, all equations are simply multiples of the same line, so clearly the
  two systems are equivalent.

  In Case 2, assume WLOG that the first two equations in the first system are
  different lines. Then \[
    \frac{a_{11}}{a_{12}} \neq \frac{a_{21}}{a_{22}}.
  \] We need to show that there exists some $(u, v)$ that solves
  \begin{align*}
    a_{11}u + a_{12}v &= b_{i1} \\
    a_{21}u + a_{22}v &= b_{i2}
  \end{align*}
  for some $i$. Solving for $u$ and $v$, \[
    u = \frac{a_{22}b_{i1} - a_{12}b_{i2}}{a_{11}a_{22} - a_{12}a_{21}}, \\
    v = \frac{a_{21}b_{i1} - a_{11}b_{i2}}{a_{12}a_{21} - a_{11}a_{22}}.
  \] Since $a_{11}a_{22} - a_{12}a_{21} \neq 0$, both $u$ and $v$ are
  well-defined, thus the two systems are equivalent.
\end{proof}

\paragraph{7.} Prove that each subfield of the field of complex numbers contains
every rational number.

\begin{proof}
  For a subfield $F$ of $C$, $1 \in F$ and $n \cdot 1 = 0 \in F \subseteq C
  \iff n = 0$ as $C$ has characteristic zero, so $n = 1, 2, 3, \ldots$ are all
  distinct elements of $F$. Each of these elements have additive inverses $-1,
  -2, -3, \ldots$ which are also distinct elements of $F$, and therefore $Z
  \subseteq F$.

  For all $n \in Z$, $n \neq 0$, their multiplicative inverses $\frac{1}{n}$ are
  also distinct elements of $F$. Consider a rational number $\frac{m}{n} \in Q$.
  Then, $m, n \in Z \implies m, \frac{1}{n} \in F$ and so $m \cdot \frac{1}{n} =
  \frac{m}{n} \in F$. Hence, we can conclude that $Q \subseteq F$.
\end{proof}

\paragraph{8.} Prove that each field of characteristic zero contains a copy of
the rational number field.

\begin{proof}
  For a field $F$ of characteristic zero, let its additive and multiplicative
  identity be $0_F$ and $1_F$ respectively, and define \[
    n_F = \underbrace{1_F + 1_F + \cdots + 1_F}_{n\text{ terms}}.
  \] Define the additive and multiplicative inverse of $n_F$ as $-n_F$ and
  $n_F^{-1}$ respectively. As $F$ has characteristic zero, if $n \neq m$ then
  $n_F \neq m_F$. For $m, n \in Z$, let $\left(\frac{m}{n}\right)_F = m_F \cdot
  n_F^{-1}$; if $\frac{m}{n} \neq \frac{m'}{n'}$ then
  $\left(\frac{m}{n}\right)_F \neq \left(\frac{m'}{n'}\right)_F$. So the map
  $h: \frac{m}{n} \to \left(\frac{m}{n}\right)_F$ is a bijection from $Q$ to a
  subset of $F$, with $h(0) = 0_F$, $h(1) = 1_F$, $h(x + y) = h(x) + h(y)$ and
  $h(xy) = h(x)h(y)$.
\end{proof}

\setcounter{subsection}{5}
\subsection{Invertible Matrices}

\paragraph{9.} An $n \times n$ matrix $A$ is called \textbf{upper-triangular} if
$A_{ij} = 0$ for $i > j$, that is, if every entry below the main diagonal is 0.
Prove that an upper-triangular (square) matrix is invertible if and only if
every entry on its main diagonal is different from 0.

\begin{proof}
  Suppose $A_{ii} \neq 0$ for all $i$. Construct the matrix $A'$ by dividing
  each row $i$ of $A$ by $A_{ii}$ such that $A'_{ii} = 1$ for all $i$. Then,
  $I_n$ can be easily obtained by applying elementary row operations on $A'$ to
  eliminate all non-zero entries above the main diagonal. Hence, $A$ is
  row-equivalent to $I_n$ and invertible by \textbf{Theorem 12}.

  Now, suppose $A_{ii} = 0$ for some $i$. If $A_{ii} = 0$ for all $i$, then the
  last row of $A$ is all zeroes and $A$ cannot be row-equivalent to $I_n$. Now,
  taking $i'$ as the largest index such that $A_{i'i'} = 0$ and $A_{ii} \neq 0$
  for all $i > i'$, construct the matrix $A'$ by dividing all rows $i$ below row
  $i'$ of $A$ by $A_{ii}$ such that $A'_{ii} = 1$ for all $i > i'$. Then add
  multiples of these rows below row $i'$ such that row $i'$ is all zeroes. Then
  $A$ cannot be row-equivalent to $I_n$.
\end{proof}

\end{document}
