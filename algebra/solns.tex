\documentclass{article}

\usepackage{amsmath,amssymb,amsthm}
\usepackage[shortlabels]{enumitem}

\title{Solutions to ``Abstract Algebra'', 3rd edition (Dummit, Foote)}
\author{Ng Wei En}

\begin{document}

\maketitle
\tableofcontents
\newpage

\section{Introduction to Groups}

\subsection{Basic Axioms and Examples}

\paragraph{1.}
\begin{enumerate}[(a)]
  \item Not associative. $(0 \star 1) \star 1 = (0 - 1) - 1 = -2 \neq 0 = 0 -
    (1 - 1) = 0 \star (1 \star 1)$.
  \item Associative. \begin{align*}
      (a \star b) \star c
      &= (a + b + ab) \star c \\
      &= (a + b + ab) + c + (a + b + ab)c \\
      &= (a + b + c) + (ab + ac + bc) + abc \\
      &= a + (b + c + bc) + a(b + c + bc) \\
      &= a \star (b + c + bc) \\
      &= a \star (b \star c).
    \end{align*}
  \item Not associative. $(0 \star 1) \star 1 = \frac{\frac{0 + 1}{5} + 1}{5} =
    \frac{6}{25} \neq \frac{2}{25} = \frac{0 + \frac{1 + 1}{5}}{5} = 0 \star (1
    \star 1)$.
  \item Associative. \begin{align*}
      ((a, b) \star (c, d)) \star (e, f)
      &= (ad + bc, bd) \star (e, f) \\
      &= ((ad + bc)f + bde, bdf) \\
      &= (adf + bcf + bde, bdf) \\
      &= (adf + b(cf + de), bdf) \\
      &= (a, b) \star (cf + de, df) \\
      &= (a, b) \star ((c, d) \star (e, f)).
    \end{align*}
  \item Not associative. $(1 \star 2) \star 2 = \frac{1}{2} \star 2 =
    \frac{\frac{1}{2}}{2} = \frac{1}{4} \neq 1 = \frac{1}{\frac{2}{2}} = 1
    \star \frac{2}{2} = 1 \star (2 \star 2)$.
\end{enumerate}

\paragraph{2.}
\begin{enumerate}[(a)]
  \item Not commutative. $0 \star 1 = 0 - 1 = -1 \neq 1 = 1 - 0 = 1 \star 0$.
  \item Commutative by symmetry.
  \item Commutative by symmetry.
  \item Commutative. \begin{align*}
      (a, b) \star (c, d)
      &= (ad + bc, bd) \\
      &= (cb + da, db) \\
      &= (c, d) \star (a, b).
    \end{align*}
  \item Not commutative. $1 \star 2 = \frac{1}{2} \neq \frac{2}{1} = 2 \star 1$.
\end{enumerate}

\paragraph{5.}
\begin{proof}
  Consider $\overline{0} \in \mathbb{Z}/n\mathbb{Z}$. Then for all $\overline{x}
  \in \mathbb{Z}/n\mathbb{Z}$, $\overline{0} \cdot \overline{x} = \overline{0x}
  = \overline{0}$. Hence, there does not exist a multiplicative inverse for
  $\overline{0}$.
\end{proof}

\paragraph{6.} Note that addition is associative with identity 0, so it suffices
to check for existence of inverse and closure.
\begin{enumerate}[(a)]
  \item Group. Any rational number $\frac{a}{b}$ in lowest terms has an inverse
    $\frac{-a}{b}$ in the set. Also, any two numbers $\frac{a}{b}$ and
    $\frac{c}{d}$ when added will yield a fraction with an odd denominator.
    Since all factors of an odd number are odd, the fraction when written in
    lowest terms will be in the set.
  \item Not a group. $\frac{1}{2} + \frac{1}{2} = \frac{2}{2} = \frac{1}{1}$
    which is not in the set.
  \item Not a group. $\frac{1}{2} + \frac{1}{2} = 1 \geq 1$.
  \item Not a group. $\frac{3}{2} + (-1) = \frac{1}{2} < 1$.
  \item Group. Any rational number $\frac{a}{b}$ with $b = 1, 2$ has an inverse
    $\frac{-a}{b}$ in the set. Also, any rational numbers $\frac{a}{b}$ and
    $\frac{c}{d}$ for $1 \leq b, d \leq 2$ can be written as $\frac{a'}{2}$ and
    $\frac{c'}{2}$ respectively, so $\frac{a}{b} + \frac{c}{d} = \frac{a'}{2} +
    \frac{c'}{2} = \frac{a' + c'}{2}$ which when written in lowest terms will be
    in the set.
  \item Not a group. $\frac{1}{2} + \frac{1}{3} = \frac{5}{6}$ which is not in
    the set.
\end{enumerate}

\paragraph{7.}
\begin{proof}
  Since $[a]$ is well defined for all $a \in \mathbb{R}$, it is clear that $x
  \star y$ is well defined. Also, for all $a \in \mathbb{R}$, we have $[a] \leq
  a < [a] + 1$ by definition. It follows that for all $x, y \in G$, \[
    0 \leq (x + y) - [x + y] = x \star y < 1
  \] and so $x \star y \in G$.

  Observe that for any $x \in \mathbb{R}$ and $y \in \mathbb{Z}$, since $[x]
  \leq x < [x] + 1 \implies [x] + y \leq x + y < ([x] + y) + 1$ and $[x] + y \in
  \mathbb{Z}$, we have $[x + y] = [x] + y$. Hence, for any $x, y, z \in
  \mathbb{G}$, \begin{align*}
    (x \star y) \star z
    &= (x + y - [x + y]) \star z \\
    &= (x + y - [x + y]) + z - [(x + y - [x + y]) + z] \\
    &= x + y + z - [x + y] - ([x + y + z] - [x + y]) \because [x + y] \in
    \mathbb{Z} \\
    &= x + y + z - [x + y + z].
  \end{align*} Similarly, \begin{align*}
    x \star (y \star z)
    &= x \star (y + z - [y + z]) \\
    &= x + (y + z - [y + z]) - [x + (y + z - [y + z])] \\
    &= x + y + z - [y + z] - ([x + y + z] - [y + z]) \\
    &= x + y + z - [x + y + z].
  \end{align*}
  Hence, $\star$ is associative. Clearly, $0 \in G$ is the identity element and 
  every $x \in G$ has an inverse $1 - x$ if $x \neq 0$ and $0$ if $x = 0$.
  Therefore $G$ is a group. Since $\star$ is commutative, $G$ is abelian.
\end{proof}

\paragraph{11.}
\begin{align*}
  \overline{0}\text{ is the additive identity}
  &\implies |\overline{0}| = 1, \\
  12(\overline{1}) = \overline{12} = \overline{0}
  &\implies |\overline{1}| = 12, \\
  6(\overline{2}) = \overline{12} = \overline{0}
  &\implies |\overline{2}| = 6, \\
  4(\overline{3}) = \overline{12} = \overline{0}
  &\implies |\overline{3}| = 4, \\
  3(\overline{4}) = \overline{12} = \overline{0}
  &\implies |\overline{4}| = 3, \\
  12(\overline{5}) = \overline{60} = \overline{0}
  &\implies |\overline{5}| = 12, \\
  2(\overline{6}) = \overline{12} = \overline{0}
  &\implies |\overline{6}| = 2, \\
  12(\overline{7}) = \overline{84} = \overline{0}
  &\implies |\overline{7}| = 12, \\
  3(\overline{8}) = \overline{24} = \overline{0}
  &\implies |\overline{8}| = 3, \\
  4(\overline{9}) = \overline{36} = \overline{0}
  &\implies |\overline{9}| = 4, \\
  6(\overline{10}) = \overline{60} = \overline{0}
  &\implies |\overline{10}| = 6, \\
  12(\overline{11}) = \overline{132} = \overline{0}
  &\implies |\overline{11}| = 12.
\end{align*}

\paragraph{15.}
\begin{proof}
  For all $a_1, a_2, \ldots, a_n \in G$, \begin{align*}
    &(a_1a_2 \cdots a_{n-1}a_n) \cdot (a_n^{-1}a_{n-1}^{-1} \cdots
    a_2^{-1}a_1^{-1}) \\
    &= (a_1a_2 \cdots a_{n-1}) \cdot (a_n a_n^{-1}) \cdot
    (a_{n-1}^{-1} \cdots a_2^{-1}a_1^{-1}) \\
    &= (a_1a_2 \cdots a_{n-1}) \cdot 1 \cdot
    (a_{n-1}^{-1} \cdots a_2^{-1}a_1^{-1}) \\
    &= (a_1a_2 \cdots a_{n-1}) \cdot (a_{n-1}^{-1} \cdots a_2^{-1}a_1^{-1}).
  \end{align*}
  This process can be repeated until the expression is reduced to the identity
  $1 \in G$. By the uniqueness of the identity, we have \[
    (a_1a_2 \cdots a_n)^{-1} = a_n^{-1}a_{n-1}^{-1} \cdots a_1^{-1}.
  \]
\end{proof}

\paragraph{17.}
\begin{proof}
  From $|x| = n$ it follows that $x \cdot x^{n-1} = x^n = 1$. By the uniqueness
  of identity, we have $x^{-1} = x^{n-1}$.
\end{proof}

\paragraph{20.}
\begin{proof}
  Suppose $|x| = n$ for some $x \in G$. Then $x^n \cdot (x^{-1})^n = 1$, but
  since $x^n = 1$ we have $(x^{-1})^n = 1 \implies |x^{-1}| \leq n$.

  Now suppose $|x^{-1}| = m < n$. Then \[
    (x^{-1})^m = 1 \implies (x^{-1})^m \cdot x^n = 1 \implies x^{n-m} = 1.
  \] But $n - m \in \mathbb{Z}^+$ and $n - m < n$ which contradicts the
  minimality of $|x|$.
\end{proof}

\paragraph{22.}
\begin{proof}
  Suppose $|x| = n$. Note that $(g^{-1}xg)^k = g^{-1}x^kg$ for all $k \in
  \mathbb{Z}^+$, so \[
    |g^{-1}xg| = k \iff g^{-1}x^kg = 1 \iff x^k = gg^{-1} = 1 \iff k = n
  \] which proves $|x| = |g^{-1}xg|$. Hence, $|ab| = |a^{-1}aba| = |ba|$.
\end{proof}

\paragraph{25.}
\begin{proof}
  Let $x, y \in G$. Then $(xy)^2 = 1$; but \[
    1 = (xy)^2 = xyxy \implies x \cdot 1 \cdot y = x^2yxy^2 \implies xy = yx
  \] since $x^2 = y^2 = 1$. Therefore $G$ is abelian.
\end{proof}

\subsection{Dihedral Groups}

\paragraph{1.}
\begin{enumerate}[(a)]
  \item \begin{align*}
      1 = 1 &\implies |1| = 1, \\
      r^6 = 1 &\implies |r| = 6, \\
      (r^2)^3 = r^6 = 1 &\implies |r^2| = 3, \\
      (r^3)^2 = r^6 = 1 &\implies |r^3| = 2, \\
      (r^4)^3 = r^{12} = 1 &\implies |r^4| = 3, \\
      (r^5)^6 = r^{30} = 1 &\implies |r^5| = 6, \\
      s^2 = 1 &\implies |s| = 2, \\
      (sr^k)^2 = sr^kr^{-k}s = 1 &\implies |sr^k| = 2, \text{ for all } 1
      \leq k \leq n.
    \end{align*}
\end{enumerate}

\paragraph{2.}
\begin{proof}
  From $rs = sr^{-1}$, any product of $s$ and $r$ can be rearranged into the
  form $s^kr^i$ for some $k, i \in \mathbb{Z}$. Since $r^n = s^2 = 1$, we must
  have $x = sr^k$ for some $0 \leq k \leq n$. Hence, \[
    rx = rsr^k = sr^{-1}r^k = sr^kr^{-1} = xr^{-1}
  \] as required.
\end{proof}

\paragraph{3.}
\begin{proof}
  From Q2, we have $x = sr^k$ for some $0 \leq k \leq n$, which is clearly not
  equal to 1. But \[
    x^2 = sr^ksr^k = sr^kr^{-k}s = s^2 = 1,
  \] so $|x| = 2$.

  Since $s(sr) = s^2r = r$, $D_{2n}$ is also generated by $s$ and $sr$, both of
  which have order 2.
\end{proof}

\paragraph{4.}
\begin{proof}
  Let $y \in D_{2n}$. If $y$ is a power of $r$, then clearly $y$ commutes with
  $z$. Otherwise, from Q2 we have $y = sr^m$ for some $0 \leq m \leq n$. So \[
    yz = sr^mr^k = sr^kr^m = r^{-k}sr^m = r^{-k}y = zy
  \] as required, since $z = r^k = r^{-k}$.

  Note that for $x = r^a$ and $y = sr^b$ for some $0 \leq a, b \leq n$, \[
    xy = yx \iff r^asr^b = sr^br^a \iff sr^{b-a} = sr^{b+a} \iff r^a = r^{-a}
  \] which implies that $a = 0\text{ or }k$. It follows that $x =
  1\text{ or }r^k$, so $z = r^k$ is the only nonidentity element of $D_{2n}$
  which commutes with all other elements.
\end{proof}

\paragraph{5.}
\begin{proof}
  Similarly to Q4, for $x = r^a$ and $y = sr^b$ for some $0 \leq a, b \leq n$, 
  $xy = yx \iff r^a = r^{-a}$. Since $n \geq 3$, we have $2a = 0\text{ or }n$.
  But since $n$ is odd, it follows that $a = 0$ and $x = 1$, so only the
  identity element commutes with all elements of $D_{2n}$.
\end{proof}

\paragraph{6.}
\begin{proof}
  Since $x^2 = 1 = y^2$, it follows that $x = x^{-1}$ and $y = y^{-1}$. Hence \[
    xt^{-1} = x(xy)^{-1}xy^{-1}x^{-1} = xyx = tx.
  \]
\end{proof}

\paragraph{7.}
\begin{proof}
  From $a^2 = b^2 = (ab)^n = 1$ we have $s^2 = a^2 = 1$, and so $r^n = (ssr)^n
  = (ab)^n = 1$. Also, \[
    b^2 = (sr)^2 = 1 \iff sr = (sr)^{-1} = r^{-1}s^{-1} = r^{-1}s
  \] since $s = s^{-1}$, and so $rs = r(sr)r^{-1} = r(r^{-1}s)r^{-1} = sr^{-1}$.

  Conversely, from $r^n = s^2 = 1$ and $rs = sr^{-1}$, we have $a^2 = s^2 = 1$,
  $b^2 = (sr)^2 = srsr = s(sr^{-1})r = s^2 = 1$, and $(ab)^n = (ssr)^n = r^n$.
\end{proof}

\subsection{Symmetric Groups}

\paragraph{1.}
\begin{align*}
  \sigma &= (1\ 3\ 5)(2\ 4) \\
  \tau &= (1\ 5)(2\ 3) \\
  \sigma^2 &= (1\ 5\ 3) \\
  \sigma\tau &= (2\ 5\ 3\ 4) \\
  \tau\sigma &= (1\ 2\ 4\ 3) \\
  \tau^2\sigma &= (1\ 3\ 5)(2\ 4)
\end{align*}

\paragraph{4.}
\begin{enumerate}[(a)]
  \item \begin{align*}
      |1| &= 1 \\
      |(1 2)| &= 2 \\
      |(1 3)| &= 2 \\
      |(2 3)| &= 2 \\
      |(1 2 3)| &= 3 \\
      |(1 3 2)| &= 3
  \end{align*}
\end{enumerate}

\paragraph{8.}
\begin{proof}
  Suppose that $S_\Omega$ is a finite group and let $n$ be the order of
  $S_\Omega$. Pick some $m \in \mathbb{Z}$ such that $n < m!$. Clearly, we have
  $S_m \subseteq S_\Omega$, since each permutation in $S_m$ can be extended to
  a permutation of $S_\Omega$ by mapping all numbers greater than $m$ to
  themselves. But this implies that $|S_m| \subseteq |S_\Omega| \iff m! \leq n$
  which is absurd.
\end{proof}

\paragraph{9.}
\begin{enumerate}[(a)]
  \item $i = 1, 5, 7, 11$
  \item $i = 1, 3, 5, 7$
  \item $i = 1, 3, 5, 9, 11, 13$
\end{enumerate}

\paragraph{10.}
\begin{proof}
  We prove this statement by induction on $i$.

  \textbf{Base case}: Suppose $i = 1$. Clearly, for $k \in
  \{1, 2, \ldots, m-1\}$ we have $\sigma(a_k) = a_{k+1}$ since $k + 1 \leq m$.
  If $k = m$, then by definition $\sigma(a_k) = \sigma(a_m) = a_1$. But $a_1 =
  a_{m+1} = a_{k+1}$ since the least residue $m+1$ mod $m$ is $1$, which
  proves the base case.

  \textbf{Inductive hypothesis}: Assume that for some $i \in \mathbb{Z}^+$,
  $\sigma^i(a_k) = a_{k+i}$ for $k \in \{1, 2, \ldots, m\}$.

  \textbf{Inductive step}: Consider $\sigma^{i+1}(a_k)$ for $k \in \{1, 2,
  \ldots, m\}$. Since $\sigma(a_k) = a_{k+1}$ by the base case, we have \[
    \sigma^{i+1}(a_k) = \sigma^i(\sigma(a_k)) = \sigma^i(a_{k+1}).
  \] If $k + 1 \leq m$, then $\sigma^i(a_{k+1}) = a_{k+1+i}$ by the inductive
  hypothesis. Otherwise, we have $k = m$, so $a_{k+1} = a_1$ by the base case
  and \[
    \sigma^i(a_{k+1}) = \sigma^i(a_1) = a_{1+i} = a_{m+i+1} = a_{k+i+1}
  \] since the least residue of $m+i+1$ mod $m$ is the same as that of $i+1$.
\end{proof}

\paragraph{11.}
\begin{proof}
  ($\Rightarrow$): Suppose $\sigma^i$ is a $m$-cycle. Let $d = \gcd(i, m)$ and
  $i = dp$, $m = dq$ for some $p, q \in \mathbb{Z}^+$. Then we have \[
    \sigma^d = (1\ d+1\ \ldots\ (q-1)d+1)
  \] since $\sigma^d(x) = x+d$ for $x \in \{1, \ldots, m\}$, where $x+d$ is
  replaced by its least residue mod $m$ when $x+d > m$. It follows that
  $dq = m = |\sigma^i| = |(\sigma^d)^p| \leq |\sigma^d| = q$, so we must have
  $d = 1$ and $i$ relatively prime to $m$.

  \textbf{($\Leftarrow$)}: Suppose $i$ is relatively prime to $m$. Let $x, y \in
  \{1, i+1, \ldots, (m-1)i+1\}$ such that the least residues of $x$ and $y$ mod
  $m$ are equal. WLOG, assume $x \geq y$. Then $m \mid x - y = di$ for some $d
  \in \{0, 1, \ldots, m-1\}$. Since $i$ and $m$ are relatively prime, $m \mid d$
  and so $d = 0$, which implies $x = y$. Therefore it follows that \[
    \{1, \sigma^i(1), \ldots, \sigma^{(m-1)i}(1)\}
    = \{1, i+1, \ldots, (m-1)i+1\}
    = \{1, 2, \ldots, m-1\}
  \] where each element is replaced by its least residue mod $m$, so $|\sigma^i|
  = m$.
\end{proof}

\paragraph{13.}
\begin{proof}
  ($\Rightarrow$): Let $\sigma \in S_n$ such that $|\sigma| = 2$. Suppose
  $\sigma = \sigma_1\tau\sigma_2$ for some disjoint $\sigma_1, \sigma_2, \tau
  \in S_n$ where $\tau$ is a $k$-cycle with $k > 2$. Then we have $|\sigma| =
  |\sigma_1\tau\sigma_2| \geq |\tau| = k > 2$ which is absurd.

  ($\Leftarrow$): Let $\sigma = \sigma_1\ldots\sigma_m \in S_n$ for some $m \in
  \mathbb{Z}^+$ and disjoint 2-cycles $\sigma_1, \ldots, \sigma_m$. Then we have
  \[
    |\sigma| \geq \max\{|\sigma_1|, \ldots, |\sigma_m|\} = 2.
  \] Consider some $x \in \{1, \ldots, n\}$. Then either $\sigma_i(x) = x$
  for all $i \in \{1, \ldots, m\}$, which implies $\sigma(x) = x$, or there
  exists exactly one $\sigma_i$ (by the disjointness of $\sigma_1, \ldots,
  \sigma_m$) such that $\sigma_i(x) \neq x$ and $\sigma_i^2(x) = x$, which
  implies that $\sigma^2(x) = x$. It follows that $|\sigma| \leq 2$.
\end{proof}

\end{document}
