\documentclass{article}

\usepackage{amsmath,amssymb,amsthm}
\usepackage{hyperref}

\newtheorem{theorem}{Theorem}
\newtheorem{lemma}{Lemma}

\begin{document}

\title{Assignment 1: Problems on "Hidden" Independence and Uniformity}
\maketitle

The exercises are available at \href{https://ocw.mit.edu/courses/mathematics/18-a34-mathematical-problem-solving-putnam-seminar-fall-2018/assignments/MIT18_A34F18PS1.pdf}{here}.

\paragraph{1.} Slips of paper with the numbers from 1 to 99 are placed in a
hat. Five numbers are randomly drawn out of the hat one at a time (without
replacement). What is the probability that the numbers are chosen in increasing
order?

\paragraph{Solution:} Noting that each way of drawing the five numbers is
equiprobable, it suffices to consider the number of ways to draw the five
numbers in any order, and in increasing order.

There is a bijection between the set of ways to draw the five numbers in
increasing order and the set of subsets of $\{1, 2, \ldots, 99\}$ with exactly
five numbers. For every such subset, there are $5! = 120$ possible arrangements
of the five numbers in the subset. Clearly, every arrangement of the subsets
is unique and represents all ways of drawing the five numbers out of the hat.
Hence, the probability is $1/120$.

\paragraph{2.} In how many ways can a positive integer $n$ be written as a sum
of positive integers, taking order into account? For example, 4 can be written
as a sum in the eight ways $4 = 3 + 1 = 1 + 3 = 2 + 2 = 2 + 1 + 1 = 1 + 2 + 1 =
1 + 1 + 2 = 1 + 1 + 1 + 1$.

\paragraph{Solution:} For any such sum, it suffices to consider all integers
except the last integer, since its value must be determined by all of the other
integers. One can represent the sum with the use of a tuple, e.g. taking $n =
12$, $3 + 2 + 4 + 1$ can be represented by $(3, 2, 4)$, while $12$ is
represented by $()$, the 0-tuple.

Construct a bijection between the set of all such tuples and the set of all
subsets of $\{1, 2, \ldots, n -  1\}$, with each element in the subset
representing the cumulative sum of the integers in the tuple from left to
right, e.g. $(3, 2, 4)$ is represented by $\{3, 3 + 2, 3 + 2 + 4\} = \{3, 5,
9\}$, and $()$ is represented by $\{\}$. Then, the number of ways must be
$2^{n - 1}$.

\end{document}

