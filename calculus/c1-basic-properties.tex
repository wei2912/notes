\documentclass{article}

\usepackage{amsmath,amssymb,amsthm}

\newtheorem{theorem}{Theorem}
\newtheorem{lemma}{Lemma}

\begin{document}

\title{Chapter 1: Basic Properties of Numbers}
\maketitle

\paragraph{Numbers.} Of the first twelve properties in this chapter, the first
nine are concerned with the fundamental operations of addition and
multiplication.

\begin{tabular}{l p{4in}}
  (P1) & (Associative law for addition) $a + (b + c) = (a + b) + c$. \\
  (P2) & (Existence of an additive identity) $a + 0 = 0 + a = a$. \\
  (P3) & (Existence of additive inverses) $a + (-a) = (-a) + a = 0$. \\
  (P4) & (Commutative law for addition) $a + b = b + a$. \\
  (P5) & (Associative law for multiplication) $a \cdot (b \cdot c) = (a \cdot
    b) \cdot c$. \\
  (P6) & (Existence of a multiplicative identity) $a \cdot 1 = 1 \cdot a = a; 1
    \neq 0$. \\
  (P7) & (Existence of a multiplicative inverse) $a \cdot a^{-1} = a^{-1} \cdot
    a = 1, \text{ for } a \neq 0$. \\
  (P8) & (Commutative law for multiplication) $a \cdot b = b \cdot a$. \\
  (P9) & (Distributive law) $a \cdot (b + c) = a \cdot b + a \cdot c$.
\end{tabular}

The last three are concerned with inequalities. Considering the collection of
all positive numbers, $P$,

\begin{tabular}{l p{4in}}
  (P10) & (Trichotomy law) For every number $a$, one and only one of the
    following holds: \begin{enumerate}
      \itemsep0em
      \item $a = 0$,
      \item $a$ is in the collection $P$,
      \item $-a$ is in the collection $P$.
    \end{enumerate} \\
  (P11) & (Closure under addition) If $a$ and $b$ are in $P$, then $a + b$ is
    in $P$. \\
  (P12) & (Closure under multiplication) If $a$ and $b$ are in $P$, then $a
    \cdot b$ is in $P$.
\end{tabular}

\section*{Exercises}

\paragraph{Problem 1-8 (abridged).} Suppose that P10-12 are replaced by the
following properties:

\begin{tabular}{l p{4in}}
  (P'10) & For every number $a$ and $b$, one and only one of the following
    holds: \begin{enumerate}
      \itemsep0em
      \item $a = b$,
      \item $a < b$,
      \item $b < a$.
    \end{enumerate} \\
  (P'11) & For any numbers $a$, $b$, and $c$, if $a < b$ and $b < c$, then $a <
    c$. \\
  (P'12) & For any numbers $a$, $b$, and $c$, if $a < b$, then $a + c < b + c$.
    \\
  (P'13) & For any numbers $a$, $b$, and $c$, if $a < b$ and $0 < c$, then $ac
    < bc$. \\
\end{tabular}

Show that P10-12 can then be deduced as theorems.

\paragraph{Solution:} P10 can be deduced by considering P'10 with $b = 0$.
Then, for every number $a$, one and only one of the following holds:
\begin{enumerate}
  \itemsep0em
  \item $a = 0$,
  \item $a < 0$ i.e. $-a$ is in $P$,
  \item $0 < a$ i.e. $a$ is in $P$.
\end{enumerate}

P11 can be deduced by considering P'12 with $a = 0$. Then, for any numbers $b$
and $c$, if $0 < b$, $0 + c < b + c$. If $0 < c$ too, then $0 < 0 + c < b + c$.
Therefore, if $0 < b$ and $0 < c$, $0 < b + c$ i.e. if $b$ and $c$ are in $P$,
then $b + c$ is in $P$.

P12 can be deduced by considering P'11 and P'13. From P'11, for any numbers $a$
and $b$, if $0 < a$ and $a < b$, then $0 < b$. Hence, from P'13, for any
numbers $a$ and $b$, if $0 < a$ and $0 < b$, then $0 \cdot a < a \cdot b$ i.e.
if $a$ and $b$ are in $P$, then $ab$ is in $P$.

\paragraph{Problem 1-15 (abridged).} Prove that if $x$ and $y$ are not both 0,
then \begin{gather*}
  x^2 + xy + y^2 > 0, \\
  x^4 + x^3y + x^2y^2 + xy^3 + y^4 > 0.
\end{gather*}

\paragraph{Solution:} From Problem 1, $x^n - y^n = (x - y)(x^{n-1} + x^{n-2}y +
\cdots + xy^{n-2} + y^{n-1})$. Without loss of generality, assume $x > y$.
Considering these two inequalities, \begin{gather*}
  x^3 - y^3 > 0, \\
  x^4 - y^4 > 0
\end{gather*} and factorising out $x - y > 0$, \begin{gather*}
  (x - y)(x^2 + xy + y^2) > 0 \iff x^2 + xy + y^2 > 0 \\
  (x - y)(x^3 + x^2y + xy^2 + y^3) > 0 \iff x^3 + x^2y + xy^2 + y^3 > 0.
\end{gather*}
This concludes the proof.

\paragraph{Problem 1-16. (a)} Show that \begin{gather*}
  (x + y)^2 = x^2 + y^2 \text{ only when } x = 0 \text{ or } y = 0, \\
  (x + y)^3 = x^3 + y^3 \text{ only when } x = 0 \text{ or } y = 0 \text{ or }
  x = -y.
\end{gather*}

\paragraph{Solution:} $(x + y)^2 = x^2 + 2xy + y^2$. Clearly, for
$(x + y)^2 = x^2 + y^2$, $2xy = 0$ which implies $x = 0$ or $y = 0$. Similarly,
$(x + y)^3 = x^3 + 3x^2y + 3xy^2 + y^3$, and for $(x + y)^3 = x^3 + y^3$,
$3x^2y + 3xy^2 = 3xy(x + y) = 0$ which implies $x = 0$ or $y = 0$ or $x = -y$.

\paragraph{(b)} Using the fact that \begin{equation*}
  x^2 + 2xy + y^2 = (x + y)^2 \geq 0,
\end{equation*} show that $4x^2 + 6xy + 4y^2 > 0$ unless $x$ and $y$ are both
0.

\paragraph{Solution:} $4x^2 + 6xy + 4y^2 = (x^2 + y^2) + 3(x + y)^2 \geq
x^2 + y^2 + 0 = x^2 + y^2$. However, $x^2 + y^2 = 0$ only when $x$ and $y$ are
both 0; otherwise, $x^2 + y^2 > 0$.  Hence, it is clear that $4x^2 + 6xy + 4y^2
> 0$ unless $x$ and $y$ are both 0.

\paragraph{(c)} Use part (b) to find out when $(x + y)^4 = x^4 + y^4$.

\paragraph{Solution:} $(x + y)^4 = x^4 + 4x^3y + 6x^2y^2 + 4xy^3 + y^4 =
(x^4 + y^4) + xy(4x^2 + 6xy + 4y^2)$. It is now obvious that for $(x + y)^4 =
x^4 + y^4$, $xy(4x^2 + 6xy + 4y^2) = 0$ which implies $x = 0$ or $y = 0$ or
$4x^2 + 6xy + 4y^2 = 0$. From (b), $4x^2 + 6xy + 4y^2 > 0$ unless $x$ and $y$
are both 0. Hence, $(x + y)^4 = x^4 + y^4$ only when $x = 0$ or $y = 0$.

\paragraph{(d)} Find out when $(x + y)^5 = x^5 + y^5$.

\paragraph{Solution:} $(x + y)^5 = x^5 + 5x^4y + 10x^3y^2 + 10x^2y^3 +
5xy^4 + y^5 = (x^5 + y^5) + 5xy(x^3 + 2x^2y + 2xy^2 + y^3)$. It is clear that
for $(x + y)^5 = x^5 + y^5$, $5xy(x^3 + 2x^2y + 2xy^2 + y^3) = 0$ which implies
$x = 0$, $y = 0$ or $x^3 + 2x^2y + 2xy^2 + y^3 = 0$. The problem simplifies to
solving $x^3 + 2x^2y + 3xy^2 + y^3 = 0$ if $xy \neq 0$.

Considering $(x + y)^3 = x^3 + 3x^2y + 3xy^2 + y^3 = (x^3 + 2x^2y + 2xy^2 +
y^3) + xy(x + y)$, for $x^3 + 2x^2y + 2xy^2 + y^3 = 0$ if $xy \neq 0$, $(x +
y)^3 = xy(x + y)$. Since $xy \neq 0$, $x = -y$.

Hence, $(x + y)^5 = x^5 + y^5$ only when $x = 0$ or $y = 0$ or $x = -y$.

\paragraph{Problem 1-21 (abridged).} Prove that if \begin{equation*}
  |x - x_0| < \mathrm{min}\left(\frac{\epsilon}{2(|y_0| + 1)}, 1\right)
    \text{ and } |y - y_0| < \frac{\epsilon}{2(|x_0| + 1)},
\end{equation*} then $|xy - x_0y_0| < \epsilon$.

\paragraph{Solution:} The problem can be rewritten in terms of three
inequalities \begin{align*}
  |x - x_0| &< \frac{\epsilon}{2(|y_0| + 1)}, \\
  |x - x_0| &< 1, \\
  |y - y_0| &< \frac{\epsilon}{2(|x_0| + 1)}
\end{align*} with the first and third inequality further rewritten into
\begin{align*}
  (|x_0| + 1)|y - y_0| &< \frac{\epsilon}{2}, \\
  (|y_0| + 1)|x - x_0| &< \frac{\epsilon}{2}.
\end{align*}

By the reverse triangle inequality, $|x| - |x_0| \leq |x - x_0| < 1$ which
yields $|x| < |x_0| + 1$. Rewritting $xy - x_0y_0 = x(y - y_0) + y_0(x - x_0)$
and applying the triangle inequality, \begin{align*}
  |xy - x_0y_0| &\leq |x(y - y_0)| + |y_0(x - x_0)| \\
    &= |x||y - y_0| + |y_0||x - x_0| \\
    &< (|x_0| + 1)|y - y_0| + |y_0||x - x_0| \\
    &< (|x_0| + 1)|y - y_0| + (|y_0| + 1)|x - x_0| \\
    &< \frac{\epsilon}{2} + \frac{\epsilon}{2} \\
    &= \epsilon.
\end{align*}

\paragraph{Problem 1-22.} Prove that if $y_0 \neq 0$ and \begin{equation*}
  |y - y_0| < \mathrm{min}\left(\frac{|y_0|}{2}, \frac{\epsilon|y_0|^2}{2}
    \right),
\end{equation*} then $y \neq 0$ and \begin{equation*}
  \left|\frac{1}{y} - \frac{1}{y_0}\right| < \epsilon.
\end{equation*}

\paragraph{Solution:} From the assumption, since $|y - y_0| <
\frac{\epsilon|y_0|^2}{2}$, \begin{align*}
  \left|\frac{1}{y} - \frac{1}{y_0}\right| &= \frac{|y_0 - y|}{|yy_0|} \\
    &= \frac{|y - y_0|}{|y||y_0|} \\
    &< \frac{\frac{\epsilon|y_0|^2}{2}}{|y||y_0|} \\
    &= \frac{\epsilon|y_0|}{2|y|}.
\end{align*}

By the triangle inequality, $|y_0| = |(y_0 - y) + y| \leq |y_0 - y| + |y| = |y
- y_0| + |y|$. Then, $|y| = |y_0| - |y - y_0|$. From the assumption, $|y - y_0|
< \frac{|y_0|}{2}$ which leads to $|y| > |y_0| - \frac{|y_0|}{2} =
\frac{|y_0|}{2} \iff 2|y| > |y_0|$. Hence, $\frac{\epsilon|y_0|}{2|y|} <
\frac{\epsilon|y_0|}{|y_0|} = \epsilon$.

\paragraph{Problem 1-23 (abridged).} Replace the question marks in the
following statement by expressions involving $\epsilon$, $x_0$, and $y_0$ so
that the conclusion will be true:

If $y_0 \neq 0$ and \begin{equation*}
  |y - y_0| < ? \text{ and } |x - x_0| < ?
\end{equation*} then $y \neq 0$ and \begin{equation*}
  \left|\frac{x}{y} - \frac{x_0}{y_0}\right| < \epsilon.
\end{equation*}

\paragraph{Solution:} Replacing $y$ and $y_0$ with $\frac{1}{y}$ and
$\frac{1}{y_0}$ respectively in the result proven by Q21, if \begin{equation*}
  |x - x_0| < \mathrm{min}\left(\frac{\epsilon}{2(|\frac{1}{y_0}| + 1)},
  1\right) \text{ and } \left|\frac{1}{y} - \frac{1}{y_0}\right| <
  \frac{\epsilon}{2(|x_0| + 1)}
\end{equation*} then $|x(\frac{1}{y}) - x_0(\frac{1}{y_0})| = \left|\frac{x}{y}
- \frac{x_0}{y_0}\right| < \epsilon$ which is what we are looking to prove.

To express the second inequality in terms of $|y - y_0|$, $\epsilon$ must be
substituted with $\frac{\epsilon}{2(|x_0| + 1)}$ in the result proven by Q22
which leads to \begin{equation*}
  |y - y_0| < \mathrm{min}\left(\frac{|y_0|}{2},
  \frac{\epsilon|y_0|^2}{4(|x_0| + 1)}\right).
\end{equation*}
Therefore, this proves that if $y_0 \neq 0$ and \begin{equation*}
  |x - x_0| < \mathrm{min}\left(\frac{\epsilon}{2(|\frac{1}{y_0}| + 1)},
  1\right) \text{ and } |y - y_0| < \mathrm{min}\left(\frac{|y_0|}{2},
  \frac{\epsilon|y_0|^2}{4(|x_0| + 1)}\right)
\end{equation*} then $y \neq 0$ and \begin{equation*}
  \left|\frac{x}{y} - \frac{x_0}{y_0}\right| < \epsilon.
\end{equation*}

\paragraph{Problem 1-24 (abridged).} Let us agree, for definiteness, that $a_1
+ \cdots + a_n$ will denote \begin{equation*}
  a_1 + (a_2 + (a_3 + \cdots + (a_{n-2} + (a_{n-1} + a_n)))\cdots).
\end{equation*}
Thus $a_1 + a_2 + a_3$ denotes $a_1 + (a_2 + a_3)$, and $a_1 + a_2 + a_3 + a_4$
denotes $a_1 + (a_2 + (a_3 + a_4))$, etc.

\paragraph{(a)} Prove that \begin{equation*}
  (a_1 + \cdots + a_k) + a_{k+1} = a_1 + \cdots + a_{k+1}.
\end{equation*}

\paragraph{Solution:} The base case $k = 1$ is trivial, since $(a_1) + a_2
= a_1 + a_2$.

Assuming that $(a_1 + \cdots + a_{k-1}) + a_k = a_1 + \cdots + a_k$ for some $k
\geq 2$, \begin{align*}
  (a_1 + \cdots + a_k) + a_{k+1} &= ((a_1 + \cdots + a_{k-1}) + a_k) + a_{k+1}
  \text{ (by assumption)} \\
    &= (a_1 + \cdots + a_{k-1}) + (a_k + a_{k+1}) \text{ (by P1)} \\
    &= a_1 + \cdots + a_{k-1} + (a_k + a_{k+1}) \text{ (by assumption)} \\
    &= a_1 + \cdots + a_{k+1} \text{ (by the definition of } a_1 + \cdots +
    a_{k+1} \text{)}
\end{align*} This proves the inductive hypothesis.

\paragraph{(b)} Prove that if $n \geq k$, then \begin{equation*}
  (a_1 + \cdots + a_k) + (a_{k+1} + \cdots + a_n) = a_1 + \cdots + a_n.
\end{equation*}

\paragraph{Solution:} The base case $n = k$ is trivial.

Assuming that $(a_1 + \cdots + a_k) + (a_{k+1} + \cdots + a_n) = a_1 + \cdots +
a_n$ for some $n \geq k + 1$, \begin{align*}
  (a_1 + \cdots + a_k) + (a_{k+1} + \cdots + a_n + a_{n+1}) &= (a_1 + \cdots +
  a_k) + ((a_{k+1} + \cdots + a_n) + a_{n+1}) \text{ (from (a))} \\
    &= ((a_1 + \cdots + a_k) + (a_{k+1} + \cdots + a_n)) + a_{n+1} \text{ (from
    P1)} \\
    &= (a_1 + \cdots + a_n) + a_{n+1} \text{ (by assumption)} \\
    &= a_1 + \cdots + a_{n+1} \text{ (from (a))}.
\end{align*} This proves the inductive hypothesis.

\paragraph{(c)} Let $s(a_1, \ldots, a_k)$ be some sum formed from
$a_1, \ldots, a_k$. Show that \begin{equation*}
  s(a_1, \ldots, a_k) = a_1 + \ldots + a_k.
\end{equation*}

\paragraph{Solution:} The base case $k = 1$ is trivial.

Suppose that this statement holds for all sums comprising less than $k$ terms.
Then, there must be two sums $s'(a_1, \ldots, a_l)$ and $s''(a_{l+1}, \ldots,
a_k)$ with $l < k$ such that \begin{align*}
  s(a_1, \ldots, a_k) &= s'(a_1, \ldots, a_l) + s''(a_{l+1}, \ldots, a_k) \\
    &= (a_1 + \ldots + a_l) + (a_{l+1} + \ldots + a_k) \text{ (by assumption)}
    \\
    &= a_1 + \ldots + a_k \text{ (from (b))}.
\end{align*} This proves the inductive hypothesis.

\end{document}
