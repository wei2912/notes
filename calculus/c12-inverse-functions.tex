\documentclass{article}

\usepackage{amsmath,amssymb,amsthm}

\newtheorem{corollary}{Corollary}
\newtheorem{definition}{Definition}
\newtheorem{theorem}{Theorem}

\begin{document}

\title{Chapter 12: Inverse Functions}
\maketitle

\begin{definition}
  A function $f$ is \emph{one-one} if $f(a) \neq f(b)$ whenever $a \neq b$.
\end{definition}

\begin{definition}
  For any function $f$, the \emph{inverse} of $f$, denoted by $f^{-1}$, is the
  set of all pairs $(a, b)$ for which the pair $(b, a)$ is in $f$.
\end{definition}

\begin{theorem}
  $f^{-1}$ is a function if and only if $f$ is one-one.
\end{theorem}

\begin{proof}
  Suppose first that $f$ is one-one. Let $(a, b)$ and $(a, c)$ be two pairs in
  $f^{-1}$. Then $(b, a)$ and $(c, a)$ are in $f$, so $a = f(b)$ and $a =
  f(c)$; since $f$ is one-one, this implies that $b = c$. Thus $f^{-1}$ is a
  function.

  Conversely, suppose that $f^{-1}$ is a function. If $f(b) = f(c)$, then $f$
  contains the pairs $(b, f(b))$ and $(c, f(c)) = (c, f(b))$, so $(f(b), b)$
  and $(f(b), c)$ are in $f^{-1}$. Since $f^{-1}$ is a function this implies
  that $b = c$. Thus $f$ is one-one.
\end{proof}

\begin{theorem}
  If $f$ is continuous and one-one of an interval, then $f$ is either
  increasing or decreasing on that interval.
\end{theorem}

\begin{proof}
  Let $a_0 < b_0$ be two numbers in the interval. Since $f$ is one-one, we know
  that \begin{align*}
    \text{either} &&\text{ (i) } &f(b_0) - f(a_0) > 0 \\
    \text{or} &&\text{ (ii) } &f(b_0) - f(a_0) < 0.
  \end{align*} We will assume that (i) is true, and show that the same
  inequality holds for any $a_1 < b_1$ in the interval, so that $f$ is
  increasing. (A similar argument shows that if (ii) is true, then $f$ is
  decreasing.)

  Let \begin{align*}
    x_t &= (1 - t)a_0 + ta_1 \\
    y_t &= (1 - t)b_0 + tb_1
  \end{align*} for $0 \leq t \leq 1$. Then $x_0 = a_0$ and $x_1 = a_1$ and the
  points $x_t$ all lie between $a_0$ and $a_1$. An analogous statement holds
  for $y_t$. So $x_t$ and $y_t$ are all in the domain of $f$. Moreover, since
  $a_0 < b_0$ and $a_1 < b_1$, we also have \begin{equation*}
    x_t < y_t \text{ for } 0 \leq t \leq 1.
  \end{equation*} Now consider the function \begin{equation*}
    g(t) = f(y_t) - f(x_t) \text{ for } 0 \leq t \leq 1.
  \end{equation*} Using Theorem 6-2, it is easy to see that $g$ is continuous
  on $[0, 1]$. Moreover, $g(t)$ is never 0, since $x_t < y_t$ and $f$ is
  one-one. Consequently, $g(t)$ is either positive for all $t$ in $[0, 1]$ or
  negative for all $t$ in $[0, 1]$ (otherwise, by the Intermediate Value
  Theorem it would also be 0 somewhere in $[0, 1]$). But $g(0) > 0$ by (i). So
  also $g(1) > 0$, which means that (i) also holds for $a_1, b_1$.
\end{proof}

Moving on, we now consider which important properties of a one-one function are
inherited by its inverse.

\begin{theorem}
  If $f$ is continuous and one-one on an interval, then $f^{-1}$ is also
  continuous.
\end{theorem}

\begin{proof}
  We know by Theorem 2 that $f$ is either increasing or decreasing. We might as
  well assume that $f$ is increasing, since we can then take care of the other
  case by applying the usual trick of considering $-f$.

  We must show that \begin{equation*}
    \lim_{x \rightarrow b}f^{-1}(x) = f^{-1}(b)
  \end{equation*} for each $b$ in the domain of $f^{-1}$. Such a number $b$ is
  of the form $f(a)$ for some $a$ in the domain of $f$. For any $\epsilon > 0$,
  we want to find a $\delta > 0$ such that, for all $x$, \begin{equation*}
    \text{if } f(a) - \delta < x < f(a) + \delta, \text{ then } a - \epsilon <
      f^{-1}(x) < a + \epsilon.
  \end{equation*} Since $a - \epsilon < a < a + \epsilon$, it follows that
  $f(a - \epsilon) < f(a) < f(a + \epsilon)$; we let $\delta$ be the smaller of
  $f(a + \epsilon) - f(a)$ and $f(a) - f(a - \epsilon)$.

  Our choice of $\delta$ ensures that \begin{equation*}
    f(a - \epsilon) \leq f(a) - \delta \text{ and } f(a) + \delta \leq f(a +
      \epsilon).
  \end{equation*} Consequently, if \begin{equation*}
    f(a) - \delta < x < f(a) + \delta.
  \end{equation*} then \begin{equation*}
    f(a - \epsilon) < x < f(a + \epsilon).
  \end{equation*} Since $f$ is increasing, $f^{-1}$ is also increasing, and we
  obtain \begin{equation*}
    f^{-1}(f(a - \epsilon)) < f^{-1}(x) < f^{-1}(f(a + \epsilon)),
  \end{equation*} i.e., \begin{equation*}
    a - \epsilon < f^{-1}(x) < a + \epsilon,
  \end{equation*} which is precisely what we want.
\end{proof}

The proof of differentiability is more involved.

\begin{theorem}
  If $f$ is a continuous one-one function defined on an interval and $f'(f^{-1}
  (a)) = 0$, then $f^{-1}$ is \emph{not} differentiable at $a$.
\end{theorem}

\begin{proof}
  We have \begin{equation*}
    f(f^{-1}(x)) = x.
  \end{equation*} If $f^{-1}$ were differentiable at $a$, the Chain Rule would
  imply that \begin{equation*}
    f'(f^{-1}(a)) \cdot (f^{-1})'(a) = 1,
  \end{equation*} hence \begin{equation*}
    0 \cdot (f^{-1})'(a) = 1,
  \end{equation*} which is absurd.
\end{proof}

\begin{theorem}
  Let $f$ be a continuous one-one function defined on an interval, and suppose
  that $f$ is differentiable at $f^{-1}(b)$, with derivative $f'(f^{-1}(b))
  \neq 0$. Then $f^{-1}$ is differentiable at $b$, and \begin{equation*}
    (f^{-1})'(b) = \frac{1}{f'(f^{-1}(b))}.
  \end{equation*}
\end{theorem}

\begin{proof}
  Let $b = f(a)$. Then \begin{equation*}
    \lim_{h \rightarrow 0}\frac{f^{-1}(b + h) - f^{-1}(b)}{h} = \lim_{h
    \rightarrow 0}\frac{f^{-1}(b + h) - a}{h}.
  \end{equation*} Now every number $b + h$ in the domain of $f^{-1}$ can be
  written in the form \begin{equation*}
    b + h = f(a + k)
  \end{equation*} for an unique $k$. Then \begin{equation*}
    \lim_{h \rightarrow 0}\frac{f^{-1}(b + h) - a}{h} = \lim_{h \rightarrow 0}
      \frac{f^{-1}(f(a + k)) - a}{f(a + k) - b} = \lim_{h \rightarrow 0}
      \frac{k}{f(a + k) - f(a)}.
  \end{equation*} It is not hard to get an explicit expression for k; since
  \begin{equation*}
    b + h = f(a + k)
  \end{equation*} we have \begin{equation*}
    f^{-1}(b + h) = a + k
  \end{equation*} or \begin{equation*}
    k = f^{-1}(b + h) - f^{-1}(b).
  \end{equation*} Now by Theorem 3, the function $f^{-1}$ is continuous at $b$.
  This means that $k$ approaches 0 as $h$ approaches 0. Since \begin{equation*}
    \lim_{k \rightarrow 0}\frac{f(a + k) - f(a)}{k} = f'(a) = f'(f^{-1}(b))
      \neq 0,
  \end{equation*} this implies that \begin{equation*}
    (f^{-1})'(b) = \frac{1}{f'(f^{-1}(b))}.
  \end{equation*}
\end{proof}

\section*{Exercises}

\paragraph{Problem 12-11 (abridged). (a)} Prove that there is a differentiable
function $f$ such that $[f(x)]^5 + f(x) + x = 0$ for all $x$.

\paragraph{Solution:} Let $f = g^{-1}$ where $g(x) = -x^5 - x$. Notice that $g$
is one-one, since $g'(x) = -5x^4 - 1 < 0$ for all $x$, and that $g$ takes on
all values. So $f$ is defined on $\mathbb{R}$, and for all $x$ we have
\begin{equation*}
  x = g(f(x)) = -[f(x)]^5 - f(x).
\end{equation*} Moreover, $f$ is differentiable, since $g'(x) \neq 0$ for all
$x$.

\paragraph{(b)} Find $f'$ in terms of $f$, using an appropriate theorem of this
chapter.

\paragraph{Solution:} By Theorem 5, as $g$ is a differentiable one-one function
defined on an interval, with derivative $g'(x) \neq 0$ for all $x$, Then for
all $b$, $f = g^{-1}$ is differentiable and \begin{equation*}
  (g^{-1})'(b) = \frac{1}{g'(g^{-1}(b))}
\end{equation*} or \begin{equation*}
  f'(b) = \frac{1}{g'(f(b))} = \frac{1}{-5[f(b)]^4 - 1}.
\end{equation*}

\paragraph{(c)} Find $f'$ in another way, by simply differentiating the
equation defining $f$.

\paragraph{Solution:} Differentiating the equation defining $f$,
\begin{equation*}
  5[f(x)]^4f'(x) + f'(x) + 1 = 0
\end{equation*} or \begin{equation*}
  f'(x) = -\frac{1}{5[f(x)]^4 + 1}
\end{equation*} which is equivalent to the function found in (b).

\paragraph{Problem 12-23 (abridged). (a)} If $f$ is a continuous function on
$\mathbb{R}$ and $f = f^{-1}$, prove that there is at least one $x$ such that
$f(x) = x$.

\paragraph{Solution:} Suppose that $f(a) > a$ for some $a$ (such an $a$ must
exist given the symmetry of $y = f(x)$ about $y = x$); then, $f(f(a)) = a <
f(a)$. Since $f(a) > a$, and $f(b) < b$ for some $b$ (namely, $f(a)$), it
follows that $f(x) = x$ for some $x$ in $[a, b]$.

\paragraph{(b)} Give several examples of continuous $f$ such that $f = f^{-1}$
and $f(x) = x$ for exactly one $x$.

\paragraph{Solution:} Let $f$ be any decreasing function on $(-\infty, a]$
which takes on all values $\geq a$, and define \begin{equation*}
  g(x) = \begin{cases}
    f(x), &x \leq a \\
    f^{-1}(x), &x > a.
  \end{cases}
\end{equation*}

\paragraph{(c)} Prove that if $f$ is an increasing function such that $f =
f^{-1}$, then $f(x) = x$ for all $x$.

\paragraph{Solution:} If $f(x) < x$ for any $x$, then $x = f^{-1}(f(x)) <
f^{-1}(x) = f(x)$, a contradiction; similarly, we cannot have $f(x) > x$.

\paragraph{Problem 12-24.} Which functions have the property that the graph is
still the graph of a function when reflected through the graph of $-I$ (the
"antidiagonal")?

\paragraph{Solution:} Reflecting through the antidiagonal is the same as
reflecting through the vertical axis, reflecting through the diagonal, and then
reflecting through the vertical axis again. Clearly, the functions with this
property are precisely the one-one functions.

\end{document}

