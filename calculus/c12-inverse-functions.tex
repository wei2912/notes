\documentclass{article}

\usepackage{amsmath,amssymb,amsthm}
\usepackage[shortlabels]{enumitem}

\newtheorem{corollary}{Corollary}
\newtheorem{definition}{Definition}
\newtheorem{theorem}{Theorem}

\begin{document}

\title{Chapter 12: Inverse Functions}
\maketitle

\begin{definition}
  A function $f$ is \emph{one-one} if $f(a) \neq f(b)$ whenever $a \neq b$.
\end{definition}

\begin{definition}
  For any function $f$, the \emph{inverse} of $f$, denoted by $f^{-1}$, is the
  set of all pairs $(a, b)$ for which the pair $(b, a)$ is in $f$.
\end{definition}

\begin{theorem}
  $f^{-1}$ is a function if and only if $f$ is one-one.
\end{theorem}

\begin{theorem}
  If $f$ is continuous and one-one of an interval, then $f$ is either
  increasing or decreasing on that interval.
\end{theorem}
\begin{proof}
  Let $a_0 < b_0$ be two numbers in the interval. Since $f$ is one-one, we know
  that
  \begin{align*}
    \text{either} &&\text{ (i) } &f(b_0) - f(a_0) > 0 \\
    \text{or} &&\text{ (ii) } &f(b_0) - f(a_0) < 0.
  \end{align*}
  We will assume that (i) is true, and show that the same inequality holds for
  any $a_1 < b_1$ in the interval, so that $f$ is increasing. (A similar
  argument shows that if (ii) is true, then $f$ is decreasing.)

  Let
  \begin{align*}
    x_t &= (1 - t)a_0 + ta_1 \\
    y_t &= (1 - t)b_0 + tb_1
  \end{align*}
  for $0 \leq t \leq 1$. Then $x_0 = a_0$ and $x_1 = a_1$ and the points $x_t$
  all lie between $a_0$ and $a_1$. An analogous statement holds for $y_t$. So
  $x_t$ and $y_t$ are all in the domain of $f$. Moreover, since $a_0 < b_0$ and
  $a_1 < b_1$, we also have \[
    x_t < y_t \text{ for } 0 \leq t \leq 1.
  \] Now consider the function \[
    g(t) = f(y_t) - f(x_t) \text{ for } 0 \leq t \leq 1.
  \] Using Theorem 6-2, it is easy to see that $g$ is continuous on $[0, 1]$.
  Moreover, $g(t)$ is never 0, since $x_t < y_t$ and $f$ is one-one.
  Consequently, $g(t)$ is either positive for all $t$ in $[0, 1]$ or negative
  for all $t$ in $[0, 1]$ (otherwise, by the Intermediate Value Theorem it
  would also be 0 somewhere in $[0, 1]$). But $g(0) > 0$ by (i). So also $g(1)
  > 0$, which means that (i) also holds for $a_1, b_1$.
\end{proof}

Moving on, we now consider which important properties of a one-one function are
inherited by its inverse.

\begin{theorem}
  If $f$ is continuous and one-one on an interval, then $f^{-1}$ is also
  continuous.
\end{theorem}
\begin{proof}
  We know by Theorem 2 that $f$ is either increasing or decreasing. We might as
  well assume that $f$ is increasing, since we can then take care of the other
  case by applying the usual trick of considering $-f$.

  We must show that \[
    \lim_{x \to b} f^{-1}(x) = f^{-1}(b)
  \] for each $b$ in the domain of $f^{-1}$. Such a number $b$ is of the form
  $f(a)$ for some $a$ in the domain of $f$. For any $\epsilon > 0$, we want to
  find a $\delta > 0$ such that, for all $x$, \[
    \text{if } f(a) - \delta < x < f(a) + \delta,
    \text{ then } a - \epsilon < f^{-1}(x) < a + \epsilon.
  \] Since $a - \epsilon < a < a + \epsilon$, it follows that $f(a - \epsilon)
  < f(a) < f(a + \epsilon)$; we let $\delta$ be the smaller of $f(a + \epsilon)
  - f(a)$ and $f(a) - f(a - \epsilon)$.

  Our choice of $\delta$ ensures that \[
    f(a - \epsilon)
    \leq f(a) - \delta \text{ and } f(a) + \delta \leq f(a + \epsilon).
  \] Consequently, if \[
    f(a) - \delta < x < f(a) + \delta.
  \] then \[
    f(a - \epsilon) < x < f(a + \epsilon).
    \] Since $f$ is increasing, $f^{-1}$ is also increasing, and we obtain \[
    f^{-1}(f(a - \epsilon)) < f^{-1}(x) < f^{-1}(f(a + \epsilon)),
  \] i.e., \[
    a - \epsilon < f^{-1}(x) < a + \epsilon,
  \] which is precisely what we want.
\end{proof}

The proof of differentiability is more involved.

\begin{theorem}
  If $f$ is a continuous one-one function defined on an interval and
  $f'(f^{-1}(a)) = 0$, then $f^{-1}$ is \emph{not} differentiable at $a$.
\end{theorem}

\begin{theorem}
  Let $f$ be a continuous one-one function defined on an interval, and suppose
  that $f$ is differentiable at $f^{-1}(b)$, with derivative $f'(f^{-1}(b))
  \neq 0$. Then $f^{-1}$ is differentiable at $b$, and \[
    (f^{-1})'(b) = \frac{1}{f'(f^{-1}(b))}.
  \]
\end{theorem}
\begin{proof}
  Let $b = f(a)$. Then \[
    \lim_{h \to 0} \frac{f^{-1}(b + h) - f^{-1}(b)}{h}
    = \lim_{h \to 0} \frac{f^{-1}(b + h) - a}{h}.
  \] Now every number $b + h$ in the domain of $f^{-1}$ can be written in the
  form \[
    b + h = f(a + k)
  \] for an unique $k$. Then \[
    \lim_{h \to 0} \frac{f^{-1}(b + h) - a}{h}
    = \lim_{h \to 0} \frac{f^{-1}(f(a + k)) - a}{f(a + k) - b}
    = \lim_{h \to 0} \frac{k}{f(a + k) - f(a)}.
  \] It is not hard to get an explicit expression for k; since \[
    b + h = f(a + k)
  \] we have \[
    f^{-1}(b + h) = a + k
  \] or \[
    k = f^{-1}(b + h) - f^{-1}(b).
  \] Now by Theorem 3, the function $f^{-1}$ is continuous at $b$. This means
  that $k$ approaches 0 as $h$ approaches 0. Since \[
    \lim_{k \to 0} \frac{f(a + k) - f(a)}{k} = f'(a) = f'(f^{-1}(b)) \neq 0,
  \] this implies that \[
    (f^{-1})'(b) = \frac{1}{f'(f^{-1}(b))}.
  \]
\end{proof}

\section*{Exercises}

\paragraph{Problem 11}
\begin{enumerate}[(a)]
  \item Let $f = g^{-1}$ where $g(x) = -x^5 - x$. Notice that $g$ is one-one,
    since $g'(x) = -5x^4 - 1 < 0$ for all $x$, and that $g$ takes on all
    values. So $f$ is defined on $\mathbb{R}$, and for all $x$ we have \[
      x = g(f(x)) = -[f(x)]^5 - f(x).
    \] Moreover, $f$ is differentiable, since $g'(x) \neq 0$ for all $x$.
  \item By Theorem 5, as $g$ is a differentiable one-one function defined on an
    interval, with derivative $g'(x) \neq 0$ for all $x$, Then for all $b$, $f
    = g^{-1}$ is differentiable and \[
      (g^{-1})'(b) = \frac{1}{g'(g^{-1}(b))}
    \] or \[
      f'(b) = \frac{1}{g'(f(b))} = \frac{1}{-5[f(b)]^4 - 1}.
    \]
  \item Differentiating the equation defining $f$, \[
      5[f(x)]^4f'(x) + f'(x) + 1 = 0
    \] or \[
      f'(x) = -\frac{1}{5[f(x)]^4 + 1}
    \] which is equivalent to the function found in (b).
\end{enumerate}

\paragraph{Problem 23}
\begin{enumerate}[(a)]
  \item Suppose that $f(a) > a$ for some $a$ (such an $a$ must exist given the
    symmetry of $y = f(x)$ about $y = x$); then, $f(f(a)) = a < f(a)$. Since
    $f(a) > a$, and $f(b) < b$ for some $b$ (namely, $f(a)$), it follows that
    $f(x) = x$ for some $x$ in $[a, b]$.
  \item Let $f$ be any decreasing function on $(-\infty, a]$ which takes on all
    values $\geq a$, and define \[
      g(x) =
      \begin{cases}
        f(x), &x \leq a \\
        f^{-1}(x), &x > a.
      \end{cases}
    \]
  \item If $f(x) < x$ for any $x$, then $x = f^{-1}(f(x)) < f^{-1}(x) = f(x)$,
    a contradiction; similarly, we cannot have $f(x) > x$.
\end{enumerate}

\paragraph{Problem 24} Reflecting through the antidiagonal is the same as
reflecting through the vertical axis, reflecting through the diagonal, and then
reflecting through the vertical axis again. Clearly, the functions with this
property are precisely the one-one functions.

\end{document}

