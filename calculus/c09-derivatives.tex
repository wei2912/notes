\documentclass{article}

\usepackage{amsmath,amssymb,amsthm}
\usepackage[shortlabels]{enumitem}

\newtheorem{definition}{Definition}
\newtheorem{theorem}{Theorem}
\newtheorem*{theorem*}{Theorem}
\newtheorem{lemma}{Lemma}

\begin{document}

\title{Chapter 9: Derivatives}
\maketitle

\begin{definition}
  The function $f$ is \emph{differentiable at $A$} if \[
    \lim_{h \to 0}\frac{f(a + h) - f(a)}{h}
  \] exists.

  In this case the limit is denoted by $f'(a)$ and is called the
  \emph{derivative of $f$ at $a$}. (We also say that $f$ is
  \emph{differentiable} if $f$ is differentiable at $a$ for every $a$ in the
  domain of $f$.)
\end{definition}

\begin{theorem}
  If $f$ is differentiable at $a$, then $f$ is continuous at $a$.
\end{theorem}
\begin{proof}
  \begin{align*}
    \lim_{h \to 0} (f(a + h) - f(a))
    &= \lim_{h \to 0} \left( \frac{f(a + h) - f(a)}{h} \cdot h \right) \\
    &= \lim_{h \to 0} \frac{f(a + h) - f(a)}{h} \cdot \lim_{h \to 0} h \\
    &= f'(a) \cdot 0 \\
    &= 0.
  \end{align*}

  As pointed out in Chapter 5, the equation $\lim_{h \to 0} (f(a + h) - f(a)) =
  0$ is equivalent to $\lim_{x \to a} f(x) = f(a)$; thus $f$ is continuous at
  $a$.
\end{proof}

\section*{Exercises}

\paragraph{Problem 14} Consider $g(x) = f(x)/x$; $g(x) = x$ if $x$ is rational,
and $g(x) = 0$ if $x$ is irrational. Hence, \[
  \lim_{h \to 0} \frac{f(h) - f(0)}{h}
  = \lim_{h \to 0} \frac{f(h)}{h}
  = \lim_{h \to 0} g(h) = 0,
\] so $f$ is differentiable at 0.

\paragraph{Problem 15}
\begin{enumerate}[(a)]
  \item Note that by the definition of $f$, $|f(0)| \leq 0^2$, so $f(0) = 0$.
    Similar to Exercise 9-14, consider $g(x) = |f(x)|/x$; from $0 \leq g(x)
    \leq x^2/x = x$, it is clear that $0 \leq \lim_{h \to 0} g(h)$ and
    $\lim_{h \to 0} g(h) \leq \lim_{h \to 0} h = 0$, so $\lim_{h \to 0} g(h) =
    0$. Hence, \[
      \lim_{h \to 0} \frac{f(h) - f(0)}{h}
      = \lim_{h \to 0} \frac{f(h)}{h}
      = \lim_{h \to 0} g(h) = 0,
    \] so $f$ is differentiable at 0.
  \item Any function $g$ with $\lim_{h \to 0} g(h)/h = 0$ i.e. a function $g$
    with $g(0) = g'(0) = 0$.
\end{enumerate}

\paragraph{Problem 18} Since $f$ is not continuous at $a$ if $a$ is rational,
it is also not differentiable at rational $a$. If $a = m.a_1a_2a_3\ldots$ is
irrational and $h$ is rational, then $a + h$ is irrational, so $f(a + h) - f(a)
= 0$. But if $h = -0.00\ldots0a_{n+1}a_{n+2}\ldots$, then $a + h = m.a_1a_2a_3
\ldots a_n000\ldots$, so $f(a + h) \geq 10^{-n}$, while $|h| < 10^{-n}$, so we
have $|[f(a + h) - f(a)]/h| \geq 1$. Thus $[f(a + h) - f(a)]/h$ is 0 for
arbitrarily small $h$ and also has absolute value $\geq 1$ for arbitarily small
$h$. It follows that $\lim_{h \to 0} [f(a + h) - f(a)]/h$ cannot exist.

\paragraph{Problem 23} Noting that $f(x) = f(-x)$,
\begin{align*}
  f'(-x) &= \lim_{h \to 0}\frac{f(-x - h) - f(-x)}{-h} \\
         &= \lim_{h \to 0}\frac{f(-x) - f(-x - h)}{h} \\
         &= \lim_{h \to 0}\frac{f(x) - f(x + h)}{h} \\
         &= -\lim_{h \to 0}\frac{f(x + h) - f(x)}{h} \\
         &= -f'(x),
\end{align*}
which can be rewritten as $f'(x) = -f'(-x)$.

\paragraph{Problem 24} Noting that $f(x) = -f(-x)$,
\begin{align*}
  f'(-x) &= \lim_{h \to 0}\frac{f(-x - h) - f(-x)}{-h} \\
         &= \lim_{h \to 0}\frac{f(-x) - f(-x - h)}{h} \\
         &= \lim_{h \to 0}\frac{-f(x) + f(x + h)}{h} \\
         &= \lim_{h \to 0}\frac{f(x + h) - f(x)}{h} \\
         &= f'(x).
\end{align*}

\paragraph{Exercise 9-28. (a)} Find $f'(x)$ if $f(x) = |x|^3$. Find $f''(x)$.
Does $f'''(x)$ exist for all $x$?

\paragraph{Problem 28}
\begin{enumerate}[(a)]
  \item Since \[
      f(x) =
      \begin{cases}
        x^3, x > 0 \\
        -x^3, x < 0,
      \end{cases}
    \] we have \[
      f'(x) =
      \begin{cases}
        3x^2, x > 0 \\
        -3x^2, x < 0
      \end{cases}
      f''(x) =
      \begin{cases}
        6x, x > 0 \\
        -6x, x < 0.
      \end{cases}
    \] Moreover, $f'(0) = f''(0) = 0$, while $f'''(0)$ does not exist.
  \item Since \[
      f(x) =
      \begin{cases}
        x^4, x > 0 \\
        -x^4, x < 0,
      \end{cases}
      \] we have \[
      f'(x) =
      \begin{cases}
        4x^3, x > 0 \\
        -4x^3, x < 0
      \end{cases}
      f''(x) =
      \begin{cases}
        12x^2, x > 0 \\
        -12x^2, x < 0.
      \end{cases}
      f'''(x) =
      \begin{cases}
        24x, x > 0 \\
        -24x, x < 0.
      \end{cases}
    \] Moreover, $f'(0) = f''(0) = f'''(0) = 0$, while $f^{(4)}(0)$ does not
    exist.
\end{enumerate}

\paragraph{Problem 29} It is clear that \[
  f^{(n-1)}(x) =
  \begin{cases}
    n!x, x > 0 \\
    0, x \leq 0.
  \end{cases}
\] So $f^{(n)}(0)$ does not exist since $\lim_{h \to 0^+} n!h/h = n!$, while
$\lim_{h \to 0^-} 0/h = 0$.

\end{document}

