\documentclass{article}

\usepackage{amsmath,amssymb,amsthm}
\usepackage[shortlabels]{enumitem}

\newtheorem{corollary}{Corollary}
\newtheorem{definition}{Definition}
\newtheorem{lemma}{Lemma}
\newtheorem*{theorem*}{Theorem}
\newtheorem{theorem}{Theorem}

\begin{document}

\title{Chapter 14: Fundamental Theorem of Calculus}
\maketitle

\begin{theorem}[The First Fundamental Theorem of Calculus]
  Let $f$ be integrable on $[a, b]$, and define $F$ on $[a, b]$ by \[
    F(x) = \int_a^x f.
  \] If $f$ is continuous at $c$ in $[a, b]$, then $F$ is differentiable at
  $c$, and \[
    F'(c) = f(c).
  \] (If $c = a$ or $b$, then $F'(c)$ is understood to mean the right- or
  left-hand derivative of $F$.)
\end{theorem}
\begin{proof}
  We will assume that $c$ is in $(a, b)$; the easy modifications for $c = a$ or
  $b$ may be supplied by the reader. By definition, \[
    F'(c) = \lim_{h \to 0}\frac{F(c + h) - F(c)}{h}.
  \] Suppose first that $h > 0$. Then \[
    F(c + h) - F(c) = \int_c^{c + h} f.
  \] Define $m_h$ and $M_h$ as follows:
  \begin{align*}
    m_h &= \inf\{f(x): c \leq x \leq c + h\}, \\
    M_h &= \sup\{f(x): c \leq x \leq c + h\}.
  \end{align*}
  It follows from Theorem 13-7 that \[
    m_h \cdot h \leq \int_c^{c + h} f \leq M_h \cdot h.
  \] Therefore \[
    m_h \leq \frac{F(c + h) - F(c)}{h} \leq M_h.
  \] If $h < 0$, only a few details of the argument have to be changed. Let
  \begin{align*}
    m_h &= \inf\{f(x): c + h \leq x \leq c\}, \\
    M_h &= \sup\{f(x): c + h \leq x \leq c\}.
  \end{align*}
  Then \[
    m_h \cdot (-h) \leq \int_{c + h}^c f \leq M_h \cdot (-h).
  \] Since \[
    F(c + h) - F(c) = \int_c^{c + h} f = -\int_{c + h}^c f
  \] this yields \[
    m_h \cdot h \geq \int_c^{c + h} f \geq M_h \cdot h.
  \] Since $h < 0$, dividing by $h$ reverses the inequality again, yielding the
  same result as before: \[
    m_h \leq \frac{F(c + h) - F(c)}{h} \leq M_h.
  \] This inequality is true for any integrable function, continuous or not.
  Since $f$ is continuous at $c$, however, \[
    \lim_{h \to 0} m_h = \lim_{h \to 0} M_h = f(c),
  \] and this proves that \[
    F'(c) = \lim_{h \to 0} \frac{F(c + h) - F(c)}{h} = f(c).
  \]
\end{proof}

Theoerm 1 has a simple corollary which frequently reduces computation of
integrals to a trivality.

\begin{corollary}
  If $f$ is continuous on $[a, b]$ and $f = g'$ for some function $g$, then \[
    \int_a^b f = g(b) - g(a).
  \]
\end{corollary}

Theorem 2 is a somewhat strong result than the corollary to Theorem 1, but
proven in an entirely different manner.

\begin{theorem}[The Second Fundamental Theorem of Calculus]
  If $f$ is integrable on $[a, b]$ and $f = g'$ for some function $g$, then \[
    \int_a^b f = g(b) - g(a).
  \]
\end{theorem}
\begin{proof}
  Let $P = \{t_0, \ldots, t_n\}$ be any partition of $[a, b]$. By the Mean
  Value Theorem there is a point $x_i$ in $[t_{i-1}, t_i]$ such that
  \begin{align*}
    g(t_i) - g(t_{i-1})
    &= g'(x_i)(t_i - t_{i-1}) \\
    &= f(x_i)(t_i - t_{i-1}).
  \end{align*}
  If
  \begin{align*}
    m_i &= \inf\{f(x): t_{i-1} \leq x \leq t_i\}, \\
    M_i &= \sup\{f(x): t_{i-1} \leq x \leq t_i\},
  \end{align*}
  then clearly \[
    m_i(t_i - t_{i-1}) \leq f(x_i)(t_i - t_{i-1}) \leq M_i(t_i - t_{i-1}),
  \] that is, \[
    m_i(t_i - t_{i-1}) \leq g(t_i) - g(t_{i-1}) \leq M_i(t_i - t_{i-1}).
  \] Adding these equations for $i = 1, \ldots, n$ we obtain \[
    \sum_{i=1}^n m_i(t_i - t_{i-1})
    \leq g(b) - g(a)
    \leq \sum_{i=1}^n M_i(t_i - t_{i-1})
  \] so that \[
    L(f, P) \leq g(b) - g(a) \leq U(f, P)
  \] for every partition $P$. But this means that \[
    g(b) - g(a) = \int_a^b f.
  \]
\end{proof}

If $f$ is any bounded function on $[a, b]$, then \[
  \sup\{L(f, P)\} \text{ and } \inf\{U(f, P)\}
\] will both exist, even if $f$ is not integrable. These numbers
are called the \emph{lowest integral} and the \emph{upper integral} of $f$ on
$[a, b]$ respectively, and will be denoted by \[
  \textbf{L}\int_a^b f \text{ and } \textbf{U}\int_a^b f.
\]

\begin{theorem*}[cf. Theorem 13-3]
  If $f$ is continuous on $[a, b]$, then $f$ is integrable on $[a, b]$.
\end{theorem*}

\begin{proof}
  Define functions $L$ and $U$ on $[a, b]$ by \[
    L(x) = \textbf{L}\int_a^x f \text{ and } U(x) = \textbf{U}\int_a^x f.
  \] Let $x$ be in $(a, b)$. If $h > 0$ and
  \begin{align*}
    m_h &= \inf\{f(t): x \leq t \leq x + h\}, \\
    M_h &= \sup\{f(t): x \leq t \leq x + h\},
  \end{align*}
  then \[
    m_h \cdot h \leq \textbf{L}\int_x^{x + h} f \leq \textbf{U}\int_x^{x + h} f
    \leq M_h \cdot h,
  \] so \[
    m_h \cdot h \leq L(x + h) - L(x) \leq U(x + h) - U(x) \leq M_h \cdot h
  \] so \[
    m_h \leq \frac{L(x + h) - L(x)}{h} \leq \frac{U(x + h) - U(x)}{h} \leq M_h.
  \] If $h < 0$ and
  \begin{align*}
    m_h &= \inf\{f(t): x + h \leq t \leq x\}, \\
    M_h &= \sup\{f(t): x + h \leq t \leq x\},
  \end{align*}
  one obtains the same inequality, precisely as in the proof of
  Theorem 1.

  Since $f$ is continuous at $x$, we have \[
    \lim_{h \to 0} m_h = \lim_{h \to 0} M_h = f(x),
  \] and this proves that \[
    L'(x) = U'(x) = f(x) \text{ for } x \text{ in } (a, b).
  \] This means that there is a number $c$ such that \[
    U(x) = L(x) + c \text{ for all } x \text{ in } [a, b].
  \] Since \[
    U(a) = L(a) = 0,
  \] the number $c$ must equal 0, so \[
    U(x) = L(x) \text{ for all } x \text{ in } [a, b].
  \] In particular, \[
    \textbf{U}\int_a^b f = U(b) = L(b) = \textbf{L}\int_a^b f,
  \] and this means that $f$ is integrable on $[a, b]$.
\end{proof}

\section*{Exercises}

\paragraph{Problem 8} Since the two sides of the desired inequality are equal
for $x = 0$, we just need to prove the same inequality for their derivatives,
i.e., \[
  f(x)^3 \leq 2f(x)\int_0^x f.
\] We have $f(x) > 0$ for $x > 0$, since $f(0) = 0$ and $0 < f'$, so this
inequality is equivalent to \[
  f(x)^2 \leq 2\int_0^x f.
\] But both sides of this inequality are true for $x = 0$, so we just need to
prove the inequality for their derivatives: \[
  2f(x)f'(x) \leq 2f(x).
\] This is true since $f(x) > 0$ and $0 < f'(x) \leq 1$.

\paragraph{Problem 9} If \[
  g(x) =
  \begin{cases}
    x^2 \sin \frac{1}{x}, &x \neq 0 \\
    0, &x = 0
  \end{cases}
\] then \[
  g'(x) =
  \begin{cases}
    2x \sin \frac{1}{x} - \cos \frac{1}{x}, &x \neq 0 \\
    0, &x = 0.
  \end{cases}
\] So if we define \[
  h(x) =
  \begin{cases}
    2x \sin \frac{1}{x}, &x \neq 0 \\
    0, &x = 0
  \end{cases}
\] we have \[
  f(x) = h(x) - g'(x) \text{ for all } x.
\] Hence \[
  F(x) = \int_0^x (h - g') = \left(\int_0^x h\right) - g,
\] using the Second Fundamental Theorem of Calculus. Since $h$ is continuous we
can then apply the First Fundamental Theorem to conclude that \[
  F'(0) = h(0) - g'(0) = 0.
\]

\paragraph{Problem 12} Since \[
  F(x) = \int_0^x f(u)(x - u) \,du = \int_0^x xf(u) \,du - \int_0^x uf(u) \,du,
\] then
\begin{align*}
  F'(x)
  &= \left[xf(x) + \int_0^x f(u) \,du\right] - xf(x) \\
  &= \int_0^x f(u) \,du.
\end{align*}
Consequently, there is a number $c$ such that \[
  \int_0^x f(u)(x - u) \,du
  = \int_0^x \left( \int_0^u f(t) \,dt \right) \,du + c
\] for all $x$, but clearly $c = 0$, since both sides are equal to 0 for $x =
0$.

\paragraph{Problem 13} Applying Problem 12 to $g(u) = f(u)(x - u)$, we obtain
\begin{align*}
  \int_0^x f(u)(x - u)^2 \,du
  &= \int_0^x [f(u)(x - u)](x - u) \,du \\
  &= \int_0^x \left(\int_0^u f(t)(x - t) \,dt\right) \,du.
\end{align*}
It suffices to prove that \[
  \int_0^x \left( \int_0^u f(t)(x - t) \,dt \right) \,du
  = 2\int_0^x \left(
    \int_0^u \left( \int_0^{u_1} f(t) \,dt \right)
  \,du_1 \right) \,du.
\] Observing that $x - t = (u - t) + (x - u)$, \[
  \int_0^u f(t)(x - t) \,dt
  = \int_0^u f(t)(u - t) \,dt + \int_0^u f(t)(x - u) \,dt
\] The first integral on the right can be rewritten as \[
  \int_0^u f(t)(u - t) \,dt
  = \int_0^u \left( \int_0^{u_1} f(t) \,dt \right) \,du_1,
\] while the second can be rewritten as \[
  \int_0^u f(t)(x - u) \,dt = (x - u)\int_0^u f(t) \,dt.
\] Combining these two integrals,
\begin{multline*}
  \int_0^x \left(\int_0^u f(t)(x - t) \,dt\right) \,du
  = \int_0^x \left(
    \int_0^u \left(\int_0^{u_1} f(t) \,dt\right) \,du_1
  \right) \\
  + \int_0^x \left(
    (x - u) \int_0^u f(t) \,dt
  \right) \,du.
\end{multline*}
Once again, applying Problem 12 to $h(x) = \int_0^u f(t) \,dt$, \[
  \int_0^x \left((x - u)\int_0^u f(t) \,dt\right) \,du
  = \int_0^x \left(
    \int_0^u \left( \int_0^{u_1} f(t) \,dt \right) \,du_1
  \right) \,du
\] which leads to the desired result.

\paragraph{Problem 15}
\begin{enumerate}[(b)]
  \item Let $g$ be periodic and continuous with $g \geq 0$ (e.g. $g(x) =
    \sin^2 x$). If $f(x) = \int_0^x g$, then $f' = g$ is periodic, but $f$ is
    increasing and is not periodic.
  \item Let $g(x) = f(x + a)$. Then $g'(x) = f'(x + a) = f'(x)$. If $f(a) =
    f(0)$, then we also have $g(0) = f(a) = f(0)$. Consequently $g = f$, i.e.,
    $f(x + a) = f(x)$ for all $x$.

    Conversely, suppose that $f$ is periodic (with some period not necessarily
    $a$). Let $g(x) = f(x + a) - f(x)$. Then $g'(x) = f'(x + a) - f'(x) = 0$,
    so $g$ has the constant value $g(0) = f(a) - f(0)$, which implies that \[
      f(x + a) = f(x) + g(x) = f(x) + f(a) - f(0).
    \] It follows that
    \begin{align*}
      f(na) &= nf(a) - (n - 1)f(0) \\
            &= n[f(a) - f(0)] + f(0);
    \end{align*}
    if $f(a) \neq f(0)$, then this would be unbounded, so for $f$ to be
    periodic, $f(a) = f(0)$.
\end{enumerate}

\paragraph{Problem 25}
\begin{enumerate}[(a)]
  \item \[
      \int_1^N x^r \,dx = \frac{1}{r + 1}(N^{r + 1} - 1^{r + 1}),
    \] but $\lim_{N \to \infty} N^{r + 1} = 0$ since $r + 1 < 0$, so
    $\int_1^{\infty} x^r \,dx = -\frac{1}{r + 1}$.
  \item Summing up $N$ terms of $\int_1^2 1/x \,dx$, Problem 13-15 leads to the
    following result: \[
      N\int_1^2 1/x \,dx = \int_1^{2^N} 1/x \,dx
    \] But $\int_1^2 1/x \,dx = \ln 2 > 0$, so $\int_1^{2^N} 1/x \,dx$ is
    unbounded as N increases and $\int_1^{\infty} 1/x \,dx$ cannot exist.
  \item The function $I(N) = \int_0^N g$ is clearly increasing, and it is
    bounded above by $\int_0^{\infty} f$. Consequently, $\lim_{N \to \infty}
    I(N)$ exists (as the least upper bound of $\{I(N): N \geq 0\}$.
  \item Clearly $\int_0^{\infty} 1/(1 + x^2) \,dx$ exists if $\int_1^{\infty}
    1/(1 + x^2) \,dx$ exists. The latter exists because $1/(1 + x^2)
    \leq x^{-2}$ for all $x \geq 1$, and so $\int_1^{\infty} 1/(1 + x^2) \,dx
    \leq \int_1^{\infty} x^{-2} \,dx = 1$ by (a).
\end{enumerate}

\paragraph{Problem 27.}
\begin{enumerate}[(a)]
  \item From Problem 14-25, $\int_0^{\infty} 1/(1 + x^2) \,dx$ exists. Clearly,
    since $1/(1 + x^2)$ is odd, $\int_{-\infty}^0 1/(1 + x^2) \,dx =
    \int_0^{\infty} 1/(1 + x^2) \,dx$, and hence $\int_{-\infty}^{\infty}
    1/(1 + x^2) \,dx$ exists.
  \item Clearly $\int_{-\infty}^0 x \,dx$ and $\int_0^{\infty} x \,dx$ both do
    not exist.
  \item We attempt to prove the following generalisation: If
    $\lim_{N \to \infty} g(N) = -\infty$ and $\lim_{N \to \infty} h(N) =
    \infty$, then \[
      \lim_{N \to \infty} \int_{g(N)}^{h(N)} f = \int_{-\infty}^{\infty} f.
    \]

    Given $\epsilon > 0$, choose $M_0$ such that \[
      \left| \int_{-\infty}^0 f - \int_{-M}^0 f \right| < \frac{\epsilon}{2}
      \text{ and } \left| \int_0^{\infty} f - \int_0^M f \right|
      < \frac{\epsilon}{2}
    \] for all $M > M_0$. Now choose $N$ such that $h(N) > M$ and $g(N) < -M$
    for all $N > N_0$. Then for $N > N_0$ we have \[
      \left| \int_{-\infty}^{\infty} f - \int_{g(N)}^{h(N)} f \right|
      \leq \left| \int_{-\infty}^0 f - \int_{g(N)}^0 f \right|
      + \left| \int_0^{\infty} f - \int_0^{h(N)} f \right|
      < \frac{\epsilon}{2} + \frac{\epsilon}{2} = \epsilon.
    \]
\end{enumerate}

\paragraph{Problem 28}
\begin{enumerate}[(a)]
  \item \[
      \int_{\epsilon}^a 1/\sqrt{x} \,dx = 2(\sqrt{a} - \sqrt{\epsilon}),
    \] which shows that $\lim_{\epsilon \to 0^+} \int_{\epsilon}^a 1/\sqrt{x}
    \,dx = 2\sqrt{a}$.
  \item Since $x^r$ for $-1 < r < 0$ is unbounded in the interval $[0, a]$, we
    consider the limit $\lim_{\epsilon \to 0^+} \int_{\epsilon}^a x^r \,dx$
    instead. Since \[
      \int_{\epsilon}^a x^r \,dx = \frac{1}{r + 1}(a^{r + 1} -
      \epsilon^{r + 1})
    \] and $r + 1 > 0$, $\lim_{\epsilon \to 0^+} \epsilon^{r + 1} = 0$, so
    $\int_0^a x^r \,dx = \frac{a^{r + 1}}{r + 1}$ for $-1 < r < 0$.
  \item Problem 13-15 implies that \[
      N\int_{1/2}^1 x^{-1} \,dx = \int_{1/2^N}^1 x^{-1} \,dx
    \] so $\lim_{\epsilon \to 0^+} \int_{\epsilon}^1 1/x \,dx$ does not exist,
    which implies that $\lim_{\epsilon \to 0^+} \int_{\epsilon}^a x^{-1} \,dx =
    \lim_{\epsilon \to 0^+} \int_{\epsilon}^1 x^{-1} \,dx + \int_1^a x^{-1}
    \,dx$ does not exist either.
  \item We first define $\int_a^0 |x|^r \,dx =
    \lim_{\epsilon \to 0^-} \int_a^{\epsilon} |x|^r \,dx$ for $a < 0$, such
    that
    \begin{align*}
      \int_a^0 |x|^r \,dx
      &= \lim_{\epsilon \to 0^-} \int_a^{\epsilon} |x|^r \,dx \\
      &= -\lim_{\epsilon \to 0^+} \int_{\epsilon}^a x^r \,dx \\
      &= -\frac{a^{r + 1}}{r + 1}
    \end{align*}
  \item Since $\lim_{x \to -1} (1 - x^2)^{-1/2} = \lim_{x \to 1}
    (1 - x^2)^{-1/2} = \infty$, we define
    \begin{align*}
      \int_{-1}^1 (1 - x^2)^{-1/2} \,dx
      &= \int_{-1}^0 (1 - x^2)^{-1/2} \,dx + \int_0^1 (1 - x^2)^{-1/2} \,dx \\
      &= \lim_{\epsilon \to -1^+} \int_{\epsilon}^0 (1 - x^2)^{-1/2} \,dx
      + \lim_{\epsilon \to 1^-} \int_0^{\epsilon} (1 - x^2)^{-1/2} \,dx \\
      &= 2\lim_{\epsilon \to -1^+} \int_{\epsilon}^0 (1 - x^2)^{-1/2} \,dx
    \end{align*}
    Now, the limit \[
      \lim_{\epsilon \to -1^+} \int_{\epsilon}^0 (1 + x)^{-1/2} \,dx
      = \lim_{\epsilon \to 0^+} \int_{\epsilon}^1 x^{-1/2} \,dx
    \] exists by part (a). For $-1 < x < 0$, we have $(1 + x)^{-1/2} >
    (1 - x^2)^{-1/2}$. It follows that \[
      \lim_{\epsilon \to -1^+} \int_{\epsilon}^0 (1 - x^2)^{-1/2} \,dx
    \] also exists.
\end{enumerate}

\end{document}

