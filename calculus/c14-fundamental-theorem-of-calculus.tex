\documentclass{article}

\usepackage{amsmath,amssymb,amsthm}

\newtheorem{corollary}{Corollary}
\newtheorem{definition}{Definition}
\newtheorem{lemma}{Lemma}
\newtheorem*{theorem*}{Theorem}
\newtheorem{theorem}{Theorem}

\begin{document}

\title{Chapter 14: Fundamental Theorem of Calculus}
\maketitle

\begin{theorem}[The First Fundamental Theorem of Calculus]
  Let $f$ be integrable on $[a, b]$, and define $F$ on $[a, b]$ by
  \begin{equation*}
    F(x) = \int_a^x f.
  \end{equation*} If $f$ is continuous at $c$ in $[a, b]$, then $F$ is
  differentiable at $c$, and \begin{equation*}
    F'(c) = f(c).
  \end{equation*} (If $c = a$ or $b$, then $F'(c)$ is understood to mean the
  right- or left-hand derivative of $F$.)
\end{theorem}

\begin{proof}
  We will assume that $c$ is in $(a, b)$; the easy modifications for $c = a$ or
  $b$ may be supplied by the reader. By definition, \begin{equation*}
    F'(c) = \lim_{h \rightarrow 0}\frac{F(c + h) - F(c)}{h}.
  \end{equation*} Suppose first that $h > 0$. Then \begin{equation*}
    F(c + h) - F(c) = \int_c^{c + h} f.
  \end{equation*} Define $m_h$ and $M_h$ as follows: \begin{align*}
    m_h &= \inf\{f(x): c \leq x \leq c + h\}, \\
    M_h &= \sup\{f(x): c \leq x \leq c + h\}.
  \end{align*} It follows from Theorem 13-7 that \begin{equation*}
    m_h \cdot h \leq \int_c^{c + h} f \leq M_h \cdot h.
  \end{equation*} Therefore \begin{equation*}
    m_h \leq \frac{F(c + h) - F(c)}{h} \leq M_h.
  \end{equation*} If $h < 0$, only a few details of the argument have to be
  changed. Let \begin{align*}
    m_h &= \inf\{f(x): c + h \leq x \leq c\},
    M_h &= \sup\{f(x): c + h \leq x \leq c\}.
  \end{align*} Then \begin{equation*}
    m_h \cdot (-h) \leq \int_{c + h}^c f \leq M_h \cdot (-h).
  \end{equation*} Since \begin{equation*}
    F(c + h) - F(c) = \int_c^{c + h} f = -\int_{c + h}^c f
  \end{equation*} this yields \begin{equation*}
    m_h \cdot h \geq \int_c^{c + h} f \geq M_h \cdot h.
  \end{equation*} Since $h < 0$, dividing by $h$ reverses the inequality again,
  yielding the same result as before: \begin{equation*}
    m_h \leq \frac{F(c + h) - F(c)}{h} \leq M_h.
  \end{equation*} This inequality is true for any integrable function,
  continuous or not. Since $f$ is continuous at $c$, however, \begin{equation*}
    \lim_{h \rightarrow 0}m_h = \lim_{h \rightarrow 0}M_h = f(c),
  \end{equation*} and this proves that \begin{equation*}
    F'(c) = \lim_{h \rightarrow 0}\frac{F(c + h) - F(c)}{h} = f(c).
  \end{equation*}
\end{proof}

Theoerm 1 has a simple corollary which frequently reduces computation of
integrals to a trivality.

\begin{corollary}
  If $f$ is continuous on $[a, b]$ and $f = g'$ for some function $g$, then
  \begin{equation*}
    \int_a^b f = g(b) - g(a).
  \end{equation*}
\end{corollary}

\begin{proof}
  Let \begin{equation*}
    F(x) = \int_a^x f.
  \end{equation*} Then $F' = f = g'$ on $[a, b]$. Consequently, there is a
  number $c$ such that \begin{equation*}
    F = g + c.
  \end{equation*} The number $c$ can be evauated easily; note that
  \begin{equation*}
    0 = F(a) = g(a) + c,
  \end{equation*} so $c = -g(a)$; thus \begin{equation*}
    F(x) = g(x) - g(a).
  \end{equation*} This is true, in particular, for $x = b$. Thus
  \begin{equation*}
    \int_a^b f = F(b) = g(b) - g(a).
  \end{equation*}
\end{proof}

Theorem 2 is a somewhat strong result than the corollary to Theorem 1, but
proven in an entirely different manner.

\begin{theorem}[The Second Fundamental Theorem of Calculus]
  If $f$ is integrable on $[a, b]$ and $f = g'$ for some function $g$, then
  \begin{equation*}
    \int_a^b f = g(b) - g(a).
  \end{equation*}
\end{theorem}

\begin{proof}
  Let $P = \{t_0, \ldots, t_n\}$ be any partition of $[a, b]$. By the Mean
  Value Theorem there is a point $x_i$ in $[t_{i - 1}, t_i]$ such that
  \begin{align*}
    g(t_i) - g(t_{i - 1}) &= g'(x_i)(t_i - t_{i - 1}) \\
      &= f(x_i)(t_i - t_{i - 1}).
  \end{align*} If \begin{align*}
    m_i &= \inf\{f(x): t_{i - 1} \leq x \leq t_i\}, \\
    M_i &= \sup\{f(x): t_{i - 1} \leq x \leq t_i\},
  \end{align*} then clearly \begin{equation*}
    m_i(t_i - t_{i - 1}) \leq f(x_i)(t_i - t_{i - 1}) \leq M_i(t_i -
      t_{i - 1}),
  \end{equation*} that is, \begin{equation*}
    m_i(t_i - t_{i - 1}) \leq g(t_i) - g(t_{i - 1}) \leq M_i(t_i - t_{i - 1}).
  \end{equation*} Adding these equations for $i = 1, \ldots, n$ we obtain
  \begin{equation*}
    \sum_{i = 1}^n m_i(t_i - t_{i - 1}) \leq g(b) - g(a) \leq
      \sum_{i = 1}^n M_i(t_i - t_{i - 1})
  \end{equation*} so that \begin{equation*}
    L(f, P) \leq g(b) - g(a) \leq U(f, P)
  \end{equation*} for every partition $P$. But this means that
  \begin{equation*}
    g(b) - g(a) = \int_a^b f.
  \end{equation*}
\end{proof}

If $f$ is any bounded function on $[a, b]$, then \begin{equation*}
  \sup\{L(f, P)\} \text{ and } \inf\{U(f, P)\}
\end{equation*} will both exist, even if $f$ is not integrable. These numbers
are called the \emph{lowest integral} and the \emph{upper integral} of $f$ on
$[a, b]$ respectively, and will be denoted by \begin{equation*}
  \textbf{L}\int_a^b f \text{ and } \textbf{U}\int_a^b f.
\end{equation*}

\begin{theorem*}[cf. Theorem 13-3]
  If $f$ is continuous on $[a, b]$, then $f$ is integrable on $[a, b]$.
\end{theorem*}

\begin{proof}
  Define functions $L$ and $U$ on $[a, b]$ by \begin{equation*}
    L(x) = \textbf{L}\int_a^x f \text{ and } U(x) = \textbf{U}\int_a^x f.
  \end{equation*} Let $x$ be in $(a, b)$. If $h > 0$ and \begin{align*}
    m_h &= \inf\{f(t): x \leq t \leq x + h\}, \\
    M_h &= \sup\{f(t): x \leq t \leq x + h\},
  \end{align*} then \begin{equation*}
    m_h \cdot h \leq \textbf{L}\int_x^{x + h} f \leq \textbf{U}\int_x^{x + h} f
      \leq M_h \cdot h,
  \end{equation*} so \begin{equation*}
    m_h \cdot h \leq L(x + h) - L(x) \leq U(x + h) - U(x) \leq M_h \cdot h
  \end{equation*} so \begin{equation*}
    m_h \leq \frac{L(x + h) - L(x)}{h} \leq \frac{U(x + h) - U(x)}{h} \leq M_h.
  \end{equation*} If $h < 0$ and \begin{align*}
    m_h &= \inf\{f(t): x + h \leq t \leq x\}, \\
    M_h &= \sup\{f(t): x + h \leq t \leq x\},
  \end{align*} one obtains the same inequality, precisely as in the proof of
  Theorem 1.

  Since $f$ is continuous at $x$, we have \begin{equation*}
    \lim_{h \rightarrow 0} m_h = \lim_{h \rightarrow 0} M_h = f(x),
  \end{equation*} and this proves that \begin{equation*}
    L'(x) = U'(x) = f(x) \text{ for } x \text{ in } (a, b).
  \end{equation*} This means that there is a number $c$ such that
  \begin{equation*}
    U(x) = L(x) + c \text{ for all } x \text{ in } [a, b].
  \end{equation*} Since \begin{equation*}
    U(a) = L(a) = 0,
  \end{equation*} the number $c$ must equal 0, so \begin{equation*}
    U(x) = L(x) \text{ for all } x \text{ in } [a, b].
  \end{equation*} In particular, \begin{equation*}
    \textbf{U}\int_a^b f = U(b) = L(b) = \textbf{L}\int_a^b f,
  \end{equation*} and this means that $f$ is integrable on $[a, b]$.
\end{proof}

\section*{Exercises}

\paragraph{Problem 14-8.} Suppose that $f$ is a differentiable function with
$f(0) = 0$ and $0 < f' \leq 1$. Prove that for all $x \geq 0$ we have
\begin{equation*}
  \int_0^x f^3 \leq \left(\int_0^x f\right)^2.
\end{equation*}

\paragraph{Solution:} Since the two sides of the desired inequality are equal
for $x = 0$, we just need to prove the same inequality for their derivatives,
i.e., \begin{equation*}
  f(x)^3 \leq 2f(x)\int_0^x f.
\end{equation*} We have $f(x) > 0$ for $x > 0$, since $f(0) = 0$ and $0 < f'$,
so this inequality is equivalent to \begin{equation*}
  f(x)^2 \leq 2\int_0^x f.
\end{equation*} But both sides of this inequality are true for $x = 0$, so we
just need to prove the inequality for their derivatives: \begin{equation*}
  2f(x)f'(x) \leq 2f(x).
\end{equation*} This is true since $f(x) > 0$ and $0 < f'(x) \leq 1$.

\paragraph{Problem 14-9 (abridged).} Let \begin{equation*}
  f(x) = \begin{cases}
    \cos\frac{1}{x}, &x \neq 0 \\
    0, &x = 0.
  \end{cases}
\end{equation*} Is the function $F(x) = \int_0^x f$ differentiable at 0?

\paragraph{Solution:} If \begin{equation*}
  g(x) = \begin{cases}
    x^2\sin\frac{1}{x}, &x \neq 0 \\
    0, &x = 0
  \end{cases}
\end{equation*} then \begin{equation*}
  g'(x) = \begin{cases}
    2x\sin\frac{1}{x} - \cos\frac{1}{x}, &x \neq 0 \\
    0, &x = 0.
  \end{cases}
\end{equation*} So if we define \begin{equation*}
  h(x) = \begin{cases}
    2x\sin\frac{1}{x}, &x \neq 0 \\
    0, &x = 0
  \end{cases}
\end{equation*} we have \begin{equation*}
  f(x) = h(x) - g'(x) \text{ for all } x.
\end{equation*} Hence \begin{equation*}
  F(x) = \int_0^x (h - g') = \left(\int_0^x h\right) - g,
\end{equation*} using the Second Fundamental Theorem of Calculus. Since $h$ is
continuous we can then apply the First Fundamental Theorem to conclude that
\begin{equation*}
  F'(0) = h(0) - g'(0) = 0.
\end{equation*}

\paragraph{Problem 14-12 (abridged).} Prove that if $f$ is continuous, then
\begin{equation*}
  \int_0^x f(u)(x - u) \,\mathrm{d}u
  = \int_0^x \left(\int_0^u f(t) \,\mathrm{d}t\right) \,\mathrm{d}u.
\end{equation*}

\paragraph{Solution:} Since \begin{equation*}
  F(x) = \int_0^x f(u)(x - u) \,\mathrm{d}u
  = \int_0^x xf(u) \,\mathrm{d}u - \int_0^x uf(u) \,\mathrm{d}u,
\end{equation*} then \begin{align*}
  F'(x) &= \left[xf(x) + \int_0^x f(u) \,\mathrm{d}u\right] - xf(x) \\
    &= \int_0^x f(u) \,\mathrm{d}u.
\end{align*} Consequently, there is a number $c$ such that \begin{equation*}
\int_0^x f(u)(x - u) \,\mathrm{d}u
= \int_0^x \left(\int_0^u f(t) \,\mathrm{d}t\right) \,\mathrm{d}u + c
\end{equation*} for all $x$, but clearly $c = 0$, since both sides are equal to
0 for $x = 0$.

\paragraph{Problem 14-13 (abridged).} Prove that \begin{equation*}
  \int_0^x f(u)(x - u)^2 \,\mathrm{d}u
  = 2\int_0^x \left(\int_0^{u_2} \left(\int_0^{u_1} f(t) \,\mathrm{d}t\right)
    \,\mathrm{d}u_1\right) \,\mathrm{d}u_2.
\end{equation*}

\paragraph{Solution:} Applying Problem 12 to $g(u) = f(u)(x - u)$, we obtain
\begin{align*}
  \int_0^x f(u)(x - u)^2 \,\mathrm{d}u
  &= \int_0^x [f(u)(x - u)](x - u) \,\mathrm{d}u \\
  &= \int_0^x \left(\int_0^u f(t)(x - t) \,\mathrm{d}t\right) \,\mathrm{d}u.
\end{align*} It suffices to prove that \begin{equation*}
  \int_0^x \left(\int_0^u f(t)(x - t) \,\mathrm{d}t\right) \,\mathrm{d}u
  = 2\int_0^x \left(\int_0^u \left(\int_0^{u_1} f(t) \,\mathrm{d}t\right)
    \,\mathrm{d}u_1\right) \,\mathrm{d}u.
\end{equation*} Observing that $x - t = (u - t) + (x - u)$, \begin{equation*}
  \int_0^u f(t)(x - t) \,\mathrm{d}t
  = \int_0^u f(t)(u - t) \,\mathrm{d}t + \int_0^u f(t)(x - u) \,\mathrm{d}t
\end{equation*} The first integral on the right can be rewritten as
\begin{equation*}
  \int_0^u f(t)(u - t) \,\mathrm{d}t
  = \int_0^u \left(\int_0^{u_1} f(t) \,\mathrm{d}t\right) \,\mathrm{d}u_1,
\end{equation*} while the second can be rewritten as \begin{equation*}
  \int_0^u f(t)(x - u) \,\mathrm{d}t
  = (x - u)\int_0^u f(t) \,\mathrm{d}t.
\end{equation*} Combining these two integrals, \begin{multline*}
  \int_0^x \left(\int_0^u f(t)(x - t) \,\mathrm{d}t\right) \,\mathrm{du}
  = \int_0^x \left(
      \int_0^u \left(\int_0^{u_1} f(t) \,\mathrm{d}t\right) \,\mathrm{d}u_1
    \right) \\
  + \int_0^x \left(
      (x - u)\int_0^u f(t) \,\mathrm{d}t
    \right) \,\mathrm{d}u.
\end{multline*} Once again, applying Problem 12 to $h(x) = \int_0^u f(t)
\,\mathrm{d}t$, \begin{equation*}
  \int_0^x \left((x - u)\int_0^u f(t) \,\mathrm{d}t\right) \,\mathrm{d}u
  = \int_0^x \left(
    \int_0^u \left(\int_0^{u_1} f(t) \,\mathrm{d}t\right) \,\mathrm{d}u_1
  \right) \,\mathrm{d}u
\end{equation*} which leads to the desired result.

\paragraph{Problem 14-15.} A function $f$ is \emph{periodic}, with
\emph{period} $a$, if $f(x + a) = f(x)$ for all $x$.

\paragraph{(b)} Find a function $f$ such that $f$ is not periodic, but $f'$ is.
Hint: Choose a periodic $g$ for which it can be guaranteed that $f(x) =
\int_0^x g$ is not periodic.

\paragraph{Solution:} Let $g$ be periodic and continuous with $g \geq 0$ (e.g.
$g(x) = sin^2 x$). If $f(x) = \int_0^x g$, then $f' = g$ is periodic, but $f$
is increasing and is not periodic.

\paragraph{(c)} Suppose that $f'$ is periodic with period $a$. Prove that $f$
is periodic if and only if $f(a) = f(0)$.

\paragraph{Solution:} Let $g(x) = f(x + a)$. Then $g'(x) = f'(x + a) = f'(x)$.
If $f(a) = f(0)$, then we also have $g(0) = f(a) = f(0)$. Consequently $g = f$,
i.e., $f(x + a) = f(x)$ for all $x$.

Conversely, suppose that $f$ is periodic (with some period not necessarily
$a$). Let $g(x) = f(x + a) - f(x)$. Then $g'(x) = f'(x + a) - f'(x) = 0$, so
$g$ has the constant value $g(0) = f(a) - f(0)$, which implies that
\begin{equation*}
  f(x + a) = f(x) + g(x) = f(x) + f(a) - f(0).
\end{equation*} It follows that \begin{align*}
  f(na) &= nf(a) - (n - 1)f(0) \\
        &= n[f(a) - f(0)] + f(0);
\end{align*} if $f(a) \neq f(0)$, then this would be unbounded, so for $f$ to
be periodic, $f(a) = f(0)$.

\paragraph{Problem 14-25 (abridged).} The limit $\lim_{N \rightarrow \infty}
\int_a^N f$, if it exists, is denoted by $\int_a^{\infty} f$, and called an
"improper integral".

\paragraph{(a)} Determine $\int_1^{\infty} x^r \,\mathrm{d}x$, if $r < -1$.

\paragraph{Solution:} \begin{equation*}
  \int_1^N x^r \,\mathrm{d}x = \frac{1}{r + 1}(N^{r + 1} - 1^{r + 1}),
\end{equation*} but $\lim_{N \rightarrow \infty} N^{r + 1} = 0$ since $r + 1 <
0$, so $\int_1^{\infty} x^r \,\mathrm{d}x = -\frac{1}{r + 1}$.

\paragraph{(b)} Use Problem 13-15 to show that $\int_1^{\infty} 1/x
\,\mathrm{d}x$ does not exist.

\paragraph{Solution:} Summing up $N$ terms of $\int_1^2 1/x \,\mathrm{d}x$,
Problem 13-15 leads to the following result: \begin{equation*}
  N\int_1^2 1/x \,\mathrm{d}x = \int_1^{2^N} 1/x \,\mathrm{d}x
\end{equation*} But $\int_1^2 1/x \,\mathrm{d}x = \ln 2 > 0$, so $\int_1^{2^N}
1/x \,\mathrm{d}x$ is unbounded as N increases and $\int_1^{\infty} 1/x
\,\mathrm{d}x$ cannot exist.

\paragraph{(c)} Suppose that $f(x) \geq 0$ for $x \geq 0$ and that
$\int_0^{\infty} f$ exists. Prove that if $0 \leq g(x) \leq f(x)$ for all $x
\geq 0$, and $g$ is integrable on each interval $[0, N]$, then $\int_0^{\infty}
g$ also exists.

\paragraph{Solution:} The function $I(N) = \int_0^N g$ is clearly increasing,
and it is bounded above by $\int_0^{\infty} f$. Consequently,
$\lim_{N \rightarrow \infty} I(N)$ exists (as the least upper bound of $\{I(N):
N \geq 0\}$.

\paragraph{(d)} Explain why $\int_0^{\infty} 1/(1 + x^2) \,\mathrm{d}x$ exists.

\paragraph{Solution:} Clearly $\int_0^{\infty} 1/(1 + x^2) \,\mathrm{d}x$
exists if $\int_1^{\infty} 1/(1 + x^2) \,\mathrm{d}x$ exists. The latter exists
because $1/(1 + x^2) \leq x^{-2}$ for all $x \geq 1$, and so $\int_1^{\infty}
1/(1 + x^2) \,\mathrm{d}x \leq \int_1^{\infty} x^{-2} \,\mathrm{d}x = 1$ by
(a).

\paragraph{Problem 14-27.} The improper integral $\int_{-\infty}^a f$ is
defined in the obvious way, as $\lim_{N \rightarrow -\infty} \int_N^a f$. But
But another kind of improper integral $\int_{-\infty}^{\infty} f$ is defined in
a nonobvious way: it is $\int_0^{\infty} f + \int_{-\infty}^0 f$, provided
these improper integrals both exist.

\paragraph{(a)} Explain why $\int_{-\infty}^{\infty} 1/(1 + x^2) \,\mathrm{d}x$
exists.

\paragraph{Solution:} From Problem 14-25, $\int_0^{\infty} 1/(1 + x^2)
\,\mathrm{d}x$ exists. Clearly, since $1/(1 + x^2)$ is odd, $\int_{-\infty}^0
1/(1 + x^2) \,\mathrm{d}x = \int_0^{\infty} 1/(1 + x^2) \,\mathrm{d}x$, and
hence $\int_{-\infty}^{\infty} 1/(1 + x^2) \,\mathrm{d}x$ exists.

\paragraph{(b)} Explain why $\int_{-\infty}^{\infty} x \,\mathrm{d}x$ does not
exist. (But notice that $\lim_{N \rightarrow \infty} \int_{-N}^N x
\,\mathrm{d}x$ does exist.)

\paragraph{Solution:} Clearly $\int_{-\infty}^0 x \,\mathrm{d}x$ and
$\int_0^{\infty} x \,\mathrm{d}x$ both do not exist.

\paragraph{(c)} Prove that if $\int_{-\infty}^{\infty} f$ exists, then
$\lim_{N \rightarrow \infty}  \int_{-N}^N f$ exists and equals
$\int_{-\infty}^{\infty} f$. Show moreover, that $\lim_{N \rightarrow \infty}
\int_{-N}^{N + 1} f$ and $\lim_{N \rightarrow \infty} \int_{-N^2}^N f$ both
exist and equal $\int_{-\infty}^{\infty} f$. Can you state a reasonable
generalisation of these facts?

\paragraph{Solution:} We attempt to prove the following generalisation: If
$\lim_{N \rightarrow \infty} g(N) = -\infty$ and
$\lim_{N \rightarrow \infty} h(N) = \infty$, then \begin{equation*}
  \lim_{N \rightarrow \infty} \int_{g(N)}^{h(N)} f = \int_{-\infty}^{\infty} f.
\end{equation*}

Given $\epsilon > 0$, choose $M_0$ such that \begin{equation*}
  \left|\int_{-\infty}^0 f - \int_{-M}^0 f\right| < \frac{\epsilon}{2}
  \text{ and } \left|\int_0^{\infty} f - \int_0^M f\right| < \frac{\epsilon}{2}
\end{equation*} for all $M > M_0$. Now choose $N$ such that $h(N) > M$ and
$g(N) < -M$ for all $N > N_0$. Then for $N > N_0$ we have \begin{equation*}
  \left|\int_{-\infty}^{\infty} f - \int_{g(N)}^{h(N)} f\right|
  \leq \left|\int_{-\infty}^0 f - \int_{g(N)}^0 f\right|
  + \left|\int_0^{\infty} f - \int_0^{h(N)} f\right|
  < \frac{\epsilon}{2} + \frac{\epsilon}{2} = \epsilon.
\end{equation*}

\paragraph{Problem 14-28 (abridged).} There is another kind of "improper
integral" in which the interval is bounded, but the \emph{function} is
unbounded:

\paragraph{(a)} If $a > 0$, find $\lim_{\epsilon \rightarrow 0^+}
\int_{\epsilon}^a 1/\sqrt{x} \,\mathrm{d}x$. This limit is denoted by $\int_0^a
1/\sqrt{x} \,\mathrm{d}x$, even though the function $f(x) = 1/\sqrt{x}$ is not
bounded on $[0, a]$, no matter how we define $f(0)$.

\paragraph{Solution:} \begin{equation*}
  \int_{\epsilon}^a 1/\sqrt{x} \,\mathrm{d}x = 2(\sqrt{a} - \sqrt{\epsilon}),
\end{equation*} which shows that $\lim_{\epsilon \rightarrow 0^+}
\int_{\epsilon}^a 1/\sqrt{x} \,\mathrm{d}x = 2\sqrt{a}$.

\paragraph{(b)} Find $\int_0^a x^r \,\mathrm{d}x$ if $-1 < r < 0$.

\paragraph{Solution:} Since $x^r$ for $-1 < r < 0$ is unbounded in the interval
$[0, a]$, we consider the limit $\lim_{\epsilon \rightarrow 0^+}
\int_{\epsilon}^a x^r \,\mathrm{d}x$ instead. Since \begin{equation*}
  \int_{\epsilon}^a x^r \,\mathrm{d}x = \frac{1}{r + 1}(a^{r + 1} -
  \epsilon^{r + 1})
\end{equation*} and $r + 1 > 0$, $\lim_{\epsilon \rightarrow 0^+}
\epsilon^{r + 1} = 0$, so $\int_0^a x^r \,\mathrm{d}x =
\frac{a^{r + 1}}{r + 1}$ for $-1 < r < 0$.

\paragraph{(c)} Use Problem 13-15 to show that $\int_0^a x^{-1} \,\mathrm{d}x$
does not make sense, even as a limit.

\paragraph{Solution:} Problem 13-15 implies that \begin{equation*}
  N\int_{1/2}^1 x^{-1} \,\mathrm{d}x = \int_{1/2^N}^1 x^{-1} \,\mathrm{d}x
\end{equation*} so $\lim_{\epsilon \rightarrow 0^+} \int_{\epsilon}^1 1/x
\,\mathrm{d}x$ does not exist, which implies that
$\lim_{\epsilon \rightarrow 0^+} \int_{\epsilon}^a x^{-1} \,\mathrm{d}x =
\lim_{\epsilon \rightarrow 0^+} \int_{\epsilon}^1 x^{-1} \,\mathrm{d}x +
\int_1^a x^{-1} \,\mathrm{d}x$ does not exist either.

\paragraph{(d)} Invent a reasonable definition of $\int_a^0 |x|^r
\,\mathrm{d}x$ for $a < 0$ and compute it for $-1 < r < 0$.

\paragraph{Solution:} We first define $\int_a^0 |x|^r \,\mathrm{d}x =
\lim_{\epsilon \rightarrow 0^-} \int_a^{\epsilon} |x|^r \,\mathrm{d}x$ for $a <
0$, such that \begin{align*}
  \int_a^0 |x|^r \,\mathrm{d}x
  &= \lim_{\epsilon \rightarrow 0^-} \int_a^{\epsilon} |x|^r \,\mathrm{d}x \\
  &= -\lim_{\epsilon \rightarrow 0^+} \int_{\epsilon}^a x^r \,\mathrm{d}x \\
  &= -\frac{a^{r + 1}}{r + 1}
\end{align*}

\paragraph{(e)} Invent a reasonable definition of $\int_{-1}^1 (1 - x^2)^{-1/2}
\,\mathrm{d}x$, as a sum of two limits, and show that the limits exist.

\paragraph{Solution:} Since $\lim_{x \rightarrow -1} (1 - x^2)^{-1/2} =
\lim_{x \rightarrow 1} (1 - x^2)^{-1/2} = \infty$, we define \begin{align*}
  \int_{-1}^1 (1 - x^2)^{-1/2}
  &= \int_{-1}^0 (1 - x^2)^{-1/2} + \int_0^1 (1 - x^2)^{-1/2} \\
  &= \lim_{\epsilon \rightarrow -1^+} \int_{\epsilon}^0 (1 - x^2)^{-1/2}
    \,\mathrm{d}x
    + \lim_{\epsilon \rightarrow 1^-} \int_0^{\epsilon} (1 - x^2)^{-1/2}
    \,\mathrm{d}x \\
  &= 2\lim_{\epsilon \rightarrow -1^+} \int_{\epsilon}^0 (1 - x^2)^{-1/2}
    \,\mathrm{d}x
\end{align*} Now, the limit \begin{equation*}
  \lim_{\epsilon \rightarrow -1^+} \int_{\epsilon}^0 (1 + x)^{-1/2}
  \,\mathrm{d}x = \lim_{\epsilon \rightarrow 0^+} \int_{\epsilon}^1 x^{-1/2}
  \,\mathrm{d}x
\end{equation*} exists by part (a). For $-1 < x < 0$, we have $(1 + x)^{-1/2} >
(1 - x^2)^{-1/2}$. It follows that \begin{equation*}
  \lim_{\epsilon \rightarrow -1^+} \int_{\epsilon}^0 (1 - x^2)^{-1/2}
    \,\mathrm{d}x
\end{equation*} also exists.

\end{document}

