\documentclass{article}

\usepackage{amsmath,amssymb,amsthm}
\usepackage[shortlabels]{enumitem}

\newtheorem{definition}{Definition}
\newtheorem{theorem}{Theorem}
\newtheorem{lemma}{Lemma}

\begin{document}

\title{Chapter 5: Limits}
\maketitle

\begin{definition}[Limit]
  The function \emph{$f$ approaches the limit $l$ near $a$} means: for every
  $\epsilon > 0$ there is some $\delta > 0$ such that, for all $x$, if $0 < |x
  - a| < \delta$, then $|f(x) - l| < \epsilon$.
\end{definition}
This is denoted later on by $\lim_{x \to a} f(x) = l$. One-sided limits can be
defined similarly, and use the notation $\lim_{x \to a^+} f(x) = l$ or $\lim_{x
\downarrow a} f(x) = l$ for limits from above, and $\lim_{x \to a^-} f(x) = l$
or $\lim_{x \uparrow a} f(x) = l$.

\begin{theorem}
  A function cannot approach two different limits near $a$. In other words, if
  $f$ approaches $l$ near $a$, and $f$ approaches $m$ near $a$, then $l = m$.
\end{theorem}
\begin{proof}
  Since $f$ approaches $l$ near $a$, we know that for any $\epsilon > 0$ there
  is some number $\delta_1 > 0$ such that, for all $x$, \[
    \text{if } 0 < |x - a| < \delta_1, \text{ then } |f(x) - l| < \epsilon.
  \] We also know, since $f$ approaches $m$ near $a$, that there is some
  $\delta_2 > 0$ such that, for all $x$, \[
    \text{if } 0 < |x - a| < \delta_2, \text{ then } |f(x) - m| < \epsilon.
  \] We have had to use two numbers, $\delta_1$ and $\delta_2$, since there is
  no guarantee that the $\delta$ which works in one definition will work in the
  other. But, in fact, it is now easy to conclude that for any $\epsilon > 0$
  there is some $\delta > 0$ such that, for all $x$, \[
    \text{if } 0 < |x - a| < \delta, \text{ then } |f(x) - l| < \epsilon
    \text{ and } |f(x) - m| < \epsilon;
  \] we simply choose $\delta = \min(\delta_1, \delta_2)$.

  To complete the proof we just have to pick a particular $\epsilon > 0$ for
  which the two conditions \[
    |f(x) - l| < \epsilon \text{ and } |f(x) - m| < \epsilon
  \] cannot both hold, if $l \neq m$. If $l \neq m$, so that $|l - m| > 0$, we
  can choose $|l - m|/2$ as our $\epsilon$. It follows that there is a $\delta
  > 0$ such that, for all $x$,
  \begin{align*}
    \text{if } 0 < |x - a| < \epsilon,
    \text{ then } &|f(x) - l| < \frac{|l - m|}{2} \\
    \text{ and } &|f(x) - m| < \frac{|l - m|}{2}.
  \end{align*}
  This implies that for $0 < |x - a| < \delta$ we have
  \begin{align*}
    |l - m| = |l - f(x) + f(x) - m|
    &\leq |l - f(x)| + |f(x) - m| \\
    &< \frac{|l - m|}{2} + \frac{|l - m|}{2} \\
    &= |l - m|,
  \end{align*}
  a contradiction.
\end{proof}

\begin{lemma}
  \begin{enumerate}
    \item If \[
        |x - x_0| < \frac{\epsilon}{2} \text{ and } |y - y_0| <
          \frac{\epsilon}{2},
      \] then \[
        |(x + y) - (x_0 + y_0)| < \epsilon.
      \]
    \item If \[
        |x - x_0| < \min\left( 1, \frac{\epsilon}{2(|y_0| + 1)} \right)
        \text{ and } |y - y_0| < \frac{\epsilon}{2(|x_0| + 1)},
      \] then \[
        |xy - x_0y_0| < \epsilon.
      \]
    \item If $y_0 \neq 0$ and \[
        |y - y_0| < \min\left(
          \frac{|y_0|}{2},
          \frac{\epsilon|y_0|^2}{2}
        \right),
      \] then $y \neq 0$ and \[
        \left| \frac{1}{y} - \frac{1}{y_0} \right| < \epsilon.
      \]
  \end{enumerate}
\end{lemma}

\begin{theorem}
  If $\lim_{x \to a}f(x) = l$ and $\lim_{x \to a}g(x) = m$, then
  \begin{enumerate}
    \item $\lim_{x \to a}(f + g)(x) = l + m;$
    \item $\lim_{x \to a}(f \cdot g)(x) = l \cdot m.$
  \end{enumerate}
  Moreover, if $m \neq 0$, then
  \begin{enumerate}
    \setcounter{enumi}{2}
    \item $\lim_{x \to a}\left(\frac{1}{g}\right)(x) = \frac{1}{m}.$
  \end{enumerate}
\end{theorem}

\section*{Exercises}

\paragraph{Problem 14}
\begin{enumerate}[(a)]
  \item By bringing out the constant $b$, \[
      \lim_{x \to 0} \frac{f(bx)}{x}
      = \lim_{x \to 0} \frac{bf(bx)}{bx}
      = b \lim_{x \to 0} \frac{f(bx)}{bx}
      = b \lim_{y \to 0} \frac{f(y)}{y}
      = bl
    \] with $y = bx$. The next to last equality can be justified as follows: If
    $\epsilon > 0$ there is a $\delta > 0$ such that if $0 < |y| < \delta$,
    then $\left| \frac{f(y)}{y} - l \right| < \epsilon$. Then if $0 < |x| <
    \frac{\delta}{b}$, we have $0 < |bx| < \delta$, so $\left| \frac{f(bx)}{bx}
    - l \right| < \epsilon$.
  \item In this case, $\lim_{x \to 0} f(bx)/x = \lim_{x \to 0} f(0)/x$ does not
    exist unless $f(0) = 0$.
  \item One could make use of $\sin{2x} = 2\sin{x}\cos{x}$ in order to derive
    the limit: \[
      \lim_{x \to 0} \frac{\sin 2x}{x}
      = \lim_{x \to 0} \frac{2 \sin x \cos x}{x}
      = 2\lim_{x \to 0} \cos x \lim_{x \to 0} \frac{\sin x}{x}
      = 2\lim_{x \to 0} \frac{\sin x}{x}.
    \]
\end{enumerate}

\paragraph{Problem 20} Consider, for simplicity, the case $a > 0$. Then, for
any choice of $\delta > 0$, there are $x$ with $0 < |x - a| < \delta$ and $f(x)
> a/2$, as well as $x$ with $0 < |x - a| < \delta$ and $f(x) < -a/2$. Since the
distance between $a/2$ and $-a/2$ is $a$, this means that we cannot have $|f(x)
- l| < a$ for all such $x$ with $0 < |x - a| < \delta$, no matter what $l$ is.

\paragraph{Problem 23}
\begin{enumerate}[(a)]
  \item Should $\lim_{x \to 0}f(x)g(x)$ exist, then $\lim_{x \to 0} g(x) =
    \lim_{x \to 0} f(x)g(x)/f(x)$ must also exist.
  \item Clearly, if $\lim_{x \to 0} f(x)g(x)$ exists, then $\lim_{x \to 0} g(x)
    = 0$.
  \item For case (1), it is clear that we cannot have $\lim_{x \to 0} f(x) =
    0$, so by the assumption in (a) the limit $\lim_{x \to 0} f(x)$ cannot
    exist. Let $g = 1/f$. Since it is not true that $\lim_{x \to 0} |f(x)| =
    \infty$, it follows that if $\lim_{x \to 0} g(x)$ does exist, then
    $\lim_{x \to 0} g(x) \neq 0$. But by (a) this would imply that
    $\lim_{x \to 0}f(x)$ exists, so $\lim_{x \to 0} g(x)$ does not exist. On
    the other hand, it is clear that $\lim_{x \to 0} f(x)g(x) = 1$.

    For case (2), choose $x_n$ as in the hint. Define $g(x) = 0$ for $x \neq
    x_n$, and $g(x) = 1$ for $x = x_n$. Then $\lim_{x \to 0} g(x)$ does not
    exist, but $\lim_{x \to 0} f(x)g(x) = 0$.
\end{enumerate}

\paragraph{Problem 24} Given $\epsilon > 0$, pick $n$ with $1/n < \epsilon$ and
let $\delta$ be the minimum distance from $a$ to all points in $A_1, \ldots,
A_n$ (except $a$ itself if $a$ is one of these points). Then, $0 < |x - a| <
\delta$ implies that $x$ is not in $A_1, \ldots, A_n$, so $f(x) = 0$ or $1/m$
for $m > n$, so $|f(x)| < \epsilon$.

\paragraph{Problem 31} Let $l = \lim_{x \to a^-}f(x)$ and $m = \lim_{x \to a^+}
f(x)$. Since $m - l > 0$, there is a $\delta > 0$ such that
\begin{align*}
  |f(x) - l| < \frac{m - l}{2} &\text{ when } a - \delta < x < a, \\
  |f(y) - m| < \frac{m - l}{2} &\text{ when } a < y < a + \delta.
\end{align*}
\[
  f(x) < l + \frac{m - l}{2} = \frac{m + l}{2} = m - \frac{m - l}{2} < f(y)
\]

The converse is false, as shown by $f(t) = t$ and any $a$. It is only possible
to conclude that $\lim_{x \to a^-} f(x) \leq \lim_{x \to a^+} f(x)$.

\paragraph{Problem 32} Naturally, we assume that $a_n, b_m \neq 0$. If $x \neq
0$ and $m < n$, we write \[
  \lim_{x \to \infty} \frac{a_nx^n + \cdots + a_0}{b_mx^m + \cdots + b_0}
  = \lim_{x \to \infty}
  \frac{a_n + \cdots + \frac{a_0}{x^n}}{\frac{b_m}{x^{n-m}} + \cdots +
  \frac{b_0}{x^n}}
  = \frac{f(x)}{g(x)}
\] for $f(x) = a_n + \cdots + \frac{a_0}{x^n}$ and $g(x) = \frac{b_m}{x^{n-m}}
+ \cdots + \frac{b_0}{x^n}$. With $m < n$, it is clear that
$\lim_{x \to \infty} f(x) = a_n$ while $\lim_{x \to \infty} g(x) = 0$, which
implies that $\lim_{x \to \infty} f(x)/g(x)$ cannot exist.

If $x \neq 0$ and $m \geq n$, we write \[
  \lim_{x \to \infty} \frac{a_nx^n + \cdots + a_0}{b_mx^m + \cdots + b_0}
  = \lim_{x \to \infty} \frac{\frac{a_n}{x^{m - n}} + \cdots +
  \frac{a_0}{x^m}}{b_m + \cdots + \frac{b_0}{x^m}}
  = \frac{f(x)}{g(x)}
\] for $f(x) = \frac{a_n}{x^{m - n}} + \cdots + \frac{a_0}{x^m}$ and $g(x) =
b_m + \cdots + \frac{b_0}{x^m}$. Then $\lim_{x \to \infty} f(x) = a_n$ if $m =
n$ and 0 if $m > n$, while $\lim_{x \to \infty} g(x) = b_m$ for $m \geq n$.
This leads to $\lim_{x \to \infty} f(x)/g(x) = a_n/b_m$ if $m = n$ and 0 if $m
> n$.

\end{document}

