\documentclass{article}

\usepackage{amsmath,amssymb,amsthm}
\usepackage[shortlabels]{enumitem}

\newtheorem{corollary}{Corollary}
\newtheorem{definition}{Definition}
\newtheorem*{lemma*}{Lemma}
\newtheorem{lemma}{Lemma}
\newtheorem*{theorem*}{Theorem}
\newtheorem{theorem}{Theorem}

\begin{document}

\title{Chapter 16: $\pi$ is Irrational}
\maketitle

Before the proof, two lemmas are required.

\begin{lemma*}
  The function \[
    f_n(x) = \frac{x^n(1 - x)^n}{n!}
  \] satisfying \[
    0 < f_n(x) < \frac{1}{n!} \text{ for } 0 < x < 1
  \] satisfies various properties:
  \begin{enumerate}
    \item $f_n^{(k)}(0) = f_n^{(k)}(1) = 0$ if $k < n$ or $k > 2n$, and
    \item $f_n^{(k)}(0)$ and $f_n^{(k)}(1)$ are integers for all $k$.
  \end{enumerate}
\end{lemma*}
\begin{proof}
  We first write $f_n$ in the form \[
    f_n(x) = \frac{1}{n!}\sum_{i = n}^{2n} c_ix^i
  \] for integers $c_1, c_2, \ldots, c_n$. Clearly, \[
    f_n^{(k)}(0) = 0 \text{ if } k < n \text{ or } k > 2n.
  \] Moreover,
  \begin{align*}
    f_n^{(n)}(x) &= \frac{1}{n!}[n!c_n + \text{ terms involving } x] \\
    f_n^{(n+1)}(x) &= \frac{1}{n!}[(n+1)!c_{n+1} +
    \text{ terms involving } x] \\
    \vdots& \\
    f_n^{(2n)}(x) &= \frac{1}{n!}[(2n)!c_{2n}].
  \end{align*}
  which implies that
  \begin{align*}
    f_n^{(n)}(0) &= c_n \\
    f_n^{(n+1)}(0) &= (n+1)c_{n+1} \\
    \vdots& \\
    f_n^{(2n)}(0) &= (2n)(2n-1)\cdots(n+1)c_{2n},
  \end{align*}
  where the numbers on the right are all integers. Thus $f_n^{(k)}(0)$ are
  integers for all $k$.

  Furthermore, the relation $f_n(x) = f_n(1 - x)$ implies that $f_n^{(k)}(x) =
  (-1)^kf_n^{(k)}(1 - x)$; therefore, $f_n^{(k)}(1) = 0$ if $k < n$ or $k >
  2n$, and is also an integer for all $k$.
\end{proof}

\begin{lemma*}
  If $a$ is any number, and $\epsilon > 0$, then for sufficiently large $n$ we
  have \[
    \frac{a^n}{n!} < \epsilon.
  \]
\end{lemma*}
\begin{proof}
  Notice that if $n \geq 2a$, then \[
    \frac{a^{n+1}}{(n+1)!} = \frac{a}{n+1} \cdot \frac{a^n}{n!} <
    \frac{1}{2} \cdot \frac{a^n}{n!}.
  \] Now let $n_0$ be any natural number with $n_0 \geq 2a$. Then, whatever
  value \[
    \frac{a^{n_0}}{(n_0)!}
  \] may have, the succeeding values satisfy
  \begin{align*}
    \frac{a^{n_0 + 1}}{(n_0 + 1)!}
    &< \frac{1}{2} \cdot \frac{a^{n_0}}{n_0!} \\
    \frac{a^{n_0 + 2}}{(n_0 + 2)!}
    &< \frac{1}{2} \cdot \frac{a^{n_0 + 1}}{(n_0 + 1)!}
    < \frac{1}{2} \cdot \frac{1}{2} \cdot \frac{a^{n_0}}{n_0!} \\
    \vdots& \\
    \frac{a^{n_0 + k}}{(n_0 + k)!}
    &< \frac{1}{2^k} \cdot \frac{a^{n_0}}{n_0!}.
  \end{align*}
  If $k$ is so large that $\frac{a^{n_0}}{(n_0)!\epsilon} < 2^k$, then \[
    \frac{a^{n_0 + k}}{(n_0 + k)!} < \epsilon.
  \]
\end{proof}

\begin{theorem}
  The number $\pi$ is irrational; in fact, $\pi^2$ is irrational. (Notice that
  the irrationality of $\pi^2$ implies the irrationality of $\pi$, for if $\pi$
  were rational, then $\pi^2$ certainly would be.)
\end{theorem}
\begin{proof}
  Suppose $\pi^2$ were rational, so that \[
    \pi^2 = \frac{a}{b}
  \] for some positive integers $a$ and $b$. Let
  \begin{equation} \label{eq:pi-irr-1}
    G(x) = b^n[
      \pi^{2n}f_n(x) - \pi^{2n - 2}f_n''(x) - \cdots + (-1)^nf_n^{(2n)}(x)
    ].
  \end{equation}
  Notice that each of the factors \[
    b^n \pi^{2n - 2k}
    = b^n (\pi^2)^{n-k}
    = b^n \left( \frac{a}{b} \right)^{n-k}
    = a^{n-k} b^k
  \] is an integer. Since $f_n^{(k)}(0)$ and $f_n^{(k)}(1)$ are integers, this
  shows that $G(0)$ and $G(1)$ are integers. Differentiating $G$ twice yields
  \begin{equation} \label{eq:pi-irr-2}
    G''(x) = b^n[
      \pi^{2n}f_n''(x) - \pi^{2n-2}f_n^{(4)}(x) - \cdots +
      (-1)^nf_n^{(2n+2)}(x)
    ].
  \end{equation}
  The last term, $(-1)^nf_n^{(2n + 2)}(x)$, is zero. Thus, adding
  \eqref{eq:pi-irr-1} and \eqref{eq:pi-irr-2} gives
  \begin{equation}
    G''(x) + \pi^2 G(x) = b^n \pi^{2n + 2} f_n(x) = \pi^2 a^n f_n(x).
  \end{equation}
  Now let \[
    H(x) = G'(x) \sin \pi x - \pi G(x) \cos \pi x.
  \] Then
  \begin{align*}
    H'(x)
    &= \pi G'(x) \cos \pi x + G''(x) \sin \pi x - \pi G'(x) \cos \pi x + \pi^2
    G(x)\sin\pi x \\
    &= [G''(x) + \pi^2 G(x)] \sin \pi x \\
    &= \pi^2 a^n f_n(x) \sin \pi x, \text{ by (3)}.
  \end{align*}
  By the Second Fundamental Theorem of Calculus,
  \begin{align*}
    \pi^2 \int_0^1 a^n f_n(x) \sin \pi x \,dx
    &= H(1) - H(0) \\
    &= G'(1) \sin \pi - \pi G(1) \cos \pi - G'(0) \sin 0 + \pi G(0) \cos 0 \\
    &= \pi[G(1) + G(0)].
  \end{align*} Thus \[
    \pi \int_0^1 a^n f_n(x) \sin \pi x \,dx \text{ is an integer}.
  \] On the other hand, $0 < f_n(x) < 1/n!$ for $0 < x < 1$, so \[
    0 < \pi a^n f_n(x) \sin \pi x < \frac{\pi a^n}{n!} \text{ for } 0 < x < 1.
  \] Consequently, \[
    0 < \pi \int_0^1 a^n f_n(x) \sin \pi x \,dx < \frac{\pi a^n}{n!}.
  \] This reasoning was completely independent of the value of $n$. Now if $n$
  is large enough, then \[
    0 < \pi \int_0^1 a^n f_n(x) \sin \pi x \,dx < \frac{\pi a^n}{n!} < 1.
  \] But this is absurd, because the integral is an integer, and there is no
  integer between 0 and 1. Thus our original assumption must have been
  incorrect: $\pi^2$ is irrational.
\end{proof}

\end{document}

