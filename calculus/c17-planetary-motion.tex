\documentclass{article}

\usepackage{amsmath,amssymb,amsthm}
\usepackage[shortlabels]{enumitem}

\newtheorem{corollary}{Corollary}
\newtheorem{definition}{Definition}
\newtheorem*{lemma*}{Lemma}
\newtheorem{lemma}{Lemma}
\newtheorem*{theorem*}{Theorem}
\newtheorem{theorem}{Theorem}

\DeclareMathOperator{\area}{area}

\begin{document}

\title{Chapter 17: Planetary Motion}
\maketitle

We describe the motion of a planet by the parametrized curve \[
  c(t) = r(t)(\cos \theta(t), \sin \theta(t)),
\] so that $r$ always gives the length of the line from the sun to
the planet, while $\theta$ gives the angle. It will be convenient to write this
as
\begin{equation} \label{eq:planet-disp}
  c(t) = r(t) \cdot \textbf{e}(\theta(t)),
\end{equation}
where \[
  \textbf{e}(t) = (\cos t, \sin t)
\] is just the parametrized curve that runs along the unit circle. Note that \[
  \textbf{e}'(t) = (-\sin t, \cos t)
\] is also a vector of unit length, but perpendicular to $\textbf{e}(t)$, and
that we also have
\begin{equation} \label{eq:e-det}
  \det(\textbf{e}(t), \textbf{e}'(t)) = 1.
\end{equation}
Differentiating \eqref{eq:planet-disp}, we obtain
\begin{equation} \label{eq:planet-vel}
  c'(t) = r'(t) \cdot \textbf{e}(\theta(t))
  + r(t)\theta'(t) \cdot \textbf{e}'(\theta(t))
\end{equation}
and combining with \eqref{eq:planet-disp}, together with the
formulas in Problem 6 of Appendix I to Chapter 4, we get
\begin{align*}
  \det(c(t), c'(t))
  &= r(t)r'(t)\det(\textbf{e}(\theta(t)), \textbf{e}(\theta(t)))
  + r(t)^2\theta'(t)\det(\textbf{e}(\theta(t)), \textbf{e}'(\theta(t))) \\
  &= r(t)^2\theta'(t)\det(\textbf{e}(\theta(t)), \textbf{e}'(\theta(t)))
\end{align*}
since $\det(v, v)$ is always 0. Using \eqref{eq:e-det} we then get
\begin{equation} \label{eq:planet-disp-det}
  \det(c, c') = r^2\theta'.
\end{equation}

Suppose that $A(t)$ is the area swept out from time 0 to $t$. The area of the
triangle $\Delta(h)$ with vertices $O$, $c(t)$, and $c(t + h)$, according to
Problems 4 and 5 of Appendix 1 to Chapter 4, is \[
  \area(\Delta(h)) = \frac{1}{2}\det(c(t), c(t + h) - c(t)).
\] Since the triangle $\Delta(h)$ has practically the same area as the region
$A(t + h) - A(t)$, this shows that
\begin{align*}
  A'(t) &= \lim_{h \to 0}\frac{A(t + h) - A(t)}{h} \\
        &= \lim_{h \to 0}\frac{\area(\Delta(h))}{h} \\
        &= \frac{1}{2}\det\left(
          c(t),
          \lim_{h \to 0}\frac{c(t + h) - c(t)}{h}
        \right) \\
        &= \frac{1}{2}\det(c(t), c'(t)).
\end{align*}

A rigorous derivation can be established with Problem 13-24, which gives a
formula for the area of a region determined by the graph of a function in polar
coordinates. We can write
\begin{equation} \label{eq:area-pol} \tag{*}
  A(t) = \frac{1}{2} \int_0^{\theta(t)} \rho(\phi)^2 \,d\phi
\end{equation}
if our parametrized curve $c(t) = r(t) \cdot \mathbf{e}(\theta(t))$ is the
graph of the function $\rho$ in polar coordinates. Now the function $\rho$ is
\[
  \rho = r \circ \theta^{-1},
\] so applying the First Fundamental Theorem of Caclulus and the Chain Rule on
\eqref{eq:area-pol} we immediately get
\begin{align*}
  A'(t) &= \frac{1}{2}\rho(\theta(t))^2 \cdot \theta'(t) \\
        &= \frac{1}{2} r(t)^2 \theta'(t).
\end{align*} Therefore, we have
\begin{equation} \label{eq:A'}
  \boxed{
    A' = \frac{1}{2}\det(c, c') = \frac{1}{2}r^2\theta'.
  }
\end{equation}

\begin{theorem}
  Kepler's second law is true if and only if the force is central, and in this
  case each planetary path $c(t) = r(t) \cdot \mathbf{e}(\theta(t))$ satisfies
  the equation
  \begin{equation} \label{eq:kep2} \tag{$K_2$}
    A' = \frac{1}{2}\det(c, c') = \text{constant},
  \end{equation} or that $A'' = 0$.
\end{theorem}
\begin{proof}
  First note that \[
    A'' = \frac{1}{2}[\det(c, c')]' = \frac{1}{2}[\det(c', c') + \det(c, c'')]
    = \frac{1}{2}\det(c, c''),
  \] so the statement is equivalent to $\det(c, c'') = 0$. A central force
  points along $c(t)$, but $c''(t)$ points in the direction of the force. This
  is equivalent to saying that $c''(t)$ always points along $c(t)$, so
  $\det(c, c'') = 0$.
\end{proof}

Newton next showed that if the gravitational force of the sun is a central
force and also satisfies an inverse square law, then the path of any object in
it will be a conic section having the sun at one focus.

\begin{theorem}
  If the gravitational force of the sun is a central force that satisfies an
  inverse square law, then the path of any body in it will be a conic section
  having the sun at one focus.
\end{theorem}
\begin{proof}
  Notice that our conclusion specifies the shape of the path, not a particular
  parametrization. But this parametrization is essentially determined by
  Theorem 1: the hypothesis of a central force implies that the area $A(t)$ is
  proportional to $t$, so determining $c(t)$ is essentially equivalent to
  determining $A$ for arbitrary points on the ellipse.

  By Theorem 1, the hypothesis of a central force implies that
  \begin{equation} \tag{$K_2$}
    A' = \frac{1}{2}r^2\theta' = \frac{1}{2}\det(c, c') = \frac{1}{2}M,
  \end{equation}
  for some constant $M$. The hypothesis of an inverse square law can be written
  \begin{equation} \label{eq:planet-acc-1} \tag{*}
    c''(t) = -\frac{H}{r(t)^2}\mathbf{e}(\theta(t)),
  \end{equation}
  for some constant $H$. Using \eqref{eq:kep2}, this can be written \[
    \frac{c''(t)}{\theta'(t)} = -\frac{H}{M}\mathbf{e}(\theta(t)).
  \] Notice that the left-hand side of this equation is \[
    [c' \circ \theta^{-1}]'(\theta(t)).
  \] So if we let \[
    D = c' \circ \theta^{-1},
  \] then this equation can be written as \[
    D'(\theta(t)) = -\frac{H}{M}\mathbf{e}(\theta(t))
    = -\frac{H}{M}(\cos \theta(t), \sin \theta(t)),
  \] and we can write this simply as \[
    D'(u) = -\frac{H}{M}(\cos u, \sin u)
    = \left(-\frac{H}{M} \cos u, -\frac{H}{M} \sin u\right),
  \] completely eliminating $\theta$. This offers a pair of equations for the
  components of $D$, which we find to be \[
    D(u) = \left(
      \frac{H \cdot \sin u}{-M} + A,
      \frac{H \cdot \cos u}{M} + B
    \right)
  \] for two constants $A$ and $B$. Letting $u = \theta(t)$ again we thus have
  an explicit formula for $c'$: \[
    c' = \left(
      \frac{H \cdot \sin \theta}{-M} + A,
      \frac{H \cdot \cos \theta}{M} + B
    \right)
  \] Substituting this with $c = r(\cos \theta, \sin \theta)$ into the equation
  \begin{equation} \tag{$K_2$}
    \det(c, c') = M,
  \end{equation} we get \[
    r\left[
      \frac{H}{M} \cos^2 \theta + B \cos \theta
      + \frac{H}{M} \sin^2 \theta - A \sin \theta
    \right] = M,
  \] which simplifies to \[
    r\left[
      \frac{H}{M^2} + \frac{B}{M} \cos \theta - \frac{A}{M} \sin \theta
    \right] = 1,
  \] Problem 15-8 shows that this can be written in the form \[
    r(t)\left[\frac{H}{M^2} + C\cos(\theta(t) + D)\right] = 1,
  \] for some constants $C$ and $D$. We can let $D = 0$, since this simply
  amounts to rotating our polar coordinate system (choosing which ray
  corresponds to $\theta = 0$), so we can write, finally, \[
    r[1 + \epsilon \cos \theta] = \frac{M^2}{H} = \Lambda.
  \] But this is the formula for a conic section derived in Appendix 3 of
  Chapter 4.
\end{proof}

In terms of the constant $M$ in the equation \[
  r^2\theta' = M
\] and the constant $\Lambda$ in the equation of the orbit \[
  r[1 + \epsilon \cos \theta] = \Lambda
\] the last equation in our proof shows that we can rewrite
\eqref{eq:planet-acc-1} as
\begin{equation} \label{eq:planet-acc-2} \tag{**}
  c''(t) = -\frac{M^2}{\Lambda} \cdot \frac{1}{r^2}\mathbf{e}(\theta(t)).
\end{equation}
Recall that the major axis $a$ of the ellipse is given by
\begin{equation} \label{eq:ell-a} \tag{a}
  a = \frac{\Lambda}{1 - \epsilon^2},
\end{equation}
while the minor axis $b$ is given by
\begin{equation} \label{eq:ell-b} \tag{b}
  b = \frac{\Lambda}{\sqrt{1 - \epsilon^2}}.
\end{equation}
Consequently,
\begin{equation} \label{eq:ell-c} \tag{c}
  \frac{b^2}{\Lambda} = a.
\end{equation}
Remember that \eqref{eq:A'} gives \[
  A'(t) = \frac{1}{2}r^2\theta' = \frac{1}{2}M.
\] and thus \[
  A(t) = \frac{1}{2}Mt.
\] We can therefore interpret $M$ in terms of the period $T$ of the orbit. This
period $T$ is, by definition, the value of $t$ for which we have $\theta(t) =
2\pi$, so that we obtain the complete ellipse. Hence \[
  \text{area of the ellipse} = A(T) = \frac{1}{2}MT
\] or \[
  M = \frac{2(\text{area of the ellipse})}{T} = \frac{2 \pi ab}{T} \quad
  \text{by Problem 13-17}.
\] Hence the constant $M^2/\Lambda$ in \eqref{eq:planet-acc-2} is
\begin{align*}
  \frac{M^2}{\Lambda}
  &= \frac{4 \pi^2 a^2 b^2}{T^2 \Lambda} \\
  &= \frac{4 \pi^2 a^3}{T^2}, \quad \text{using (c)}.
\end{align*} This completes the final step of Newton's analysis.

\begin{theorem}
  Kepler's third law is true if and only if the acceleration $c''(t)$ of any
  planet, moving on an ellipse, satisfies \[
    c''(t) = -G \cdot \frac{1}{r^2} \mathbf{e}(\theta(t))
  \] for a constant $G$ that does not depend on the planet.
\end{theorem}

It should be mentioned that the converse of Theorem 2 is also true. To prove
this, we first want to establish one further consequence of Kepler's second
law. Recall that for \[
  \mathbf{e}(t) = (\cos t, \sin t)
\] we have \[
  \mathbf{e}'(t) = (-\sin t, \cos t).
\] Consequently, \[
  \mathbf{e}''(t) = (-\cos t, -\sin t) = -\mathbf{e}(t).
\] Now differentiating \eqref{eq:planet-vel} gives
\begin{multline*}
  c''(t)
  = r''(t) \cdot \mathbf{e}(\theta(t))
  + r'(t)\theta'(t) \cdot \mathbf{e}'(\theta(t)) \\
  + r'(t)\theta'(t) \cdot \mathbf{e}'(\theta(t))
  + r(t)\theta''(t) \cdot \mathbf{e}'(\theta(t))
  + r(t)\theta'(t)\theta'(t) \cdot \mathbf{e}''(\theta(t)).
\end{multline*}
Using $\mathbf{e}''(t) = -\mathbf{e}(t)$ we get \[
  c''(t)
  = [r''(t) - r(t)\theta'(t)^2] \cdot \mathbf{e}(\theta(t))
  + [2r'(t)\theta'(t) + r(t)\theta''(t)] \cdot \mathbf{e}'(\theta(t)).
\] Since Kepler's second law implies central forces, hence that $c''(t)$ is
always a multiple of $c(t)$, and thus always a multiple of
$\mathbf{e}(\theta(t))$, the coefficient of $\mathbf{e}'(\theta(t))$ must be 0.
Thus Kepler's second law implies that
\begin{equation} \label{eq:planet-acc}
  c''(t) = [r''(t) - r(t)\theta'(t)^2] \cdot \mathbf{e}(\theta(t)).
\end{equation}

\begin{theorem}
  If the path of a planet moving under a central gravitational force is an
  ellipse with the sun as focus, then the force must satisfy an inverse square
  law.
\end{theorem}
\begin{proof}
  Once again, the hypothesis of a central force implies that
  \begin{equation} \label{eq:k3-1} \tag{$K_2$}
    r^2\theta' = M,
  \end{equation} for some constant $M$, and the hypothesis that the path is an
  ellipse with the sun as focus implies that it satisfies the equation
  \begin{equation} \label{eq:k3-2} \tag{A}
    r[1 + \epsilon \cos \theta] = \Lambda,
  \end{equation} for some $\epsilon$ and $\Lambda$. For our proof, we will keep
  differentiating and substituting from these two equations.

  First, we differentiate \eqref{eq:k3-2} to obtain \[
    r'[1 + \epsilon \cos \theta] - \epsilon r \theta' \sin \theta = 0.
  \] Multiplying by $r$ this becomes \[
    rr'[1 + \epsilon \cos \theta] - \epsilon r^2 \theta' \sin \theta = 0.
  \] Using both \eqref{eq:k3-1} and \eqref{eq:k3-2}, this becomes \[
    \Lambda r' - \epsilon M \sin \theta = 0.
  \] Differentiating again, we get \[
    \Lambda r'' - \epsilon M \theta' \cos \theta = 0.
  \] Using \eqref{eq:k3-1} we get \[
    \Lambda r'' - \frac{\epsilon M^2}{r^2} \cos \theta = 0,
  \] and then using \eqref{eq:k3-2} we get \[
    \Lambda r'' - \frac{M^2}{r^2}\left[\frac{\Lambda}{r} - 1\right] = 0.
  \] Substituting from \eqref{eq:k3-1} yet again, we get \[
    \Lambda[r'' - r(\theta')^2] + \frac{M^2}{r^2} = 0,
  \] or \[
    r'' - r(\theta')^2 = -\frac{M^2}{\Lambda r^2}.
  \] Comparing with \eqref{eq:planet-acc}, we obtain \[
    c''(t) = -\frac{M^2}{\Lambda r^2} \mathbf{e}(\theta(t)),
  \] which is precisely what we wanted to show: the force is inversely
  proportional to the square of the distance from the sun to the planet.
\end{proof}

\end{document}

