\documentclass{article}

\usepackage{amsmath,amssymb,amsthm}
\usepackage[shortlabels]{enumitem}

\newtheorem{corollary}{Corollary}
\newtheorem{definition}{Definition}
\newtheorem*{lemma*}{Lemma}
\newtheorem{lemma}{Lemma}
\newtheorem*{theorem*}{Theorem}
\newtheorem{theorem}{Theorem}

\begin{document}

\title{Chapter 15: The Trigonometric Functions}
\maketitle

\begin{definition}
  $\pi$ is the area of the unit circle, or more precisely, twice the area of a
  semicircle: \[
    \pi = 2 \cdot \int_{-1}^1 \sqrt{1 - x^2} \,dx.
  \]
\end{definition}

\begin{definition}
  For $-1 \leq x \leq 1$, the area $A(x)$ of the sector bounded by the unit
  circle, the horizontal axis, and the half-line through $(x, \sqrt{1 - x^2})$
  is \[
    A(x) = \frac{x\sqrt{1 - x^2}}{2} + \int_x^1 \sqrt{1 - t^2} \,dt.
  \]
\end{definition}

Notice that if $-1 < x < 1$, then $A$ is differentiable at $x$ and (using the
Fundamental Theorem of Calculus),
\begin{align*}
  A'(x)
  &= \frac{1}{2}\left[
    x \cdot \frac{-2x}{2\sqrt{1 - x^2}}
    + \sqrt{1 - x^2}
  \right] - \sqrt{1 - x^2} \\
  &= \frac{1}{2}\left[ \frac{-x^2 + (1 - x^2)}{\sqrt{1 - x^2}} \right] -
  \sqrt{1 - x^2} \\
  &= \frac{1 - 2x^2}{2\sqrt{1 - x^2}} - \sqrt{1 - x^2} \\
  &= \frac{1 - 2x^2 - 2(1 - x^2)}{2\sqrt{1 - x^2}} \\
  &= \frac{-1}{2\sqrt{1 - x^2}}.
\end{align*} Notice also that on the interval $[-1, 1]$ the function $A$
decreases from \[
  A(-1) = 0 + \int_{-1}^1 \sqrt{1 - t^2} \,dt = \frac{\pi}{2}
\] to $A(1) = 0$.

\begin{definition}
  For $0 \leq x \leq \pi$ we wish to define $\cos x$ and $\sin x$ as the
  coordinates of a point $P = (\cos x, \sin x)$ on the unit circle which
  determines a sector whose area is $x/2$.

  In other words, $\cos x$ is the unique number in $[-1, 1]$ such that
  \[
    A(\cos x) = \frac{x}{2};
  \] and \[
    \sin x = \sqrt{1 - \cos^2 x}.
  \]
\end{definition}

\begin{theorem}
  If $0 < x < \pi$, then
  \begin{align*}
    \cos'(x) &= -\sin x, \\
    \sin'(x) &= \cos x.
  \end{align*}
\end{theorem}

The values of $\sin x$ and $\cos x$ for $x$ not in $[0, \pi]$ are defined by a
two-step piecing together process:
\begin{enumerate}
  \item If $\pi \leq x \leq 2\pi$, then
    \begin{align*}
      \sin x &= -\sin(2\pi - x), \\
      \cos x &= \cos(2\pi - x).
    \end{align*}
  \item If $x = 2\pi k + x'$ for some integer $k$, and some $x'$ in $[0,
    2\pi]$, then
    \begin{align*}
      \sin x &= \sin x', \\
      \cos x &= \cos x'.
    \end{align*}
\end{enumerate}

The other standard trigonometric functions are defined by
\begin{align*}
  &\left.
    \begin{array}{c}
      \sec x = \frac{1}{\cos x} \\
      \tan x = \frac{\sin x}{\cos x} \\
    \end{array}
  \right\} x \neq k\pi + \pi/2, \\
  &\left.
    \begin{array}{c}
      \csc x = \frac{1}{\sin x} \\
      \cot x = \frac{\cos x}{\sin x}
    \end{array}
  \right\} x \neq k\pi.
\end{align*}

\begin{theorem}
  If $x \neq k\pi + \pi/2$, then
  \begin{align*}
    \sec'(x) &= \sec x \tan x, \\
    \tan'(x) &= \sec^2 x.
  \end{align*}
  If $x \neq k\pi$, then
  \begin{align*}
    \csc'(x) &= -\csc x \cot x, \\
    \cot'(x) &= -\csc^2 x. \\
  \end{align*}
\end{theorem}

The trigonometric functions are not one-one, so it is first necessary to
restrict them to suitable intervals in order to work with their inverses. The
intervals usually chosen are
\begin{align*}
  [-\pi/2, \pi/2] &\text{ for } \sin, \\
  [0, \pi]        &\text{ for } \cos, \\
  (-\pi/2, \pi/2) &\text{ for } \tan.
\end{align*}

\begin{theorem}
  If $-1 < x < 1$, then
  \begin{align*}
    \arcsin'(x) &= \frac{1}{\sqrt{1 - x^2}}, \\
    \arccos'(x) &= \frac{-1}{\sqrt{1 - x^2}}.
  \end{align*}
  Moreover, for all $x$ we have \[
    \arctan'(x) = \frac{1}{1 + x^2}.
  \]
\end{theorem}

To derive the addition formulas, we require a lemma.

\begin{lemma*}
  Suppose $f$ has a second derivative everywhere and that
  \begin{align*}
    f'' + f &= 0, \\
    f(0) &= 0, \\
    f'(0) &= 0.
  \end{align*}
  Then $f = 0$.
\end{lemma*}

\begin{theorem}
  If $x$ and $y$ are any two numbers, then \begin{align*}
    \sin(x + y) &= \sin x \cos y + \cos x \sin y, \\
    \cos(x + y) &= \cos x \cos y - \sin x \sin y.
  \end{align*}
\end{theorem}

\begin{proof}
  For any particular number $y$ we can define a function $f$ by \[
    f(x) = \sin(x + y).
  \] Then
  \begin{align*}
    f'(x) &= \cos(x + y) \\
    f''(x) &= -\sin(x + y).
  \end{align*}
  Consequently,
  \begin{align*}
    f'' + f &= 0, \\
    f(0)    &= \sin y, \\
    f'(0)   &= \cos y.
  \end{align*}
  It follows from Theorem 4 that \[
    f = (\cos y ) \cdot \sin{} + (\sin y) \cdot \cos{};
  \] that is, \[
    \sin(x + y) = \cos y \sin x + \sin y \cos x, \text{ for all } x.
  \] Since any number $y$ could have been chosen to begin with, this proves the
  first formula for all $x$ and $y$.

  The second formula is proven similarly.
\end{proof}

\section*{Exercises}

\paragraph{Problem 26}
\begin{enumerate}[(a)]
  \item \[
      \lim_{\lambda \to \infty} \int_c^d \sin \lambda x \,dx
      = \lim_{\lambda \to \infty} \frac{-1}{\lambda}
      (\cos \lambda d - \cos \lambda c)
      = 0
    \]
  \item Consider a partition $P = \{t_0, \ldots, t_n\}$ of $[a, b]$ such that
    $s$ is a constant on each $(t_{i - 1}, t_i)$. Then each term in the sum \[
      \sum_{i=1}^n \lim_{\lambda \to \infty} \int_{t_{i-1}}^{t_i} s(x)
        \sin \lambda x
      \,dx = \lim_{\lambda \to \infty} \int_a^b
        s(x) \sin \lambda x
      \,dx
    \] is equal to 0 by part (a), so the limit must be 0.
  \item From Problem 13-26, for any $\epsilon > 0$ there is a step function $s
    \leq f$ with $\int_a^b (f - s) < \epsilon$. Now
    \begin{align*}
      \left|
        \int_a^b f(x) \sin \lambda x \,dx - \int_a^b s(x) \sin \lambda x \,dx
      \right|
      &= \left|
        \int_a^b [f(x) - s(x)] \sin \lambda x \,dx
      \right| \\
      &\leq \int_a^b [f(x) - s(x)] \cdot |\sin \lambda x| \,dx \\
      &\leq \int_a^b [f(x) - s(x)] \,dx < \epsilon.
    \end{align*}
    With part (b), we have shown that \[
      \lim_{\lambda \to \infty} \left| \int_a^b f(x)\sin \lambda x \,dx \right|
      < \epsilon \text{ for any } \epsilon > 0,
    \] and so the limit $\lim_{\lambda \to \infty} \int_a^b f(x)\sin \lambda x
    \,dx = 0$.
\end{enumerate}

\paragraph{Problem 28}
\begin{enumerate}[(a)]
  \item From Problem 13-25,
    \begin{align*}
      \mathcal{L}(x)
      &= \int_x^1 \sqrt{1 + f'(t)^2} \,dt \\
      &= \int_x^1 \sqrt{1 +
      \left( \frac{-2t}{2\sqrt{1 - t^2}} \right)^2} \,dt \\
      &= \int_x^1 \sqrt{1 + \frac{t^2}{1 - t^2}} \,dt \\
      &= \int_x^1 \frac{1}{\sqrt{1 - t^2}} \,dt.
    \end{align*}
    (A more detailed proof to deal with the indefinite integral should be
    outlined.)
  \item Rewriting \[
      \mathcal{L}(x) = \int_1^x \frac{-1}{\sqrt{1 - t^2}} \,dt
    \] it is clear by the Fundamental Theorem of Calculus that \[
      \mathcal{L'}(x) = -\frac{1}{\sqrt{1 - x^2}} \text{ for } -1 < x < 1.
    \]
  \item By the definition, $\cos = \mathcal{L}^{-1}$, so
    \begin{align*}
      \cos'(x)
      &= (\mathcal{L}^{-1})'(x)
      = \frac{1}{\mathcal{L}'(\mathcal{L}^{-1}(x))} \\
      &= -\frac{1}{\frac{1}{\sqrt{1 - \cos^2 x}}}
      = -\sin x.
    \end{align*}
    The proof for $\sin' x$ is the same as in the text.
\end{enumerate}

\paragraph{Problem 30}
\begin{enumerate}[(a)]
  \item \[
      (y_0^2 + (y_0')^2)' = 2y_0y_0' + 2y_0'y_0'' = 2y_0'(y_0'' + y_0) = 0,
    \] so $y_0^2 + (y_0')^2$ must be equal to some constant $c$. But $y_0$ is
    not always 0, and so $y_0(x)^2 + y_0'(x)^2 = c \neq 0$ for all $x$. It
    follows that either $y_0(0) \neq 0$ or $y_0'(0) \neq 0$.
  \item Any function $s = ay_0 + by_0'$ satisfies $s'' + s = 0$. Hence, it
    suffices to choose $a$ and $b$ such that:
    \begin{align*}
      ay_0(0) + by_0'(0) &= 0, \\
      ay_0'(0) - by_0(0) &= 1.
    \end{align*}
    This is always possible, since from part (a), either $y_0(0) \neq 0$ or
    $y_0'(0) \neq 0$.
  \item Suppose that $\cos x > 0$ for all $x \geq 0$. Then, as $\cos'(0) =
    -\sin 0 = 0$, by Problem 7 of the Appendix to Chapter 11, $\cos''(x) = 0$
    for some $x > 0$. But $\cos'' = -\cos$, implying that $\cos x = 0$ for some
    $x > 0$ which is a contradiction.
  \item Suppose $\cos x > 0$ for $0 < x < x_0 = \pi/2$, the function $\sin$ is
    increasing on $[0, \pi/2]$ (as $\sin' = \cos$). Since $\sin 0 = 0$, it
    follows that $\sin \pi/2 > 0$, so $\sin \pi/2 = 1$.
  \item A straightforward substitution.
  \item \begin{align*}
      \cos(x + 2\pi) &= \cos x \cos 2\pi - \sin x \sin 2\pi = \cos x, \\
      \sin(x + 2\pi) &= \sin x \cos 2\pi + \cos x \sin 2\pi = \sin x.
    \end{align*}
\end{enumerate}

\paragraph{Problem 32}
\begin{enumerate}[(a)]
  \item
    \begin{align*}
      \phi_1''\phi_2 - \phi_2''\phi_1 - (g_2 - g_1)\phi_1\phi_2
      &= (-g_1\phi_1)\phi_2 - (-g_2\phi_2)\phi_1
        - (g_2\phi_2)\phi_1 + (g_1\phi_1)\phi_2 \\
      &= g_1(-\phi_1\phi_2 + \phi_1\phi_2) + g_2(\phi_2\phi_1 - \phi_2\phi_1) \\
      &= 0.
    \end{align*}
  \item From part (a), \[
      \int_a^b [\phi_1''\phi_2 - \phi_2''\phi_1]
      = \int_a^b (g_2 - g_1)\phi_1\phi_2.
    \] Since $\phi_1(x) > 0$, $\phi_2(x) > 0$ for all $x$ in $(a, b)$, $(g_2 -
    g_1)\phi_1\phi_2 > 0$ for all $x$ and the integral must be positive for $a
    < b$. Note too that
    \begin{align*}
      (\phi_1'\phi_2)' &= \phi_1''\phi_2 + \phi_1'\phi_2', \\
      (\phi_1\phi_2')' &= \phi_2''\phi_1 + \phi_1'\phi_2',
    \end{align*}
    which suggests that
    \begin{align*}
      \int_a^b [\phi_1''\phi_2 - \phi_2''\phi_1]
      &= \int_a^b [\phi_1''\phi_2 + \phi_1'\phi_2']
      - \int_a^b [\phi_2''\phi_1 + \phi_1'\phi_2'] \\
      &= [\phi_1'(b)\phi_2(b) - \phi_1'(a)\phi_2(a)]
      + [\phi_1(b)\phi_2'(b) - \phi_1(a)\phi_2'(a)]
    \end{align*}
    is positive.
  \item From part (b), with $\phi_1(a) = \phi_1(b) = 0$, we have \[
      \phi_1'(b)\phi_2(b) - \phi_1'(a)\phi_2(a) > 0.
    \] But clearly
    \begin{align*}
      \phi_2(a) \geq 0,  \: &\phi_2(b) \geq 0, \\
      \phi_1'(a) \geq 0, \: &\phi_1'(b) \leq 0,
    \end{align*}
    which leads to a contradiction.
  \item This follows from part (c) by replacing $\phi_1$ with $-\phi_1$ or
    $\phi_2$ with $-\phi_2$ where necessary.
\end{enumerate}

The net result of this problem may be stated as follows: if $a$ and $b$ are
consecutive zeros of $\phi_1$, then $\phi_2$ must have a zero somewhere between
$a$ and $b$. This result, in a slightly more general form, is known as the
Sturm Comparison Theorem. As a particular example, any solution of the
differential equation \[
  y'' + (x + 1)y = 0
\] must have zeros on the positive horizontal axis which are within $\pi$ of
each other.

\end{document}

