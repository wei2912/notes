\documentclass{article}

\usepackage{amsmath,amssymb,amsthm}

\newtheorem{definition}{Definition}
\newtheorem{theorem}{Theorem}
\newtheorem*{theorem*}{Theorem}
\newtheorem{lemma}{Lemma}

\begin{document}

\title{Chapter 9: Derivatives}
\maketitle

\begin{definition}
  The function $f$ is \emph{differentiable at $A$} if \begin{equation*}
    \lim_{h \rightarrow 0}\frac{f(a + h) - f(a)}{h}
  \end{equation*} exists.

  In this case the limit is denoted by $f'(a)$ and is called the
  \emph{derivative of $f$ at $a$}. (We also say that $f$ is
  \emph{differentiable} if $f$ is differentiable at $a$ for every $a$ in the
  domain of $f$.)
\end{definition}

\begin{theorem}
  If $f$ is differentiable at $a$, then $f$ is continuous at $a$.
\end{theorem}

\begin{proof}
  \begin{align*}
    \lim_{h \rightarrow 0}{f(a + h) - f(a)} &= \lim_{h \rightarrow 0}{\frac
      {f(a + h) - f(a)}{h} \cdot h} \\
      &= \lim_{h \rightarrow 0}{\frac{f(a + h) - f(a)}{h} \cdot \lim_{h
        \rightarrow 0}h} \\
      &= f'(a) \cdot 0 \\
      &= 0.
  \end{align*}

  As pointed out in Chapter 5, the equation $\lim_{h \rightarrow 0}{f(a + h) -
  f(a)} = 0$ is equivalent to $\lim_{x \rightarrow a}{f(x)} = f(a)$; thus $f$
  is continuous at $a$.
\end{proof}

\section*{Exercises}

\paragraph{Exercise 9-14 (abridged).} Let $f(x) = x^2$ if $x$ is rational, and
$f(x) = 0$ if $x$ is irrational. Prove that $f$ is differentiable at 0.

\paragraph{Solution:} Consider $g(x) = f(x)/x$; $g(x) = x$ if $x$ is rational,
and $g(x) = 0$ if $x$ is irrational. Hence, \begin{equation*}
  \lim_{h \rightarrow 0}{\frac{f(h) - f(0)}{h}} = \lim_{h \rightarrow 0}{\frac
  {f(h)}{h}} = \lim_{h \rightarrow 0}{g(h)} = 0,
\end{equation*} so $f$ is differentiable at 0.

\paragraph{Exercise 9-15 (abridged). (a)} Let $f$ be a function such that
$|f(x)| \leq x^2$ for all $x$. Prove that $f$ is differentiable at 0.

\paragraph{Solution:} Note that by the definition of $f$, $|f(0)| \leq 0^2$, so
$f(0) = 0$. Similar to Exercise 9-14, consider $g(x) = |f(x)|/x$; from $0 \leq
g(x) \leq x^2/x = x$, it is clear that $0 \leq \lim_{h \rightarrow 0}{g(h)}$
and $\lim_{h \rightarrow 0}{g(h)} \leq \lim_{h \rightarrow 0}{h} = 0$, so
$\lim_{h \rightarrow 0}{g(h)} = 0$. Hence, \begin{equation*}
  \lim_{h \rightarrow 0}{\frac{f(h) - f(0)}{h}} = \lim_{h \rightarrow 0}{\frac
  {f(h)}{h}} = \lim_{h \rightarrow 0}{g(h)} = 0,
\end{equation*} so $f$ is differentiable at 0.

\paragraph{(b)} This result can be generalized if $x^2$ is replaced by
$|g(x)|$, where $g$ has what property?

\paragraph{Solution:} Any function $g$ with $\lim_{h \rightarrow 0}g(h)/h = 0$
i.e. a function $g$ with $g(0) = g'(0) = 0$.

\paragraph{Exercise 9-18.} Let $f(x) = 0$ for irrational $x$, and $1/q$ for $x
= p/q$ in lowest terms. Prove that $f$ is not differentiable at $a$ for any
$a$.

\paragraph{Solution:} Since $f$ is not continuous at $a$ if $a$ is rational, it
is also not differentiable at rational $a$. If $a = m.a_1a_2a_3\ldots$ is
irrational and $h$ is rational, then $a + h$ is irrational, so $f(a + h) - f(a)
= 0$. But if $h = -0.00\ldots0a_{n+1}a_{n+2}\ldots$, then $a + h = m.a_1a_2a_3
\ldots a_n000\ldots$, so $f(a + h) \geq 10^{-n}$, while $|h| < 10^{-n}$, so we
have $|[f(a + h) - f(a)]/h| \geq 1$. Thus $[f(a + h) - f(a)]/h$ is 0 for
arbitrarily small $h$ and also has absolute value $\geq 1$ for arbitarily small
$h$. It follows that $\lim_{h \rightarrow 0}{[f(a + h) - f(a)]/h}$ cannot
exist.

\paragraph{Exercise 9-23 (abridged).} Prove that if $f$ is even, then $f'(x) =
-f'(-x)$.

\paragraph{Solution:} Noting that $f(x) = f(-x)$, \begin{align*}
  f'(-x) &= \lim_{h \rightarrow 0}\frac{f(-x - h) - f(-x)}{-h} \\
    &= \lim_{h \rightarrow 0}\frac{f(-x) - f(-x - h)}{h} \\
    &= \lim_{h \rightarrow 0}\frac{f(x) - f(x + h)}{h} \\
    &= -\lim_{h \rightarrow 0}\frac{f(x + h) - f(x)}{h} \\
    &= -f'(x),
\end{align*} which can be rewritten as $f'(x) = -f'(-x)$.

\paragraph{Exercise 9-24 (abridged).} Prove that if $f$ is odd, then $f'(x) =
f'(-x)$.

\paragraph{Solution:} Noting that $f(x) = -f(-x)$, \begin{align*}
  f'(-x) &= \lim_{h \rightarrow 0}\frac{f(-x - h) - f(-x)}{-h} \\
    &= \lim_{h \rightarrow 0}\frac{f(-x) - f(-x - h)}{h} \\
    &= \lim_{h \rightarrow 0}\frac{-f(x) + f(x + h)}{h} \\
    &= \lim_{h \rightarrow 0}\frac{f(x + h) - f(x)}{h} \\
    &= f'(x).
\end{align*}

\paragraph{Exercise 9-28. (a)} Find $f'(x)$ if $f(x) = |x|^3$. Find $f''(x)$.
Does $f'''(x)$ exist for all $x$?

\paragraph{Solution:} Since \begin{equation*}
  f(x) = \begin{cases}
    x^3, x > 0 \\
    -x^3, x < 0,
  \end{cases}
\end{equation*} we have \begin{equation*}
  f'(x) = \begin{cases}
    3x^2, x > 0 \\
    -3x^2, x < 0
  \end{cases}
  f''(x) = \begin{cases}
    6x, x > 0 \\
    -6x, x < 0.
  \end{cases}
\end{equation*}
Moreover, $f'(0) = f''(0) = 0$, while $f'''(0)$ does not exist.

\paragraph{(b)} Analyze $f$ similarly if $f(x) = x^4$ for $x \geq 0$ and $f(x)
= -x^4$ for $x \leq 0$.

\paragraph{Solution:} Since \begin{equation*}
  f(x) = \begin{cases}
    x^4, x > 0 \\
    -x^4, x < 0,
  \end{cases}
\end{equation*} we have \begin{equation*}
  f'(x) = \begin{cases}
    4x^3, x > 0 \\
    -4x^3, x < 0
  \end{cases}
  f''(x) = \begin{cases}
    12x^2, x > 0 \\
    -12x^2, x < 0.
  \end{cases}
  f'''(x) = \begin{cases}
    24x, x > 0 \\
    -24x, x < 0.
  \end{cases}
\end{equation*}
Moreover, $f'(0) = f''(0) = f'''(0) = 0$, while $f^{(4)}(0)$ does not exist.

\paragraph{Exercise 9-29.} Let $f(x) = x^n$ for $x \geq 0$ and let $f(x) = 0$
for $x \leq 0$. Prove that $f^{(n-1)}$ exists (and find a formula for it), but
that $f^(n)(0)$ does not exist.

\paragraph{Solution:} It is clear that \begin{equation*}
  f^{(n-1)}(x) = \begin{cases}
    n!x, x > 0 \\
    0, x \leq 0.
  \end{cases}
\end{equation*}
So $f^{(n)}(0)$ does not exist since $\lim_{h \rightarrow 0^+}n!h/h = n!$,
while $\lim_{h \rightarrow 0^-}0/h = 0$.

\end{document}

