\documentclass{article}

\usepackage{amsmath,amssymb,amsthm}

\newtheorem{theorem}{Theorem}
\newtheorem{lemma}{Lemma}

\begin{document}

\title{Chapter 3: Functions}
\maketitle

\paragraph{Functions.} A \textbf{function} is a collection of pairs of numbers
with the following properties: if $(a, b)$ and $(a, c)$ are both in the
collection, then $b = c$; in other words, the collection must not contain two
different pairs with the same first element.

\paragraph{Domain.} If $f$ is a function, the \textbf{domain} of $f$ is the set
of all $a$ for which there is some $b$ such that $(a, b)$ is in $f$. If $a$ is
in the domain of $f$, it follows from the definition of a function that there
is, in fact, a \emph{unique} number $b$ such that $(a, b)$ is in $f$. This
unique $b$ is denoted by $f(a)$.

\section*{Exercises}

\paragraph{Problem 3-10. (c)} For which functions $b$ and $c$ can we find a function $x$ such
that \begin{equation*}
  (x(t))^2 + b(t)x(t) + c(t) = 0
\end{equation*} for all numbers $t$?

\paragraph{Solution:} (The scope is limited to real functions.) Applying the
quadratic formula, \begin{equation*}
  x(t) = \frac{-b(t) \pm \sqrt{(b(t))^2 - 4(1)(c(t))}}{2(1)}
\end{equation*}. It is clear that for $x(t)$ to be real for all $t$, $(b(t))^2
- 4(1)(c(t)) \geq 0 \iff (b(t))^2 \geq 4c(t)$ for all $t$.

\paragraph{(d)} What conditions must the functions $a$ and $b$ satisfy if there
is to be a function $x$ such that \begin{equation*}
  a(t)x(t) + b(t) = 0
\end{equation*} for all numbers $t$? How many such functions $x$ will there be?

\paragraph{Solution:} If $x(t) = 0$ for all $t$, then the only condition is
$b(t) = 0$ for all $t$. Otherwise, $x = b/a$ and the condition is $a(t) \neq 0$
for all $t$. There will only be one function $x$ in either case.

\paragraph{Problem 3-11 (abridged) (d)} Find a function $H$ such that $H(H(x))
= H(x)$ for all numbers $x$, and such that $H(1) = 36, H(2) = \pi/3, H(13) =
47, H(36) = 36, H(\pi/3) = \pi/3, H(47) = 47$.

\paragraph{Solution:} It is clear that the six conditions can be expressed as
$H(H(1)) = H(1) = 36$, $H(H(2)) = H(2) = \pi/3$ and $H(H(13)) = H(13) = 47$.
What is left is to construct a rule for the remaining numbers in the domain. A
simple way of constructing a rule that fits the condition $H(H(x)) = H(x)$
would be to yield the number $47 = H(13)$. Hence, the function $H$ can be
defined in this manner: \begin{equation*}
  H(x) = \begin{cases}
    36, \text{ if } x = 1 \text{ or } 36 \\
    \pi/3, \text{ if } x = 2 \text{ or } \pi/3 \\
    47, \text{ otherwise.} \\
  \end{cases}
\end{equation*}

\paragraph{(e)} Find a function $H$ such that $H(H(x)) = H(x)$ for all $x$, and
such that $H(1) = 7$, $H(17) = 18$.

\paragraph{Solution:} Using a similar method to the previous part, one simply
needs to impose the condition $H(H(1)) = H(1) = 7$ and to yield the number $18
= H(17)$. The function $H$ can be defined in this manner:
\begin{equation*}
  H(x) = \begin{cases}
    7, \text{ if } x = 1 \text{ or } 7 \\
    18, \text{ otherwise} \\
  \end{cases}
\end{equation*}

\paragraph{Problem 3-13 (abridged). (a)} Prove that any function $f$ with
domain $\mathbb{R}$ can be written $f = E + O$, where $E$ is even and $O$ is
odd. (A function $g$ is \emph{even} if $g(x) = g(-x)$ and \emph{odd} if $g(x) =
-g(-x)$.

\paragraph{Solution:} Consider $f(x)$ and $f(-x)$: \begin{align*}
  f(x)  &= E(x) + O(x), \\
  f(-x) &= E(-x) + O(-x) \\
        &= E(x) - O(x).
\end{align*} It is clear that $E(x)$ and $O(x)$ can be solved to obtain these
two unique solutions: \begin{align*}
  E(x) = \frac{f(x) + f(-x)}{2}, O(x) = \frac{f(x) - f(-x)}{2}.
\end{align*}

\paragraph{(b)} Prove that this way of writing $f$ is unique.

\paragraph{Solution:} Solved in (a).

\paragraph{Problem 3-16.} Suppose $f$ satisfies $f(x + y) = f(x) + f(y)$ for
all $x$ and $y$.

\paragraph{(a)} Prove that $f(x_1 + \cdots + x_n) = f(x_1) + \cdots + f(x_n)$.

\paragraph{Solution:} By definition, the base case $n = 2$ is trivially
proved. Suppose that the statement holds for $n = k$ for some integer $k \geq
2$. Then, \begin{align*}
  f(x_1 + \cdots + x_k + x_{k+1}) &= f(x_1 + \cdots + x_k) + f(x_{k+1})
  \text{ (by definition)} \\
    &= (f(x_1) + \cdots + f(x_k)) + f(x_{k+1}) \text{ (by assumption)} \\
    &= f(x_1) + \cdots + f(x_{k+1}).
\end{align*}
This proves the statement for all $n \geq 2$.

\paragraph{(b)} Prove that there is some number $c$ such that $f(x) = cx$ for
all rational numbers $x$.

\paragraph{Solution:} Let $c = f(1)$. Then, it is clear that $f(n) = cn$ for
any natural number $n$.

Since $f(x) + f(0) = f(x + 0) = f(x)$, it follows that $f(0) = 0$. Since $f(x)
+ f(-x) = f(x + (-x)) = f(0) = 0$, it also follows that $f(x) = -f(-x)$. In
particular, for any natural number $n$, $f(-n) = -f(n) = -cn = c(-n)$.

Moreover, $nf(1/n) = f(1) = c$ which implies that $f(1/n) = c(1/n)$.
Consequently, $f(-1/n) = -f(1/n) = c(-1/n)$. Finally, writing any rational
number as $m/n$ with a natural number $m$ and integer $n$, $f(m/n) = mf(1/n) =
mc/n = c(m/n)$. This proves the statement $f(x) = cx$ for all rational numbers
$x$.

\paragraph{Problem 3-17.} If $f(x) = 0$ for all $x$, then $f$ satisfies $f(x +
y) = f(x) + f(y)$ for all $x$ and $y$, and also $f(x \cdot y) = f(x) \cdot
f(y)$ for all $x$ and $y$. Now suppose that $f$ satisfies these two properties,
but that $f(x)$ is not always 0. Prove that $f(x) = x$ for all $x$, as follows:

\paragraph{(a)} Prove that $f(1) = 1$.

\paragraph{Solution:} From the property $f(x \cdot y) = f(x) \cdot f(y)$,
\begin{equation*}
  f(1) = f(1 \cdot 1) = f(1) \cdot f(1)
\end{equation*} which implies $f(1) = 0, 1$. However, if $f(1) = 0$, then for
any $y$, $f(y) = f(1 \cdot y) = f(1) \cdot f(y) = 0$ which implies that $f(y) =
0$ for all $y$ contrary to the assumption made. Hence, $f(1) = 1$.

\paragraph{(b)} Prove that $f(x) = x$ if $x$ is rational.

\paragraph{Solution:} Let $c = f(1)$. The proof that $f(x) = cx$ for all
rational $x$ is similar to that in Q16. Since $c = 1$, the statement is proven.

\paragraph{(c)} Prove that $f(x) > 0$ if $x > 0$.

\paragraph{Solution:} If $x > 0$, then $x = y^2$ for some real $y$, so $f(x) =
f(y)^2 = (f(y))^2 \geq 0$ by definition. Since $y \neq 0$ (from (a)), it must
follow that $f(y) \neq 0$ and thus $f(x) > 0$.

\paragraph{(d)} Prove that $f(x) > f(y)$ if $x > y$.

\paragraph{Solution:} If $x > y$, then $x - y > 0$. By (c), $f(x) - f(y) > 0$,
which proves $f(x) > f(y)$.

\paragraph{(e)} Prove that $f(x) = x$ for all $x$.

\paragraph{Solution:} Suppose that $f(x) > x$ for some $x$. Choose a rational
number $r$ with $x < r < f(x)$. Then, by (b) and (d), \begin{equation*}
  f(x) < f(r) = r < f(x)
\end{equation*} which is a contradiction. Similarly, if $f(x) < x$ for some
$x$, by choosing a rational number $r'$ with $f(x) < r' < x$, by (b) and (d),
\begin{equation*}
  r' = f(r') < f(x) < r'
\end{equation*} which is a contradiction. Hence, $f(x) = x$ for all $x$.

\paragraph{Problem 3-18.} Precisely what conditions must $f$, $g$, $h$, and $k$
satisfy in order that $f(x)g(y) = h(x)k(y)$ for all $x$ and $y$?

\paragraph{Solution:} If either $f = 0$ or $g = 0$ holds, and also either
$h = 0$ or $k = 0$, then the equation certainly holds. If not, then there
exists some $x'$ with $f(x') \neq 0$ and some $y'$ with $g(y') \neq 0$. Then,
$0 \neq f(x')g(y') = h(x')k(y')$, so we have $h(x') \neq 0$ and $k(y') \neq 0$.
Letting $\alpha = h(x')/f(x')$, $g(y) = \alpha k(y)$ for all $y$. Similarly,
from $\alpha = g(y')/k(y')$, $h(x) = \alpha f(x)$ for all $x$. Hence, $h =
\alpha f$ and $g = \alpha k$ for some $\alpha \neq 0$.

\paragraph{Problem 3-19 (abridged). (a)} Prove that there do not exist
functions $f$ and $g$ with either of the following properties: \begin{align*}
  f(x) + g(y) = xy \text{ for all } x \text{ and } y, \\
  f(x) \cdot g(y) = x + y \text{ for all } x \text{ and } y.
\end{align*}

\paragraph{Solution:} Assuming that there exist functions meeting the first
condition, $f(x) + g(0) = x \cdot 0 = 0$ for all $x$, so $f(x) = -g(0)$.
Likewise, $f(0) + g(y) = 0 \cdot y = 0$ for all $y$, so $g(y) = -f(0) = g(0)$.
This would imply that $f(x) + g(y) = -g(0) + g(0) = 0 = xy$ for all $x$ and
$y$, which is absurd.

Assuming that there exist functions meeting the second condition, $f(x) \cdot
g(0) = x + 0 = x$ for all $x$, so $f(x) = x/g(0)$. Likewise, $f(0) \cdot g(y) =
0 + y = y$ for all $y$, so $g(y) = y/f(0)$. This would imply that $f(x) \cdot
g(y) = x/g(0) \cdot y/f(0) = x + y$ for all $x$ and $y$. For $y = 0$, $x = 0$
for all $x$ which is absurd.

\paragraph{(b)} Find functions $f$ and $g$ such that $f(x + y) = g(xy)$ for all
$x$ and $y$.

\paragraph{Solution:} Let $f$ and $g$ be the same constant function.

\paragraph{Problem 3-20 (abridged). (a)} Find a function $f$, other than a
constant function, such that $|f(y) - f(x)| \leq |y - x|$.

\paragraph{Solution:} Since $0 \leq |f(y) - f(x)| \leq |y - x|$, the
function $f(x) = cx$ for some $0 < c \leq 1$ will satisfy the condition.

\paragraph{(b)} Suppose that $f(y) - f(x) \leq (y - x)^2$ for all $x$ and $y$.
(Why does this imply that $|f(y) - f(x)| \leq (y - x)^2$?) Prove that $f$ is a
constant function.

\paragraph{Solution:} Suppose that $f(y) < f(x)$. Naturally, $f(y) - f(x) < 0
\leq (y - x)^2$. Swapping $x$ and $y$, for $f(x) < f(y)$, the statement $0 \leq
f(x) - f(y) < (x - y)^2$ must also hold. Hence, $|f(y) - f(x)| = |f(x) - f(y)|
\leq (y - x)^2$ for all $x$ and $y$.

Let $m = \frac{x + y}{2}$. Then, $f(y) - f(m) \leq (y - m)^2 = \left(y -
\frac{x + y}{2}\right)^2 = \left(\frac{y - x}{2}\right)^2 = \frac{1}{4}(y -
x)^2$. Similarly, $f(m) - f(x) \leq (m - x)^2 = \left(\frac{x + y}{2} -
x\right)^2 = \left(\frac{y - x}{2}\right)^2 = \frac{1}{4}(y - x)^2$. This
suggests that $f(y) - f(x) \leq \frac{1}{2}(y - x)^2$, providing a stricter
bound than $(y - x)^2$.

The same process can be repeated indefinitely, with the bound tending towards 0
while remaining positive. Hence, $f(y) - f(x) = 0$ and $f$ is a constant
function.

\paragraph{Problem 3-22. (a)} Suppose $g = h \circ f$. Prove that if $f(x) =
f(y)$, then $g(x) = g(y)$.

\paragraph{Solution:} Consider $g(x) = h(f(x))$. Since $f(x) = f(y)$, $g(x) =
h(f(y))$. However, $g(y) = h(f(y))$ too. Hence, $g(x) = g(y)$.

\paragraph{(b)} Conversely, suppose that $f$ and $g$ are two functions such
that $g(x) = g(y)$ whenever $f(x) = f(y)$. Prove that $g = h \circ f$ for some
function $h$.

\paragraph{Solution:} If $z = f(x)$, define $h(z) = g(x)$; this definition
makes sense, as if $z = f(x')$, then $g(x) = g(x')$ by part (a). For $z$ not of
the form $f(x)$, leave $h$ undefined. Then for all $x$ in the domain of $f$, we
have $g(x) = h(f(x))$, proving $g = h \circ f$.

\paragraph{Problem 3-23.} Suppose that $f \circ g = I$, where $I(x) = x$.

\paragraph{(a)} Prove that if $x \neq y$, then $g(x) \neq g(y)$.

\paragraph{Solution:} Suppose $x \neq y$. Then, $g(x) = g(y)$ would imply $x =
f(g(x)) = f(g(y)) = y$ leading to a contradiction.

\paragraph{(b)} Prove that every number $b$ can be written $b = f(a)$ for some
number $a$.

\paragraph{Solution:} From $b = f(g(b))$, let $a = g(b)$.

\paragraph{Problem 3-24. (a)} Suppose $g$ is a function with the property that
$g(x) \neq g(y)$ if $x \neq y$. Prove that there is a function $f$ such that $f
\circ g = I$.

\paragraph{Solution:} The hypothesis can be stated as the contrapositive: If $x
= y$, then $g(x) = g(y)$. Then, applying Q22(b) to $g$ and $I$, $I = h \circ g$
for some function $h$.

\paragraph{(b)} Suppose that $f$ is a function such that every number $b$ can
be written $b = f(a)$ for some number $a$. Prove that there is a function $g$
such that $f \circ g = I$.

\paragraph{Solution:} For each $x$, pick some number $a$ such that $x = f(a)$.
Call this number $g(x)$. Then, $f(g(x)) = x = I(x)$.

\paragraph{Problem 3-25.} Find a function $f$ such that $g \circ f = I$ for
some $g$, but such that there is no function $h$ with $f \circ h = I$.

\paragraph{Solution:} It suffices to find a function $f$ such that $f(x) \neq
f(y)$ if $x \neq y$ (as by Q24(a) there will be a function $g$ such that $g
\circ f = I$), but such that not every number is of the form $f(x)$ (as by
Q22(b) there will not be a function $g$ such that $f \circ g = I$). One such
function is: \begin{equation*}
  f(x) =
  \begin{cases}
    x + 1 & \text{ if } x \geq 0 \\
    x     & \text{ if } x < 0
  \end{cases}
\end{equation*} which ensures that no number in the range $(0, 1)$ is of the
form $f(x)$.

\paragraph{Problem 3-26.} Suppose $f \circ g = I$ and $h \circ f = I$. Prove
that $g = h$.

\paragraph{Solution:} Consider $h \circ (f \circ g) = h \circ I = h$. By the
associative property of composition, as $(h \circ f) \circ g = I \circ g = g$,
this proves $h = g$.

\paragraph{Problem 2-27. (c)} Show that P10-P12 cannot hold. In other words,
show that there is no collection $P$ of functions in $F$, such that P10-P12
hold for $P$. (It is sufficient, and will simplify things, to consider only
functions which are 0 except at two points $x_0$ and $x_1$.)

\paragraph{Solution:} Let $f$ and $g$ be two functions which are 0 except at
$x_0$ and $x_1$, with $f(x_0) = 1$, $f(x_1) = 0$ and $g(x_0) = 0$, $g(x_1) =
1$. Taking P10, neither $f$ nor $g$ is 0, so $f$ or $-f$ would have to be in
the collection $P$, and likewise for $g$ or $-g$. But $(\pm f)(\pm g) = 0$,
which contradicts P12.

\end{document}

