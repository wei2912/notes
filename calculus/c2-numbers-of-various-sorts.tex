\documentclass{article}

\usepackage{amsmath,amssymb,amsthm}
\usepackage[margin=0.5in]{geometry}

\setlength{\parindent}{0pt}
\setlength{\parskip}{1.5pt}

\newtheorem{theorem}{Theorem}
\newtheorem{lemma}{Lemma}

\begin{document}

\section{Chapter 2: Numbers of Various Sorts}

\paragraph{Mathematical Induction.} That's the whole chapter, it's just about natural numbers and mathematical induction.

\section{Exercises}

\setcounter{subsection}{6}
\subsection{Finding general sum of $\sum^n_{k=1} k^p$}

\paragraph{Problem (abridged).} Show that $\sum^n_{k=1} k^p$ can always be written in the form \begin{equation*}
  \frac{n^{p+1}}{p + 1} + An^p + Bn^{p-1} + Cn^{p-2} + \cdots.
\end{equation*}

\paragraph{Solution.} The base case $p = 0$ is trivial. To construct an inductive hypothesis, consider the formula \begin{equation*}
  (k + 1)^{p+1} - k^{p+1} = \binom{p+1}{1} \cdot k^p + \binom{p+1}{2} \cdot k^{p-1} + \cdots + \binom{p+1}{p-1} \cdot k^2 + \binom{p+1}{p} \cdot k + 1.
\end{equation*}

Writing this formula for $k = 1, \ldots, n$, yields \begin{align*}
  2^{p+1} - 1^{p+1} &= \binom{p+1}{1} \cdot 1^p + \binom{p+1}{2} \cdot 1^{p-1} + \cdots + \binom{p+1}{p-1} \cdot 1^2 + \binom{p+1}{p} \cdot 1 + 1 \\
  3^{p+1} - 2^{p+1} &= \binom{p+1}{1} \cdot 2^p + \binom{p+1}{2} \cdot 2^{p-1} + \cdots + \binom{p+1}{p-2} \cdot 2^2 + \binom{p+1}{p-1} \cdot 2 + 1 \\
  \vdots & \\
  (n + 1)^{p+1} - n^{p+1} &= \binom{p+1}{1} \cdot n^p + \binom{p+1}{2} \cdot n^{p-1} + \cdots + \binom{p+1}{p-2} \cdot n^2 + \binom{p+1}{p-1} \cdot n + 1.
\end{align*}. Adding together, we obtain \begin{multline*}
  (n + 1)^{p+1} - 1^{p+1} = \binom{p+1}{1} \cdot (1^p + 2^p + \cdots + n^p) + \binom{p+1}{2} \cdot (1^{p-1} + 2^{p-1} + \cdots + n^{p-1}) \\ + \cdots + \binom{p+1}{p-1} \cdot (1^2 + 2^2 + \cdots + n^2) + \binom{p+1}{p} \cdot (1 + 2 + \cdots + n) + n
\end{multline*} which can be rewritten as \begin{equation*}
  (n + 1)^{p+1} - 1^{p+1} = \binom{p+1}{1} \sum^n_{k=1} k^p + \binom{p+1}{2} \sum^n_{k=1} k^{p-1} + \cdots + \binom{p+1}{p-1} \sum^n_{k=1} k^2 + \binom{p+1}{p} \sum^n_{k=1} k + n.
\end{equation*} It then follows naturally that, by expansion of $(n + 1)^{p+1}$ and assuming that the statement to be proven holds for all sums of lower order than $p$, that \begin{equation*}
  n^{p+1} + f(n) = \binom{p+1}{1} \sum^n_{k=1} k^p
\end{equation*} where $f(n)$ is a polynomial of degree $p$ in $n$. Clearly, the statement is proven true.

\setcounter{subsection}{16}
\subsection{Proving properties of roots and primes}

\paragraph{Problem (a) (abridged).} Rigorously prove, by complete induction, that any natural number can be written as a product of primes by factorising it into smaller numbers repeatedly.

\paragraph{Solution (a).} Suppose that every number at least 2 and less than $n$ can be written as a product of primes. Clearly, if $n$ is a prime, it can be written as a product of primes (namely, itself). If $n$ is not a prime, then $n = ab$ for some natural numbers $2 \leq a, b < n$. By the assumption that $a$ and $b$ can be written as products of primes, $n = ab$ can also be written as a product of primes.

\paragraph{Problem (b) (abridged).} Using the unique factorisation theorem (as stated in (a)), prove that $\sqrt{n}$ is irrational unless $n = m^2$ for some natural number $m$.

\paragraph{Solution (b).} For some rational $\sqrt{n}$ i.e. $\sqrt{n} = a/b$ for some natural numbers $a, b$, then $b\sqrt{n} = a$ and their prime factorisations must be the same. Squaring both sides, $b^2n = a^2$. Since every prime appears an even number of times in the factorisation of $a^2$, and of $b^2$, the same must be true of the factorisation of $n$. Hence, $n$ must be a square. This proves the statement.

\paragraph{Problem (c) (abridged.)} Prove more generally that $\sqrt[k]{n}$ is irrational unless $n = m^k$.

\paragraph{Solution (c).} Applying the same argument in (b), for some rational $\sqrt[k]{n}$ i.e. $\sqrt[k]{n} = a/b$ for some natural numbers $a, b$, $b^kn = a^k$. Since every prime appears a multiple of $k$ times in the factorisation of $a^k$, and of $b^k$, the same must be true of the factorisation of $n$. Hence, $n = m^k$ for some natural number $m$. This proves the statement.

\paragraph{Problem (d) (abridged.)} Prove that there cannot be only finitely many prime numbers $p_1, p_2, p_3, \ldots, p_n$ by considering $p_1 \cdot p_2 \cdot \ldots \cdot p_n + 1$ (Euclid's proof of infinitude of primes).

\paragraph{Solution (d).} Suppose that there are a finite number of primes, as stated in (d). Then, $p_1 \cdot p_2 \cdot \ldots \cdot p_n + 1$ would divide each prime $p_1, p_2, p_3, \ldots, p_n$ with remainder 1. Should this be a prime, then there is a contradiction; should this be a composite number, by the unique factorisation theorem, the number can be rewritten into a product of primes none of which are $p_1, p_2, p_3, \ldots, p_n$, leading into a contradiction. Hence, there must be an infinite number of primes.

\setcounter{subsection}{17}
\subsection{Proving properties of polynomials}

\paragraph{Problem (a) (abridged).} Prove that if $x$ satisfies \begin{equation*}
  x^n + a_{n-1}x^{n-1} + \cdots + a_0 = 0
\end{equation*} for some integers $a_{n-1}, \ldots, a_0$, then $x$ is irrational unless $x$ is an integer.

\paragraph{Solution (a).} For some rational solution $x$ i.e. $x = b/c$ for some integers $b, c$ with no common factors, \begin{equation*}
  \frac{b^n}{c^n} + a_{n-1} \cdot \frac{b^{n-1}}{c^{n-1}} + \cdots + a_0 \cdot \frac{b^0}{c^0} = 0
\end{equation*} which can be multiplied by $c^n$ and rewritten to yield \begin{equation*}
  b^n + a_{n-1} \cdot b^{n-1}c^1 + \cdots + a_0 \cdot c^n = 0.
\end{equation*} If $c \neq \pm1$, then $c$ has some prime number as a factor. This prime factor divides every term other than $b^n$, so it must divide $b^n$ too. However, since $b$ and $c$ share no common factors, the prime factor cannot divide $b^n$. Hence, $c = \pm1$. This proves that $x$ is an integer.

\paragraph{Problem (b).} Prove that $\sqrt{6} - \sqrt{2} - \sqrt{3}$ is irrational.

\paragraph{Solution (b).} Let $x = \sqrt{6} - \sqrt{2} - \sqrt{3}$. Then, \begin{align*}
  x^2 &= 6 + (\sqrt{2} + \sqrt{3})^2 - 2\sqrt{6}(\sqrt{2} + \sqrt{3}) \\
      &= 11 + 2\sqrt{6}[1 - (\sqrt{2} + \sqrt{3})],
\end{align*} which implies \begin{align*}
  (x^2 - 11)^2 &= 24[1 - (\sqrt{2} + \sqrt{3})]^2 \\
               &= 24[1 + (\sqrt{2} + \sqrt{3})^2 - 2(\sqrt{2} + \sqrt{3})] \\
               &= 24[6 + 2\sqrt{2}\sqrt{3} - 2(\sqrt{2} + \sqrt{3})] \\
               &= 24(6 + 2x).
\end{align*} From (a), $x$ is irrational unless it is an integer. It is easy to check that $0 < \sqrt{6} - \sqrt{2} - \sqrt{3} < 1$, and so $x$ must be irrational.

\paragraph{Problem (c).} Prove that $\sqrt{2} + \sqrt[3]{2}$ is irrational.

\paragraph{Solution (c).} Let $x = \sqrt{2} + \sqrt[3]{2} = \gamma^2 + \gamma^3$ where $\gamma = 2^{1/6}$. Considering the first sixth powers of $x$ and constructing a table for the coefficients of each term,

\begin{tabular}{c | c c c c c c}
  & $\gamma^0$ & $\gamma^1$ & $\gamma^2$ & $\gamma^3$ & $\gamma^4$ & $\gamma^5$ \\
  \hline
  $x^0$ & 1 & & & & & \\
  $x^1$ & & & 1 & 1 & & \\
  $x^2$ & 2 & & & & 1 & 2 \\
  $x^3$ & 2 & 6 & 6 & 2 & & \\
  $x^4$ & 4 & & 2 & 8 & 12 & 8 \\
  $x^5$ & 40 & 40 & 20 & 4 & 2 & 10 \\
  $x^6$ & 12 & 24 & 60 & 80 & 60 & 24 \\
\end{tabular}

We can then find numbers $a_0, \ldots, a_5$ such that \begin{equation*}
  x^6 + a_5x^5 + \cdots + a_0 = 0,
\end{equation*} by substituting in $x^0, x^1, \ldots, x^6$ in terms of $\gamma$. This leaves us with the following matrix: \begin{equation*}
  \begin{bmatrix}
    1 & 0 & 2 & 2 & 4 & 40 & 12 \\
    0 & 0 & 0 & 6 & 0 & 40 & 24 \\
    0 & 1 & 0 & 6 & 2 & 20 & 60 \\
    0 & 1 & 0 & 2 & 8 & 4 & 80 \\
    0 & 0 & 1 & 0 & 12 & 2 & 60 \\
    0 & 0 & 2 & 0 & 8 & 10 & 24 \\
  \end{bmatrix}
  \begin{bmatrix}
    a_0 \\
    a_1 \\
    a_2 \\
    a_3 \\
    a_4 \\
    a_5 \\
    1 \\
  \end{bmatrix}
  =
  0
\end{equation*} which turns out to have a solution corresponding to the equation \begin{equation*}
  x^6 + 6x^5 + 4x^4 - 12x^3 + 24x^2 + 4x = 0.
\end{equation*} Part (a) implies that either $x$ is irrational or $x$ is an integer, and it is easy to see that $x$ is not an integer, so $x$ must be irrational.

\end{document}

