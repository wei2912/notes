\documentclass{article}

\usepackage{amsmath,amssymb,amsthm}
\usepackage[shortlabels]{enumitem}

\newtheorem{corollary}{Corollary}
\newtheorem{definition}{Definition}
\newtheorem{lemma}{Lemma}
\newtheorem{theorem}{Theorem}

\begin{document}

\title{Chapter 13: Integrals}
\maketitle

The number which we will eventually assign as the area of $R(f, a, b)$ given a
function $f$ is called the \emph{integral} of $f$ on $[a, b]$.

\begin{definition}
  Let $a < b$. A \emph{partition} of the interval $[a, b]$ is a finite
  collection of points in [a, b], one of which is $a$, and one of which is $b$.
\end{definition}

The points in a partition can be numbered $t_0, \ldots, t_n$ so that \[
  a = t_0 < t_1 < \cdots < t_{n - 1} < t_n = b;
\] we shall always assume that such a numbering has been assigned.

\begin{definition}
  Suppose $f$ is bounded on $[a, b]$ and $P = \{t_0, \ldots, t_n\}$ is a
  partition of $[a, b]$. Let
  \begin{align*}
    m_i &= \inf\{f(x): t_{i - 1} \leq x \leq t_i\}, \\
    M_i &= \sup\{f(x): t_{i - 1} \leq x \leq t_i\},
  \end{align*}

  The \emph{lower sum} of $f$ for $P$, denoted by $L(f, P)$, is defined as \[
    L(f, P) = \sum_{i = 1}^n m_i(t_i - t_{i - 1}).
  \]

  The \emph{upper sum} of $f$ for $P$, denoted by $U(f, P)$, is defined as \[
    U(f, P) = \sum_{i = 1}^n M_i(t_i - t_{i - 1}).
  \]
\end{definition}

We begin by considering two partitions $P$ and $Q$.

\begin{lemma}
  If $Q$ contains $P$ (i.e., if all points of $P$ are also in $Q$), then
  \begin{align*}
    L(f, P) \leq L(f, Q), \\
    U(f, P) \geq U(f, Q).
  \end{align*}
\end{lemma}
\begin{proof}
  Consider first the special case in which $Q$ contains just one more point
  than $P$:
  \begin{align*}
    P &= \{t_0, \ldots, t_n\}, \\
    Q &= \{t_0, \ldots, t_{k - 1}, u, t_{k + 1}, \ldots, t_n\},
  \end{align*} where \[
    a = t_0 < t_1 < \cdots < t_{k - 1} < u < t_k < \cdots < t_n = b.
  \]

  Let
  \begin{align*}
    m'  &= \inf\{f(x): t_{k - 1} \leq x \leq u\}, \\
    m'' &= \inf\{f(x): u \leq x \leq t_k\}.
  \end{align*}
  Then
  \begin{align*}
    L(f, P) &= \sum_{i=1}^n m_i(t_i - t_{i-1}), \\
    L(f, Q) &= \sum_{i=1}^{k-1} m_i(t_i - t_{i-1}) + m'(u - t_{k-1}) +
    m''(t_k - u) + \sum_{i=k+1}^n m_i(t_i - t_{i-1}).
  \end{align*} To prove that $L(f, P) \leq L(f, Q)$ it therefore suffices to
  show that \[
    m_k(t_k - t_{k-1}) \leq m'(u - t_{k-1}) + m''(t_k - u).
  \] Now the set $\{f(x): t_{k-1} \leq x \leq t_k\}$ contains all the numbers
  in $\{f(x): t_{k-1} \leq x \leq u\}$, and possibly some smaller ones, so the
  greatest lower bound of the first set is \emph{less than or equal to} the
  greatest lower bound of the second; thus \[
    m_k \leq m'.
  \] Similarly, \[
    m_k \leq m''.
  \] Therefore, \[
    m_k(t_k - t_{k-1}) = m_k(u - t_{k-1}) + m_k(t_k - u) \leq m'(u - t_{k-1}) +
    m''(t_k - u).
  \] This proves, in the special case, that $L(f, P) \leq L(f, Q)$. The proof
  that $U(f, P) \geq U(f, q)$ is similar, and is left to you as an easy, but
  valuable, exercise.

  The general case can now be deduced quite easily. The partition $Q$ can be
  obtained from $P$ by adding one point at a time; in other words, there is a
  sequence of partitions \[
    P = P_1, P_2, \ldots, P_{\alpha} = Q
    \] such that $P_{j + 1}$ contains just one more point than $P_j$. Then \[
    L(f, P) = L(f, P_1) \leq L(f, P_2) \leq \cdots \leq L(f, P_{\alpha}) = L(f,
      Q),
  \] and \[
    U(f, P) = U(f, P_1) \geq U(f, P_2) \geq \cdots \geq U(f, P_{\alpha})
    = U(f, Q).
  \]
\end{proof}

The theorem we wish to prove is a simple consequence of this lemma.

\begin{theorem}
  Let $P_1$ and $P_2$ be partitions of $[a, b]$, and let $f$ be a function
  which is bounded on $[a, b]$. Then \[
    L(f, P_1) \leq U(f, P_2).
  \]
\end{theorem}

Theorem 1 leads to the consequence that $\sup\{L(f, P)\} \leq \inf\{U(f, P)\}$.
It may well happen that $\sup\{L(f, P)\} = \inf\{U(f, P)\}$ which is an ideal
candidate for the area of $R(f, a, b)$, and we could maintain that whenever
$\sup\{L(f, P)\} \neq \inf\{U(f, P)\}$ the region $R(f, a, b)$ is too
unreasonable to have an area.

\begin{definition}
  A function $f$ which is bounded on $[a, b]$ is \emph{integrable} on $[a, b]$
  if \[
    \sup\{L(f, P): P \text{ a partition of } [a, b]\}
    = \inf\{U(f, P): P \text{ a partition of } [a, b]\}.
  \] In this case, this common number is called the \emph{integral} of $f$ on
  $[a, b]$ and is denoted by \[
    \int_a^b f.
  \] The integral $\int_a^b f$ is also called the \emph{area} of $R(f, a, b)$
  when $f(x) \geq 0$ for all $x$ in $[a, b]$.
\end{definition}

If $f$ is integrable, then according to this definition, \[
  L(f, P) \leq \int_a^b f
  \leq U(f, P) \text{ for all partitions } P \text{ of } [a, b].
\] Moreover, $\int_a^b f$ is the \emph{unique} number with this property.

\begin{theorem}
  If $f$ is bounded on $[a, b]$, then $f$ is integrable on $[a, b]$ if and only
  if for every $\epsilon > 0$ there is a partition $P$ of $[a, b]$ such that \[
    U(f, P) - L(f, P) < \epsilon.
  \]
\end{theorem}
\begin{proof}
  Suppose first that for every $\epsilon > 0$ there is a partition $P$ with \[
    U(f, P) - L(f, P) < \epsilon.
  \] Since
  \begin{align*}
    \inf\{U(f, P')\} &\leq U(f, P), \\
    \sup\{L(f, P')\} &\geq L(f, P),
  \end{align*}
  it follows that \[
    \inf\{U(f, P')\} - \sup\{L(f, P')\} < \epsilon.
  \] Since this is true for all $\epsilon > 0$, it follows that \[
    \sup\{L(f, P')\} = \inf\{U(f, P')\};
  \] by definition, then, $f$ is integrable. The proof of the converse
  assertion is similar: If $f$ is integrable, then \[
    \sup\{L(f, P)\} = \inf\{U(f, P)\}.
  \] This means that for each $\epsilon > 0$ there are partitions $P'$, $P''$
  with \[
    U(f, P'') - L(f, P') < \epsilon.
  \] Let $P$ be a partition which contains both $P'$ and $P''$. Then, according
  to the lemma,
  \begin{align*}
    U(f, P) \leq U(f, P''), \\
    L(f, P) \geq L(f, P');
  \end{align*}
  consequently, \[
    U(f, P) - L(f, P) \leq U(f, P'') - L(f, P') < \epsilon.
  \]
\end{proof}

\begin{theorem}
  If $f$ is continuous on $[a, b]$, then $f$ is integrable on $[a, b]$.
\end{theorem}
\begin{proof}
  Notice, first, that $f$ is bounded on $[a, b]$, because it is continuous on
  $[a, b]$. To prove that $f$ is integrable on $[a, b]$, we want to use Theorem
  2, and show that for every $\epsilon > 0$ there is a partition $P$ of $[a,
  b]$ such that \[
    U(f, P) - L(f, P) < \epsilon.
  \] Now we know, by Theorem 1 of the Appendix to Chapter 8, that $f$ is
  uniformly continuous on $[a, b]$. So there is some $\delta > 0$ such that for
  all $x$ and $y$ in $[a, b]$, \[
    \text{if } |x - y| < \delta,
    \text{ then } |f(x) - f(y)| < \frac{\epsilon}{2(b - a)}.
  \] The trick is simply to choose a partition $P = \{t_0, \ldots, t_n\}$ such
  that each $|t_i - t_{i - 1}| < \delta$. Then for each $i$ we have \[
    |f(x) - f(y)|
    < \frac{\epsilon}{2(b - a)} \text{ for all } x, y \in [t_{i - 1}, t_i],
    \] and it follows easily that \[
    M_i - m_i \leq \frac{\epsilon}{2(b - a)} < \frac{\epsilon}{b - a}.
  \] Since this is true for all $i$, we then have
  \begin{align*}
    U(f, P) - L(f, P)
    &= \sum_{i = 1}^n (M_i - m_i)(t_i - t_{i - 1}) \\
    &< \frac{\epsilon}{b - a}\sum_{i = 1}^n t_i - t_{i - 1} \\
    &= \frac{\epsilon}{b - a} \cdot (b - a) \\
    &= \epsilon,
  \end{align*}
  which is what we wanted.
\end{proof}

Proofs of integrability could use the criteria of either Theorem 2 or 3.

\begin{theorem}
  Let $a < c < b$. If $f$ is integrable on $[a, b]$, then $f$ is integrable on
  $[a, c]$ and on $[c, b]$. Conversely, if $f$ is integrable on $[a, c]$ and on
  $[c, b]$, then $f$ is integrable on $[a, b]$. Finally, if $f$ is integrable
  on $[a, b]$, then \[
    \int_a^b f = \int_a^c f + \int_c^b f.
  \]
\end{theorem}

\begin{theorem}
  If $f$ and $g$ are integrable on $[a, b]$, then $f + g$ is integrable on $[a,
  b]$ and \[
    \int_a^b (f + g) = \int_a^b f + \int_a^b g.
  \]
\end{theorem}

\begin{theorem}
  If $f$ is integrable on $[a, b]$, then for any number $c$, the function $cf$
  is integrable on $[a, b]$ and \[
    \int_a^b cf = c \cdot \int_a^b f.
  \]
\end{theorem}

\begin{theorem}
  Suppose $f$ is integrable on $[a, b]$ and that \[
    m \leq f(x) \leq M \text{ for all } x \text{ in } [a, b].
  \] Then \[
    m(b - a) \leq \int_a^b f \leq M(b - a).
  \]
\end{theorem}

With an integrable function $f$, one can define a function $F(x) = \int_a^x f$.

\begin{theorem}
  If $f$ is integrable on $[a, b]$ and $F$ is defined on $[a, b]$ by
  \begin{equation*}
    F(x) = \int_a^b f,
  \end{equation*} then $F$ is continuous on $[a, b]$.
\end{theorem}

\begin{proof}
  Suppose $c$ is in $[a, b]$. Since $f$ is integrable on $[a, b]$ it is, by
  definition, bounded on $[a, b]$; let $M$ be a number such that \[
    |f(x)| \leq M \text{ for all } x \text{ in } [a, b].
  \]

  If $h > 0$, then \[
    F(c + h) - F(c) = \int_a^{c + h} f - \int_a^c f = \int_c^{c + h} f.
  \] Since \[
    -M \leq f(x) \leq M \text{ for all } x,
  \] it follows from Theorem 7 that \[
    -Mh \leq \int_c^{c + h} f \leq Mh;
  \] in other words, \[
    -Mh \leq F(c + h) - F(c) \leq Mh.
  \] If $h < 0$, a similar inequality can be derived: Note that \[
    F(c + h) - F(c) = \int_c^{c + h} f = -\int_{c + h}^c f.
  \] Applying Theorem 7 to the interval $[c + h, c]$, of length $-h$, we obtain
  \[
    Mh \leq \int_{c + h}^c f \leq -Mh;
  \] multiplying by -1, which reverses all the inequalities, we have \[
    Mh \leq F(c + h) - F(c) \leq -Mh.
  \]

  Combining these two cases: \[
    |F(c + h) - F(c)| \leq M \cdot |h|.
  \] Therefore, if $\epsilon > 0$, we have \[
    |F(c + h) - F(c)| < \epsilon,
  \] provided that $|h| < \epsilon/M$. This proves that \[
    \lim_{h \to 0} F(c + h) = F(c);
  \] in other words $F$ is continuous at $c$.
\end{proof}

\section*{Exercises}

\paragraph{Problem 3}
\begin{enumerate}[(a)]
  \item Rewriting the sum,
    \begin{align*}
      \sum_{k=1}^n k^p/n^{p+1}
      &= \frac{1}{n^{p+1}} \sum_{k=1}^n k^p \\
      &= \frac{1}{n^{p+1}} \left(
        \frac{n^{p+1}}{p+1} + An^p + Bn^{p-1} + \cdots
      \right) \\
      &= \frac{1}{p+1} + \frac{A}{n} + \frac{B}{n^2} + \cdots.
    \end{align*}
    As \[
      \lim_{n \to \infty} \left(
        \frac{A}{n} + \frac{B}{n^2} + \cdots
      \right) = 0,
    \] it follows that \[
      \lim_{n \to \infty} \sum_{k=1}^n k^p/n^{p+1} = \frac{1}{p+1}.
    \]
  \item Let $f(x) = x^p$. For any partition $P_n = \{0, \frac{b}{n},
    2\frac{b}{n}, \ldots, (n - 1)\frac{b}{n}, b\}$, we have
    \begin{align*}
      L(f, P_n)
      &= \frac{b}{n} \sum_{k=0}^{n-1} \left( \frac{kb}{n} \right)^p
      = \frac{b^{p+1}}{n^{p+1}} \sum_{k=0}^{n-1}k^p, \\
      U(f, P_n)
      &= \frac{b}{n} \sum_{k=1}^n \left( \frac{kb}{n} \right)^p
      = \frac{b^{p+1}}{n^{p+1}} \sum_{k=1}^n k^p. \\
    \end{align*}
    Part (a) shows that $L(f, P_n)$ and $U(f, P_n)$ can be made as close to
    $b^{p+1}/(p+1)$ as desired by choosing $n$ sufficiently large. As in
    Problem 1, this implies that $\int_0^b x^p \,dx = b^{p + 1}/(p + 1)$.
\end{enumerate}

\paragraph{Problem 4}
\begin{enumerate}[(a)]
  \item The product \[
      \frac{t_1}{t_0} \frac{t_2}{t_1} \cdots \frac{t_i}{t_{i - 1}}
      = \frac{t_i}{t_0}
    \] can be rewritten into \[
      r^i = \frac{t_i}{a}
    \] for some ratio $r$. With $i = n$, $r^n = \frac{b}{a} = c$ and $r =
    c^{1/n}$ which naturally follows to the above result for $t_i$.
  \item We have
    \begin{align*}
      U(f, P)
      &= \sum_{i=1}^n [a \cdot c^{i/n}]^p \cdot
      [a \cdot c^{i/n} - a \cdot c^{(i-1)/n}] \\
      &= a^{p+1}(1 - c^{1/n}) \sum_{i=1}^n (c^{(p+1)/n})^i \\
      &= \cdots \\
      &= (b^{p+1} - a^{p+1})c^{p/n} \cdot \frac{1}{1 + c^{1/n} + \cdots +
        c^{p/n}}
    \end{align*}
    which leads to the formulae
    \begin{align*}
      U(f, P)
      &= (b^{p + 1} - a^{p + 1})c^{p/n} \cdot \frac{1}{1 + c^{1/n} + \cdots +
      c^{p/n}} \\
      L(f, P) = c^{-p/n}U(f, P)
      &= (b^{p + 1} - a^{p + 1})\frac{1}{1 + c^{1/n} + \cdots + c^{p/n}}.
    \end{align*}
  \item By making $n$ large enough, we can make $c^{1/n}$ as close to 1 as
    desired. The same holds for each of the $p$ numbers $c^{1/n}, c^{2/n},
    \ldots, c^{p/n}$. So $U(f, P)$ and $L(f, P)$ can both be made as close as
    desired to \[
      \frac{b^{p + 1} - a^{p + 1}}{p + 1}.
    \]
\end{enumerate}

\paragraph{Problem 15} Let $m_i = \inf\{\frac{1}{x}: t_{i - 1} \leq x \leq
t_i\}$ and $m'_i = \inf\{\frac{1}{x}: bt_{i - 1} \leq x \leq bt_i\}$. Then,
noting that $m_i/b = m'_i$, we have
\begin{align*}
  L(f, P')
  &= \sum_{i = 1}^n m'_i(bt_i - bt_{i - 1}) \\
  &= \sum_{i = 1}^n bm'_i(t_i - t_{i - 1}) \\
  &= \sum_{i = 1}^n m_i(t_i - t_{i - 1}) \\
  &= L(f, P).
\end{align*}
So \[
  \int_b^{ab} 1/t \,dt
  = \sup\{L(f, P')\}
  = \sup\{L(f, P)\}
  = \int_1^a 1/t \,dt.
\]

\paragraph{Problem 16} If $P = \{t_0, \ldots, t_n\}$ is a partition of $[a,
b]$, and $P' = \{ct_0, \ldots, ct_n\}$, then \[
  m_i
  = \inf\{f(ct): t_{i-1} \leq t \leq t_i\}
  = \inf\{f(t): ct_{i-1} \leq t \leq ct_i\} = m'_i.
\] So if $g(t) = f(ct)$, then
\begin{align*}
  cL(g, P) &= c\sum_{i=1}^n m_i(t_i - t_{i-1}) \\
           &= \sum_{i=1}^n m'_i(ct_i - ct_{i-1}) \\
           &= L(f, P').
\end{align*}
Hence \[
  \int_{ca}^{cb} f(t) \,dt
  = \sup\{L(f, P')\}
  = c \cdot \sup\{L(g, P)\}
  = c \cdot \int_a^b f(ct) \,dt.
\] (The problem is only solved for $c \geq 0$, but the case $c < 0$ can be
easily deduced.)

\paragraph{Problem 21}
\begin{enumerate}[(a)]
  \item
    \begin{align*}
      L(f^{-1}, P) + U(f, P')
      &= \sum_{i=1}^n f^{-1}(t_{i-1})(t_i - t_{i-1})
      + \sum_{i=1}^n t_i(f^{-1}(t_i) - f^{-1}(t_{i-1})) \\
      &= \sum_{i=1}^n [t_if^{-1}(t_i) - t_{i-1}f^{-1}(t_{i-1})] \\
      &= t_nf^{-1}(t_n) - t_0f^{-1}(t_0) = bf^{-1}(b) - af^{-1}(a).
    \end{align*}
  \item It follows from (a) that
    \begin{align*}
      \int_a^b f^{-1} = \sup\{L(f^{-1}, P)\}
      &= \sup\{bf^{-1}(b) - af^{-1}(a) - U(f, P')\} \\
      &= bf^{-1}(b) - af^{-1}(a) - \inf\{U(f, P')\} \\
      &= bf^{-1}(b) - af^{-1}(a) - \int_{f^{-1}(a)}^{f^{-1}(b)} f.
    \end{align*}
  \item
    \begin{align*}
      \int_a^b \sqrt[n]{x}
      &= b\sqrt[n]{x} - a\sqrt[n]{x} - \int_{\sqrt[n]{a}}^{\sqrt[n]{b}} x^n \\
      &= b\sqrt[n]{x} - a\sqrt[n]{x} - \frac{1}{n+1}[(\sqrt[n]{b})^{n+1} -
      (\sqrt[n]{a})^{n+1}] \\
      &= b\sqrt[n]{x} - a\sqrt[n]{x} - \frac{1}{n+1}(b\sqrt[n]{b} -
      a\sqrt[n]{a}) \\
      &= \frac{n}{n+1}(b\sqrt[n]{b} - a\sqrt[n]{a}).
    \end{align*}
\end{enumerate}

\paragraph{Problem 25}
\begin{enumerate}[(a)]
  \item If $f(x) = \alpha x + \beta$, then $f(t_i) - f(t_{i-1}) = \alpha (t_i
    - t_{i-1})$, and so
    \begin{align*}
      l(f, P)
      &= \sum_{i=1}^n \sqrt{(t_i - t_{i-1})^2 + (\alpha(t_i - t_{i-1}))^2} \\
      &= \sqrt{1 + \alpha^2}\sum_{i=1}^n (t_i - t_{i-1}) \\
      &= \sqrt{1 + \alpha^2}(t_n - t_0) = \sqrt{1 + \alpha^2}(b - a).
    \end{align*}
    The distance from $(a, \alpha a + \beta)$ to $(b, \alpha b + \beta)$ is \[
      \sqrt{(b - a)^2 + ((\alpha b + \beta) - (\alpha a + \beta))^2}
    \] which simplifies to the same expression.
  \item By the triangle inequality,
    \begin{align*}
      l(f, P)
      &= \sqrt{(t - a)^2 + (f(t) - f(a))^2} + \sqrt{(b - t)^2 + (f(b) -
      f(t))^2} \\
      &\geq \sqrt{(b - a)^2 + (f(b) - f(a))^2}
    \end{align*}
    which is the distance from $(a, f(a))$ to $(b, f(b))$. But $f$ is not
    linear, so there must exist a $t$ such that $(a, f(a))$, $(t, f(t))$ and
    $(b, f(b))$ do not lie on a straight line, so that $l(f, P) \neq
    \sqrt{(b - a)^2 + (f(b) - f(a))^2}$.
  \item This follows immediately from (b).
  \item  By the Mean Value Theorem, there is some $x_i$ in $(t_{i-1}, t_i)$
    with \[
      f'(x_i)(t_i - t_{i-1}) = f(t_i) - f(t_{i-1}),
    \] so \[
      L(\sqrt{1 + (f')^2}, P)
      \leq \sum_{i=1}^n \sqrt{1 + [f'(x_i)]^2}(t_i - t_{i-1})
      \leq U(\sqrt{1 + (f')^2}, P)
    \] and
    \begin{align*}
      \sum_{i=1}^n \sqrt{1 + [f'(x_i)]^2}(t_i - t_{i-1})
      &= \sum_{i=1}^n \sqrt{(t_i - t_{i-1})^2 + [f'(x_i)(t_i - t_{i-1})]^2} \\
      &= \sum_{i=1}^n \sqrt{(t_i - t_{i-1})^2 + [f(t_i) - f(t_{i-1})]^2} \\
      &= l(f, P).
    \end{align*}
  \item Since $\sup\{l(f, P)\}$ is an upper bound for the set of all $l(f, P)$,
    it is also an upper bound for the set of all $L(\sqrt{1 + (f')^2}, P)$ by
    part (d).
  \item It suffices to show that \[
      \sup\{l(f, P)\} \leq U(\sqrt{1 + (f')^2}, P'')
    \] for any partition $P''$, and to prove this it suffices to show that \[
      l(f, P') \leq U(\sqrt{1 + (f')^2}, P'')
    \] for any partition $P'$. If $P$ contains the points of $P'$ and $P''$,
    then $l(f, P) \geq l(f, P')$ by part (b), putting in one point at a time,
    so \[
      l(f, P')
      \leq l(f, P)
      \leq U(\sqrt{1 + (f')^2}, P)
      \leq U(\sqrt{1 + (f')^2}, P'').
    \]
  \item We are considering \[
      \lim_{x \to a} \frac{\int_a^x \sqrt{1 + (f')^2}}{\sqrt{(x - a)^2 +
      [f(x) - f(a)]^2}} = 1.
    \] By the Mean Value Theorem, $f(x) - f(a) = (x - a)f'(b)$ for some $b$ in
    $(a, x)$, and by the Mean Value Theorem for integrals (Problem 23), the
    numerator is $(x - a)\sqrt{1 + (f'(c))^2}$ for some $c$ in $(a, x)$. So we
    are considering \[
      \frac{(x - a)\sqrt{1 + f'(c)^2}}{\sqrt{(x - a)^2 + f'(b)^2(x - a)^2}}
      = \frac{\sqrt{1 + f'(c)^2}}{\sqrt{1 + f'(b)^2}}
    \] which approaches 1 as $x \to a$ (assuming $f'$ is continuous at $a$).
\end{enumerate}

\paragraph{Problem 30}
\begin{enumerate}[(a)]
  \item Noting that
    \begin{align*}
      U(f, P) - L(f, P)
      &= \sum_{i=1}^n M_i(t_i - t_{i-1}) - \sum_{i=1}^n m_i (t_i - t_{i-1}) \\
      &= \sum_{i=1}^n (M_i - m_i)(t_i - t_{i-1}),
    \end{align*}
    should $M_i - m_i \geq 1$ for all $i$, clearly $U(f, P) - L(f, P) \geq b -
    a$ which is a contradiction.
  \item It suffices to pick $[a_1, b_1] = [t_{i-1}, t_i]$ from part (a),
    unless $i = 1$ or $n$. For $i = 1$, it suffices to pick $b_1 = t_1$ and
    choose any $a_1$ with $t_0 < a_1 < t_1$. Similarly, for $i = n$, it
    suffices to pick $a_1 = t_{n-1}$ and choose any $b_1$ with $t_{n-1} <
    b_1 < t_n$.
  \item Choose a partition $P$ of $[a_1, b_1]$ with $U(f, P) - L(f, P) < (b_1 -
    a_1)/2$. Then $M_i - m_i < 1/2$ for some $i$. Choose $[a_2, b_2] =
    [t_{i-1}, t_i]$ unless $i = 1$ or $n$, in which case use the modification
    in part (b).
  \item Let $x$ be a point in each $I_n$. For all $n$, $a_n < x < b_n$. If
    $\epsilon > 0$, there is some $n$ such that \[
      \sup\{f(y): y \in I_n\} - \inf\{f(y): y \in I_n\} < \epsilon.
    \] Then $|f(y) - f(x)| < \epsilon$ for all $y$ in $I_n$; since $x$ is in
    $(a_n, b_n)$, this means that $|f(y) - f(x)| < \epsilon$ for all $y$
    satisfying $|y - x| < \delta$ where $\delta > 0$ is the minimum of $x -
    a_n$ and $b_n - x$. Thus $f$ is continuous at $x$.
  \item $f$ must be continuous at some point in every interval contained in
    $[a, b]$, since it is integrable on every such interval.
\end{enumerate}

\paragraph{Problem 31}
\begin{enumerate}[(a)]
  \item Choose $x_0$ in $[a, b]$, and let $f(x) = 0$ for all $x \neq x_0$ and
    $f(x_0) = 1$.
  \item There is a partition $P$ of $[a, b]$ such that $f(x) > x_0/2$ for all
    $x$ in some $[t_{i-1}, t_i]$. Then $L(f, P) > x_0(t_i - t_{i-1})/2$.
  \item This follows from part (b), since $f$ is continuous at some $x_0$, by
    Problem 30.
\end{enumerate}

\paragraph{Problem 32}
\begin{enumerate}[(a)]
  \item Choose $g = f$; then $\int_a^b f^2 = 0$. Since $f$ is continuous, this
    implies that $f = 0$.
  \item Suppose that for some $x_0$ in $[a, b]$, $f(x_0) > 0$. As $f$ is
    continuous, $f(x) > 0$ for all $x$ in $(x_0 - \delta, x_0 + \delta)$ for
    some $\delta > 0$. Choose a continuous $g$ with $g > 0$ on $(x_0 - \delta,
    x_0 + \delta)$ and 0 elsewhere. Then $\int_a^b fg > 0$, a contradiction. A
    similar argument can be made for $f(x_0) < 0$.
\end{enumerate}

\paragraph{Problem 34} Let $\epsilon > 0$. Choose $n$ such that $1/n <
\epsilon/2$. Let $x_0 < x_1 < \ldots < x_m$ be those rational points $p/q$ in
$[0, 1]$ for $q < n$. Choose a partition with $\{t_0, \ldots, t_k\}$ such that
the intervals $[t_{i-1}, t_i]$ which contain some $x_j$ have total length $<
\epsilon/2$. On all intervals we have $M_i \leq 1$, with intervals not
containing $x_j$ having $M_i \leq 1/n < \epsilon/2$. Let $I_1$ denote all
these $i$ from $1, \ldots, n$ for which $[t_{i-1}, t_i]$ contains some $x_j$,
and $I_2$ denote all the other $i$ from $1, \ldots, n$. Then \begin{align*}
  U(f, P)
  &= \sum_{i \in I_1} M_i(t_i - t_{i-1}) + \sum_{i \in I_2} M_i(t_i - t_{i-1})
  \\
  &\leq 1 \cdot \sum_{i \in I_1} (t_i - t_{i-1}) + \frac{\epsilon}{2} \cdot
  \sum_{i \in I_2} (t_i - t_{i-1}) \\
  &\leq 1 \cdot \frac{\epsilon}{2} + \frac{\epsilon}{2} \cdot 1 = \epsilon.
\end{align*}

\paragraph{Problem 35} Let $f$ be the function in Problem 34, and let $g(x) =
0$ for $x = 0$, and $g(x) = 1$ for $x \neq 0$. Then $(f \circ g)(x) = 0$ if $x$
is irrational, and 1 if $x$ is rational.

\paragraph{Problem 36}
\begin{enumerate}[(a)]
  \item If $f \geq 0$ on $[t_{i-1}, t_i]$, then $M'_i = M_i$ and $m'_i = m_i$.
    If $f \leq 0$ on $[t_{i-1}, t_i]$, then $M'_i = -m_i$ and $m'_i = -M_i$.
    Either ways, $M'_i - m'_i \leq M_i - m_i$. Now suppose that $f$ has both
    positive and negative values on $[t_{i-1}, t_i]$, so that $m_i < 0 < M_i$.
    There are two cases to consider. If $-m_i \leq M_i$ then \[
      M'_i = M_i,
    \] so \[
      M'_i - m'_i \leq M'_i = M_i < M_i - m_i
    \] since $m_i < 0$. A similar argument works if $-m_i \geq M_i$.
  \item If $P$ is a partition of $[a, b]$, then
    \begin{align*}
      U(|f|, P) - L(|f|, P)
      &= \sum_{i=1}^n (M'_i - m'_i)(t_i - t_{i-1}) \\
      &\leq \sum_{i=1}^n (M_i - m_i)(t_i - t_{i-1}) \\
      &= U(f, P) - L(f, P).
    \end{align*} So integrability of $f$ implies integrability of $|f|$, by
    Theorem 2.
  \item This follows from part (b) and with the application of Theorem 5 on \[
      \max(f, g) = \frac{f + g + |f - g|}{2},
      \min(f, g) = \frac{f + g - |f - g|}{2}.
    \]
  \item If $f$ is integrable, then $\max(f, 0)$ and $\min(f, 0)$ are integrable
    by part (c). Conversely, if $\max(f, 0)$ and $\min(f, 0)$ are integrable,
    then $f = \max(f, 0) + \min(f, 0)$ is integrable too by Theorem 5.
\end{enumerate}

\paragraph{Problem 38}
\begin{enumerate}[(a)]
  \item For all $x$ in $[t_{i - 1}, t_i]$, \[
      0 \leq m'_i \leq f(x) \leq M'_i \text{ and } 0 \leq m''_i \leq g(x) \leq
        M''_i.
    \] Hence, we have \[
      m'_im''_i \leq f(x)g(x) \leq M'_iM''_i
    \] which implies that $m'_im''_i \leq m_i$ and $M_i \leq M'_iM''_i$.
  \item This follows immediately from part (a).
  \item By part (b),
    \begin{align*}
      U(fg, P) - L(fg, P)
      &\leq \sum_{i=1}^n (M'_iM''_i - m'_im''_i)(t_i - t_{i-1}) \\
      &= \sum_{i=1}^n M''_i(M'_i - m'_i)(t_i - t_{i-1})
      + \sum_{i=1}^n m'_i(M''_i - m''_i)(t_i - t_{i-1}) \\
      &\leq M\left[
        \sum_{i=1}^n (M'_i - m'_i)(t_i - t_{i-1})
        + \sum_{i=1}^n (M''_i - m''_i)(t_i - t_{i - 1})
      \right].
    \end{align*}
  \item Integrability of $fg$ follows immediately from part (c) and Theorem 2.
  \item The result clearly holds if $f \leq 0$ or $g \leq 0$ on $[a, b]$. Now
    write $f = \max(f, 0) + \min(f, 0)$ and $g = \max(g, 0) + \min(g, 0)$, so
    that $fg$ is the sum of four products, each of which is integrable.
\end{enumerate}

\paragraph{Problem 40} If $\epsilon > 0$, pick $N \geq 0$ so that $|f(t) - a| <
\epsilon$ for $t \geq N$. Then for $N \geq 0$ we have \[
  \left| \int_N^{N + M} f(t) - a \,dt \right|
  = \left| \int_N^{N + M} f(t) \,dt - Ma \right| < \epsilon M,
\] and so \[
  \left|
    \frac{1}{N + M}\int_N^{N + M} f(t) \,dt - \frac{Ma}{N + M}
  \right| < \frac{\epsilon M}{N + M} < \epsilon.
\] Choose $M$ so that \[
  \left| \frac{Ma}{N + M} - a \right| < \epsilon
  \text{ and } \left| \frac{1}{N + M}\int_0^N f(t) \,dt \right| < \epsilon.
\] Then \[
    \left| \frac{1}{N + M}\int_0^{N + M} f(t) \,dt - a \right| < 3\epsilon.
\]

\end{document}

