\documentclass{article}

\usepackage{amsmath,amssymb,amsthm}

\newtheorem{definition}{Definition}
\newtheorem{theorem}{Theorem}
\newtheorem{lemma}{Lemma}

\begin{document}

\title{Chapter 6: Continuous Functions}
\maketitle

\begin{definition}[Continuity]
  The function $f$ is \emph{continuous at $a$} if $\lim_{x \rightarrow a}f(x)
  = f(a)$.

  If $f$ is continuous at $x$ for all $x$ in $(a, b)$, then $f$ is called
  \emph{continuous on $(a, b)$}. A function $f$ is called \emph{continuous on
  $[a, b]$} if \begin{enumerate}
    \item $f$ is continuous at $x$ for all $x$ in $(a, b)$,
    \item $\lim_{x \rightarrow a^+}f(x) = f(a)$ and $\lim_{x \rightarrow b^-}
      f(x) = f(b)$.
  \end{enumerate}
\end{definition}

\begin{theorem}
  If $f$ and $g$ are continuous at $a$, then \begin{enumerate}
    \item $f + g$ is continuous at $a$,
    \item $f \cdot g$ is continuous at $a$.
  \end{enumerate}
  Moreover, if $g(a) \neq 0$, then \begin{enumerate}
    \setcounter{enumi}{2}
    \item $1/g$ is continuous at $a$.
  \end{enumerate}
\end{theorem}

\begin{proof}
  Proofs are self-evident from Theorem 5-2.
\end{proof}

\begin{theorem}
  If $g$ is continuous at $a$, and $f$ is continuous at $g(a)$, then $f \circ
  g$ is continuous at $a$.
\end{theorem}

\begin{proof}
  Let $\epsilon > 0$. We wish to find a $\delta > 0$ such that for all $x$,
  \begin{align*}
    \text{if } |x - a| < \delta, &\text{ then } |(f \circ g)(x) -
    (f \circ g)(a)| < \epsilon, \\
      &\text{i.e., } |f(g(x)) - f(g(a))| < \epsilon.
  \end{align*}
  Since $f$ is continuous at $g(a)$, there is a $\delta' > 0$ such that for all
  $x$, \begin{equation*}
    \text{if } |g(x) - g(a)| < \delta', \text{ then } |f(g(x)) - f(g(a))| <
      \epsilon.
  \end{equation*}
  We also conclude, by the continuity of $g$ at $a$, that there is a $\delta >
  0$ such that for all $x$, \begin{equation*}
    \text{if } |x - a| < \delta, \text{ then } |g(x) - g(a)| < \delta'.
  \end{equation*}
  Combining these two statements, we see that for all $x$, \begin{equation*}
    \text{if } |x - a| < \delta, \text{ then } |f(g(x)) - f(g(a))| < \epsilon.
  \end{equation*}
\end{proof}

\begin{theorem}
  Suppose $f$ is continuous at $a$, and $f(a) > 0$. Then there is a number
  $\delta > 0$ such that $f(x) > 0$ for all $x$ satisfying $|x - a| < \delta$.
  Similarly, if $f(a) < 0$, then there is a number $\delta > 0$ such that $f(x)
  < 0$ for all $x$ satisfying $|x - a| < \delta$.
\end{theorem}

\begin{proof}
  Consider the case $f(a) > 0$. Since $f$ is continuous at $a$, if $\epsilon >
  0$ there is a $\delta > 0$ such that, for all $x$, \begin{equation*}
    \text{if } |x - a| < \delta, \text{ then } |f(x) - f(a)| < \epsilon.
  \end{equation*}
  Since $f(a) > 0$ we can take $f(a)$ as the $\epsilon$. Thus there is a
  $\delta > 0$ so that for all $x$, \begin{equation*}
    \text{if } |x - a| < \delta, \text{ then } |f(x) - f(a)| < f(a),
  \end{equation*}
  and this last inequality implies $f(x) > 0$. A similar proof can be given in
  the case $f(a) < 0$.
\end{proof}

\section*{Exercises}

\paragraph{Problem 6-12. (a)} Prove that if $f$ is continuous at $l$ and
$\lim_{x \rightarrow a}g(x) = l$, then $\lim_{x \rightarrow a}f(g(x)) = f(l)$.

\paragraph{Solution:} Consider a function $G$ with $G(x) = g(x)$ for $x
\neq a$, and $G(a) = l$. Such a function is guaranteed to be continuous at $a$.

Then, from Theorem 2, it is clear that as $G$ is continuous at $a$ and $f$ is
continuous at $G(a)$, then $f \circ G$ is continuous at $a$. Thus, $f(l) =
f(G(a)) = \lim_{x \rightarrow a}f(G(x)) = \lim_{x \rightarrow a}(f \circ G)(x)$
which proves the statement.

\paragraph{(b)} Show that if continuity of $f$ at $l$ is not assumed,
then it is not generally true that $\lim_{x \rightarrow a}f(g(x)) = f(\lim_{x
\rightarrow a}g(x))$.

\paragraph{Solution:} Let $g = l + x - a$ and \begin{equation*}
  f(x) = \begin{cases}
    0, &x \neq l, \\
    1, &x = l. \\
  \end{cases}
\end{equation*}
Then, as $\lim_{x \rightarrow a}g(x) = l$, $f(\lim_{x \rightarrow a}g(x)) = 1$.
Yet, since $g(x) \neq l$ for $x \neq a$, $\lim_{x \rightarrow a}f(g(x)) = 0$.

\paragraph{Problem 6-16.} If $\lim_{x \rightarrow a}f(x)$ exists, but is $\neq
f(a)$, then $f$ is said to have a \emph{removable discontinuity} at $a$.

\paragraph{(d)} Let $f$ be a function with the property that every point of
discontinuity is a removable discontinuity. This means that $\lim_{y
\rightarrow x}f(y)$ exists for all $x$, but $f$ may be discontinuous at some
(even infinitely many) numbers $x$. Define $g(x) = \lim_{y \rightarrow x}f(y)$.
Prove that $g$ is continuous.

\paragraph{Solution:} Let $g(a) = \lim_{y \rightarrow a}f(y)$. By definition,
for any $\epsilon > 0$, there is a $\delta > 0$ such that $|f(y) - g(a)| <
\epsilon$ for $|y - a| < \delta$. Similarly, $|g(x) - g(a)| = |\lim_{y
\rightarrow x}f(y) - g(a)| < \epsilon$ for $|x - a| < \delta$, which shows that
$g$ is continuous at $a$.

\end{document}

