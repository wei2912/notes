\documentclass{article}

\usepackage{amsmath,amssymb,amsthm}
\usepackage[shortlabels]{enumitem}

\newtheorem{definition}{Definition}
\newtheorem{theorem}{Theorem}
\newtheorem{lemma}{Lemma}

\begin{document}

\title{Chapter 3: Functions}
\maketitle

\begin{definition}
  A \emph{function} is a collection of pairs of numbers with the following
  properties: if $(a, b)$ and $(a, c)$ are both in the collection, then $b =
  c$; in other words, the collection must not contain two different pairs with
  the same first element.
\end{definition}

If $f$ is a function, the \emph{domain} of $f$ is the set of all $a$ for which
there is some $b$ such that $(a, b)$ is in $f$. If $a$ is in the domain of $f$,
it follows from the definition of a function that there is, in fact, a
\emph{unique} number $b$ such that $(a, b)$ is in $f$. This unique $b$ is
denoted by $f(a)$.

\section*{Exercises}

\paragraph{Problem 10}
\begin{enumerate}[(c)]
  \item (The scope is limited to real functions.) Applying the quadratic
    formula, \[
      x(t) = \frac{-b(t) \pm \sqrt{(b(t))^2 - 4(1)(c(t))}}{2(1)}.
    \] It is clear that for $x(t)$ to be real for all $t$, $(b(t))^2 -
    4(1)(c(t)) \geq 0 \iff (b(t))^2 \geq 4c(t)$ for all $t$.
  \item If $x(t) = 0$ for all $t$, then the only condition is $b(t) = 0$ for
    all $t$. Otherwise, $x = b/a$ and the condition is $a(t) \neq 0$ for all
    $t$. There will only be one function $x$ in either case.
\end{enumerate}

\paragraph{Problem 11}
\begin{enumerate}[(d)]
  \item It is clear that the six conditions can be expressed as $H(H(1)) = H(1)
    = 36$, $H(H(2)) = H(2) = \pi/3$ and $H(H(13)) = H(13) = 47$. What is left
    is to construct a rule for the remaining numbers in the domain. A simple
    way of constructing a rule that fits the condition $H(H(x)) = H(x)$ would
    be to yield the number $47 = H(13)$. Hence, the function $H$ can be defined
    in this manner: \[
      H(x) =
      \begin{cases}
        36, \text{ if } x = 1 \text{ or } 36 \\
        \pi/3, \text{ if } x = 2 \text{ or } \pi/3 \\
        47, \text{ otherwise.} \\
      \end{cases}
    \]
  \item Using a similar method to the previous part, one simply needs to impose
    the condition $H(H(1)) = H(1) = 7$ and to yield the number $18 = H(17)$.
    The function $H$ can be defined in this manner: \[
      H(x) =
      \begin{cases}
        7, \text{ if } x = 1 \text{ or } 7 \\
        18, \text{ otherwise} \\
      \end{cases}
    \]
\end{enumerate}

\paragraph{Problem 13}
\begin{enumerate}[(a)]
  \item Consider $f(x)$ and $f(-x)$:
    \begin{align*}
      f(x)  &= E(x) + O(x), \\
      f(-x) &= E(-x) + O(-x) \\
            &= E(x) - O(x).
    \end{align*}
    It is clear that $E(x)$ and $O(x)$ can be solved to obtain these two unique
    solutions: \[
      E(x) = \frac{f(x) + f(-x)}{2}, O(x) = \frac{f(x) - f(-x)}{2}.
    \]
  \item Solved in (a).
\end{enumerate}

\paragraph{Problem 16}
\begin{enumerate}[(a)]
  \item By definition, the base case $n = 2$ is trivially proved. Suppose that
    the statement holds for $n = k$ for some integer $k \geq 2$. Then,
    \begin{multline*}
      f(x_1 + \cdots + x_k + x_{k+1}) \\
      \begin{aligned}
        &= f(x_1 + \cdots + x_k) + f(x_{k+1}) &\text{ (by definition)} \\
        &= (f(x_1) + \cdots + f(x_k)) + f(x_{k+1}) &\text{ (by assumption)} \\
        &= f(x_1) + \cdots + f(x_{k+1}).
      \end{aligned}
    \end{multline*}
    This proves the statement for all $n \geq 2$.
  \item Let $c = f(1)$. Then, it is clear that $f(n) = cn$ for any natural
    number $n$.

    Since $f(x) + f(0) = f(x + 0) = f(x)$, it follows that $f(0) = 0$. Since
    $f(x) + f(-x) = f(x + (-x)) = f(0) = 0$, it also follows that $f(x) =
    -f(-x)$. In particular, for any natural number $n$, $f(-n) = -f(n) = -cn =
    c(-n)$.

    Moreover, $nf(1/n) = f(1) = c$ which implies that $f(1/n) = c(1/n)$.
    Consequently, $f(-1/n) = -f(1/n) = c(-1/n)$. Finally, writing any rational
    number as $m/n$ with a natural number $m$ and integer $n$, $f(m/n) =
    mf(1/n) = mc/n = c(m/n)$. This proves the statement $f(x) = cx$ for all
    rational numbers $x$.
\end{enumerate}

\paragraph{Problem 17.}
\begin{enumerate}[(a)]
  \item From the property $f(x \cdot y) = f(x) \cdot f(y)$, \[
      f(1) = f(1 \cdot 1) = f(1) \cdot f(1)
    \] which implies $f(1) = 0, 1$. However, if $f(1) = 0$, then for any $y$,
    $f(y) = f(1 \cdot y) = f(1) \cdot f(y) = 0$ which implies that $f(y) = 0$
    for all $y$ contrary to the assumption made. Hence, $f(1) = 1$.
  \item Let $c = f(1)$. The proof that $f(x) = cx$ for all rational $x$ is
    similar to that in Q16. Since $c = 1$, the statement is proven.
  \item If $x > 0$, then $x = y^2$ for some real $y$, so $f(x) = f(y)^2 =
    (f(y))^2 \geq 0$ by definition. Since $y \neq 0$ (from (a)), it must follow
    that $f(y) \neq 0$ and thus $f(x) > 0$.
  \item If $x > y$, then $x - y > 0$. By (c), $f(x) - f(y) > 0$, which proves
    $f(x) > f(y)$.
  \item Suppose that $f(x) > x$ for some $x$. Choose a rational number $r$ with
    $x < r < f(x)$. Then, by (b) and (d), \[
      f(x) < f(r) = r < f(x)
    \] which is a contradiction. Similarly, if $f(x) < x$ for some $x$, by
    choosing a rational number $r'$ with $f(x) < r' < x$, by (b) and (d), \[
      r' = f(r') < f(x) < r'
    \] which is a contradiction. Hence, $f(x) = x$ for all $x$.
\end{enumerate}

\paragraph{Problem 18} If either $f = 0$ or $g = 0$ holds, and also either $h =
0$ or $k = 0$, then the equation certainly holds. If not, then there exists
some $x'$ with $f(x') \neq 0$ and some $y'$ with $g(y') \neq 0$. Then, $0 \neq
f(x')g(y') = h(x')k(y')$, so we have $h(x') \neq 0$ and $k(y') \neq 0$. Letting
$\alpha = h(x')/f(x')$, $g(y) = \alpha k(y)$ for all $y$. Similarly, from
$\alpha = g(y')/k(y')$, $h(x) = \alpha f(x)$ for all $x$. Hence, $h = \alpha f$
and $g = \alpha k$ for some $\alpha \neq 0$.

\paragraph{Problem 19}
\begin{enumerate}[(a)]
  \item Assuming that there exist functions meeting the first
    condition, $f(x) + g(0) = x \cdot 0 = 0$ for all $x$, so $f(x) = -g(0)$.
    Likewise, $f(0) + g(y) = 0 \cdot y = 0$ for all $y$, so $g(y) = -f(0) =
    g(0)$. This would imply that $f(x) + g(y) = -g(0) + g(0) = 0 = xy$ for all
    $x$ and $y$, which is absurd.

    Assuming that there exist functions meeting the second condition, $f(x)
    \cdot g(0) = x + 0 = x$ for all $x$, so $f(x) = x/g(0)$. Likewise, $f(0)
    \cdot g(y) = 0 + y = y$ for all $y$, so $g(y) = y/f(0)$. This would imply
    that $f(x) \cdot g(y) = x/g(0) \cdot y/f(0) = x + y$ for all $x$ and $y$.
    For $y = 0$, $x = 0$ for all $x$ which is absurd.
  \item Let $f$ and $g$ be the same constant function.
\end{enumerate}

\paragraph{Problem 20}
\begin{enumerate}[(a)]
  \item Since $0 \leq |f(y) - f(x)| \leq |y - x|$, the function $f(x) = cx$ for
    some $0 < c \leq 1$ will satisfy the condition.
  \item Suppose that $f(y) < f(x)$. Naturally, $f(y) - f(x) < 0 \leq
    (y - x)^2$. Swapping $x$ and $y$, for $f(x) < f(y)$, the statement $0 \leq
    f(x) - f(y) < (x - y)^2$ must also hold. Hence, $|f(y) - f(x)| = |f(x) -
    f(y)| \leq (y - x)^2$ for all $x$ and $y$.

    Let $m = \frac{x + y}{2}$. Then, $f(y) - f(m) \leq (y - m)^2 = \left( y -
    \frac{x + y}{2} \right)^2 = \left( \frac{y - x}{2}\right)^2 = \frac{1}{4}
    (y - x)^2$. Similarly, $f(m) - f(x) \leq (m - x)^2 = \left( \frac{x + y}{2}
    - x \right)^2 = \left( \frac{y - x}{2} \right)^2 = \frac{1}{4}(y - x)^2$.
    This suggests that $f(y) - f(x) \leq \frac{1}{2}(y - x)^2$, providing a
    stricter bound than $(y - x)^2$.

    The same process can be repeated indefinitely, with the bound tending
    towards 0 while remaining positive. Hence, $f(y) - f(x) = 0$ and $f$ is a
    constant function.
\end{enumerate}

\paragraph{Problem 22}
\begin{enumerate}[(a)]
  \item Consider $g(x) = h(f(x))$. Since $f(x) = f(y)$, $g(x) = h(f(y))$.
    However, $g(y) = h(f(y))$ too. Hence, $g(x) = g(y)$.
  \item If $z = f(x)$, define $h(z) = g(x)$; this definition makes sense, as if
    $z = f(x')$, then $g(x) = g(x')$ by part (a). For $z$ not of the form
    $f(x)$, leave $h$ undefined. Then for all $x$ in the domain of $f$, we have
    $g(x) = h(f(x))$, proving $g = h \circ f$.
\end{enumerate}

\paragraph{Problem 23}
\begin{enumerate}[(a)]
  \item Suppose $x \neq y$. Then, $g(x) = g(y)$ would imply $x = f(g(x)) =
    f(g(y)) = y$ leading to a contradiction.
  \item From $b = f(g(b))$, let $a = g(b)$.
\end{enumerate}

\paragraph{Problem 24}
\begin{enumerate}[(a)]
  \item The hypothesis can be stated as the contrapositive: If $x = y$, then
    $g(x) = g(y)$. Then, applying Q22(b) to $g$ and $I$, $I = h \circ g$ for
    some function $h$.
  \item For each $x$, pick some number $a$ such that $x = f(a)$. Call this
    number $g(x)$. Then, $f(g(x)) = x = I(x)$.
\end{enumerate}

\paragraph{Problem 25} It suffices to find a function $f$ such that $f(x) \neq
f(y)$ if $x \neq y$ (as by Q24(a) there will be a function $g$ such that $g
\circ f = I$), but such that not every number is of the form $f(x)$ (as by
Q22(b) there will not be a function $g$ such that $f \circ g = I$). One such
function is: \[
  f(x) =
  \begin{cases}
    x + 1 & \text{ if } x \geq 0 \\
    x     & \text{ if } x < 0
  \end{cases}
\] which ensures that no number in the range $(0, 1)$ is of the form $f(x)$.

\paragraph{Problem 26} Consider $h \circ (f \circ g) = h \circ I = h$. By the
associative property of composition, as $(h \circ f) \circ g = I \circ g = g$,
this proves $h = g$.

\paragraph{Problem 27}
\begin{enumerate}[(c)]
  \item Let $f$ and $g$ be two functions which are 0 except at $x_0$ and $x_1$,
    with $f(x_0) = 1$, $f(x_1) = 0$ and $g(x_0) = 0$, $g(x_1) = 1$. Taking P10,
    neither $f$ nor $g$ is 0, so $f$ or $-f$ would have to be in the collection
    $P$, and likewise for $g$ or $-g$. But $(\pm f)(\pm g) = 0$, which
    contradicts P12.
\end{enumerate}

\end{document}

