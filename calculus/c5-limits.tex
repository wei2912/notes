\documentclass{article}

\usepackage{amsmath,amssymb,amsthm}

\newtheorem{definition}{Definition}
\newtheorem{theorem}{Theorem}
\newtheorem{lemma}{Lemma}

\begin{document}

\title{Chapter 5: Limits}
\maketitle

\begin{definition}[Limit]
  The function \emph{$f$ approaches the limit $l$ near $a$} means: for every
  $\epsilon > 0$ there is some $\delta > 0$ such that, for all $x$, if $0 < |x
  - a| < \delta$, then $|f(x) - l| < \epsilon$.
\end{definition}
This is denoted later on by $\lim_{x \rightarrow a} f(x) = l$. One-sided limits
can be defined similarly, and use the notation $\lim_{x \rightarrow a^+} f(x) =
l$ or $\lim_{x \downarrow a} f(x) = l$ for limits from above, and $\lim_{x
\rightarrow a^-} f(x) = l$ or $\lim_{x \uparrow a} f(x) = l$.

\begin{theorem}
  A function cannot approach two different limits near $a$. In other words, if
  $f$ approaches $l$ near $a$, and $f$ approaches $m$ near $a$, then $l = m$.
\end{theorem}
\begin{proof}
  Since this is our first theorem about limits it will certainly be necessary
  to translate the hypotheses according to the definition.

  Since $f$ approaches $l$ near $a$, we know that for any $\epsilon > 0$ there
  is some number $\delta_1 > 0$ such that, for all $x$, \begin{equation*}
    \text{if } 0 < |x - a| < \delta_1, \text{ then } |f(x) - l| < \epsilon.
  \end{equation*} 
  We also know, since $f$ approaches $m$ near $a$, that there is some $\delta_2
  > 0$ such that, for all $x$, \begin{equation*}
    \text{if } 0 < |x - a| < \delta_2, \text{ then } |f(x) - m| < \epsilon.
  \end{equation*}
  We have had to use two numbers, $\delta_1$ and $\delta_2$, since there is no
  guarantee that the $\delta$ which works in one definition will work in the
  other. But, in fact, it is now easy to conclude that for any $\epsilon > 0$
  there is some $\delta > 0$ such that, for all $x$, \begin{equation*}
    \text{if } 0 < |x - a| < \delta, \text{ then } |f(x) - l| < \epsilon
      \text{ and } |f(x) - m| < \epsilon;
  \end{equation*} we simply choose $\delta = \mathrm{min}(\delta_1, \delta_2)$.

  To complete the proof we just have to pick a particular $\epsilon > 0$ for
  which the two conditions \begin{equation*}
    |f(x) - l| < \epsilon\text{ and }|f(x) - m| < \epsilon
  \end{equation*} cannot both hold, if $l \neq m$. If $l \neq m$, so that $|l -
  m| > 0$, we can choose $|l - m|/2$ as our $\epsilon$. It follows that there
  is a $\delta > 0$ such that, for all $x$, \begin{align*}
    \text{if } 0 < |x - a| < \epsilon,
      &\text{ then } |f(x) - l| < \frac{|l - m|}{2} \\
      &\text{ and } |f(x) - m| < \frac{|l - m|}{2}.
  \end{align*}
  This implies that for $0 < |x - a| < \delta$ we have \begin{align*}
    |l - m| = |l - f(x) + f(x) - m| &\leq |l - f(x)| + |f(x) - m| \\
      &< \frac{|l - m|}{2} + \frac{|l - m|}{2} \\
      &= |l - m|,
  \end{align*} a contradiction.
\end{proof}

\begin{lemma}
  \begin{enumerate}
    \item If \begin{equation*}
        |x - x_0| < \frac{\epsilon}{2} \text{ and } |y - y_0| <
          \frac{\epsilon}{2},
      \end{equation*} then \begin{equation*}
        |(x + y) - (x_0 + y_0)| < \epsilon.
      \end{equation*}
    \item If \begin{equation*}
        |x - x_0| < \mathrm{min}\left(1, \frac{\epsilon}{2(|y_0| + 1)}\right)
          \text{ and } |y - y_0| < \frac{\epsilon}{2(|x_0| + 1)},
      \end{equation*} then \begin{equation*}
        |xy - x_0y_0| < \epsilon.
      \end{equation*}
    \item If $y_0 \neq 0$ and \begin{equation*}
        |y - y_0| < \mathrm{min}\left(\frac{|y_0|}{2},
          \frac{\epsilon|y_0|^2}{2}\right),
      \end{equation*} then $y \neq 0$ and \begin{equation*}
        \left|\frac{1}{y} - \frac{1}{y_0}\right| < \epsilon.
      \end{equation*}
  \end{enumerate}
\end{lemma}

\begin{proof}
  The proof has been omitted as the statements were proven in Problem 1-20,
  1-21, and 1-22.
\end{proof}

\begin{theorem}
  If $\lim_{x \rightarrow a}f(x) = l$ and $\lim_{x \rightarrow a}g(x) = m$,
  then \begin{enumerate}
    \item $\lim_{x \rightarrow a}(f + g)(x) = l + m;$
    \item $\lim_{x \rightarrow a}(f \cdot g)(x) = l \cdot m.$
  \end{enumerate} Moreover, if $m \neq 0$, then \begin{enumerate}
    \setcounter{enumi}{2}
    \item $\lim_{x \rightarrow a}\left(\frac{1}{g}\right)(x) = \frac{1}{m}.$
  \end{enumerate}
\end{theorem}

\begin{proof}
  The hypothesis means that for every $\epsilon > 0$ there are $\delta_1,
  \delta_2 > 0$ such that, for all $x$, \begin{align*}
    &\text{if } 0 < |x - a| < \delta_1, \text{ then } |f(x) - l| < \epsilon, \\
    \text{and } &\text{if } 0 < |x - a| < \delta_2, \text{ then } |g(x) - m| <
      \epsilon.
  \end{align*} This means (since, after all, $\epsilon/2$ is also a positive
  number) that there are $\delta_1, \delta_2 > 0$ such that, for all $x$,
  \begin{align*}
    &\text{if } 0 < |x - a| < \delta_1, \text{ then } |f(x) - l| <
      \frac{\epsilon}{2}, \\
    \text{and } &\text{if } 0 < |x - a| < \delta_2, \text{ then } |g(x) - m| <
      \frac{\epsilon}{2}.
  \end{align*} Now let $\delta = \mathrm{min}(\delta_1, \delta_2)$. If $0 < |x
  - a| < \delta$, then $0 < |x - a| < \delta_1$ and $0 < |x - a| < delta_2$ are
  both true, so both \begin{equation*}
    |f(x) - l) < \frac{\epsilon}{2} \text{ and } |g(x) - m| <
      \frac{\epsilon}{2}
  \end{equation*} are true. But by part (1) of the lemma this implies that $|(f
  + g)(x) - (l + m)| < \epsilon$. This proves (1).

  To prove (2) we proceed similarly, after consulting part (2) of the lemma. If
  $\epsilon > 0$ there are $\delta_1, \delta_2 > 0$ such that, for all $x$,
  \begin{align*}
    &\text{if } 0 < |x - a| < \delta_1, \text{ then } |f(x) - l| < 
      \mathrm{min}\left(1, \frac{\epsilon}{2(|m| + 1)}\right), \\
    \text{and } &\text{if } 0 < |x - a| < \delta_2, \text{ then } |g(x) - m| <
      \frac{\epsilon}{2(|l| + 1)}.
  \end{align*} Again let $\delta = \mathrm{min}(\delta_1, \delta_2)$. If $0 <
  |x - a| < \delta$, then \begin{equation*}
    |f(x) - l| < \mathrm{min}\left(1, \frac{\epsilon}{2(|m| + 1)}\right)
      \text{ and } |g(x) - m| < \frac{\epsilon}{2(|l| + 1)}.
  \end{equation*} So, by the lemma, $|(f \cdot g)(x) - l \cdot m| < \epsilon|$,
  and this proves (2).

  Finally, if $\epsilon > 0$ there is a $\delta > 0$ such that, for all $x$,
  \begin{equation*}
    \text{if } 0 < |x - a| < \delta, \text{ then } |g(x) - m| <
      \mathrm{min}\left(\frac{|m|}{2}, \frac{\epsilon|m|^2}{2}\right).
  \end{equation*} But according to part (3) of the lemma this means, first,
  that $g(x) \neq 0$, so $(1/g)(x)$ makes sense, and second that
  \begin{equation*}
    \left|\left(\frac{1}{g}\right)(x) - \frac{1}{m}\right| < \epsilon.
  \end{equation*} This proves (3).
\end{proof}

\section{Exercises}

\setcounter{subsection}{13}
\subsection{Limit when divisor is $x$}

\paragraph{Problem (a) (abridged).} Prove that if $\lim_{x \rightarrow 0}
f(x)/x = l$ and $b \neq 0$, then $\lim_{x \rightarrow 0}f(bx)/x = bl$.

\paragraph{Solution (a).} By bringing out the constant $b$, \begin{equation*}
  \lim_{x \rightarrow 0}\frac{f(bx)}{x} = \lim_{x \rightarrow 0}
    \frac{bf(bx)}{bx} = b\lim_{x \rightarrow 0}\frac{f(bx)}{bx} =
    b\lim_{y \rightarrow 0}\frac{f(y)}{y} = bl
\end{equation*} with $y = bx$.
The next to last equality can be justified as follows: If $\epsilon > 0$ there
is a $\delta > 0$ such that if $0 < |y| < \delta$, then $\left|\frac{f(y)}{y} -
l\right| < \epsilon$. Then if $0 < |x| < \frac{\delta}{b}$, we have $0 < |bx| <
\delta$, so $\left|\frac{f(bx)}{bx} - l\right| < \epsilon$.

\paragraph{Problem (b).} What happens if $b = 0$?

\paragraph{Solution (b).} In this case, $\lim_{x \rightarrow 0} (bx)/x =
\lim_{x \rightarrow 0} f(0)/x$ does not exist as no upper bound can be derived
for $|bx|$, unless $f(0) = 0$.

\paragraph{Problem (c).} Part (a) enables us to find $\lim_{x \rightarrow 0}
(\sin{x})/x$. Find this limit in another way.

\paragraph{Solution (c).} One could make use of $\sin{2x} = 2\sin{x}\cos{x}$ in
order to derive the limit: \begin{equation*}
  \lim_{x \rightarrow 0}\frac{\sin{2x}}{x} = \lim_{x \rightarrow 0}\frac{2
    \sin{x}\cos{x}}{x} = 2\lim_{x \rightarrow 0}\cos{x}\lim_{x \rightarrow 0}
    \frac{\sin{x}}{x} = 2\lim_{x \rightarrow 0}\frac{\sin{x}}{x}.
\end{equation*}

\setcounter{subsection}{19}
\subsection{Limit of a weird function I}

\paragraph{Problem.} Prove that if $f(x) = x$ for rational $x$, and $f(x) = -x$
for irrational $x$, then $\lim_{x \rightarrow a}f(x)$ does not exist if $a \neq
0$.

\paragraph{Solution.} Consider, for simplicity, the case $a > 0$. Then, for any
choice of $\delta > 0$, there are $x$ with $0 < |x - a| < \delta$ and $f(x) >
a/2$, as well as $x$ with $0 < |x - a| < \delta$ and $f(x) < -a/2$. Since the
distance between $a/2$ and $-a/2$ is $a$, this means that we cannot have $|f(x)
- l| < a$ for all such $x$ with $0 < |x - a| < \delta$, no matter what $l$ is.

\setcounter{subsection}{22}
\subsection{Limit of product of functions (abridged)}

\paragraph{Problem (a).} Suppose that $\lim_{x \rightarrow 0}f(x)$ exists and
is $\neq 0$. Prove that if $\lim_{x \rightarrow 0}g(x)$ does not exist, then
$\lim_{x \rightarrow 0}f(x)g(x)$ also does not exist.

\paragraph{Solution (a).} Should $\lim_{x \rightarrow 0}f(x)g(x)$ exist, then
$\lim_{x \rightarrow 0}g(x) = \lim_{x \rightarrow 0}f(x)g(x)/f(x)$ must also
exist.

\paragraph{Problem (b).} Prove the same result if $\lim_{x \rightarrow 0}|f(x)|
= \infty$.

\paragraph{Solution (b).} Clearly, if $\lim_{x \rightarrow 0}f(x)g(x)$ exists,
then $\lim_{x \rightarrow 0}g(x) = 0$.

\paragraph{Problem (c).} Prove that if neither of these two
conditions holds, then there is a function $g$ such that
$\lim_{x \rightarrow 0}g(x)$ does not exist, but $\lim_{x \rightarrow 0}f(x)
g(x)$ does exist.

Hint: Consider separately the following two cases: (1) for some $\epsilon > 0$
we have $|f(x)| > \epsilon$ for all sufficiently small $x$. (2) For every
$\epsilon > 0$, there are arbitrarily small $x$ with $|f(x)| < \epsilon$. In
the second case, begin by choosing points $x_n$ with $|x_n| < 1/n$ and
$|f(x_n)| < 1/n$.

\paragraph{Solution (c).} For case (1), it is clear that we cannot have
$\lim_{x \rightarrow 0}f(x) = 0$, so by the assumption in (a) the limit
$\lim_{x \rightarrow 0}f(x)$ cannot exist. Let $g = 1/f$. Since it is not true
that $\lim_{x \rightarrow 0}|f(x)| = \infty$, it follows that if $\lim_{x
\rightarrow 0}g(x)$ does exist, then $\lim_{x \rightarrow 0}g(x) \neq 0$. But
by (a) this would imply that $\lim_{x \rightarrow 0}f(x)$ exists, so $\lim_{x
\rightarrow 0}g(x)$ does not exist. On the other hand, it is clear that
$\lim_{x \rightarrow 0}f(x)g(x) = 1$.

For case (2), choose $x_n$ as in the hint. Define $g(x) = 0$ for $x \neq x_n$,
and $g(x) = 1$ for $x = x_n$. Then $\lim_{x \rightarrow 0}g(x)$ does not exist,
but $\lim_{x \rightarrow 0}f(x)g(x) = 0$.

\setcounter{subsection}{23}
\subsection{Limit of a weird function II}

\paragraph{Problem.} Suppose that $A_n$ is, for each natural number $n$, some
\emph{finite} set of numbers in $[0, 1]$, and that $A_n$ and $A_m$ have no
members in common if $m \neq n$. Define $f$ as follows: \begin{equation*}
  f(x) = \begin{cases}
    1/n, &x \text{ in } A_n \\
    0, &x \text{ not in } A_n \text{ for any } n.
  \end{cases}
\end{equation*} Prove that $\lim_{x \rightarrow a}f(x) = 0$ for all $a$ in $[0,
1]$.

\paragraph{Solution.} Given $\epsilon > 0$, pick $n$ with $1/n < \epsilon$ and
let $\delta$ be the minimum distance from $a$ to all points in $A_1, \ldots,
A_n$ (except $a$ itself if $a$ is one of these points). Then, $0 < |x - a| <
\delta$ implies that $x$ is not in $A_1, \ldots, A_n$, so $f(x) = 0$ or $1/m$
for $m > n$, so $|f(x)| < \epsilon$.

\setcounter{subsection}{30}
\subsection{One-sided limits}

\paragraph{Problem (abridged).} Suppose that $\lim_{x \rightarrow a^-}f(x) <
\lim_{x \rightarrow a^+}f(x)$. Prove that there is some $\delta > 0$ such that
$f(x) < f(y)$ whenever $x < a < y$ and $|x - a| < \delta$ and $|y - a| < \delta
$. Is the converse true?

\paragraph{Solution.} Let $l = \lim_{x \rightarrow a^-}f(x)$ and $m = \lim_{x
\rightarrow a^+}f(x)$. Since $m - l > 0$, there is a $\delta > 0$ such that
\begin{align*}
  |f(x) - l| < \frac{m - l}{2} &\text{ when } a - \delta < x < a, \\
  |f(y) - m| < \frac{m - l}{2} &\text{ when } a < y < a + \delta.
\end{align*}
\begin{equation*}
  f(x) < l + \frac{m - l}{2} = \frac{l}{2} = m - \frac{m - l}{2} < f(y)
\end{equation*}

The converse is false, as shown by $f(t) = t$ and any $a$. It is only possible
to conclude that $\lim_{x \rightarrow a^-}f(x) \leq \lim_{x \rightarrow a^+}
f(x)$.

\setcounter{subsection}{31}
\subsection{Limits of rational fractions}

\paragraph{Problem.} Prove that $\lim_{x \rightarrow \infty}(a_nx^n + \cdots +
a_0)/(b_mx^m + \cdots + b_0)$ exists if and only if $m \geq n$. What is the
limit when $m = n$? When $m > n$? Hint: the one easy limit is $\lim_{x
\rightarrow \infty}1/x^k = 0$; do some algebra so that this is the only
information you need.

\paragraph{Solution.} Naturally, we assume that $a_n, b_m \neq 0$. If $x \neq
0$ and $m < n$, we write \begin{equation*}
  \lim_{x \rightarrow \infty}\frac{a_nx^n + \cdots + a_0}{b_mx^m + \cdots + b_0
    } = \lim_{x \rightarrow \infty}\frac{a_n + \cdots + \frac{a_0}{x^n}}{\frac{
    b_m}{x^{n - m}} + \cdots + \frac{b_0}{x^n}} = \frac{f(x)}{g(x)}
\end{equation*} for $f(x) = a_n + \cdots + \frac{a_0}{x^n}$ and $g(x) = \frac{
b_m}{x^{n - m}} + \cdots + \frac{b_0}{x^n}$. With $m < n$, it is clear that
$\lim_{x \rightarrow \infty}f(x) = a_n$ while $\lim_{x \rightarrow \infty}g(x)
= 0$, which implies that $\lim_{x \rightarrow \infty}f(x)/g(x)$ cannot exist,
as if it were to exist, then \begin{equation*}
  \lim_{x \rightarrow \infty}f(x) = \lim_{x \rightarrow \infty}\frac{f(x)}{g(x)
  } \cdot \lim_{x \rightarrow \infty}g(x) = 0
\end{equation*} which is contradictory.

If $x \neq 0$ and $m \geq n$, we write \begin{equation*}
  \lim_{x \rightarrow \infty}\frac{a_nx^n + \cdots + a_0}{b_mx^m + \cdots + b_0
    } = \lim_{x \rightarrow \infty}\frac{\frac{a_n}{x^{m - n}} + \cdots +
    \frac{a_0}{x^m}}{b_m + \cdots + \frac{b_0}{x^m}} = \frac{f(x)}{g(x)}
\end{equation*} for $f(x) = \frac{a_n}{x^{m - n}} + \cdots + \frac{a_0}{x^m}$
and $g(x) = b_m + \cdots + \frac{b_0}{x^m}$. Then $\lim_{x \rightarrow \infty}
f(x) = a_n$ if $m = n$ and 0 if $m > n$, while $\lim_{x \rightarrow \infty}g(x)
= b_m$ for $m \geq n$. This leads to $\lim_{x \rightarrow \infty}\frac{f(x)}{
g(x)} = \frac{a_n}{b_m}$ if $m = n$ and 0 if $m > n$.

\end{document}

