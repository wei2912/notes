\documentclass{article}

\usepackage{amsmath,amssymb,amsthm}

\newtheorem{theorem}{Theorem}
\newtheorem{lemma}{Lemma}

\begin{document}

\title{Week 1: Stream Ciphers}
\maketitle

\section{Symmetric Ciphers}

\paragraph{Symmetric Ciphers.} A symmetric cipher defined over $(K, M, C)$ is a
pair of algorithms $(E, D)$ where \begin{gather*}
  E: K \times M \rightarrow C, \\
  D: K \times C \rightarrow M,
\end{gather*} such that for all $m \in M$ and $k \in K$, $D(k, E(k, m)) = m$.

\subsection{One Time Pad (OTP)}

\paragraph{One Time Pad (OTP).} For some random key $k$, \begin{gather}
  c = E(k, m) = k \oplus m, \label{otp:enc} \\
  m = D(k, c) = k \oplus c. \label{otp:dec}
\end{gather}

\paragraph{Retrieval of Key.} For some plaintext-ciphertext pair $(m, c)$,
\begin{align*}
  m \oplus c &= (k \oplus c) \oplus c &&\text{(from \eqref{otp:dec})} \\
    &= k \oplus (c \oplus c) \\
    &= k \oplus 0 \\
    &= k.
\end{align*}

Hence, the key can be retrieved from a plaintext-ciphertext pair.

\paragraph{Proof that OTP is a Cipher.} For some key-plaintext pair $(k, m)$,
\begin{align*}
  D(k, E(k, m)) &= D(k, k \oplus m) &&\text{(from \eqref{otp:enc})} \\
    &= k \oplus (k \oplus m) &&\text{(from \eqref{otp:dec})} \\
    &= (k \oplus k) \oplus m \\
    &= 0 \oplus m \\
    &= m.
\end{align*}

Hence, OTP is a cipher.

\subsection{Information Theoretic Security.} A cipher $(E, D)$ over $(K, M, C)$
has perfect secrecy if and only if: \begin{enumerate}
  \item $\forall m_0, m_1 \in M, |m_0| = |m_1|$
  \item $\forall c \in C, P[E(k, m_0) = c] = P[E(k, m_1) = c]$ where $k$ is
  randomly drawn from $K$.
\end{enumerate}

No information is revealed about the plaintext from the ciphertext alone.
However, $|K| \geq |M|$!

\begin{lemma}
  OTP has perfect secrecy.
\end{lemma}

\begin{proof}
  For a randomly drawn $k$ from $K$, $$P[E(k, m) = c] = \frac{|{k \in K: E(k,
  m) = c}|}{|K|}$$ for all $m \in M$, $c \in C$.

  In a OTP, $|\{k \in K: E(k, m) = c\}| = 1$, which implies that $P[E(k, m_0) =
  c] = P[E(k, m_1) = c] = \frac{1}{|k|}$.
\end{proof}

\subsection{Pseudorandom Generators (PRGs)}

\paragraph{Pseudorandom Generators (PRGs).} A function $G: \{0, 1\}^s
\rightarrow \{0, 1\}^n$ for positive integers $s$, $n$, such that $n \gg s$.
$G$ must be efficiently computable and output a "random" string of bits.

$G$ is predictable if and only if there exists an efficient algorithm $A$ and
an integer $1 \leq i \leq n-1$ such that \begin{equation}
  P[A(G(k)|_{1,2,\ldots,i}) = G(k)|_{i+1}] \geq \frac{1}{2} + \epsilon
  \label{prg:pred}
\end{equation} for some non-negligible $\epsilon$ (roughly $\epsilon \geq
1/2^{30}$).

\paragraph{Non-negligible.} Let $\epsilon: \mathbb{Z}^+_0 \rightarrow
\mathbb{R}^+_0$. For a non-negligible $\epsilon$, there exists a $d$ such that
$\epsilon(\lambda) \geq 1/\lambda^d$ for some sufficiently large $\lambda$.

\subsection{Attacks on OTP}

\paragraph{Two Time Pad.} Given two plaintexts encrypted with the same key,
$c_1 = m_1 \oplus k$, $c_2 = m_2 \oplus k$, then $c_1 \oplus c_2 = m_1 \oplus
m_2$. There is enough redundancy in English and ASCII encoding to retrieve
$m_1$ and $m_2$.

\paragraph{Malleability.} Given a ciphertext $c = E(k, m)$ and a modification
$p$, $D(k, c \oplus p) = m \oplus p$, resulting in modified contents.

\subsection{Secure PRG}

\paragraph{Statistical Test.} A statistical test is an algorithm that returns
`0' if the input is not random and `1' if the input is random.

\paragraph{Advantage.} Let $G: K \rightarrow \{0, 1\}^n$ be a PRG and $A$ be a
statistical test on $\{0, 1\}^n$. Then, \begin{equation}
  \text{Adv}_\text{PRG}[A, G] = |P[A(G(k)) = 1] - P[A(r) = 1]| \in [0, 1]
  \label{prg:adv}
\end{equation} where $k$ is randomly drawn from $K$ and $r$ is randomly drawn
from $\{0, 1\}^n$. If the advantage of the adversary is close to 1, $A$ can
distinguish $G$ from random.

\paragraph{Secure PRG.} A PRG is secure if and only if for all efficient
statistical tests $A$, $\text{Adv}_\text{PRG}[A, G]$ is negligible. No provably
secure PRG exists (unless $P = NP$).

\begin{lemma}
  A secure PRG is unpredictable.
\end{lemma}

\begin{proof}
  Suppose there exists an efficient algorithm $A$ such that a PRG $G$ is
  predictable (i.e. \eqref{prg:pred} holds). Define statistical test $B$ such
  that \begin{equation}
    B(X) = \begin{cases}
      0 \text{ if } A(X|_{1,2,\ldots,i}) = X_{i+1} \\
      1 \text{ otherwise}
    \end{cases}
  \end{equation}
  For a $r$ randomly drawn from $\{0, 1\}^n$, $P[B(r) = 1] = \frac{1}{2}$,
  whereas for a $r$ randomly drawn from $K$, $P[B(G(k)) = 1] \geq \frac{1}{2} +
  \epsilon$. Hence, $\text{Adv}_\text{PRG}[B, G] \geq \epsilon$.
\end{proof}

\begin{theorem}
  An unpredictable RPG is secure. (Yao, 1982)
\end{theorem}

\begin{proof}
  Proof by contradiction. If for all $i \in \{0,1,\ldots,n-1\}$, a PRG $G$ is
  unpredictable at position $i$, then $G$ is a secure PRG. (If next-bit
  predictors cannot distinguish $G$ from random, no statistical test can.)
\end{proof}

\paragraph{Computationally Indistinguishable.} For all efficient statistical
tests $A$, drawing $x_1$ and $x_2$ from distribution $P_1$ and $P_2$ over $\{0,
1\}$ respectively, the distributions are computationally indistinguishable if
they satisfy \begin{equation}
  |P[A(x_1) = 1] - P[A(x_2) = 1]| < \epsilon
\end{equation} for some negligible $\epsilon$. A secure PRG is computationally
indistinguishable from the uniform distribution.

\paragraph{Semantically Secure.} $E$ is semantically secure if for all
efficient $A$, $\text{Adv}_\text{SS}[A,E]$ is negligible.

\begin{theorem}
  A stream cipher $E$ derived from a secure PRG $G$ is semantically secure.
\end{theorem}

For all semantically secure $E$, there exists a PRG adversary $B$ such that
$\text{Adv}_\text{SS}[A, E] \leq 2\text{Adv}_\text{PRG}[B, E]$ for some
efficient algorithm $A$.

\end{document}

